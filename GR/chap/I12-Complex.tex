% !TeX encoding = UTF-8

\chapter{复几何初步}\label{chcx}

%本章初步介绍复流形和复几何.
我们先回顾一些复变函数的基本概念;
然后补充一些复线性空间的知识;
之后,引入复结构、线性空间的复化;
最后介绍复流形和殆复流形.


\section{复变函数极简概要}\label{chcx:sec_cxf}
本节回顾一些复变函数基本内容.%这些知识可在任意一本复变函数教科书中查到.

在复平面某子区域$D$中,如果单值、有限复变函数
\begin{equation}\label{chcx:eqn_complex-function}
    f(z)\equiv f(x,y) = u(x,y) + \mathbbm{i} v(x,y); \quad
    z\equiv x+ \mathbbm{i} y, \ \forall x,y\in \mathbb{R}.
\end{equation}
其中$u$、$v$是实函数.
每一点都是可微的(或可导),
那么称$f(z)$是$D$内的{\heiti 解析函数}或{\heiti 全纯函数}(holomorphic).
实或复函数在$p$点解析是指存在$p$点小邻域使得在此邻域内该函数的幂级数是收敛的.

\index[physwords]{全纯函数}

函数$f(z)$可微的充分必要条件是:在区域$D$中,满足{\heiti 柯西-黎曼条件}:
\begin{equation}\label{chcx:eqn_Cauchy-Riemann}
    \frac{\partial u}{\partial x} = \frac{\partial v}{\partial y},\qquad
    \frac{\partial u}{\partial y} = -\frac{\partial v}{\partial x} .
\end{equation}

利用柯西-黎曼条件可将$f(z)$的导数表示成如下形式
\begin{equation}
    \frac{{\rm d} f(z)}{{\rm d}z}=\frac{\partial u}{\partial x}+\mathbbm{i}\frac{\partial v}{\partial x}
    =\frac{\partial u}{\partial x} - \mathbbm{i}\frac{\partial u}{\partial y}
    =\frac{\partial v}{\partial y} - \mathbbm{i}\frac{\partial u}{\partial y}
    =\frac{\partial v}{\partial y} + \mathbbm{i}\frac{\partial v}{\partial x}.
\end{equation}

我们有如下变换关系
\begin{equation}
    z= x+ \mathbbm{i} y,\quad \bar{z}= x - \mathbbm{i} y
    \quad \Leftrightarrow \quad
    x = (z+\bar{z})/2, \quad y=(z-\bar{z})/2\mathbbm{i}  .
\end{equation}
由上式容易得到
\begin{align}
    \frac{\partial }{\partial z}= &
       \frac{\partial }{\partial x} \frac{\partial x}{\partial z}
      +\frac{\partial }{\partial y} \frac{\partial y}{\partial z}
      =\frac{1}{2}\left( \frac{\partial }{\partial x} 
        -\mathbbm{i}\frac{\partial }{\partial y} \right) . \\
     \frac{\partial }{\partial \bar{z}}= &
     \frac{\partial }{\partial x} \frac{\partial x}{\partial \bar{z}}
     +\frac{\partial }{\partial y} \frac{\partial y}{\partial \bar{z}}
     =\frac{1}{2}\left( \frac{\partial }{\partial x} 
     +\mathbbm{i}\frac{\partial }{\partial y} \right) .   
\end{align}
利用上式,容易得到
\begin{equation}\label{chcx:eqn_Cauchy-Riemann-2}
    \frac{\partial f }{\partial \bar{z}}
    =\frac{1}{2}\left( \frac{\partial f}{\partial x} 
    +\mathbbm{i}\frac{\partial f}{\partial y} \right) 
    =\frac{1}{2}\left( \frac{\partial u}{\partial x} -\frac{\partial v}{\partial y}\right) 
    +\frac{\mathbbm{i}}{2}\left(\frac{\partial v}{\partial x} + \frac{\partial u}{\partial y} \right) .
\end{equation}
由式\eqref{chcx:eqn_Cauchy-Riemann}可见:柯西-黎曼条件等价于$\frac{\partial f }{\partial \bar{z}}=0$;
也就是说$f(z)$是$D$上的全纯函数等价于$\frac{\partial f }{\partial \bar{z}}=0$.
这个纯粹形式化的结论大致说明:全纯函数$f(z)$与$\bar{z}$无关,{\kaishu 它仅是$z$的函数}.
故全纯函数确实是{\kaishu 复变量}($z$)的函数,而不是两个实变数($x,y$)的函数.
一般说来含两个实变量的复值函数有如下关系:
\begin{equation}
    f(x,y)=f\left(\frac{z+\bar{z}}{2},\ \frac{z-\bar{z}}{2\mathbbm{i}}\right),
\end{equation}
同时与$z,\bar{z}$有关.

%复变函数在区域$D$中一阶可微,则无穷阶可微.

\section{复线性空间}\label{chcx:sec_cls}
\S\ref{chmla:sec_linear-space}至\S\ref{chmla:sec_double-linear-map}中的内容是建立在一般数域$\mathbb{F}$上的,
那里的结论基本都适用于本节的复线性空间$V$(见定义\eqref{chmla:def_linear-space});
此处,我们补充些内容.

\index[physwords]{复线性空间}

例\ref{chmla:exm_rc1}和例\ref{chmla:exm_rc2}指出了数域不同,空间维数可能不同;
下面命题作了推广.

\begin{proposition}\label{chcx:thm_rc}
    设$V$是数域$\mathbb{C}$上$m$维线性空间.数域$\mathbb{R}$是数域$\mathbb{C}$的一个子集,
    并且$\mathbb{C}$可以看成数域$\mathbb{R}$上的线性空间,其维数是$2$.
    那么$V$可以看成数域$\mathbb{R}$上的$2m$维线性空间.
\end{proposition}
\begin{proof}
由于数域$\mathbb{R}$是数域$\mathbb{C}$的一个子集,故$\mathbb{R}$中元素与$V$中矢量可以
按照$\mathbb{C}$与$V$的数量乘法来作为$\mathbb{R}$与$V$的数量乘法.
这种定义方式满足定义\ref{chmla:def_linear-space}中的8个条件(请读者自行验证);
最终,集合$V$可以看成数域$\mathbb{R}$上的线性空间.

下面求证其维数是$2m$.
作为$\mathbb{C}$上的线性空间$V$,我们取它的一组基底$\{\boldsymbol{e}_1,\cdots,\boldsymbol{e}_m\}$.
$\mathbb{C}$作为域$\mathbb{R}$上的二维空间,我们取定一组基矢$\{\epsilon_1=1,\ \epsilon_2=\mathbbm{i}\}$.
$\forall \alpha \in V$,有
\begin{equation}
    \alpha = \sum_{j=1}^{m} c_j \boldsymbol{e}_j, \qquad
    c_j\in \mathbb{C},\  j=1,2,\cdots,m .
\end{equation}
对于$\mathbb{C}$中的元素$c_j$可以表示为
\begin{equation}
    c_j= a_j \epsilon_1 + b_j \epsilon_2 = a_j \cdot 1 + b_j \cdot \mathbbm{i},
    \qquad a_j,b_j \in \mathbb{R},\quad j=1,2,\cdots,m .
\end{equation}
因此
\begin{equation}
    \alpha= \sum_{j=1}^{m} c_j \boldsymbol{e}_j = \sum_{j=1}^{m} (a_j \epsilon_1 + b_j \epsilon_2) \boldsymbol{e}_j 
    = \sum_{j=1}^{m} \bigl(a_j (\epsilon_1 \boldsymbol{e}_j )+ b_j (\epsilon_2 \boldsymbol{e}_j )\bigr) . 
\end{equation}
上式表明$\forall \alpha \in V$可以由
\begin{equation*}
B_m\equiv\{\epsilon_1 \boldsymbol{e}_1,\cdots, \epsilon_1 \boldsymbol{e}_m;\ 
\epsilon_2 \boldsymbol{e}_1,\cdots, \epsilon_2 \boldsymbol{e}_m\}
= \{\boldsymbol{e}_1,\cdots, \boldsymbol{e}_m;\ 
\mathbbm{i} \boldsymbol{e}_1,\cdots, \mathbbm{i} \boldsymbol{e}_m\}
\end{equation*}
线性表出,且表出系数都是数域$\mathbb{R}$中的常数.
$B_m$更确切的表述是两个基矢量的张量积,
即$B_m\equiv\{\epsilon_1 \otimes \boldsymbol{e}_1,\cdots, \epsilon_1 \otimes \boldsymbol{e}_m;\ 
\epsilon_2 \otimes\boldsymbol{e}_1,\cdots, \epsilon_2 \otimes\boldsymbol{e}_m\}$.

下面用反证法证明集合$B_m$是线性无关的.
假设存在非零实数$a_{j},b_j,\ j=1,\cdots,m$使得
$\sum_{j=1}^{m}\bigl(a_j (\epsilon_1 \boldsymbol{e}_j )+ b_j (\epsilon_2 \boldsymbol{e}_j )\bigr)= 0$;
将此式稍作变化
\begin{equation}
    0=\sum_{j=1}^{m}\bigl(a_j \epsilon_1 + b_j \epsilon_2 \bigr) \boldsymbol{e}_j
     =\sum_{j=1}^{m}\bigl(a_j \cdot 1 + b_j \cdot \mathbbm{i}\bigr) \boldsymbol{e}_j .
\end{equation}
我们已知在数域$\mathbb{C}$上$\{\boldsymbol{e}_1,\cdots,\boldsymbol{e}_m\}$是线性无关的;
因此从上式可以得到
\begin{equation}
    a_j \cdot 1 + b_j \cdot \mathbbm{i}=0\quad \Leftrightarrow \quad
    a_j=0,\ b_j =0; \quad j=1,2,\cdots,m.
\end{equation}
与假设矛盾,故集合$B_m$是线性无关的;
因此它可以作为数域$\mathbb{R}$上线性空间$V$的基矢量,
这也就说明了数域$\mathbb{R}$上线性空间$V$的维数是$2m$.
\end{proof}


\index[physwords]{厄米双线性函数}
\index[physwords]{Hermite函数}

\subsection{厄米双线性函数}
实线性空间中有对称双线性函数及其内积,如果直接将其(见定义\ref{chmla:def_Euclidean-space})推广到复线性空间中,
尤其令复空间内积满足定义\ref{chmla:def_Euclidean-space}中的
第一条:$\left<\alpha,\beta\right>=\left<\beta,\alpha\right>$;
那么在正交归一基矢下,复矢量$\alpha$的内积是
\begin{equation}
    \left<\alpha,\alpha\right> = \sum_{i,j}\left<a^i \boldsymbol{e}_i, a^j\boldsymbol{e}_j\right>
    =\sum_{i,j} a^i a^j \delta_{ij} = \sum_{i=1}^{m} (a^i)^2 .
\end{equation}
注意复矢量$\alpha$的系数$a^i$是复数,上式求和后一般不是实数;
也就是说在复空间中矢量长度无法定义;可见这种内积定义方式必须修改. 

Hermite{\footnote{Charles Hermite(1822-1901)法国数学家,译为厄尔米特,简称厄米.}}
克服了此种困难;我们先叙述厄米双线性函数概念.
\begin{definition}
    $m$维复线性空间$V$上的复值函数$h(\cdot,\cdot)$称为{\bfseries\heiti 厄米双线性函数},
    是指它满足如下条件(下式中$\forall c\in \mathbb{C},\  \forall \alpha,\beta,\gamma\in V$,
    $\bar{c}$是$c$的共轭复数):
    \begin{equation*}
        h(c\cdot \alpha+\beta,\gamma) = \bar{c}\cdot h(\alpha,\gamma)+ h(\beta,\gamma),\quad
        h(\gamma,c\cdot \alpha+\beta) = c\cdot h(\gamma,\alpha)+ h(\gamma,\beta).
    \end{equation*}
\end{definition}
可见厄米{\kaishu 双线性}函数对于第一个位置矢量不是线性的,
此性质称为\CJKunderwave{反线性}或{\heiti 共轭线性};第二个位置矢量是线性的.
这种对称性可称为{\heiti 厄米对称}.

\index[physwords]{厄米线性}
\index[physwords]{厄米对称}

在复线性空间$V$中取定一组基矢$\boldsymbol{e}_i,\cdots,\boldsymbol{e}_m$.
对任意厄米双线性函数,有
\begin{equation}
    h(\alpha,\beta)=\bar{x}^i y^j h(\boldsymbol{e}_i,\boldsymbol{e}_j)
     = \bar{x}^T A y;\qquad A \equiv  h(\boldsymbol{e}_i,\boldsymbol{e}_j). 
\end{equation}
其中$\alpha=\sum_i x^i \boldsymbol{e}_i$,$\beta=\sum_j y^j \boldsymbol{e}_j$,
$x=(x^1,\cdots,x^m)^T$,$y=(y^1,\cdots,y^m)^T$;
$A$称为厄米双线性函数$h$在基矢$\{e_i\}$上的\CJKunderwave{度量矩阵}(metric,也可译成{\kaishu 度规}),
显然它是一个复矩阵.

反之,任意给定一个复矩阵$A$,则
\begin{equation}
    h(\alpha,\beta)\equiv h\left(x^i \boldsymbol{e}_i,\ y^j \boldsymbol{e}_j \right)
    \overset{def}{=} \bar{x}^T A y, \qquad \forall \alpha,\beta\in V
\end{equation}
定义了复空间$V$上的一个厄米双线性函数$h(\cdot,\cdot)$.
因此厄米双线性函数与$m\times m$复数矩阵之间有\CJKunderwave{双射}关系.
%需要注意的是:我们还未要求{\kaishu 度量矩阵}是非退化的.

%\begin{definition}
%    若$m$维复线性空间$V$上的厄米双线性函数$h(\cdot,\cdot)$满足
%    $h(\alpha,\beta)= \overline{h(\beta,\alpha)},\ \forall \alpha,\beta\in V$,
%    则称$h$是{\heiti  厄米函数}.
%    若厄米双线性函数$h(\cdot,\cdot)$满足
%    $h(\alpha,\beta)= -\overline{h(\beta,\alpha)},\ \forall \alpha,\beta\in V$,
%    则称$h$是{\heiti  反厄米函数}.
%\end{definition}


\begin{definition}
    满足$A  = \bar{A}^T$的复数矩阵称为{\heiti 厄米矩阵};
    满足$A = -\bar{A}^T$的复数矩阵称为{\heiti 反厄米矩阵};
    通常记$A^\dagger \equiv \bar{A}^T$,上标$T$表示矩阵转置,横线表示复共轭.
\end{definition}

%很明显在任意一组基底下,厄米函数对应的矩阵是厄米矩阵,
%反厄米函数对应的矩阵是反厄米矩阵.

\begin{example}
    设$A$是任意$m$维复数矩阵,则它可唯一地分解成$A= H+K$,其中
    \begin{equation}
        H=\frac{1}{2}(A + \bar{A}^T),\quad \text{ 厄米矩阵};\quad
        K=\frac{1}{2}(A - \bar{A}^T),\quad \text{ 反厄米矩阵}.
    \end{equation}
\end{example}

\begin{example}
    在$m$维复线性空间$V$中取定两组基矢$\boldsymbol{e}_1,\cdots,\boldsymbol{e}_m$和
    $\boldsymbol{\epsilon}_1,\cdots,\boldsymbol{\epsilon}_m$;
    存在非奇异复数矩阵$C$使得$(\boldsymbol{\epsilon}_1,\cdots,\boldsymbol{\epsilon}_m)
    =(\boldsymbol{e}_1,\cdots,\boldsymbol{e}_m)C$.设$h$是$V$上厄米双线性函数,则有
    \begin{align*}
        A\equiv h(\boldsymbol{\epsilon}_i, \boldsymbol{\epsilon}_j)
        =h\left(\sum\nolimits_k \boldsymbol{e}_k C_{ki},\ \sum\nolimits_l \boldsymbol{e}_l C_{lj}\right)
        =\sum\nolimits_{kl} \bar{C}_{ki} C_{lj} h(\boldsymbol{e}_k, \boldsymbol{e}_l)
        =C^\dagger B C. 
    \end{align*}
    其中$B\equiv h(\boldsymbol{e}_k, \boldsymbol{e}_l)$.    
    
    若两个不同复矩阵($A$、$B$)变换关系是$A= C^\dagger B C$($C$非奇异),则称
    $A$、$B$是{\heiti 复相似的}.
    
    本例说明$V$上厄米双线性函数$h$
    在不同基矢下对应的矩阵是\CJKunderwave{复相似的}.\qed
\end{example}



\subsection{酉空间}\label{chcx:sec_unitary}
\begin{definition} \label{chcx:def_unitary-space}
    设$V$是复数域$\mathbb{C}$上的矢量空间,在$V$上定义一个二元运算,有复数$c$与之对应,
    称为{\heiti 酉(内)积};或{\heiti 幺正内积};
    或{\heiti \bfseries Hermite内积};
    记成$c=\left<\alpha,\beta\right>$;它需满足如下性质:
    
    {\bfseries (1)} $\left<\alpha,\beta\right>=\overline{\left<\beta,\alpha\right>}$;
    
    {\bfseries (2)} $\left<\alpha,k\beta\right>=\left<\alpha,\beta\right>k$;
    
    {\bfseries (3)} $\left<\alpha,\beta+\gamma\right>=\left<\alpha,\beta\right>+\left<\alpha,\gamma\right>$; 
    
    {\bfseries (4)} $\left<\alpha,\alpha\right>$是非负实数,当且仅当$\alpha=0$时,$\left<\alpha,\alpha\right> = 0$.
    
    \noindent 其中$\forall\alpha,\beta,\gamma \in V$,$\forall k\in \mathbb{C}$.
    这样的线性空间称为{\heiti 酉空间}或{\heiti 幺正空间}. \qed
\end{definition}

\index[physwords]{酉空间} \index[physwords]{幺正空间|see{酉空间}}
\index[physwords]{酉内积} \index[physwords]{Hermite内积}
\index[physwords]{厄米内积}  


这个定义适用于有限维和无限维线性空间;
当把复数域$\mathbb{C}$换成实数域$\mathbb{R}$时,它就退化为欧氏内积.
代数学上习惯称为酉内积,几何学上习惯称为Hermite内积,物理学上习惯称为幺正内积.
由酉空间定义容易得到
$\left<k\alpha,\beta\right>=\overline{\left<\beta,k\alpha\right>}
=\overline{\left<\beta,\alpha\right>} \bar{k} = \left<\alpha,\beta\right>\bar{k}$;
容易看出酉内积是正定、厄米双线性函数.
可见酉内积\ref{chcx:def_unitary-space}不是对称的;是厄米对称的.

给定任意$m$维复线性空间$V$,再给定一组基矢量$\{{e}_i\}$;
$V$中任意两个矢量$\alpha,\beta$可表示
为$\alpha=\sum_i x^i \boldsymbol{e}_i$,$\beta=\sum_j y^j \boldsymbol{e}_j$,
$x=(x^1,\cdots,x^m)^T$,$y=(y^1,\cdots,y^m)^T$.
我们已知系数$\{x^i,y^j\}$肯定是复数;但对基矢$\{{e}_i\}$知之甚少,
一般说来只知道它是个符号,不了解具体表征什么内容.
我们可以规定基矢$\{{e}_i\}$的酉内积是:
$    \left<\boldsymbol{e}_i,\boldsymbol{e}_j\right> = \delta_{ij}$.
也就是规定基矢$\{{e}_i\}$是正交归一的;此种情况下,酉内积的表达式较为简单
\begin{equation}\label{chcx:eqn_Hermite-Product}
    \left<\alpha,\beta\right>=\bar{x}^i y^j \left<\boldsymbol{e}_i,\boldsymbol{e}_j\right>
     =\bar{x}^i y^j \delta_{ij} = \sum\nolimits_{i=1}^{m}\bar{x}^i y^i.
\end{equation}
上式为最常用的酉内积形式.

\begin{example}
    在$m$维复线性空间$V$中取定两组基矢$\boldsymbol{e}_1,\cdots,\boldsymbol{e}_m$和
   $\boldsymbol{\epsilon}_1,\cdots,\boldsymbol{\epsilon}_m$;已知两组基矢都是正交归一的;
   存在非奇异复数矩阵$U$使得$(\boldsymbol{\epsilon}_1,\cdots,\boldsymbol{\epsilon}_m)
   =(\boldsymbol{e}_1,\cdots,\boldsymbol{e}_m)U$.那么有
   \begin{equation*}
       \left<\boldsymbol{e}_i,\boldsymbol{e}_j\right> = \delta_{ij} 
       =\left<\boldsymbol{\epsilon}_i,\boldsymbol{\epsilon}_j\right> 
       =\sum\nolimits_{kl}\bar{U}_{ki} U_{lj}\left<\boldsymbol{e}_k,\boldsymbol{e}_l\right>
       =\sum\nolimits_{kl}\bar{U}_{ki} U_{lj}\delta_{kl}
       =\sum\nolimits_{k}\bar{U}_{ki} U_{kj} .
   \end{equation*}
   上式说明两组正交归一基矢量间的变换系数满足
   \begin{equation}
       U^\dagger U = I \quad \Leftrightarrow\quad
       U \text {可逆且} U^{-1} = U^\dagger \quad \Leftrightarrow\quad
       U U^\dagger = I .
   \end{equation}
   满足上述关系的矩阵称为{\heiti 幺正矩阵}(或酉矩阵);
   即,联系两组正交归一基矢量的变换矩阵是幺正矩阵.
   若$U$是无穷维的,则需满足$U^\dagger U = I=U U^\dagger$.   \qed
\end{example}


\begin{definition}
    设有$m$维酉空间$V$,$\forall \alpha \in V$,
    $|\alpha|\equiv\sqrt{\left<\alpha,\alpha\right>}$称为矢量$\alpha$的{\heiti 长度}.
\end{definition}

\begin{definition}
    若酉空间$V$上线性变换$\hat{U}$满足
    $\left<\hat{U}\alpha,\hat{U}\beta\right>=\left<\alpha,\beta\right>$,$\forall \alpha,\beta \in V$;
    则称$\hat{U}$为酉空间$V$上的{\heiti 幺正变换}.
\end{definition}


\begin{theorem}
    设$V$是酉空间,幺正变换$\hat{U}$在正交归一基底下对应幺正矩阵$U$.
\end{theorem}
\begin{proof}
    $\forall \alpha,\beta \in V$,
    我们已知$\left<\hat{U}\alpha,\hat{U}\beta\right>=\left<\alpha,\beta\right>$,即
    \begin{align*}
        &\left<\alpha,\beta\right>=\sum_{kl}\left<\alpha^i \boldsymbol{e}_k U_{ki}, \beta^j \boldsymbol{e}_l U_{lj} \right>
        =\sum_{kl}\bar{U}_{ki} U_{lj} \bar{\alpha}^i \beta^j \left< \boldsymbol{e}_k, \boldsymbol{e}_l\right>
        =\sum_k \bar{U}_{ki} U_{kj} \bar{\alpha}^i \beta^j \\
        \Rightarrow\ & \bar{\alpha}^i \beta^j \delta_{ij}=\sum\nolimits_k \bar{U}_{ki} U_{kj} \bar{\alpha}^i \beta^j 
        \quad \Rightarrow \quad U^\dagger U = I.
    \end{align*}
    可见在正交归一基底下幺正变换对应幺正矩阵.
\end{proof}

\begin{theorem}\label{chcx:thm_UD}
	设$A$是任意$m$维幺正矩阵;则一定存在一个幺正矩阵$P$使得$P^{-1}A P=D$,
	其中$D={\rm diag}\{e^{\mathbbm{i}\theta_1},\cdots,e^{\mathbbm{i}\theta_m}\}$,
	$0\leqslant \theta_j <2\pi$,$j=1,2,\cdots,m$.
\end{theorem}
\begin{proof}
	证明可参考\parencite[p.522]{qiuws-2019-v2}定理16;或类似文献.
\end{proof}


\begin{definition}
    酉空间$V$上线性变换$\hat{A}$称为{\heiti 厄米变换}是
    指$\left<\hat{A}\alpha,\beta\right>=\left<\alpha,\hat{A}\beta\right>$.
%    $\forall \alpha,\beta\in V$.
\end{definition}
厄米变换$\hat{A}$在正交归一基底下对应的矩阵是厄米矩阵(证明留给读者).



\begin{definition}\label{chcx:def_Hilbert-Space}
	完备的酉空间称为Hilbert空间.
\end{definition}
完备性是指第\pageref{chtop:def_complete-metric}页的定义\ref{chtop:def_complete-metric}(距离由酉积定义).
大多数情形下,物理中的Hilbert空间是指$L^2(\mathbb{R}^3)$:定义在$\mathbb{R}^4_1$上,且平方可积的复函数集,
即$\{f(t,\boldsymbol{x}) \mid \int_{-\infty}^{\infty} 
\bar{f}(t,\boldsymbol{x}) f(t,\boldsymbol{x}) {\rm d}^3x  < +\infty \}$.
可以证明$L^2(\mathbb{R}^3)$是完备的酉空间(略).




\section{复结构}\label{chcx:sec_cpstru}
给定数域$\mathbb{C}$上复线性空间$V$,取定其中一组基矢$\{\boldsymbol{e}_i\}$;
再令$V^*$表示其对偶空间,基矢是$\{\boldsymbol{e}^{*i}\}$.
复空间的基底$\{\boldsymbol{e}_i\}$是抽象的,我们只知道它是一个符号,
一般说来不能对它进行取复共轭、取复角等操作.
参考命题\ref{chcx:thm_rc},可知空间
\begin{equation}\label{chcx:eqn_VR}
    V_{\mathbb{R}} \equiv {\rm Span}_{\mathbb{R}}\{\boldsymbol{e}_1,\cdots,\boldsymbol{e}_m;\  
    \mathbbm{i}\boldsymbol{e}_1,\cdots,\mathbbm{i}\boldsymbol{e}_m\} .
\end{equation}
是复线性空间$V$限制在实数域$\mathbb{R}$上的具体表现形式;
很明显实数域上线性空间$V_{\mathbb{R}}$是$2m$维的;
%一般称实线性空间$V_{\mathbb{R}}$是复线性空间$V$的{\heiti 实形},
其实就是同一集合$V$在不同数域($\mathbb{C}$和$\mathbb{R}$)上的具体表现.

\subsection{复结构定义}

\index[physwords]{复结构}  


\begin{definition}
    设$W$是一个实线性空间,若其上存在一个实线性变换$\mathbb{J}:W\to W$满
    足$\mathbb{J}^2\equiv \mathbb{J}\circ \mathbb{J}=-{\rm id}$,
    那么称$\mathbb{J}$是$W$上的一个{\heiti 复结构}.
\end{definition}
由例\ref{chmla:exm_T11TR}容易看出$\mathbb{J}$是实线性空间$W$上的一个$\binom{1}{1}$型张量.


在$V_{\mathbb{R}}$定义\CJKunderwave{实线性变换}$\mathbb{J}:V_{\mathbb{R}}\to V_{\mathbb{R}}$,
它作用在基矢上的具体表达式为
\begin{equation}\label{chcx:eqn_cxstrut1}
    \mathbb{J} (\boldsymbol{e}_i) = \mathbbm{i} \boldsymbol{e}_i, \quad
    \mathbb{J} (\mathbbm{i}\boldsymbol{e}_i) = - \boldsymbol{e}_i; \qquad 1\leqslant i \leqslant m.
\end{equation}
容易算出
\begin{equation}
    \mathbb{J}\circ \mathbb{J} (\boldsymbol{e}_i) = \mathbb{J} (\mathbbm{i}\boldsymbol{e}_i) = - \boldsymbol{e}_i, \quad
    \mathbb{J}\circ \mathbb{J} (\mathbbm{i}\boldsymbol{e}_i) = - \mathbb{J} \boldsymbol{e}_i = -\mathbbm{i}\boldsymbol{e}_i.
\end{equation}
需要注意$\mathbb{J}$是\CJKunderwave{实}线性变换,虚数单位$\mathbbm{i}$不能移进、移出括号,
即$\mathbb{J} (\mathbbm{i}\boldsymbol{e}_i)\neq \mathbbm{i}\mathbb{J} (\boldsymbol{e}_i)$;
实数是可以移进、移出的,即$\mathbb{J} (a\boldsymbol{e}_i)=a \mathbb{J} (\boldsymbol{e}_i)$,$\forall a\in \mathbb{R}$.
变换矩阵如下:
\begin{equation}\label{chcx:eqn_J-matrix}
    \mathbb{J} \begin{pmatrix}
        \boldsymbol{e}_1 \\  \vdots\\ \boldsymbol{e}_m \\   
        \mathbbm{i}\boldsymbol{e}_1\\  \vdots\\ \mathbbm{i}\boldsymbol{e}_m
    \end{pmatrix}^T
    =\begin{pmatrix}
        \boldsymbol{e}_1 \\  \vdots\\ \boldsymbol{e}_m \\   
        \mathbbm{i}\boldsymbol{e}_1\\  \vdots\\ \mathbbm{i}\boldsymbol{e}_m
    \end{pmatrix}^T
    \begin{pmatrix}
        0&\cdots&0 & -1 &  \cdots & 0\\
        \vdots&\ddots&\vdots &\vdots&\ddots&\vdots\\
        0&\cdots&0 & 0 &  \cdots & -1\\
        1&\cdots&0 &0&\cdots&0 \\
        \vdots&\ddots&\vdots &\vdots&\ddots&\vdots\\
        0&\cdots&1 & 0 & \cdots & 0
    \end{pmatrix} .
\end{equation}
上式中除标注$\pm 1$的矩阵元外,所有矩阵元都是零.
当把复线性空间$V$看成$V_{\mathbb{R}}$时,
由式\eqref{chcx:eqn_cxstrut1}定义的映射$\mathbb{J}$是$V_{\mathbb{R}}$上
的{\kaishu 复结构},称为$V_{\mathbb{R}}$的{\heiti 典型复结构}.


下面计算$\mathbb{J}$作用在$V_{\mathbb{R}}$中任意矢量$v=a^j \boldsymbol{e}_j 
+ b^j \mathbbm{i}\boldsymbol{e}_j$($\forall a^j,b^j \in \mathbb{R}$)的结果.
\begin{align}
   \mathbb{J} v &= \mathbb{J} \left( a^j \boldsymbol{e}_j + b^j \mathbbm{i}\boldsymbol{e}_j \right)
     = a^j \mathbb{J}( \boldsymbol{e}_j) + b^j \mathbb{J} (\mathbbm{i}\boldsymbol{e}_j ) 
     = a^j ( \mathbbm{i} \boldsymbol{e}_j) + b^j (-\boldsymbol{e}_j ) \notag\\
     &\xlongequal[-1=\mathbbm{i}^2]{\text{令}}
      a^j  \mathbbm{i} \boldsymbol{e}_j + b^j (\mathbbm{i}\mathbbm{i}\boldsymbol{e}_j ) 
     = \mathbbm{i} \left( a^j \boldsymbol{e}_j + b^j \mathbbm{i}\boldsymbol{e}_j \right)
     = \mathbbm{i} v. \label{chcx:eqn_Jv=iv} \\
   \mathbb{J}^2 v &=  \mathbb{J}\circ \mathbb{J} v = \mathbb{J} \left(a^j ( \mathbbm{i} \boldsymbol{e}_j) + b^j (-\boldsymbol{e}_j ) \right)
     = a^j ( - \boldsymbol{e}_j) - b^j (\mathbbm{i}\boldsymbol{e}_j ) 
     = -v . \label{chcx:eqn_JJv=-v}
\end{align}
$\mathbb{J}$作用在$V_{\mathbb{R}}$中的元素$v$相当于把$v$看成复空间$V$中的元素乘以虚数单位$\mathbbm{i}$,
然后再把它看作$V_{\mathbb{R}}$中的元素.
由式\eqref{chcx:eqn_Jv=iv}可知实线性变换$\mathbb{J}$在$V_\mathbb{R}$中的定义是良好的,
与$V$中基底$\{\boldsymbol{e}_j\}$选取无关.
反之,我们有如下定理.
\begin{theorem}\label{chcx:thm_2mm}
    设$W$是有复结构$\mathbb{J}$的$n$维实线性空间,那么$n$必是偶数,即$n=2m$.
    并且在$W$中存在一组基底$\{\boldsymbol{\epsilon}_\alpha,\ 1\leqslant \alpha \leqslant 2m\}$使得
    \begin{equation}
        \mathbb{J} \boldsymbol{\epsilon}_i = \boldsymbol{\epsilon}_{m+i}, \quad
    \mathbb{J} \boldsymbol{\epsilon}_{m+i} = - \boldsymbol{\epsilon}_i; \qquad 1\leqslant i \leqslant m.
    \end{equation}
    因此,存在一个$m$维复线性空间$V$,把它限制在实数域$\mathbb{R}$上得到的空间$V_{\mathbb{R}}=W$,
    并且$\mathbb{J}$是$V_{\mathbb{R}}$上的{\kaishu 典型复结构}.
\end{theorem}
\begin{proof}
    首先定义复数和线性空间$W$中的元素的{\kaishu 数量乘法}如下,
    $\forall a,b\in \mathbb{R}$有$a+\mathbbm{i}b\in \mathbb{C}$,
    $\forall \alpha \in W$,令
    \begin{equation}\label{chcx:eqn_tmptime}
        (a+\mathbbm{i}b)\cdot \alpha \overset{def}{=} a \alpha +b \mathbb{J} \alpha .
        \quad \text{数量乘法记号“}\cdot\text{”通常会被省略}
    \end{equation}
    可以验证,$\forall c, c_1, c_2\in \mathbb{C}$和$\forall \alpha,\beta\in W$有
    \begin{equation}
        c_1 (c_2 \alpha) = (c_1 c_2)\alpha, \quad
        c (\alpha +\beta) = c \alpha + c \beta, \quad
        (c_1+c_2)\alpha = c_1\alpha + c_2\alpha. 
    \end{equation}
    故线性空间$W$关于数量乘法\eqref{chcx:eqn_tmptime}构成一个复线性空间,
    现将这个复线性空间记作$V$.很明显$V_{\mathbb{R}}=W$,
    并且$\mathbb{J}$是$V_{\mathbb{R}}$的典型复结构.
    
    如果复数域上$V$的复维数是$m$,那么由命题\ref{chcx:thm_rc}容易看到$n=2m$.
    在复线性空间$V$中任意取定一个基底$\{\boldsymbol{\epsilon}_1,\cdots,\boldsymbol{\epsilon}_m\}$,
    则$\mathbb{J}\boldsymbol{\epsilon}_i = \mathbbm{i}\boldsymbol{\epsilon}_i,\ 1\leqslant i \leqslant m$;
    并且$\{\boldsymbol{\epsilon}_1,\cdots,\boldsymbol{\epsilon}_m;\  
    \mathbb{J} \boldsymbol{\epsilon}_1,\cdots,\mathbb{J}\boldsymbol{\epsilon}_m\}$是实线性空间$V_{\mathbb{R}}=W$的基底;
    这个基底便可满足定理要求.
\end{proof}

综和命题\ref{chcx:thm_rc}和定理\ref{chcx:thm_2mm}可得如下定理:
\begin{theorem}\label{chcx:thm_CRJ}
$m$维复线性空间$V$等价于有确定复结构的$2m$维实线性空间$V_{\mathbb{R}}$.
\end{theorem}

\begin{proposition}\label{chcx:thm_ABC}
    设$\mathbb{J}$是$2m$维实线性空间$V_{\mathbb{R}}$的复结构,$V$是其相应的$m$维复线性空间.
    $V$上的复线性变换$\mathcal{A}$自然是$V_{\mathbb{R}}$上的线性变换.
    如果$\mathcal{A}$关于$V$的基$\{\boldsymbol{e}_1,\cdots,\boldsymbol{e}_m\}$的矩阵
    是$B+\mathbbm{i}C$(其中$B$、$C$是实矩阵),那么$\mathcal{A}$作为$V_{\mathbb{R}}$上
    的线性变换关于基$\{\boldsymbol{e}_1,\cdots,\boldsymbol{e}_m;
    \mathbb{J}\boldsymbol{e}_1,\cdots,\mathbb{J}\boldsymbol{e}_m\}$的矩阵是:
    $    \left(\begin{smallmatrix}
        B& -C \\ C & B
    \end{smallmatrix}\right)$.
\end{proposition}
\begin{proof}
    我们设实矩阵$B=\{b_{ij}\}$、$C=\{c_{ij}\}$,则
    \begin{align*}
        \mathcal{A} \boldsymbol{e}_{j} =& \sum_i \boldsymbol{e}_{i} (b_{ij}+\mathbbm{i}c_{ij}) 
        =\sum_i\boldsymbol{e}_{i} b_{ij} + \sum_i (\mathbb{J}\boldsymbol{e}_{i}) c_{ij} ,\\
        \mathcal{A} (\mathbb{J}\boldsymbol{e}_{j}) =& \mathbbm{i}( \mathcal{A} \boldsymbol{e}_{j})
        =\mathbbm{i} \sum_i \boldsymbol{e}_{i} (b_{ij}+\mathbbm{i}c_{ij}) 
        =\sum_i (\mathbb{J}\boldsymbol{e}_{i}) b_{ij} - \sum_i \boldsymbol{e}_{i} c_{ij} .
    \end{align*}
    把上式写成矩阵形式,便证明了命题.
\end{proof}

%反之,$V_{\mathbb{R}}$上的实线性变换$\phi$未必是$V$上的变换.但有:

\begin{proposition}\label{chcx:thm_pJJp}
    设有两个复线性空间$U,V$,它们分别对应$(U_{\mathbb{R}},\mathbb{J})$和$(V_{\mathbb{R}},\tilde{\mathbb{J}})$;
    有实线性映射$\phi:U_{\mathbb{R}}\to V_{\mathbb{R}}$.
    那么$\phi$是复线性映射的充要条件是:$\phi\circ \mathbb{J}=\tilde{\mathbb{J}}\circ \phi$.
\end{proposition}
\begin{proof}
    $\forall \alpha \in U_{\mathbb{R}}$,
    已知$\mathbb{J},\tilde{\mathbb{J}}$是典型复结构;自然有$\mathbbm{i}\alpha = \mathbb{J}\alpha$
    和$\mathbbm{i}\phi(\alpha)= \tilde{\mathbb{J}}\phi(\alpha)$.
    
    先证“$\Rightarrow$”.已知$\phi$是复线性映射;
    $\phi(\mathbb{J}\alpha)=\phi(\mathbbm{i} \alpha)=\mathbbm{i}\phi(\alpha)=\tilde{\mathbb{J}}\phi(\alpha)$.
    
    再证“$\Leftarrow$”.$\forall a,b\in \mathbb{R}$,
    $\phi\bigl((a+\mathbbm{i}b)\alpha \bigr)=a \phi(\alpha) + b \phi(\mathbb{J}\alpha)
    =a \phi(\alpha) + b \tilde{\mathbb{J}}\phi(\alpha)
    =a \phi(\alpha) + b \mathbbm{i}\phi(\alpha) 
    =(a + b \mathbbm{i})\phi(\alpha)  $;
    故$\phi$是复线性映射.
    
    如果$\phi$可逆,则$\phi\circ \mathbb{J}=\tilde{\mathbb{J}}\circ \phi$通常
    写为$ \mathbb{J}= \phi^{-1}\circ\tilde{\mathbb{J}}\circ \phi$.
    
    如果$U_{\mathbb{R}}\equiv V_{\mathbb{R}}$,则线性映射$\phi$退化
    为$V_{\mathbb{R}}$上的线性变换.
\end{proof}

设$\mathbb{J}$是$2m$维实线性空间$V_{\mathbb{R}}$的复结构,
其上的线性变换$\phi$在基矢组$\{\boldsymbol{e}_1,\cdots,\boldsymbol{e}_m;
\mathbb{J}\boldsymbol{e}_1,\cdots,\mathbb{J}\boldsymbol{e}_m\}$上的矩阵是:
$    \left(\begin{smallmatrix}
    A& B \\ C & D
\end{smallmatrix}\right)$.
那么可以得到:  %根据命题\ref{chcx:thm_pJJp}
$\phi\circ \mathbb{J}=\mathbb{J}\circ \phi \ \Leftrightarrow \ 
A=D, B=-C$.此时$\phi$作为$V_{\mathbb{R}}$所对应的复线性空间$V$上
的基$\{\boldsymbol{e}_1,\cdots,\boldsymbol{e}_m\}$的矩阵是:$A+\mathbbm{i}C$.
本自然段计算不难,留给读者当练习.




\subsection{复化与实形}\label{chcx:sec_cr}

%取自 许以超《线性代数与矩阵论》169-170,定理5.2.6

\index[physwords]{复化}
  
\begin{definition}\label{chcx:def_cxn}
    设$V_r$是实数域$\mathbb{R}$上的$m$维实线性空间,
    $\{\boldsymbol{\epsilon}_1,\cdots,\boldsymbol{\epsilon}_m\}$是
    它的一组基矢.形式地引入集合
    \begin{equation}\label{chcx:eqn_cxn}
        V^{\mathbb{C}}_r \equiv \left\{ \sum_{k=1}^{m} c_k \boldsymbol{\epsilon}_k \mid
        c_k = a_k+\mathbbm{i} b_k,\quad a_k,b_k \in \mathbb{R},
        \quad 1\leqslant k \leqslant m \right\} .
    \end{equation}
    则$V^{\mathbb{C}}_r$是复线性空间,称为实线性空间$V_r$的{\heiti 复化}.
\end{definition}

由定义\ref{chcx:def_cxn}可以看出$\{\boldsymbol{\epsilon}_1,\cdots,\boldsymbol{\epsilon}_m\}$是
$V^{\mathbb{C}}_r$的一组基矢,那么$V^{\mathbb{C}}_r$的{\kaishu 复维数}与$V_r$的{\kaishu 实维数}是
相等的,都是$m$.为了说明定义\ref{chcx:def_cxn}是合理的,我们还需证明该
取法与实线性空间$V_r$的基矢选取无关.
在$V_r$中另取一组基$\{\boldsymbol{\eta}_1,\cdots,\boldsymbol{\eta}_m\}$,
两组基矢间的变换关系是
\begin{equation}
    \boldsymbol{\eta}_j = \sum_{k=1}^{m} \boldsymbol{\epsilon}_k d^k_{\hphantom{k}j},\qquad
    d^k_{\hphantom{k}j} \in \mathbb{R},\quad 1\leqslant j,k \leqslant m .
\end{equation}
故有
\begin{equation}
    \sum_{j=1}^{m} c_j \boldsymbol{\eta}_j = \sum_{k=1}^{m}
    \left( \sum_{j=1}^{m} c_j d^k_{\hphantom{k}j} \right)\boldsymbol{\epsilon}_k .
\end{equation}
因$c_j= a_j+\mathbbm{i} b_j\in \mathbb{C}$,$d^k_{\hphantom{k}j} \in \mathbb{R}$,
故$\sum_{j} c_j d^k_{\hphantom{k}j} \in \mathbb{C}$;
所以由上式可知$\sum_{j} c_j \boldsymbol{\eta}_j \in V^{\mathbb{C}}_r$.
上式说明了由$\{\boldsymbol{\eta}\}$定义的复化与$\{\boldsymbol{\epsilon}\}$定义的复化是相同的复线性空间.


\begin{definition}\label{chcx:def_real}
    设$L_c$是复数域$\mathbb{C}$上的$m$维复线性空间,
    $\{\boldsymbol{\alpha}_1,\cdots,\boldsymbol{\alpha}_m\}$是
    它的一组基矢.形式地引入集合
    \begin{equation}
        L^{\mathbb{R}}_c \equiv \left\{ \sum_{k=1}^{m} a_k \boldsymbol{\alpha}_k \mid
        a_k \in \mathbb{R},     \quad 1\leqslant  k \leqslant m \right\} .
    \end{equation}
    则$L^{\mathbb{R}}_c$是实线性空间,称为复线性空间$L_c$的{\heiti 实形式}.
\end{definition}

因复数包含实数,故矢量组$\{\boldsymbol{\alpha}_1,\cdots,\boldsymbol{\alpha}_m\}$是实线性无关的;
而且由实系数$a_k$组合得到的矢量$\sum_{k=1}^{m} a_k \boldsymbol{\alpha}_k$满足线性空间的八条公理;
这便验证了定义\ref{chcx:def_real}是合理的.
$L^{\mathbb{R}}_c$的{\kaishu 实维数}与$L_c$的{\kaishu 复维数}是相等的,都是$m$.
$L_c$的实形式$L^{\mathbb{R}}_c$的定义和复线性空间$L_c$的基矢选取有关;
不同的基很可能给出不同的实形式.这与实线性空间的复化不同.
不难看出$L^{\mathbb{R}}_c$的复化$(L^{\mathbb{R}}_c)^{\mathbb{C}} \equiv L_c$.










\section{复流形}\label{chcx:sec_cxmanifold}
实流形是实空间$\mathbb{R}^m$的一些开子集以$C^r$方式拼接在一起的结果.
与此类似,复流形是复空间$\mathbb{C}^m$的一些开子集以\CJKunderwave{全纯}方式拼接在一起的结果.
在\S\ref{chcx:sec_cxf}中已经初步介绍了全纯的概念,下面把它推广到更一般的情形.


\index[physwords]{复流形}  

\begin{definition}
    设$U,V$分别是$\mathbb{C}^m,\mathbb{C}^n$的开子集,映射$\phi:U\to V$可以
    表示成$w^\alpha=\phi^\alpha(z^1,\cdots,z^m),\ 1\leqslant \alpha \leqslant n$.
    若每一个$\phi^\alpha$都是$U$上的全纯函数,则称$\phi$是从$U$到$V$的{\heiti 全纯映射}.
    
    若$m=n$,$\phi$和$\phi^{-1}$是全纯,且$\phi$是$C^m$上同胚,
    则称$\phi$是{\heiti 全纯变换}.  \qed
\end{definition}

有了这些准备,我们给出复流形的定义.
\begin{definition}\label{chcx:def_cxmanifold}
    设$M$是$2m$维实流形.若$M$有一个坐标卡集$\Sigma=\{(U_\alpha,\phi_\alpha);\ \alpha\in I\}$,
    其中$\phi_\alpha: U_\alpha \to \mathbb{C}^m(=\mathbb{R}^{2m})$是从$M$开子集$U_\alpha$到$\mathbb{C}^m$的
    同胚映射,使得$\{U_\alpha\}$是$M$的一个开覆盖,并且对于任意的$\alpha,\beta\in I$,
    当$U_\alpha \cap U_\beta \neq \varnothing$时,复合映射$\phi_\alpha \circ \phi^{-1}_\beta$是
    从$\mathbb{C}^m$开子集$\phi_\beta(U_\alpha \cap U_\beta)$到$\mathbb{C}^m$开
    子集$\phi_\alpha(U_\alpha \cap U_\beta)$的\CJKunderwave{全纯变换},则称坐标卡集$\Sigma$是流形$M$的
    一个{\heiti 复坐标覆盖}.$M$的极大复坐标覆盖称为它的{\heiti 复流形结构};
    指定了复流形结构的$2m$维实流形$M$称为$m$维{\heiti 复流形};
    属于复流形结构的坐标卡$(U_\alpha,\phi_\alpha)$称为$M$的{\heiti 容许复坐标卡}. \qed
\end{definition}

我们知道任一$m$维复流形都有自然方式看成$2m$维实流形(参见命题\ref{chcx:thm_rc}).


\index[physwords]{殆复流形}  

\begin{definition}\label{chcx:def_acm}
    设$M$是$2m$维光滑实流形.若在$M$上存在一个光滑$\binom{1}{1}$型张量场$\mathbb{J}$,
    使得$\forall p\in M$,$\mathbb{J}_p\equiv \mathbb{J}(p)$是切空间$T_p M$上的复结构;
    则称$(M,\mathbb{J})$是一个{\heiti 殆复流形}(almost complex manifold),
    $\mathbb{J}$称为$M$上的{\heiti 复结构}. \qed
\end{definition}

可以证明复流形肯定是殆复流形.

殆复流形何时是复流形?这是艰深的课题,请参阅本章末文献.




\section*{小结}
本章内容主要参考了\parencite{cc2001-zh}第七章. %\parencite{chen-li-2004v2}第八章.
基于复流形,还可以引入复矢量丛的概念,进而引入复联络以及复曲率等概念,有兴趣读者请参考上述文献.

\printbibliography[heading=subbibliography,title=第\ref{chcx}章参考文献]

\endinput











