% !TeX encoding = UTF-8
% 此文件从2021.8开始

\chapter{黎曼几何}\label{chrg}



截止到目前,在微分流形中还没有引入度规这个结构;即便如此也定义了联络、黎曼曲率、
测地线等等重要几何量.从本章开始引入度规结构,这使得微分几何内容更加丰富;
一件很重要的事:可以定义曲线长度了.

\index[physwords]{黎曼几何}
\index[physwords]{黎曼流形}
\index[physwords]{黎曼结构}

\section{黎曼流形}\label{chrg:sec_riemann}
\subsection{黎曼流形定义}
\begin{definition}
    给$m$维$C^\infty$微分流形$M$指定一个处处非退化、光滑、对称$\Tpq{0}{2}$型张量场$g$后,
    称$(M,g)$为广义{\heiti 黎曼流形},$g$称为其{\heiti 度规}或{\heiti 基本度量}或{\heiti 黎曼结构}.
\end{definition}

\index[physwords]{度规!黎曼度规}

\begin{remark}
    文献\parencite[\S 5.1]{cc2001-zh}中的定理1.1表明光滑流形必存在正定的黎曼度规;
    需注意此定理不是平庸的,没有这个定理,正定的黎曼度规有点像空中楼阁.
    并不是任何光滑流形都能定义广义Lorentz型度规,需满足一些条件;
    文献\parencite[\S 4.1]{kriele-1999}给出了广义Lorentz型度规的存在性定理.
\end{remark}

\begin{remark}
    黎曼度规$g$的抽象指标记号与\S\ref{chdm:sec_metric-abi}相同.
    指定度规后,点$p\in M$切空间的切矢量与余切空间的对偶矢量存在自然同构,
    即认为$u^a\in T_pM$和$T^*_pM \ni u_a = g_{ab}u^b$是同一矢量的不同表现形式.
    定理\ref{chmla:thm_vvss_zrtg}证明了$V^{**}$自然同构于$V$,继而将两者认为等同,即$V^{**}\equiv V$;
    然而,由于黎曼流形上的这种自然同构是基于附加结构$g_{ab}$的(逐点不同),与前者的自然同构略有区别,
    故不能记$T_pM=T^*_pM$,只能记作$T_pM\cong T^*_pM$.
\end{remark}

黎曼流形专指正定度规情形.
广义黎曼流形(Generalized Riemannian)是指度规特征值可能全是正的,可能全是负,也可能有正有负.
广义黎曼流形还被称为半黎曼流形(Semi-Riemannian)或伪黎曼流形(Pesudo-Riemannian,也被翻译成“赝”或“准”).
需要指出的是有的文献中广义黎曼流形专指非正定度规流形,即不包含正定度规情形;各个文献中此称谓内涵不完全一致.
之后的行文中,我们将尽量严格区分“黎曼流形”和“广义黎曼流形”;
度规场$g_{ab}$在正交归一基矢下的非对角元全是零,对角元是$\pm 1$;
所有对角元素的和(即这些$\pm 1$的和)称为度规场$g_{ab}$的{\heiti 号差};
比如我们采用的闵氏时空度规号差是$+2$($g={\rm diag}(-1,1,1,1)$).

\begin{definition}
    给定$m$维广义黎曼流形$(M,g)$,$\nabla_a$是$M$上的仿射联络,如果有$\nabla_a g_{bc}=0$,
    则称仿射联络$\nabla_a$为黎曼流形的{\heiti 容许联络}.
\end{definition}
下面我们导出容许联络的一个常用表达式,$\forall X^a,Y^a,Z^a\in \mathfrak{X}(M)$,
\begin{align}
    &\nabla_{X}(g_{ab}Y^a Z^b)= (\nabla_{X}g_{ab})Y^a Z^b  +g_{ab}( \nabla_{X}Y^a) Z^b
      +g_{ab}Y^a\nabla_{X}( Z^b)  \notag \\
    &\xLongrightarrow{\nabla_a g_{bc}=0}
    {X}(g_{ab}Y^a Z^b)= g_{ab}( \nabla_{X}Y^a) Z^b
    +g_{ab}Y^a\nabla_{X}( Z^b).  \label{chgd:eqn_connection-compatibility}
\end{align}


\index[physwords]{联络!Levi-Civita}
\begin{theorem}\label{chrg:thm_Levi-Civita-Connetion}
    给定$m$维广义黎曼流形$(M,g)$,则在$M$上存在唯一的无挠容许联络,
    称为黎曼流形的{\bfseries \heiti Levi-Civita联络}.
\end{theorem}
\begin{proof}
    设$\nabla_a$是$M$上的无挠、容许、仿射联络,在局部坐标$(U;x^i)$下,有
    \begin{equation} \label{chrg:eqn_Dg=0}
        \nabla_a g_{bc} =0  {\quad \color{red}\Rightarrow \quad }
        \partial_a g_{bc} = \Gamma_{ba}^e g_{ec} + \Gamma_{ca}^e g_{be} .
    \end{equation}
    将上式进行指标轮换
    \begin{equation*}
        \partial_b g_{ca} = \Gamma_{cb}^e g_{ea} + \Gamma_{ab}^e g_{ce}, \qquad
        -\partial_c g_{ab} = -\Gamma_{ac}^e g_{eb} - \Gamma_{bc}^e g_{ae}  .
    \end{equation*}
    上面三式相加,得
    \begin{small}     \setlength{\mathindent}{0em}
    \begin{subequations}\label{chrg:eqn_Christoffel-naturalbases}
        \begin{align}
            \Gamma_{ab}^c &= \frac{1}{2}g^{ce}\left(
                  \frac{\partial g_{ae}}{\partial x^b}
                + \frac{\partial g_{eb}}{\partial x^a}
                - \frac{\partial g_{ab}}{\partial x^e} \right)
                {\quad \color{red}\Leftrightarrow \quad}
            \Gamma_{ij}^k = \frac{1}{2}{g^{kl}}\left(
                  \frac{\partial g_{il}}{\partial x^j}
                + \frac{\partial g_{lj}}{\partial x^i}
                - \frac{\partial g_{ij}}{\partial x^l} \right),
            \label{chrg:eqn_Christoffel-2-naturalbases} \\
            \Gamma _{dab} &= {g_{dc}}\Gamma _{ab}^c = \frac{1}{2}\left(
                  \frac{\partial g_{ad}}{\partial x^b}
                + \frac{\partial g_{db}}{\partial x^a}
                - \frac{\partial g_{ab}}{\partial x^d} \right)
               {\color{red}\Leftrightarrow}
            \Gamma _{kij} = \frac{1}{2}\left(
                  \frac{\partial g_{ik}}{\partial x^j}
                + \frac{\partial g_{kj}}{\partial x^i}
                - \frac{\partial g_{ij}}{\partial x^k} \right) .
            \label{chrg:eqn_Christoffel-1-naturalbases}
        \end{align}
    \end{subequations}    \setlength{\mathindent}{2em}
    \end{small}
    式\eqref{chrg:eqn_Christoffel-1-naturalbases}为局部坐标系下的第一类Christoffel记号(给出了抽象指标及分量表示方式);
    第一类Christoffel记号用的较少.
    式\eqref{chrg:eqn_Christoffel-2-naturalbases}为局部坐标系下的第二类Christoffel记号(给出了抽象指标及分量表示方式),
    简称(第二类)克氏符,\uwave{第二类克氏符与度规号差无关}.由此可见无挠容许联络系数由度规唯一确定.
    需注意,两个克氏符都不是张量,所以上指标降下位置是人为约定的.

%\begin{equation}
%    \Gamma^\mu_{\lambda\nu}=\frac{1}{2}g^{\mu\rho}\left(
%    \frac{\partial g_{\rho\nu}}{\partial x^\lambda}
%    +\frac{\partial g_{\lambda\rho}}{\partial x^\nu}
%    -\frac{\partial g_{\lambda\nu}}{\partial x^\rho} \right) .
%\end{equation}

\index[physwords]{Christoffel记号}  \index[physwords]{克氏符}

    反过来,给定联络系数\eqref{chrg:eqn_Christoffel-2-naturalbases},
    经计算不难确定它满足式\eqref{chccr:eqn_Exchange-Christoffel}的变换关系;
    由注解\ref{chccr:remark_con-Nabla}可知,
    这相当于在流行$M$上定义一个联络$\nabla_a$.通过计算不难确定
    由式\eqref{chrg:eqn_Christoffel-2-naturalbases}定义的联络是无挠、容许、仿射联络.
\end{proof}

%\subsubsection
\paragraph{一般仿射联络}\label{chrg:sec_clc}
在例\ref{chccr:exm_c-c=t}中,我们指出两个仿射联络系数差是张量场;
现在有了一个特殊的联络——Levi-Civita联络,我们再次讨论此问题.
设$\nabla_a$是流形$M$的Levi-Civita联络,它与度规场$g_{bc}$相容,即$\nabla_a g_{bc}=0$;
再设一无挠、仿射联络$\oversetmy{\circ}{\nabla}_a$,它与度规$g_{ab}$不相容.
设两者关系是:
\begin{equation}
    \nabla_{a} v^c = \oversetmy{\circ}{\nabla}_{a} v^c + C^c_{ea} v^e .
\end{equation}
由例\ref{chccr:exm_c-c=t}的分析可知$C^c_{ea}$是张量场;与第一类克氏符相似,我们
约定它的上指标降在第一个位置.
由两个联络的关系可以得到
\begin{equation}
    \oversetmy{\circ}{\nabla}_{a} g_{bc} - C^e_{ba} g_{ec} -C^e_{ca} g_{be}
    = \nabla_{a} g_{bc} =0 .
\end{equation}
仿照定理\ref{chrg:thm_Levi-Civita-Connetion}中导出克氏符的操作,进行指标轮换后相加减,可得
\begin{equation}
    {C}^c_{ab}=\frac{1}{2}g^{cf}\left(\oversetmy{\circ}{\nabla}_a g_{fb} + \oversetmy{\circ}{\nabla}_b g_{fa} 
    - \oversetmy{\circ}{\nabla}_f g_{ab}\right) .    
\end{equation}

%\begin{align*}
%    & \oversetmy{\circ}{\nabla}_{a} g_{bc}+\oversetmy{\circ}{\nabla}_{b} g_{ca}-\oversetmy{\circ}{\nabla}_{c} g_{ab} 
%    =  C^e_{ba} g_{ec} +C^e_{ca} g_{be}
%    + C^e_{cb} g_{ea} +C^e_{ab} g_{ce}
%    - C^e_{ac} g_{eb} -C^e_{bc} g_{ae}\\
%    \Rightarrow \  &\oversetmy{\circ}{\nabla}_{a} g_{bc}+\oversetmy{\circ}{\nabla}_{b} g_{ca}
%    -\oversetmy{\circ}{\nabla}_{c} g_{ab}  =   2 C^e_{ab} g_{ec} \\
%    \Rightarrow \  & 
%    {C}^c_{ab}=\frac{1}{2}g^{cf}\left(\oversetmy{\circ}{\nabla}_a g_{fb} + \oversetmy{\circ}{\nabla}_b g_{fa} 
%    - \oversetmy{\circ}{\nabla}_f g_{ab}\right) %\\
%    % \Rightarrow \   &    \oversetmy{\circ}{\nabla}_a g_{bc} = g_{ec} {C}^e_{ab} + g_{be} {C}^e_{ac} .
%\end{align*}

%    = &  \nabla_{a} \nabla_{b} v^c - C^c_{fa}\nabla_{b} v^f  -  C^c_{eb}\nabla_{a} v^e
%    + C^f_{ba} \nabla_{f}v^c     +  C^c_{eb} C^e_{fa} v^f 
%    - v^e \left(\nabla_{a}  C^c_{eb} - C^c_{fa} C^f_{eb} + C^f_{ea} C^c_{fb} + C^f_{ba} C^c_{ef} \right) \\
%    = &  \nabla_{a} \nabla_{b} v^c - C^c_{ea}\nabla_{b} v^e  -  C^c_{eb}\nabla_{a} v^e
%    + C^f_{ba} \nabla_{f}v^c   
%    - v^e \nabla_{a}  C^c_{eb} + v^e C^c_{fa} C^f_{eb} - v^e C^f_{ba} C^c_{ef} 

切矢量场$v^c$的两次导数是
\setlength{\mathindent}{0em}
\begin{equation}\label{chrg:eqn_DDva}
    \begin{aligned}
        &\oversetmy{\circ}{\nabla}_{a} \oversetmy{\circ}{\nabla}_{b}v^c = 
        \oversetmy{\circ}{\nabla}_{a} \left(\nabla_{b} v^c - C^c_{eb} v^e \right) 
        =  \oversetmy{\circ}{\nabla}_{a} \nabla_{b} v^c 
        - v^e \oversetmy{\circ}{\nabla}_{a}  C^c_{eb} 
        -  C^c_{eb} \oversetmy{\circ}{\nabla}_{a} v^e \\
        %        = &  \nabla_{a} \nabla_{b} v^c - C^c_{fa}\nabla_{b} v^f + C^f_{ba} \nabla_{f}v^c
        %        - v^e \oversetmy{\circ}{\nabla}_{a}  C^c_{eb} 
        %        -  C^c_{eb} (\nabla_{a} v^e - C^e_{fa} v^f)  \\
        &=   \nabla_{a} \nabla_{b} v^c - C^c_{fa}\nabla_{b} v^f  -  C^c_{eb}\nabla_{a} v^e
        + C^f_{ba} \nabla_{f}v^c     +  C^c_{eb} C^e_{fa} v^f 
        - v^e \oversetmy{\circ}{\nabla}_{a}  C^c_{eb}  \\
        \Rightarrow &
        \oversetmy{\circ}{\nabla}_{[a} \oversetmy{\circ}{\nabla}_{b]}v^c 
        =   \nabla_{[a} \nabla_{b]} v^c - C^c_{f[a}\nabla_{b]} v^f  -  C^c_{e[b}\nabla_{a]} v^e
        +  C^c_{e[b} C^e_{a]f} v^f - v^e \oversetmy{\circ}{\nabla}_{[a}  C^c_{b]e}    .
    \end{aligned}
\end{equation}\setlength{\mathindent}{2em}
上式中“$\Rightarrow$”这一步是
对下标$ab$取反对称操作,
并利用了$\oversetmy{\circ}{\nabla}_{a}$的无挠属性(即$C^c_{[ab]}=0$).
由上式最后一步,
参考式\eqref{chccr:eqn_Riemannian13-Vec-commutator}
( $\nabla_{[a} \nabla_{b]} Z^d = \frac{1}{2}R_{cab}^d Z^c $,
${R}_{eab}^c$是${\nabla}_{a}$的黎曼曲率);
并注意到$v^c$的任意性,经计算可得两个黎曼曲率间的关系:
\begin{equation}\label{chrg:eqn_DDR}
    \oversetmy{\circ}{R}_{eab}^c  
    ={R}_{eab}^c - 2 \oversetmy{\circ}{\nabla}_{[a}  C^c_{b]e} - 2 C^c_{f[a} C^f_{b]e}
    ={R}_{eab}^c - 2 \nabla_{[a}  C^c_{b]e} + 2 C^c_{f[a} C^f_{b]e} .
\end{equation}
其中$\oversetmy{\circ}{R}_{eab}^c$是$\oversetmy{\circ}{\nabla}_{a}$的黎曼曲率,
即$\oversetmy{\circ}{\nabla}_{[a} \oversetmy{\circ}{\nabla}_{b]} Z^d 
= \frac{1}{2}\oversetmy{\circ}{\nabla}{R}_{cab}^d Z^c $.

%
%\begin{align*}
%%    & \oversetmy{\circ}{\nabla}_{a} \oversetmy{\circ}{\nabla}_{b}v^c 
%%    =   \nabla_{a} \nabla_{b} v^c - C^c_{fa}\nabla_{b} v^f  -  C^c_{eb}\nabla_{a} v^e
%%    + C^f_{ba} \nabla_{f}v^c     +  C^c_{eb} C^e_{fa} v^f 
%%    - v^e \oversetmy{\circ}{\nabla}_{a}  C^c_{eb} \\
%\Rightarrow &
%    \oversetmy{\circ}{\nabla}_{[a} \oversetmy{\circ}{\nabla}_{b]}v^c 
%=   \nabla_{[a} \nabla_{b]} v^c - C^c_{f[a}\nabla_{b]} v^f  -  C^c_{e[b}\nabla_{a]} v^e
%     +  C^c_{e[b} C^e_{a]f} v^f - v^e \oversetmy{\circ}{\nabla}_{[a}  C^c_{b]e}    \\ 
%     \Rightarrow &
%     \oversetmy{\circ}{R}_{eab}^c  = {R}_{eab}^c
%     - 2 \oversetmy{\circ}{\nabla}_{[a}  C^c_{b]e} - 2 C^c_{f[a} C^f_{b]e}
%\end{align*}

%式\eqref{chrg:eqn_DDR}中关于$\oversetmy{\circ}{\nabla}_{a}$部分的推导如下
%\begin{align*}
%    &\oversetmy{\circ}{\nabla}_{[a} \oversetmy{\circ}{\nabla}_{b]}v^c 
%    =   \nabla_{[a} \nabla_{b]} v^c - C^c_{f[a}\nabla_{b]} v^f  -  C^c_{e[b}\nabla_{a]} v^e
%    + C^f_{[ba]} \nabla_{f}v^c
%    - v^e {\nabla}_{[a}  C^c_{b]e} 
%    + v^e C^c_{f[a} C^f_{b]e} - v^e C^f_{[ba]} C^c_{ef} \\
%    \Rightarrow&  \oversetmy{\circ}{R}_{eab}^c v^e = {R}_{eab}^c v^e
%    - 2v^e {\nabla}_{[a}  C^c_{b]e} 
%    +  2 v^e C^c_{f[a} C^f_{b]e} .
%\end{align*}

\begin{remark}
    约定:如无特殊声明,在此以后所用联络皆是Levi-Civita联络.
\end{remark}

\index[physwords]{Koszul公式}
\begin{example}\label{chrg:exam_Koszul}
    证明Koszul公式:
    \begin{equation*}
        2 X_a \nabla_Y Z^a = Y(Z_a X^a) + Z(X_a Y^a) - X(Y_a Z^a)
        - Y_a [Z,X]^a + Z_a [X,Y]^a + X_a [Y,Z]^a .
    \end{equation*}
\end{example}
\begin{proof}
    从相容性条件\eqref{chgd:eqn_connection-compatibility}出发,将它们写出并轮换
\begin{align*}
    X(Y^a Z_a)=& Z_a \nabla_{X}Y^a +Y_a\nabla_{X} Z^a.  \\
    Y(X^a Z_a)=& Z_a \nabla_{Y}X^a +X_a\nabla_{Y} Z^a.  \\
    Z(X^a Y_a)=& X_a \nabla_{Z}Y^a +Y_a\nabla_{Z} X^a.  
\end{align*}
    后两式相加再减掉第一式,得
    \begin{align*}
        &Y(X^a Z_a)+Z(X^a Y_a)-X(Y^a Z_a)= Z_a \nabla_{Y}X^a +X_a\nabla_{Y} Z^a
          +X_a \nabla_{Z}Y^a  \\
        &  \hphantom{Y(X^a Z_a)+Z(X^a Y_a)-X(Y^a Z_a)= }
          +Y_a\nabla_{Z} X^a-Z_a \nabla_{X}Y^a -Y_a\nabla_{X} Z^a \\
        &\xlongequal{\ref{chccr:eqn_XYcommutator}}  
         X_a\nabla_{Y} Z^a  +X_a \nabla_{Z}Y^a  +Z_a [Y,X]^a + Y_a [Z, X]^a \\
        &= 2X_a\nabla_{Y} Z^a  +X_a [Z,Y]^a  +Z_a [Y,X]^a + Y_a [Z, X]^a .
    \end{align*}
    将上式简单整理一下便可得到Koszul公式.
\end{proof}


\begin{example}\label{chrg:exm_S3}
    三维欧几里得空间球坐标系.
\end{example}
球坐标系$r,\theta,\phi$与平面直角坐标系$x,y,z$变换关系是
\begin{equation}\label{chrg:eqn_rtpxyz}
    x  = r\sin\theta \cos\phi ,\quad
    y  = r\sin\theta \sin\phi ,\quad
    z  = r\cos\theta .
\end{equation}
球坐标系$r,\theta,\phi$下的度规是
\begin{equation}\label{chgd:eqn_S3gab}
    g=\begin{pmatrix}  
        1 &0 &0 \\ 0& r^2 & 0 \\ 0& 0 & r^2 \sin^2\theta  
    \end{pmatrix} , \qquad
    g^{-1}=\begin{pmatrix}  
        1 &0&0 \\ 0& r^{-2} & 0 \\ 0& 0 & r^{-2} \sin^{-2}\theta  
    \end{pmatrix} .
\end{equation}
由\eqref{chrg:eqn_Christoffel-naturalbases}可得非零克氏符
\begin{equation}\label{chrg:eqn_S3Gamma}
    \begin{aligned}
        &\Gamma^r_{\theta\theta}=-r,\quad \Gamma^r_{\phi\phi}=-r \sin^2\theta,\quad
        \Gamma^\theta_{r\theta}=\Gamma^\theta_{\theta r}=\frac{1}{r}, \\
        & \Gamma^\theta_{\phi\phi}= - \sin \theta \cos\theta ,\quad
        \Gamma^\phi _{r \phi} = \Gamma^\phi _{\phi r}= \frac{1}{r},\quad
        \Gamma^\phi_{\theta\phi}= \Gamma^\phi_{\phi\theta} = \cot \theta .
    \end{aligned}
\end{equation}
利用上式,由式\eqref{chccr:eqn_Riemannian13-component}可得黎曼曲率所有分量都是零.
\qed


\index[physwords]{曲线长度}
\subsection{曲线长度}\label{chrg:sec_curve-length}

给定$m$维广义黎曼流形流形$(M,g)$,其任一点的切空间便是
通常的线性空间,依照推论\ref{chmla:thm_pm1num},其度规
有式\eqref{chmla:eqn_gmetric}的形式且没有非零分量(因非退化).

\begin{definition}\label{chrg:def_Minkowski-space}
    给定光滑广义黎曼流形$(M,g)$,如果其每点切空间的度规特征值(即式\eqref{chmla:eqn_gmetric})只有一个负号,
    其余皆为正号(本书使用度规符号是$(-+\cdots+)$,其它文献可能与此整体差一负号),
    则称$M$为{\heiti 广义Minkowski流形}.此度规称为{\heiti 广义Lorentz度规}或者广义闵氏度规.
    当没有“广义”两个字时,专指\uwave{四维}Minkowski流形(或其子流形),
    Lorentz度规的符号是$(-+++)$.
\end{definition}

{\kaishu 需注意,上述定义与定义\ref{chdm:def_MinkowskiSpace}有重复部分.
定义\ref{chdm:def_MinkowskiSpace}专指平直空间;
而定义\ref{chrg:def_Minkowski-space}可以是平直的,也可以是弯曲的.
平直与弯曲的定义见\ref{chrg:def_flat-curved}.
从此以后,定义\ref{chrg:def_Minkowski-space}包含定义\ref{chdm:def_MinkowskiSpace}.}

当度规场是非退化时,那么流形$M$度规场特征值的正负号逐点相同.
如果度规场特征值会改变,比如$M$上有点$p$和$q$,
其中$p$点度规是正定的,而$q$点度规是$(-+\cdots+)$;
那么根据度规是连续函数可以肯定存在一点的度规特征值为零(分量$g_{11}$),因而退化,矛盾!

\begin{definition}\label{chrg:def_vector-property}
    给定$m$维广义黎曼流形$(M,g)$,$\forall X^a \in \mathfrak{X}(M)$;
    如果$X^a$在任意坐标下的每个分量都为零,则称为{\heiti 零矢量}(Zero Vector);
    如果$g_{ab}X^a X^b<0$,则称为{\heiti 类时矢量}(time-like);
    如果$g_{ab}X^a X^b>0$,则称为{\heiti 类空矢量}(sapce-like);
    如果$g_{ab}X^a X^b=0$且$X^a$不是零矢量,则称为{\heiti 零模矢量}(Null Vector).
\end{definition}

\begin{definition}
    给定{\heiti 四维}Minkowski流形$(M,g)$,$\forall X^a \in \mathfrak{X}(M)$,
    若$X^a$是零模矢量,也可称之为{\heiti 类光矢量}(light-like).
    我们不在$m(>4)$维广义闵氏流形上使用类光矢量这一词语,
    我们只在四维流形及其子流形上使用类光矢量一词儿.
\end{definition}


在古典微分几何$\mathbb{R}^3$中,分段光滑曲线$\boldsymbol{r}(t)$弧长被定义为
\begin{equation*}
    s=\sum_{i} \int_{t_i}^{t_{i+1}} \left|\frac{{\rm d}\boldsymbol{r}}{{\rm d}t}\right| {\rm d} t
    =\sum_{i} \int_{t_i}^{t_{i+1}} \sqrt{\frac{{\rm d}\boldsymbol{r}}{{\rm d}t}\cdot
        \frac{{\rm d}\boldsymbol{r}}{{\rm d}t} } {\rm d} t
    =\sum_{i} \int_{t_i}^{t_{i+1}} \sqrt{{\rm d}\boldsymbol{r}\cdot{\rm d}\boldsymbol{r} }.
\end{equation*}
其中点${t_i}$是分段点;由于曲线的导数可正可负,所以加了绝对值;
弧长参数$s$被称为{\heiti 自然参数}.这个定义可以推广到广义黎曼流形中.


设广义黎曼流形$(M,g)$中有分段光滑曲线$\gamma:\mathbb{R}\to M$,曲线$\gamma(t)$的
切线切矢量是$\left(\frac{{\rm d}  }{{\rm d} t}\right)^a$(见式\eqref{chdm:eqn_DiffCurve-2}),
微分流形中的内积需要用度规来描述.曲线$\gamma(t)${\heiti 弧长}的定义为
\begin{equation}\label{chrg:eqn_arc-length}
    L(\gamma)\overset{def}{=}\sum_{i} \int_{t_i}^{t_{i+1}} \sqrt{\left| g_{ab} \left(\frac{{\rm d}  }
        {{\rm d} t}\right)^a_{\gamma}\left(\frac{{\rm d}  }{{\rm d} t}\right)^b_{\gamma}\right|} {\rm d}t .
\end{equation}
因度规可能是不定的,故上式中加了绝对值.
如果度规是正定的,即式\eqref{chmla:eqn_gmetric}中特征值全部是$+1$,
则没有必要加这个绝对值.

对于正定度规而言,式\eqref{chrg:eqn_arc-length}是个良定义,对于不定度规而言要复杂一些.
比如对于广义闵氏时空中曲线可能一段类时,与其毗邻的一段变成类空曲线;
虽然这种不伦不类曲线的线长(从数学角度)可以采取分段方式用式\eqref{chrg:eqn_arc-length}来定义,
但是这种不伦不类的曲线长度在物理上貌似没有什么意义.
%所以,我们\uwave{约定}式\eqref{chrg:eqn_arc-length}仅限于正定度规或者广义闵氏度规;
%当是后者时,整条曲线必须是类时或类空或零模的,不能是不伦不类的.
%当讨论物理问题时,我们不去涉及其它类型度规的曲线线长了,这样可以避免不必要的麻烦.

很明显,如果曲线是零模的,那么线长恒为零.

设流形$M$有局域坐标系$(U;x^i)$,分段光滑曲线$\gamma(t)$的分段$[t_i,t_{i+1}]$完全落
在一个坐标域内(若不是,增加分段);
则式\eqref{chrg:eqn_arc-length}在局部坐标系的表示为
\setlength{\mathindent}{0em}
\begin{equation}\label{chrg:eqn_arc-length-local}
    L(\gamma)=\sum_{i} \int_{t_i}^{t_{i+1}} \sqrt{\left| g_{jk}
        \frac{{\rm d} x^j\circ\gamma }{{\rm d} t}\frac{{\rm d} x^k\circ\gamma }{{\rm d} t}
        \right|} {\rm d}t
    =\sum_{i} \int_{t_i}^{t_{i+1}} \sqrt{\left| g_{jk} {\rm d} x^j {\rm d} x^k  \right|}  .
\end{equation}\setlength{\mathindent}{2em}
其中包括度规分量$g_{jk}= \left.\left[g_{ab}\left(\frac{\partial}{\partial x^j}\right)^a
\left(\frac{\partial}{\partial x^k}\right)^b \right] \right|_{\gamma(t)}$在内的所有
量必须在曲线$\gamma(t)$上取值.
曲线线长公式中被积分部分
\begin{equation}
    {\rm d}s^2 \equiv g_{jk} {\rm d} x^j {\rm d} x^k ,
\end{equation}
一般称为{\heiti 线元}.虽然是平方形式,但线元可正可负.可见给出线元就
相当于给出了局部坐标系$(U;x^i)$下的度规场,线元无非是度规场的一种记号而已.

\begin{example}
    在流形$\mathbb{R}^2$中,给定线元
    \begin{equation}
        {\rm d}s^2 = -x^2 {\rm d} t^2 + t {\rm d} x^2 + 4 {\rm d} t{\rm d} x.
    \end{equation}
    我们来读取度规场信息,
    在坐标系$\{t,x\}$下的度规分量是$g_{tt}=-x^2,\ g_{xx}=t,\ g_{tx}=g_{xt}=2 $.
    度规可用抽象指标记号表示为
    \begin{equation}
        g_{ab}= -x^2 ({\rm d} t)_a ({\rm d} t)_b +t ({\rm d} x)_a ({\rm d} x)_b
         + 2 ({\rm d} t)_a ({\rm d} x)_b + 2({\rm d} x)_a ({\rm d} t)_b .
    \end{equation}
    请读者熟悉从线元读取度规信息的方式、方法. \qed
\end{example}

为简单起见,下面只讨论曲线属性不变的一段$[r_1,r_2]$,然后拼接所有分段即可.
若曲线$\gamma(t)$重参数化为$\tilde{\gamma}(u)$,其中$u=u(t)$,
变换$u(t)$是单调的(见定义\ref{chdm:def_fmapCurve}后的讨论),
即$\frac{{\rm d} t }{{\rm d} u}>0$或$\frac{{\rm d} t }{{\rm d} u}<0$;
则曲线线长公式\eqref{chrg:eqn_arc-length}是不变的
\setlength{\mathindent}{0em}
\begin{align*}
    L(\tilde{\gamma}){=}\int_{u(r_1)}^{u(r_2)} \sqrt{\left| g_{ab} \left(\frac{{\rm d}  }
        {{\rm d} u}\right)^a\left(\frac{{\rm d}  }{{\rm d} u}\right)^b \right|} {\rm d}u
    =\int_{u(r_1)}^{u(r_2)} \sqrt{\left| g_{ab} \left(\frac{{\rm d}  }
        {{\rm d} t}\right)^a\left(\frac{{\rm d}  }{{\rm d} t}\right)^b\right|}
    \left|\frac{{\rm d} t }{{\rm d} u}\right| \cdot {\rm d}u .
\end{align*}\setlength{\mathindent}{2em}
不论$\frac{{\rm d} t }{{\rm d} u}>0$还是$\frac{{\rm d} t }{{\rm d} u}<0$,
带入上式后都会得到弧长公式在形式上是不变的,即与重参数化无关,
或者说在重参数化后曲线线长不变.

如果我们将曲线$\gamma(t)$参数$t$选为线长$s$,那么从上式可以看到必有
\begin{equation}\label{chrg:eqn_arc-unit}
    \sqrt{\left| g_{ab} \left(\frac{{\rm d}  } {{\rm d} s}\right)^a
        \left(\frac{{\rm d}  }{{\rm d} s}\right)^b \right|_{\gamma(s)}} = 1.
\end{equation}
也就是当光滑曲线$\gamma$的参数是弧长$s$时,其切线切矢
量$(\frac{{\rm d}  }{{\rm d} s})^a$的长度是$1$.

\index[physwords]{笛卡尔积!广义黎曼流形}
\subsection{广义黎曼流形之积}
设$(M,g)$和$(N,h)$是两个广义黎曼流形,维数分别是$m$和$n$;
两者笛卡尔积是$M\times N$,$\forall(p,q)\in M\times N$,其中$p\in M$,$q\in N$.
有自然投影$\pi_M: M\times N\to M$,$\pi_N: M\times N\to N$,
定义如下映射$\alpha:M\to M\times N$,$\beta:N\to M\times N$
(下式中$p\in M$,$q\in N$是任取的,取定后固定不变):
\begin{equation}
    \alpha(x)=(x,q),\ \forall x \in M; \qquad  \beta(y)=(p,y),\ \forall y \in N .
\end{equation}
显然有$\pi_M \circ \alpha = {\rm id}: M\to M$和$\pi_N \circ \beta = {\rm id}: N\to N$.
$\alpha$和$\beta$相当于包含映射,自然是正则嵌入的.

假设$T_p M$的局部坐标系是$\{x^i\}$,对应的自然坐标基矢场是$\{(\frac{\partial }{\partial x^i})^a\}$;
$T_p N$的局部坐标系是$\{y^\alpha\}$,对应的自然坐标基矢场是$\{(\frac{\partial }{\partial y^\alpha})^a\}$;
那么,$T_p (M\times N)$的局部坐标系是$\{x^i,y^\alpha\}$,
对应的自然坐标基矢场是$\{(\frac{\partial }{\partial x^i})^a,(\frac{\partial }{\partial y^\alpha})^a\}$;
很明显坐标$x$与$y$相互独立.
切映射$\alpha_{p*} : T_p M \to T_{(p,q)}(M\times N)$产生的基矢映射关系为:
\begin{equation}
    \alpha_{p*} \left. \left(\frac{\partial }{\partial x^i}\right)^a \right|_{p}=
    \frac{\partial x^k}{\partial x^i}\left(\frac{\partial }{\partial x^k}\right)^a
    +\frac{\partial y^\beta}{\partial x^i}\left(\frac{\partial }{\partial y^\beta}\right)^a
    =\left. \left(\frac{\partial }{\partial x^i}\right)^a \right|_{{(p,q)}}
\end{equation}
上式最左端的基矢是在流形$M$中取值;最右端基矢是在积流形$M\times N$中取值.
与上式类似,切映射$\beta_{q*} : T_q N \to T_{(p,q)}(M\times N)$产生的基矢映射关系为:
\begin{equation}
    \beta_{q*} \left. \left(\frac{\partial }{\partial y^\alpha}\right)^a \right|_{q}=
    \frac{\partial x^k}{\partial y^\alpha}\left(\frac{\partial }{\partial x^k}\right)^a
    +\frac{\partial y^\beta}{\partial y^j}\left(\frac{\partial }{\partial y^\beta}\right)^a
    =\left. \left(\frac{\partial }{\partial y^\alpha}\right)^a \right|_{{(p,q)}}
\end{equation}
上式最左端的基矢是在流形$N$中取值;最右端基矢是在积流形$M\times N$中取值.

上两式可以说明$(\alpha_{p*} T_p M) \cap (\beta_{q*}  T_q N) =\boldsymbol{0}$,即
两个切空间交集只有零矢量.而积流形$M\times N$的维数是$m+n$,自然它的切空间$T_{(p,q)} (M\times N)$维数也如此;
$T_p M$的维数是$m$,$T_p N$的维数是$n$;由此不难得到
\begin{equation}
    T_{(p,q)} (M\times N) = \alpha_{p*} T_p M \ \oplus\ \beta_{q*}  T_q N ,
\end{equation}
即积$M\times N$的切空间可以表示成$M$和$N$的切空间(用映射推前后)的\uwave{直和}.


$\forall X_M^a, Y_M^b \in T_p M$和$\forall X_N^a, Y_N^b \in T_q N$;记
\begin{equation}
    X^a=\alpha_{p*} X_M^a \ \oplus\ \beta_{q*}  X_N^a, \qquad Y^b=\alpha_{p*} Y_M^b \ \oplus\ \beta_{q*}  Y_N^b .
\end{equation}
定义积流形$M\times N$的度规为
\begin{equation}\label{chrg:eqn_metric-product}
    G_{ab}X^a Y^b \overset{def}{=} g_{ab}X^a_M Y^b_M + h_{ab}X^a_N Y^b_N .
\end{equation}
由$g_{ab},h_{ab}$的对称性可以得到$G_{ab}$是对称张量;光滑性也可类似得到.
下面验证它的非退化性,只需验证$G_{ab}$在基矢下的矩阵是非退化的即可;
设在给定基矢下$g$和$h$的矩阵分别是$g_{ij}$和$h_{\alpha\beta}$,
不难看出$G_{ab}$的$m+n$维方矩阵是
%(或相差一个非退化合同变换,即有非退化矩阵$X$,使得$G'=X^{-1}GX$)
\begin{equation}
    G_{(i\alpha)(j\beta)}=\begin{pmatrix}
        g_{ij} & 0 \\ 0 & h_{\alpha\beta}
    \end{pmatrix},
\end{equation}
因$\det(G_{(i\alpha)(j\beta)})=\det(g_{ij}) \cdot \det(h_{\alpha\beta})$,
故$G_{(i\alpha)(j\beta)}$非退化性是显然的.
这样便证明了$G_{ab}$符合广义黎曼度规要求.


显然本节所有结论都可以推广到有限个广义黎曼流形间的笛卡尔积.

\begin{exercise}
	证明式\eqref{chrg:eqn_Christoffel-2-naturalbases}定义的联络是无挠、容许、仿射联络.
\end{exercise}

\begin{exercise}
	证明式\eqref{chrg:eqn_DDR}.
\end{exercise}

\begin{exercise}
	详尽给出例题\ref{chrg:exam_Koszul}中略掉的步骤.
\end{exercise}


\section{曲率张量}
在\S \ref{chccr:sec_Curvatures}中给出了黎曼曲率张量定义,同时也描述了它的数条性质.
在引入度规后,再次研究这个重要的几何量;同时给出几个其它几个曲率的定义与性质.

\index[physwords]{黎曼曲率}
\subsection{$\Tpq{0}{4}$型黎曼曲率张量}\label{chrg:sec_R04}

广义黎曼流形$(M,g)$中已指定度规,约定$R_{\cdot cab}^d$上标降在第一个位置:
\begin{equation}\label{chrg:eqn_RiemannianCurvature-d4}
    R_{dcab} \overset{def}{=} g_{de}R_{\cdot cab}^e.
\end{equation}
由克氏符可求$\Tpq{0}{4}$型黎曼张量的分量表达式,
由式\eqref{chccr:eqn_Riemannian13-component}和\eqref{chrg:eqn_Christoffel-naturalbases}得
\setlength{\mathindent}{0em}
\begin{equation}\label{chrg:eqn_Riemannian04-component}
    R_{ijkn} =\frac{1}{2}\left(
           \frac{\partial^2 g_{in}} {\partial x^j\partial x^k}
         - \frac{\partial^2 g_{jn}} {\partial x^i\partial x^k}
         - \frac{\partial^2 g_{ik}} {\partial x^j\partial x^n}
         + \frac{\partial^2 g_{jk}} {\partial x^i\partial x^n} \right)
         + \Gamma_{jk}^l\Gamma _{lin} - \Gamma _{jn}^l\Gamma _{lik} .
\end{equation}\setlength{\mathindent}{2em}
计算过程并不困难,请读者补齐.
由式\eqref{chrg:eqn_Riemannian04-component}可见$\Tpq{0}{4}$型黎曼张量
对度规$g_{ij}$而言是半线性的,非线性部分只体现在一阶导数上
{\footnote{我们只以$\Tpq{0}{4}$型黎曼张量这个二阶偏导数式子来说明
线性、半线性、拟线性、非线性的描述性概念.如果此张量的一阶偏导数和
二阶偏导数本身是线性的(没有
$\frac{\partial^2 g_{jn}} {\partial  x^h\partial  x^q}\frac{\partial^2 g_{ik}} {\partial  x^l\partial  x^p}$,
$\frac{\partial^2 g_{jn}} {\partial  x^h\partial  x^q}\frac{\partial g_{ik}} {\partial  x^l}$,
$\frac{\partial^2 g_{jn}} {\partial  x^h\partial  x^i}g_{kl}$,
$\frac{\partial g_{jn}} {\partial  x^h}\frac{\partial g_{ik}} {\partial  x^l}$,
$\frac{\partial g_{kn}} {\partial  x^h}g_{ij}$,$g_{ij}g_{kn}$等项),
并且这些偏导数前的系数是常数或$x^i$的函数,那么称为线性.
所有不是线性的皆称为非线性,但是非线性内部又可以略作区分.
如果最高阶偏导数(此处指二阶偏导数,例如$\frac{\partial^2 g_{ik}}{\partial x^j\partial x^n} $)前的
系数是常数或者只是$x^i$的函数,那么称为半线性(semi-linear).
如果最高阶偏导数前的
系数除了包含$x^i$,还包含$g_{ij}$或$\frac{\partial g_{ij}} {\partial  x^k}$,
那么称为拟线性(quasi-linear).}}.
但此张量还包含度规的逆($g^{ij}$).


\index[physwords]{黎曼曲率!对称性}

\begin{theorem}\label{chrg:thm_Riemann-Sym-Properties}
广义黎曼流形$M$的黎曼曲率有如下对称性或反对称性:
\begin{align}
    R_{abcd} &= R_{cdab}, \label{chrg:eqn_Riemann-sym-abcd=cdab} \\
    R_{abcd} &= -R_{bacd} = -R_{abdc}, \label{chrg:eqn_Riemann-anti-sym}\\
    R_{a[bcd]} &= 0 \ \Leftrightarrow \ R_{[b|a|cd]} = 0
       \ \Leftrightarrow \ R_{[bc|a|d]} = 0 \ \Leftrightarrow \ R_{[bcd]a} = 0, \\
    R_{ab[cd;e]} &= 0, \\
    R_{[abcd]} &= 0 \label{chrg:eqn_Riemann-cycle-4}.
\end{align}
\end{theorem}
\begin{proof}
    由于已经证明过黎曼曲率的分量表达式是坐标卡变换协变的(见\S \ref{chccr:sec_rit}),所以
    可以由式\eqref{chrg:eqn_Riemannian04-component}的对称性直接得到式\eqref{chrg:eqn_Riemann-sym-abcd=cdab}.

    借用式\eqref{chrg:eqn_Riemann-sym-abcd=cdab},
    由式\eqref{chccr:eqn_Rab=-Rba}可证明式\eqref{chrg:eqn_Riemann-anti-sym}的全部.

    接下来是第一、第二Bianchi恒等式(已由式\eqref{chccr:eqn_Bianchi-I-global}和\eqref{chccr:eqn_Bianchi-II-global}证明).

    下面证明式\eqref{chrg:eqn_Riemann-cycle-4};
    由式\eqref{chmla:eqn_gkd-25}和式\eqref{chmla:eqn_gkd-95}可得
    \begin{align*}
        24 \cdot R_{[abcd]}  =& \delta_{abcd}^{ijkl} R_{ijkl}
        =\Bigl(-\delta_{a}^{l} \cdot \delta_{bcd}^{ijk} +\delta_{b}^{l} \cdot \delta_{acd}^{ijk}
         -\delta_{c}^{l} \cdot \delta_{abd}^{ijk}  +\delta_{d}^{l} \cdot \delta_{abc}^{ijk} \Bigr)R_{ijkl} \\
        =& -R_{[bcd]a} + R_{[acd]b} - R_{[abd]c} + R_{[abc]d} =0 .
    \end{align*}
    上式最后一步用到了第一Bianchi恒等式.
\end{proof}

\index[physwords]{黎曼曲率!数目}
\begin{example}
    $m$维黎曼流形$(M,g)$中黎曼曲率张量$R_{abcd}$的独立取值数目.
\end{example}
\noindent{\heiti 解}:由式\eqref{chrg:eqn_Riemann-sym-abcd=cdab}知可把
黎曼曲率$R_{abcd}$当成“指标”为$(ab)$和$(cd)$的矩阵$R_{(ab)(cd)}$,
由式\eqref{chrg:eqn_Riemann-anti-sym}的反对称性可以看出,每一个“指标”
取独立值的数目是:$\frac{1}{2}m(m-1)$.而“指标”为$(ab)$和$(cd)$是对称
的,所以单由式\eqref{chrg:eqn_Riemann-sym-abcd=cdab}和\eqref{chrg:eqn_Riemann-anti-sym}使
张量$R_{abcd}$留下的独立分量数目等于$\frac{1}{2}m(m-1)$维对称矩阵的独立矩阵元数目,即
$\frac{1}{2}\bigl[\frac{1}{2}m(m-1)\bigr] \bigl[\frac{1}{2}m(m-1)+1\bigr]$.
而全反对称\eqref{chrg:eqn_Riemann-cycle-4}又给附加上
$m(m-1)(m-2)(m-3)/4!$个限制,最终黎曼曲率张量$R_{abcd}$留下的独立分量数目等于
\begin{align}
    C_m &= \frac{1}{2}\left[\frac{1}{2}m(m-1)\right] \left[\frac{1}{2}m(m-1)+1\right]
             - \frac{1}{4!}m(m-1)(m-2)(m-3)  \notag\\
      &= \frac{1}{12} m^2(m^2-1) . \label{chrg:eqn_Number-Of-Riemannd4}
\end{align}
需要强调的是,上述独立分量个数只是代数意义上的,由于还有第二Bianchi微分恒等式,
独立分量数目还会减少. \qed

当维数$m=1$时$C_1=0$,这说明曲线的内秉曲率$R_{1111}$恒为零.当曲线嵌入到
高维空间中,可以有非零的外曲率,但这不是内蕴性质. %可参考\S.

当维数$m=2$时$C_2=1$,只有一个独立的分量,本质上是高斯曲率. %可参考\S.

当维数$m=3$时$C_3=6$;$m=4$时$C_4=20$.


\index[physwords]{Ricci曲率}
\index[physwords]{标量曲率}
\subsection{Ricci曲率、标量曲率}
定义Ricci曲率张量$R_{cb}$和标量曲率$R$:
\begin{align}
    R_{cb} \overset{def}{=}& R_{\cdot cab}^a \equiv g^{ad}R_{dcab}=-g^{ad}R_{dcba} . \label{chrg:eqn_RicciCurvatured2}\\
    R \overset{def}{=}& g^{ab} R_{ab}. \label{chrg:eqn_ScalarCurvature}
\end{align}
从Ricci曲率张量定义可以看到:它是一个\uwave{\kaishu 对称张量}.
由式\eqref{chrg:eqn_Riemann-anti-sym}和\eqref{chrg:eqn_Riemann-sym-abcd=cdab}可知:
\begin{align*}
    g^{ab}R_{abcd} &=0= g^{cd}R_{abcd}, \\
    R_{bd} & = g^{ac}R_{abcd} = -g^{ac}R_{abdc} = -g^{ac}R_{bacd} = + g^{ac}R_{badc} .
\end{align*}
所以Ricci曲率$R_{bd}$本质上是唯一能从$R_{abcd}$构造出来的二阶张量.
同样,标量曲率本质上也是唯一的.
\begin{equation*}
    0 = g^{ab} g^{bd}R_{abcd}, \qquad     R  = g^{bd} g^{ac}R_{abcd} .
\end{equation*}
从式\eqref{chrg:eqn_Riemann-cycle-4}可得$\epsilon^{abcd}R_{abcd} =0$;
故,由反对称性构造标量的途径被否定了.




\index[physwords]{爱因斯坦张量}
\subsection{爱因斯坦张量}
定义爱因斯坦张量$G_{ab}$为
\begin{equation}\label{chrg:eqn_Einstein-tensor}
    G_{ab} \overset{def}{=} R_{ab} -\frac{1}{2}g_{ab} R .
\end{equation}
其中$R_{ab},R$分别是Ricci曲率和标量曲率.

由第二Bianchi恒等式\eqref{chccr:eqn_Bianchi-II-global}可以证明爱因斯坦张量的一个重要特征.
首先,利用度规$g_{ab}$与联络$\nabla_c$相容,先将式\eqref{chccr:eqn_Bianchi-II-global}中的上指标“$d$”降下来,
再对式\eqref{chccr:eqn_Bianchi-II-global}两端进行缩并运算,得
\begin{align}
   0=&  g^{ec}g^{ad}(\nabla_a R_{debc} + \nabla_b R_{deca} + \nabla_c R_{deab} )
   = g^{ec}(\nabla^a R_{aebc} - \nabla_b R_{ec} + \nabla_c R_{eb} ) \notag \\
   =& \nabla^a R_{ab} - \nabla_b R + \nabla^e R_{eb}
   = 2\nabla^a R_{ab} - \nabla_b R . \label{chrg:eqn_Bianchi-contract}
\end{align}
由此式可得如下重要公式(\uwave{缩并Bianchi微分恒等式}):
\begin{equation}\label{chrg:eqn_Div-Einstein-tensor=0}
    \nabla^a G_{ab} \equiv \nabla^a R_{ab} -\frac{1}{2}g_{ab} \nabla^a R \equiv 0 .
\end{equation}
需注意,微分恒等式\eqref{chrg:eqn_Div-Einstein-tensor=0}对任意$m$维广义黎曼流形都成立;
并且只要$G_{ab}$满足可微条件即可,不论$G_{ab}$是否满足爱氏方程组
($G_{ab}=8\pi T_{ab}$).

%\begin{definition}\label{chrg:def_Einstein-manifold}
%    设有$m$维广义黎曼流形$(M,g)$,如果存在实常数$\lambda \in \mathbb{R}$使得
%    \begin{equation}\label{chrg:eqn_Einstein-manifold}
%        R_{ab} = \lambda \cdot g_{ab},
%    \end{equation}
%    在流形$M$上每一点都成立,则称$(M,g)$是{\heiti 爱因斯坦流形}.
%\end{definition}

\index[physwords]{Weyl张量}
\subsection{Weyl张量}
设有$m$维广义黎曼流形$(M,g)$,定义Weyl张量,也称为共形张量.
\begin{equation}\label{chrg:eqn_WeylConform-d4}
    \begin{aligned}
        C_{abcd}\overset{def}{=}& R_{abcd} - \frac{1}{m-2}\left( g_{ac}R_{bd} - g_{ad}R_{bc}
        + g_{bd}R_{ac} - g_{bc}R_{ad}\right)  \\
        &  + \frac{1}{(m-1)(m-2)}\left(g_{ac}g_{bd}-g_{ad}g_{bc}\right) R .
    \end{aligned}
\end{equation}
要求$m\geqslant3$;一维、二维流形中无法定义Weyl张量.

$C_{abcd}$有$R_{abcd}$全部对称性(见定理\ref{chrg:thm_Riemann-Sym-Properties}).
此外,Weyl张量的迹恒零.
\begin{equation}
        g^{ac}C_{abcd}{=} 0.
\end{equation}
直接计算即可证明上式.当只考虑Weyl张量与黎曼张量有相同的对称性,
可以得到独立个数相同于黎曼张量独立个数;但
考虑到“指标”$(ab)$和$(cd)$对称性,上式对独立Weyl张量分量个数给出的限制是
$\frac{1}{2}m(m+1)$,最终Weyl张量独立分量个数是
\begin{equation}\label{chrg:eqn_Number-Of-Weyld4}
    C_m = \frac{1}{12} m^2(m^2-1) - \frac{1}{2}m(m+1)
      =\frac{1}{12} m(m+1)(m+2)(m-3) .
\end{equation}
当$m=3$时,独立Weyl张量是零;当$m=4$时,独立Weyl张量是$C_4=10$个.

三维空间中,黎曼曲率和Ricci曲率都有6个独立分量,上面的公式
也说明此时独立Weyl张量的分量个数是零,所以Ricci曲率代表了全部
黎曼曲率,Weyl张量恒为零.


\subsection{四维流形的曲率}
%由于相对论是四维时空,我们单独讨论一下.
\paragraph{黎曼曲率}
由式\eqref{chrg:eqn_Number-Of-Riemannd4}可知四维黎曼曲率
独立分量个数是20个.从定理\ref{chrg:thm_Riemann-Sym-Properties}经过
对称性分析,可将$R_{abcd}$指标取为$(ab)$和$(cd)$的矩阵,
写成表格\ref{chrg:tab-riemann}.
因为此矩阵是对称的,所以只需写出一半;同时这个表格忽略了恒为零的分量.
由式\eqref{chccr:eqn_Bianchi-I-global},
有$R_{1234}+R_{1342}+R_{1423}=0$,可令$R_{1234}$不独立,已在表中消掉.
\begin{table}[htb]
    \centering
    \caption{独立黎曼曲率分量} \label{chrg:tab-riemann}
    \begin{tabular}{|*{7}{c|}}
        \hline
        \diagbox{行}{分量}{列} & \makecell{N=2\\(12)} &
        \multicolumn{2}{c|}{\makecell{N=3\\(13) \quad (23)}} &
        \multicolumn{3}{c|}{\makecell{N=4\\ (14) \quad (24) \quad (34) }}   {}\\        \hline
        (12) & 1212 & 1213  & 1223 & 1214 & 1224 & \xcancel{1234}  \\ \hline
        \multirowcell{2}{(13)\\(23)}
        & {} & 1313  & 1323 & 1314 & 1324 & 1334  \\ \cline{2-7}
        & {} &  {}  & 2323 & 2314 & 2324 & 2334  \\ \hline
        \multirowcell{3}{(14)\\(24)\\(34)}
        & {} & {}  & {} & 1414 & 1424 & 1434  \\ \cline{2-7}
        & {} & {}  & {} & {} & 2424 & 2434  \\ \cline{2-7}
        & {} & {}  & {} & {} & {} & 3434  \\ \hline
    \end{tabular}
\end{table}

%\subsubsection{Weyl、Ricci张量}
%Ricci曲率和Weyl张量独立分量个数都是10个;结合起来正号是20个,
%与独立黎曼曲率分量相等.
%\begin{equation}
%    C_{abcd}{=} R_{abcd} - \frac{1}{2}\left( g_{ac}R_{bd} - g_{ad}R_{bc}
%    + g_{bd}R_{ac} - g_{bc}R_{ad}\right)
%    + \frac{1}{6}\left(g_{ac}g_{bd}-g_{ad}g_{bc}\right) R .
%\end{equation}


\paragraph{Bianchi恒等式}\label{chrg:sec_NumOfBianchi}
此处,我们只讨论四维流形上的Bianchi恒等式\eqref{chccr:eqn_Bianchi-II-global}$R_{ij[kl;s]} = 0$ .
前两个指标$(ij)$是对称的,所以有6个可区分;后三个指标
要求$k\neq l \neq s$且反对称循环相加,所以只有$C_4^3=4$个可区分,
综上,共有24个可区分的Bianchi恒等式.但是由于各种对称性,下列
四个式子是恒为零的(请读者自行验证),不能算作方程等式.
\begin{equation*}
    \begin{aligned}
        R_{12[34;1]} - R_{13[41;2]} + R_{14[12;3]} &=0,  \quad
        R_{23[41;2]} - R_{24[12;3]} + R_{21[23;4]} =0,  \\
        R_{34[12;3]} - R_{31[23;4]} + R_{32[34;1]} &=0,  \quad
        R_{41[23;4]} - R_{42[34;1]} + R_{43[41;2]} =0.
    \end{aligned}
\end{equation*}
这导致Bianchi恒等式只有20个线性独立.


\begin{exercise}
	证明式\eqref{chrg:eqn_Riemannian04-component}.
\end{exercise}

\begin{exercise}
	证明二维爱因斯坦张量恒为零(见式\eqref{chrg:eqn_Einstein-tensor}).
\end{exercise}

\begin{exercise}
	证明三维Weyl张量恒为零(见式\eqref{chrg:eqn_WeylConform-d4}).
\end{exercise}




\index[physwords]{等距}
\section{等距映射}\label{chrg:sec_isometry}

%本节初步涉及对称性知识,在第\ref{chhss}章将进一步讲述对称性.

\subsection{线性空间等距}
%我们先从线性空间讲起,然后再叙述一般流形上的情形.

\begin{definition}\label{chrg:def_isometry-VW}
设在实数域上有两个线性空间$V$和$W$,$V$上定义了度规$g$,$W$上定义了度规$h$;
两个空间存在同构映射$T:V\to W$;若$T$还保度规不变,即
\begin{equation}\label{chrg:eqn_isometry-VW}
    g_{ab}v^a u^b = h_{ab} T(v^a) T(u^b)    \ \Leftrightarrow \ 
    \left<v,u\right>_g = \left<T(v),T(u)\right>_h,
    \quad    \forall v^a, u^b \in V.
\end{equation}
则称$T$是线性空间上的{\heiti 等距映射}.线性空间度规定义见\S\ref{chdm:sec_Euclidean-space}.
\end{definition}

\begin{proposition}\label{chrg:thm_isometry-VW}
设有两个线性空间$V$和$W$,$V$上定义了度规$g$,$W$上定义了度规$h$.
存在等距映射$T:V\to W$的充分必要条件是:
$V$、$W$有相同的维数,$g$的正(负)特征值个数等于$h$的正(负)特征值个数.    
\end{proposition}
\begin{proof}
        度规定义要求它是非退化的.
    在$V$中选定正交归一基矢$\{(e_i)^a\}$,在$W$中选定正交归一基矢$\{(\epsilon_j)^a\}$.
    在正交归一基矢下,度规特征值参见式\eqref{chmla:eqn_gmetric}.
    
    “$\Rightarrow$”.$T$等距,则它首先是线性同构,故$V$、$W$的维数相等.
    将$V$中基矢$\{(e_i)^a\}$带入式\eqref{chrg:eqn_isometry-VW}可
    得(其中$\eta={\rm diag}(-\cdots -+\cdots +)$)
    \begin{equation}
         h_{ab} T\bigl((e_i)^a\bigr) T\bigl((e_j)^b\bigr) = g_{ab} (e_i)^a(e_j)^b = \eta_{ij} .
    \end{equation}
    $\{(e_i)^a\}$在等距、线性同构下的像是$T\bigl((e_i)^a\bigr)$,显然它是$W$中的基底;
    故在$W$中存在基底使得$h$的正(负)特征值个数等于$g$的正(负)特征值个数.  
    
    “$\Leftarrow$”.因$V$、$W$有相同的维数,故必然存在线性同构映射$\sigma:V\to W$,
    我们将具体表达式选为$\sigma\bigl((e_i)^a\bigr) = (\epsilon_i)^a$
    (这个线性映射是人为选定的,如果选为其它形式,
    比如$\sigma\bigl((e_i)^a\bigr) =-9.4 (\epsilon_i)^a$,那么会给证明造成困难).
    由于$g$的正(负)特征值个数等于$h$的正(负)特征值个数,故
    可以令$g_{ab}(e_i)^a(e_j)^b = h_{ab}(\epsilon_i)^a(\epsilon_j)^b$
    (因$(e_i)^a$和$(\epsilon_i)^a$都是正交归一的,故可作此种选择).
    在上述线性同构映射约定下,以及度规$g$、$h$关系约定下,$\sigma$是等距的,
    即$h_{ab} \sigma(v^a) \sigma(u^b)=h_{ab} \sigma\bigl(v^i(e_i)^a\bigr) \sigma(u^j(e_j)^b)
    =v^i u^j h_{ab} \sigma\bigl((e_i)^a\bigr) \sigma((e_j)^b) 
    =v^i u^j h_{ab} (\epsilon_i)^a (\epsilon_j)^b 
    =v^i u^j g_{ab} (e_i)^a (e_j)^b  = g_{ab}v^a u^b$.
    令$T=\sigma$,证毕.
\end{proof}



\begin{theorem}\label{chmla:thm_Witt-isometry}
    设在实数域上有两个线性空间$V$和$W$,$V$上定义了度规$g$,$W$上定义了度规$h$;$V$和$W$是等距的.
    $U$是$V$的非零、非退化子空间,且映射$\sigma:U\to \sigma(U)\subset W$是等距映射;
    那么$\sigma$能够延拓成$V$到$W$的等距映射.    
\end{theorem}

\begin{proof}
    由定理\ref{chmla:thm_bifunmatrix}可知,线性空间上度规与非退化、对称实矩阵有双射关系.
    设$g$在$U$上的限制所对应的矩阵是$A_1$,$h$在$\sigma(U)$上的限制所对应的矩阵是$A_2$,
    很明显$A_1,A_2$的维数是相同的.因$\sigma$是$U$到$\sigma(U)$的等距,不难证明
    矩阵$A_1\overset{c}{\sim} A_2$,即两个矩阵是合同关系.
    
    因$g$在$U$上限制非退化,故$\sigma(U)$也非退化;
    $V$和$W$可作正交直和分解:
    \begin{equation}\label{chrg:eqn_tmp-wioc}
        V = U \oplus U^\perp, \qquad
        W = \sigma(U) \oplus \bigl(\sigma(U)\bigr)^\perp .
    \end{equation}   
    因$V$和$W$是等距的,那么$g$、$h$所对应的矩阵自然也是合同的:
    \begin{equation}
        (g)=\begin{pmatrix}  A_1 &0 \\ 0 & B_1  \end{pmatrix} \overset{c}{\sim}
        \begin{pmatrix}  A_2 &0 \\ 0 & B_2  \end{pmatrix} =(h) .
    \end{equation}
    其中$B_1$是$g$在$U^\perp$上对应的非退化、对称矩阵;
    $B_2$是$h$在$\bigl(\sigma(U)\bigr)^\perp$上对应的非退化、对称矩阵;
    $B_1,B_2$维数相同.
    由Witt定理\ref{chmla:thm_Witt-cancel}可知必有$B_1 \overset{c}{\sim} B_2$,
    既然是合同关系,我们可以认为存在等距映射$\mu:U^\perp \to \bigl(\sigma(U)\bigr)^\perp$.
    因有正交补关系\eqref{chrg:eqn_tmp-wioc},故$\forall v^a\in V$都可以唯一分解
    成:$v^a = u^a + u_p^a$,其中$u^a\in U, u_p^a \in U^\perp$.
    现在定义新的等距映射$\nu:V\to W$:
    \begin{equation}
        \nu (v^a)= \nu(u^a + u_p^a) \overset{def}{=}  \sigma(u^a) + \mu(u_p^a) .
    \end{equation}
    当$v^a\in U$时,即$u_p^a=0$时,$\nu=\sigma$;故$\nu$是$\sigma$的延拓.
    
    本定理一般称作{\bfseries \heiti Witt等距延拓定理}.
    \index[physwords]{Witt等距延拓定理}
\end{proof}

\subsection{微分流形等距}
给定两个广义黎曼流形$(M,g)$和$(N,h)$,它们的维数分别是$m$和$n$,且$m \leqslant n$;
两个流形间存在光滑浸入(或嵌入)映射$\phi:M\to N$.
\begin{definition}\label{chrg:def_isometry-immersion}
    如果$g_{ab}=\phi^{*}h_{ab}$,即$\forall p \in M$,以及$\forall u^a,v^b \in T_{p}M$都有
    \begin{equation}
        g_{ab} u^a v^b = h_{ab} (\phi_{*} u^a) (\phi_{*} v^b) 
        \quad \Leftrightarrow \quad
        \left<u,v\right>_g = \left<\phi_{*}u,\  \phi_{*}v\right>_h,
    \end{equation}
    则称$\phi$是从$(M,g)$到$(N,h)$的一个{\heiti 等距浸入}(或嵌入)映射.
\end{definition}
等距浸入在局部上必是等距嵌入.等距浸入不要求两个流形维数相等.
浸入映射$\phi$会将$(N,h)$中的度规拉回到$M$中,这时$M$中有两个
度规,但$g_{ab}$未必等于$\phi^{*}h_{ab}$;只有两者相等才是等距的.

\begin{example}
    在等距浸入\ref{chrg:def_isometry-immersion}中,两个流形维数未必相同.
    比如给定流形$M$是实直线,和流形$N$是二维实平面,两者都是正定欧氏度规;
    自然存在将实直线等距浸入(其实是正则嵌入)到二维平面的包含映射$\imath$.
\end{example}
\begin{example}
    把二维球面正则嵌入到$\mathbb{R}^3$中也是等距嵌入,详见例\ref{chsm:exm_S2}.
\end{example}

\index[physwords]{等距!整体同胚}
\index[physwords]{等距!局部同构}
\index[physwords]{等距!局部同胚}
\index[physwords]{等距!浸入、嵌入}

\begin{definition}\label{chrg:def_isometry-diffeomorphism}
    若$\phi$是等距浸入映射,再要求$\phi$是微分同胚,那么称$\phi$是
    从$(M,g)$到$(N,h)$的一个{\heiti 整体等距同胚}映射,简称{\heiti 等距}.
\end{definition}


\begin{definition}\label{chrg:def_isometry-local-isomorphism}
    若$\phi$是等距浸入映射,再要求$\forall p\in M$,$\phi_{*}:T_{p}M \to T_{\phi(p)}N$是
    线性同构,则称$\phi$是从$(M,g)$到$(N,h)$的{\heiti 局部等距同构}映射,
    简称{\heiti 局部等距}.
\end{definition}
整体等距{\kaishu 同胚}自然可以得到流形$M$和$N$的维数相等.
局部等距{\kaishu 同构}要求$M$和$N$的切空间是同构,
自然得到$M$和$N$的切空间维数相等,从而流形$M$和$N$的维数也相等.
局部等距同构映射并不要求整体上是微分同胚,只要求每点切空间线性等距同构即可;
由于局部等距时两个流形的维数相等,此时的浸入是局部微分同胚,
即$p\in M$点某开邻域到$\phi(p)\in N$点某开邻域间的等距同胚;
故局部等距同构也可以称为{\heiti 局部等距同胚}.

\begin{example}
    设$S^1$是$\mathbb{R}^2$中的单位圆,指数映射$\exp\equiv(\cos t, \sin t): \mathbb{R}^1 \to S^1$,
    将$\mathbb{R}^1$中的非紧致开集映射到$\mathbb{R}^2$中的紧致闭子集$S^1$.
    指数映射$\exp$只是切空间的局部等距同构映射,不是流形上的整体等距同胚映射.
    流形$\mathbb{R}$和$S^1$之间不存在微分同胚映射.
    这个例子清楚地显示了局部等距概念和整体等距概念都是需要的.
\end{example}

\begin{example}\label{chrg:exam_isometry-MNcoord}
    设两个黎曼流形$(M,g)$和$(N,h)$间存在整体等距同胚映射$\phi:M\to N$,
    再设$(U;x^i)$和$(V;y^\alpha)$分别是$M$和$N$的局部坐标.
    依照等距同胚的定义,参考式\eqref{chdm:eqn_push-bases},有
    \begin{small}
    \setlength{\mathindent}{1em}
    \begin{equation}\label{chrg:eqn_isometry-MNcoord}
        g_{ab} \left(\frac{\partial }{\partial x^i}\right)^a
         \left(\frac{\partial }{\partial x^j}\right)^b = h_{ab}
        \left(\phi_{*} \left(\frac{\partial }{\partial x^i} \right)^a \right)
        \left(\phi_{*} \left(\frac{\partial }{\partial x^j} \right)^b \right)
        {\color{red} \Leftrightarrow }  g_{ij}=\frac{\partial y^\alpha}{\partial x^i}
        \frac{\partial y^\beta}{\partial x^j} h_{\alpha\beta} .
    \end{equation}\setlength{\mathindent}{2em}
    \end{small}
    这便是等距同胚(或局部等距同构)在局部坐标系下的具体表示.
\end{example}


%有时候,这种维数的不同可能会造成一种误解:认为我们在研究两个维数不同的流形;实际上并不是.
%上面那个例子中,我们只研究$M$与$\imath(M)$间的等距,而不会关心$\imath(M)$之外的部分;
%也就是不关心二维平面中不属于$\imath(M)$(一维实直线)的部分.
%因此即便是等距浸入(非局部同构):我们也{\kaishu 只研究维数相同流形间的等距};
%因此可以认为所有等距映射都是{\kaishu 局部等距同构}的.

%g_{ab} \left(\frac{\partial }{\partial x^i}\right)^a
%\left(\frac{\partial }{\partial x^j}\right)^b = g_{ab}
%\left(\phi_{*} \left(\frac{\partial }{\partial x^i} \right)^a \right)
%\left(\phi_{*} \left(\frac{\partial }{\partial x^j} \right)^b \right)
%{\color{red} \Leftrightarrow }




\index[physwords]{等距!保Levi-Civita联络}
\subsection{等距同胚保联络不变}\label{chrg:sec_isometry-connection}
\begin{theorem}\label{chrg:thm_isometry-connection-vector}
    给定两个广义黎曼流形及Levi-Civita联络$(M,\nabla_a,g)$和$(N,{\rm D}_a,h)$,
    它们之间存在局部等距映射$\phi:M\to N$;那么,$\forall p\in M$,
    和$\forall X^a \in T_{p}M$,$ \forall Y^a \in \mathfrak{X}(M)$,都有
    \begin{equation}\label{chrg:eqn_isometry-connection-vec}
    \phi_{*} \left(\nabla _X Y^a \right) = {\rm D}_ {\phi_* X} \left({\phi_*}Y^a \right) .
    \end{equation}
    即,作用在切矢场上的Levi-Civita联络$\nabla_a$和${\rm D}_a$在局部等距下形式不变.
\end{theorem}
\begin{proof}
    设$(U;x^i)$和$(V;y^\alpha)$分别是$M$和$N$的局部坐标.
    式\eqref{chrg:eqn_isometry-connection-vec}左端为
    \begin{align*}
    \phi_{*} (\nabla _X Y^a) %& = \phi_{*} \left[\nabla_{(X^i \frac{\partial}{\partial x^i})}
%     \left(Y^j \Bigl(\frac{\partial} {\partial x^j}\Bigr)^a \right) \right]  \\
%     &= \phi_{*} \left[\Bigl( X^i \frac{\partial Y^j}{\partial x^i} \Bigr)
%        \left(\frac{\partial} {\partial x^j}\right)^a
%        + ( X^i Y^j ) \nabla _{\frac{\partial}{\partial x^i}}
%        \left(\frac{\partial} {\partial x^j}\right)^a\right] \\
     &=  \Bigl( X^i \frac{\partial Y^j}{\partial x^i} \Bigr)
\phi_{*}\left(\frac{\partial} {\partial x^j}\right)^a
+ ( X^i Y^j ) \phi_{*}\left[ \nabla _{\frac{\partial}{\partial x^i}}
\left(\frac{\partial} {\partial x^j}\right)^a\right] .
    \end{align*}
    式\eqref{chrg:eqn_isometry-connection-vec}右端为(参考式\eqref{chdm:eqn_push-bases})
    \begin{align*}
    {\rm D}_ {\phi_* X} ({\phi_*}Y^a) = %& X^i \left( {\rm D}_ {\phi_* \frac{\partial}{\partial x^i}}
%     Y^j(x) \right) {\phi_*}\Bigl(\frac{\partial} {\partial x^j}\Bigr)^a
%    +X^i  Y^j(x)  {\rm D}_ {\phi_* \frac{\partial}{\partial x^i}}
%    \left({\phi_*}\Bigl(\frac{\partial} {\partial x^j}\Bigr)^a \right) \\
%    \xlongequal{\ref{chdm:eqn_push-bases}}&
     X^i \left( \frac{\partial y^\alpha}{\partial x^i}
     \frac{\partial Y^j(x)}{\partial y^\alpha} \right)
    {\phi_*}\Bigl(\frac{\partial} {\partial x^j}\Bigr)^a
    +X^i  Y^j(x)  {\rm D}_ {\phi_* \frac{\partial}{\partial x^i}}
    \left({\phi_*}\Bigl(\frac{\partial} {\partial x^j}\Bigr)^a \right) .
    \end{align*}
    综合上面两式,可知要证明式\eqref{chrg:eqn_isometry-connection-vec},等价于证明
    \begin{equation}\label{chrg:eqn_isometry-connection-vec2}
      \phi_{*}\left( \nabla _{\frac{\partial}{\partial x^i}}
      \Bigl(\frac{\partial} {\partial x^j}\Bigr)^a\right)
      ={\rm D}_ {\phi_* \frac{\partial}{\partial x^i}}
      \left({\phi_*}\Bigl(\frac{\partial} {\partial x^j}\Bigr)^a \right) .
    \end{equation}
    式\eqref{chrg:eqn_isometry-connection-vec2}左端为
    \begin{equation*}
       \phi_{*}\left( \nabla _{\frac{\partial}{\partial x^i}}\Bigl(\frac{\partial} {\partial x^j}\Bigr)^a\right)
        =\phi_{*}\left( \Gamma^{k}_{ji}(x) \Bigl(\frac{\partial} {\partial x^k}\Bigr)^a\right)
        =\Gamma^{k}_{ji}(x) \frac{\partial y^\alpha}{\partial x^k}
         \left( \frac{\partial }{\partial y^\alpha}\right)^a .
    \end{equation*}
    其中$\Gamma(x)$是$\nabla_a$的联络系数.
    式\eqref{chrg:eqn_isometry-connection-vec2}右端为
    \begin{align*}
        {\rm D}_ {\phi_* \frac{\partial}{\partial x^i}} \left({\phi_*}\Bigl(\frac{\partial}
        {\partial x^j}\Bigr)^a \right) &= \frac{\partial y^\alpha}{\partial x^i}
         {\rm D}_{\frac{\partial}{\partial y^\alpha}}
         \left( \frac{\partial y^\beta}{\partial x^j}
          \Bigl(\frac{\partial}{\partial y^\beta}\Bigr)^a\right)\\
%        &=\Bigl(\frac{\partial}{\partial y^\beta}\Bigr)^a
%        \frac{\partial y^\alpha}{\partial x^i}
%        {\rm D}_{\frac{\partial}{\partial y^\alpha}}
%        \left( \frac{\partial y^\beta}{\partial x^j} \right) +
%        \frac{\partial y^\alpha}{\partial x^i} \frac{\partial y^\beta}{\partial x^j}
%        {\rm D}_{\frac{\partial}{\partial y^\alpha}}
%        \left( \Bigl(\frac{\partial}{\partial y^\beta}\Bigr)^a\right)\\
        &=\Bigl(\frac{\partial}{\partial y^\beta}\Bigr)^a
        \frac{\partial y^\alpha}{\partial x^i} {\frac{\partial}{\partial y^\alpha}}
        \left( \frac{\partial y^\beta}{\partial x^j} \right) +
        \frac{\partial y^\alpha}{\partial x^i} \frac{\partial y^\beta}{\partial x^j}
        \gamma^{\rho}_{\beta\alpha} \Bigl(\frac{\partial}{\partial y^\rho}\Bigr)^a .
    \end{align*}
    其中$\gamma$是${\rm D}_a$联络系数.
    综合上述两式,要证明式\eqref{chrg:eqn_isometry-connection-vec2}等价于证明
    \begin{equation}\label{chrg:eqn_isometry-connection-tmp}
       \Gamma^{k}_{ji}(x) \frac{\partial y^\rho}{\partial x^k} =
         \frac{\partial^2 y^\rho}{\partial x^j \partial x^i}+
         \frac{\partial y^\alpha}{\partial x^i}
         \frac{\partial y^\beta}{\partial x^j}
         \gamma^{\rho}_{\beta\alpha}(y) .
    \end{equation}
    要证明上式,需要用到局部等距的定义,以及式\eqref{chrg:eqn_isometry-MNcoord}和它的逆变换(即下式):
    \begin{equation}
       h_{\alpha\beta}(y) =\frac{\partial x^i}{\partial y^\alpha}
        \frac{\partial x^j}{\partial y^\beta} g_{ij}(x), \qquad
       h^{\alpha\beta}(y)=\frac{\partial y^\alpha}{\partial x^i}
        \frac{\partial y^\beta}{\partial x^j} g^{ij}(x) .
    \end{equation}
    式\eqref{chrg:eqn_isometry-connection-tmp}右端第二项为(注意将上式带入)
    \setlength{\mathindent}{0em}
    \begin{align*}
      &\frac{\partial y^\alpha}{\partial x^i}
      \frac{\partial y^\beta}{\partial x^j}
      \Gamma^{\rho}_{\beta\alpha}(y)
      = \frac{1}{2}\frac{{\partial {y^\alpha }}}{{\partial {x^i}}}\frac{{\partial {y^\beta }}}{{\partial {x^j}}}
      {{h}^{\pi \rho }}\left( \frac{\partial h_{\alpha \pi   }} {\partial y^\beta}
           +\frac{\partial h_{\pi \beta   }} {\partial y^\alpha }
           -\frac{\partial h_{\alpha \beta}} {\partial y^\pi } \right)\\
      =& \frac{1}{2}\frac{{\partial {y^\alpha }}}{{\partial {x^i}}}\frac{{\partial {y^\beta }}}{{\partial {x^j}}}
      \frac{{\partial {y^\rho }}}{{\partial {x^k}}}\frac{{\partial {y^\pi }}}{{\partial {x^l}}}
      {g^{kl}} \biggl[ \frac{\partial }{{\partial {y^\beta }}}
      \left( {\frac{{\partial {x^o}}}{{\partial {y^\alpha }}}\frac{{\partial {x^n}}}{{\partial {y^\pi }}}{g_{on}}} \right)
       +\frac{\partial }{{\partial {y^\alpha }}}\left( {\frac{{\partial {x^o}}}{{\partial {y^\pi }}}
           \frac{{\partial {x^n}}}{{\partial {y^\beta }}}{g_{on}}} \right)
       - \frac{\partial }{{\partial {y^\pi }}}\left( {\frac{{\partial {x^o}}}{{\partial {y^\alpha }}}
           \frac{{\partial {x^n}}}{{\partial {y^\beta }}}{g_{on}}} \right)  \biggr] \\
      =& \cdots =
      \frac{{\partial {y^\alpha }}}{{\partial {x^i}}}\frac{{\partial {y^\beta }}}{{\partial {x^j}}}
      \frac{{\partial {y^\rho }}}{{\partial {x^o}}}\frac{{{\partial ^2}{x^o}}}{{\partial {y^\beta }\partial {y^\alpha }}}
       + \frac{{\partial {y^\rho }}}{{\partial {x^k}}}\Gamma _{ij}^k\left( x \right) .
    \end{align*}\setlength{\mathindent}{2em}
    上式计算冗长,中间省略了一些步骤,请读者补齐.
    将上式带入式\eqref{chrg:eqn_isometry-connection-tmp},得
    \begin{equation}
        \Gamma^{k}_{ji}(x) \frac{\partial y^\rho}{\partial x^k} =
        \frac{\partial^2 y^\rho}{\partial x^j \partial x^i}+
        \frac{{\partial {y^\alpha }}}{{\partial {x^i}}}\frac{{\partial {y^\beta }}}{{\partial {x^j}}}
        \frac{{\partial {y^\rho }}}{{\partial {x^l}}}\frac{{{\partial ^2}{x^l}}}{{\partial {y^\beta }\partial {y^\alpha }}}
         + \frac{{\partial {y^\rho }}}{{\partial {x^k}}}\Gamma _{ij}^k\left( x \right) .
    \end{equation}
    利用式\eqref{chccr:eqn_tmp21}发现上式是个恒等式,
    这样便证明了式\eqref{chrg:eqn_isometry-connection-tmp},
    那么自然可以得到式\eqref{chrg:eqn_isometry-connection-vec2},
    最终证明了式\eqref{chrg:eqn_isometry-connection-vec}.
\end{proof}

\begin{remark}
    其实还可以更简单地证明.因$\phi$局部等距,则可认为两个坐标卡是恒等的,
    即$y^i\equiv x^i$;进而度规完全相同,见式\eqref{chrg:eqn_isometry-MNcoord};
    故式\eqref{chrg:eqn_isometry-connection-tmp}为恒等式.
\end{remark}

\begin{remark}
    把定理\ref{chrg:thm_isometry-connection-vector}中的\uwave{局部等距}换成等距浸入,
    定理未必成立(见例\ref{chsm:exm_Dni}).
\end{remark}

\begin{example}
	定理\ref{chrg:sec_isometry-connection}的另一种证明途径.
\end{example}
所有符号与定理\ref{chrg:sec_isometry-connection}相同.
在$N$中已有Levi-Civita联络${\rm D}_a$,利用它,
可在$M$中定义一个算符$\bar{\nabla}_a$如下($\bar{\nabla}_X \equiv X^a\bar{\nabla}_a $):
\begin{equation}\label{chrg:eqn_bND}
	\bar{\nabla}_X Y^a \overset{def}{=} \phi^{-1}_* \left( {\rm D}_{\phi_* X} (\phi_*Y^a) \right).
\end{equation}
如果$\phi$仅是浸入映射,则没有逆,无法进行上述定义.
我们先验证它是$M$上的一个仿射联络,先验证$\mathbb{R}$-线性
(设$Z^a \in \mathfrak{X}(M)$,$\lambda \in \mathbb{R}$):
\begin{align*}
	\bar{\nabla}_X (Y^a + \lambda Z^a) =& \phi^{-1}_* \left( {\rm D}_{\phi_* X} (\phi_*Y^a +\lambda \phi_*Z^a)\right)
	= \phi^{-1}_* \left( {\rm D}_{\phi_* X}\phi_*Y^a +\lambda  {\rm D}_{\phi_* X} \phi_*Z^a\right)\\
	=& \phi^{-1}_* \left( {\rm D}_{\phi_* X}\phi_*Y^a\right) +\lambda  \phi^{-1}_* \left({\rm D}_{\phi_* X} \phi_*Z^a\right)
	=  \bar{\nabla}_X Y^a  + \lambda  \bar{\nabla}_X Z^a .
\end{align*}
再验证加法($f\in C^\infty(M)$):
\begin{align*}
	\bar{\nabla}_{f\cdot Y + Z} X^a =& \phi^{-1}_* \left( {\rm D}_{\phi_* (f\cdot Y+Z)} \phi_*X^a\right)
	=\phi^{-1}_* \left( {\rm D}_{\phi_* (f Y)} \phi_*X^a+ {\rm D}_{\phi_*Z} \phi_*X^a\right)\\
	=&\phi^{-1}_* \left( {\rm D}_{\phi_* (f Y)} \phi_*X^a\right)+\phi^{-1}_* \left( {\rm D}_{\phi_*Z} \phi_*X^a\right)
	=   \bar{\nabla}_{f\cdot Y } X^a  +\bar{\nabla}_{ Z} X^a .
\end{align*}
接着验证Leibniz法则:
\begin{align*}
	\bar{\nabla}_X (f\cdot Y^a ) =& \phi^{-1}_* \left( {\rm D}_{\phi_* X} \phi_* (fY^a) \right)
	= \phi^{-1}_* \left( {\rm D}_{\phi_* X} ((f\circ \phi^{-1})(\phi_* Y^a) )\right) \\
	=& \phi^{-1}_* \left(({\phi_* X} (f\circ \phi^{-1})) (\phi_* Y^a)\right) + 
	\phi^{-1}_* \left((f\circ \phi^{-1}) {\rm D}_{\phi_* X} (\phi_* Y^a )\right)  \\
	=& X(f) Y^a  +   f\cdot \phi^{-1}_* \left({\rm D}_{\phi_* X} (\phi_* Y^a )\right)  
	= X(f) Y^a  +   f\cdot \bar{\nabla}_X Y^a  . 
\end{align*}
这样便证明了式\eqref{chrg:eqn_bND}中定义的算符$\bar{\nabla}_a$是$M$上的一个仿射联络.

下面验证仿射联络\eqref{chrg:eqn_bND}是无挠的,自然要利用${\rm D}_a$的无挠性.
\begin{small}
\begin{align*}
	T^a_{bc} X^b Y^c = & \bar{\nabla}_X Y^a - \bar{\nabla}_Y X^a -[X,Y]^a 
	=  \phi^{-1}_* \left( {\rm D}_{\phi_* X} \phi_*Y^a \right) 
	-\phi^{-1}_* \left( {\rm D}_{\phi_* Y} \phi_*X^a \right) -[X,Y]^a \\
	=&\phi^{-1}_* \left( {\rm D}_{\phi_* X} \phi_*Y^a 
	-  {\rm D}_{\phi_* Y} \phi_*X^a -[\phi_*X,\phi_*Y]^a \right) 
	=\phi^{-1}_*(0) = 0.
\end{align*} %\setlength{\mathindent}{2em}
\end{small}
最后验证仿射联络\eqref{chrg:eqn_bND}与度规相容条件\eqref{chgd:eqn_connection-compatibility},
这步需用局部等距条件.
\begin{align*}
	X(g_{ab} Y^a Z^b) = & \phi^{-1}_*\circ \phi_* \left(X(g_{ab} Y^a Z^b) \right)
	=\phi^{-1}_*  \left(\phi_* X( (\phi^{-1*}g_{ab}) \phi_*Y^a \phi_*Z^b) \right) \\
	=& \phi^{-1}_*  \left(\phi_* X(h_{ab} \phi_*Y^a \phi_*Z^b) \right)\\
	=& \phi^{-1}_*  \left(h_{ab} ({\rm D}_{\phi_* X} \phi_*Y^a) \phi_*Z^b
	 + h_{ab} \phi_*Y^a {\rm D}_{\phi_* X}\phi_*Z^b\right) \\
	=&g_{ab} (\bar{\nabla}_{X} Y^a) Z^b + g_{ab} Y^a \bar{\nabla}_{X} Z^b .
\end{align*}
上式说明了仿射联络\eqref{chrg:eqn_bND}与度规$g_{ab}$相容.
根据定理\ref{chrg:thm_Levi-Civita-Connetion}可知由\eqref{chrg:eqn_bND}式定义
的仿射联络$\bar{\nabla}_a$就是$M$上的Levi-Civita联络${\nabla}_a$.证毕.\qed


\begin{proposition}\label{chrg:thm_isometry-connection-scalar}
    给定两个广义黎曼流形及Levi-Civita联络$(M,\nabla_a,g)$和$(N,{\rm D}_a,h)$,
    它们之间存在局部等距映射$\phi:M\to N$;那么,$\forall p\in M$,
    和$\forall X^a \in T_{p}M$,$ \forall f \in C^\infty(M)$,都有
    \begin{equation}\label{chrg:eqn_isometry-connection-scalar}
      \Phi_{*} (\nabla _X f) = {\rm D}_ {\phi_* X} ({\phi^{-1*}}f) .
    \end{equation}
    也就是说,作用在标量场上的Levi-Civita联络在局部等距映射下不变.
    $\Phi_{*}$定义与定理\ref{chdm:def_push-pull-tensor}中的$\Phi^{*}$定义类似,
    矢量$X^a \in T_{p}M$需要用$\phi_{*}$推前,而$f \in C^\infty(M)$需要
    用$\phi^{-1*}$拉回到$ C^\infty(N)$.
\end{proposition}
\begin{proof}
    式\eqref{chrg:eqn_isometry-connection-scalar}左端为
    \begin{equation*}
      \Phi_{*} (\nabla _X f) = \phi^{-1*} \bigl(X (f)\bigr)
      =( X^i \circ \phi^{-1}) \cdot \frac{\partial f\circ \phi^{-1}\bigl(y(x)\bigr)}{\partial x^i}
      = (X^i \circ \phi^{-1})  \frac{\partial y^\alpha}{\partial x^i}
      \frac{\partial f\circ \phi^{-1}}{\partial y^\alpha}.
    \end{equation*}
    式\eqref{chrg:eqn_isometry-connection-scalar}右端为
    \begin{equation*}
      {\rm D}_ {\phi_* X} ({\phi^{-1*}}f) = {\phi_* X} \bigl(\phi^{-1*} (f)\bigr)
      = (X^i \circ \phi^{-1})  \frac{\partial y^\alpha}{\partial x^i}
       \frac{\partial f\circ \phi^{-1}}{\partial y^\alpha}.
    \end{equation*}
    综合以上两式,可见命题正确.
\end{proof}

\begin{proposition}\label{chrg:thm_isometry-connection-cov}
    给定两个广义黎曼流形及Levi-Civita联络$(M,\nabla_a,g)$和$(N,{\rm D}_a,h)$,
    它们之间存在局部等距映射$\phi:M\to N$;那么,$\forall p\in M$,
    和$\forall X^a \in T_{p}M$,$ \forall \omega_a \in \mathfrak{X}^{*}(M)$,都有
    \begin{equation}\label{chrg:eqn_isometry-connection-cov}
    \Phi_{*} (\nabla _X \omega_a) = {\rm D}_ {\phi_* X} ({\phi^{-1*}}\omega_a) .
    \end{equation}
    也就是说,作用在协变矢量场上的Levi-Civita联络在局部等距映射下不变.
\end{proposition}
\begin{proof}
    利用命题\ref{chrg:thm_isometry-connection-scalar}和
    定理\ref{chrg:thm_isometry-connection-vector}容易证明本命题.
    $\forall Y^a \in \mathfrak{X}(M)$有
    \begin{align*}
        \Phi_{*} \bigl(\nabla _X (\omega_a Y^a) \bigr) =&
        \Phi_{*} \bigl((\nabla _X \omega_a )Y^a + \omega_a \nabla _X  Y^a\bigr) \\
        =&\bigl(\Phi_{*} (\nabla _X \omega_a ) \bigr)\phi_{*}Y^a
        + \phi^{-1*}(\omega_a) {\rm D} _{\phi_* X}  (\phi_* Y^a) .
    \end{align*}
    而$\omega_a Y^a$本身还是标量场,还可用式\eqref{chrg:eqn_isometry-connection-scalar}操作
    \begin{align*}
       \Phi_{*} \bigl(\nabla _X (\omega_a Y^a) \bigr) =&  {\rm D}_ {\phi_* X} \bigl({\phi^{-1*}} (\omega_a Y^a)\bigr)
       ={\rm D}_ {\phi_* X} \bigl((\phi^{-1*}\omega_a ) (\phi_{*}Y^a)\bigr) \\
       =&\bigl( {\rm D}_ {\phi_* X} (\phi^{-1*}\omega_a ) \bigr) (\phi_{*}Y^a)
       +(\phi^{-1*}\omega_a ) \bigl({\rm D}_ {\phi_* X}  (\phi_{*}Y^a)\bigr) .
    \end{align*}
    综合以上两式,可证式\eqref{chrg:eqn_isometry-connection-cov}.
\end{proof}
综上可见,局部等距映射保Levi-Civita联络(可作用在任何$\Tpq{r}{s}$型张量场上)不变;
尤其作用在黎曼曲率、Ricci曲率、标量曲率、Weyl曲率和爱因斯坦张量上是保形式不变的.
如果把{\kaishu 局部等距同构}变成整体{\kaishu 等距同胚},上面所有定理仍旧正确.

鉴于黎曼曲率的重要性,我们还是将其叙述成定理形式.
\begin{theorem}\label{chrg:thm_isometry-Riemann}
    设有局部等距映射$\phi:M\to N$.$\forall X^a,Y^a,Z^a,W^a\in \mathfrak{X}(M)$,有
    \begin{equation}\label{chrg:eqn_isometry-Riemann}
        \Phi_{*}\bigl(R_{dcab}^M X^a Y^b Z^c W^d\bigr) = \left(R^N_{dcab}
        (\phi_{*} X^a) (\phi_{*} Y^b) (\phi_{*} Z^c) (\phi_{*} W^d)\right)\circ \phi .
    \end{equation}
    其中$R_{dcab}^M$、$R_{dcab}^N$分别是$M$、$N$上的黎曼曲率.
    式\eqref{chrg:eqn_isometry-Riemann}简记为$R_{dcab}^M=\phi^* R_{dcab}^N$.
\end{theorem}
%由于$\phi$是局部等距同构,所以两个曲率有如下关系:$R_{dcab}^M=\phi^* R_{dcab}^N$.
%定理\eqref{chrg:thm_isometry-Riemann}的叙述形式取自\parencite{wuhx2014cb}第二章习题7,33页.

一般情形下,定理\ref{chrg:thm_isometry-Riemann}的逆命题是不成立的,见例题\ref{chhss:exam_NOiso}.

\begin{theorem}\label{chrg:thm_geodesic-MN}
    设$(M,g,\nabla_a)$和$(N,h,{\rm D}_a)$间存在局部等距$\phi:M\to N$.则

    {\bfseries (1)} $M$上曲线$\gamma(t)$的切矢的等距切映射是$\gamma(t)$的等距像$\tilde{\gamma}\equiv\phi\circ \gamma$的切矢.

    {\bfseries (2)} $\gamma(t)$是$M$上测地线的充要条件是$\tilde{\gamma}$是$N$上测地线.
\end{theorem}
\begin{proof}
    流形$M$上有光滑曲线$\gamma(t)$,它在流形$N$中的像曲线是$\tilde{\gamma}(t)=\phi\circ\gamma(t)$.
    通过式\eqref{chdm:eqn_phiD=Dphi}可知
    两条曲线的切矢量关系是(因$\phi$是局部等距,下式不只在$t=0$点成立,
    而在整条曲线上成立,即参数$t$可取定义域内所有值)
    \begin{equation}\label{chrg:eqn_dgM-gdN}
        \phi_{*}\left(\left.\frac{{\rm d}  }{{\rm d} t}\right|_{\gamma(t)}\right)^a
        = \left(\left.\frac{{\rm d}  }{{\rm d} t}\right|_{\phi\circ\gamma(t)}\right)^a
        = \left(\left.\frac{{\rm d}  }{{\rm d} t}\right|_{\tilde{\gamma}(t)}\right)^a.
    \end{equation}
    这便证明了(1).那么由局部等距保联络不变性可知
    \begin{equation}\label{chrg:eqn_geodesic-MN}
       \phi_{*}\left( \nabla_{\left.\frac{{\rm d}  }{{\rm d} t}\right|_{\gamma}}
         \left(\left.\frac{{\rm d}  }{{\rm d} t}\right|_{\gamma}\right)^a \right)
       ={\rm D}_{\phi_{*}\left.\frac{{\rm d}  }{{\rm d} t}\right|_{\gamma}}
         \phi_{*}\left(\left.\frac{{\rm d}  }{{\rm d} t}\right|_{\gamma}\right)^a
       ={\rm D}_{\left.\frac{{\rm d}  }{{\rm d} t}\right|_{\tilde{\gamma}}}
       \left(\left.\frac{{\rm d}  }{{\rm d} t}\right|_{\tilde{\gamma}}\right)^a .
    \end{equation} %\setlength{\mathindent}{2em}
    流形$M$上测地线是$\gamma(t)$(定义见\ref{chccr:def_geodesic})自然是一条特殊的光滑曲线;
    故式\eqref{chrg:eqn_geodesic-MN}适用于测地线,
    由此可知(2)成立:$\gamma(t)$是测地线等价于$\tilde{\gamma}(t)$是测地线.
\end{proof}


\begin{exercise}
	补全定理\ref{chrg:thm_isometry-connection-vector}证明过程中漏掉的步骤.
\end{exercise}

\begin{exercise}
	证明定理\ref{chrg:thm_isometry-Riemann},详尽写出每一步.
\end{exercise}


\index[physwords]{Killing切矢量场}

\section{Killing切矢量场}\label{chrg:sec_killing}
德国数学家Wilhelm Killing (1847 - 1923) 给出了广义黎曼流形中
的局部等距同构变换无穷小生成元,即现今被称为{Killing切矢量场}的知识.
%如果读者不想了解更多对称性知识,本节内容已足够.

\begin{definition}\label{chrg:def_killing}
    广义黎曼流形$(M,g)$上处处非零的光滑切矢量场$\xi^a$被称
    为{\bfseries \heiti Killing切矢量场}是指:
    如果$\xi^a$诱导出的局部单参数变换群是局部等距同构变换.
\end{definition}
设切矢量场是$\xi^a$诱导出的(局部)单参数微分同胚群
是$\phi_{t}:M\to M$(见\S\ref{chdm:sec_One-Parameter-Transformations-Groups}),
由定义\ref{chrg:def_isometry-local-isomorphism}可知,局部等距同构是
指$g_{ab}=\phi_t^{*}g_{ab}$对任意$t$成立.

\begin{theorem}\label{chrg:thm_killing}
    $\xi^a$为Killing矢量场的充要条件是满足如下条件之一即可.
    \begin{align}
        { (1)}:&\quad \Lie_{\xi} g_{ab} =0. \label{chrg:eqn_killing-1} \\
        { (2)}:&\quad \nabla_a \xi_b + \nabla_b \xi_a =0   {\quad \Leftrightarrow \quad}
        \nabla_{(a}\xi_{b)} =0   {\quad \Leftrightarrow \quad}
        \nabla_{a}\xi_{b} =\nabla_{[a}\xi_{b]}. \label{chrg:eqn_killing-2} %\\
%        (3):&\quad \left<Y, \nabla_X \xi \right> = \left< X , \nabla_Y \xi \right>,
%           \qquad \forall X, Y\in \mathfrak{X}(M) . \label{chrg:eqn_killing-2-bak}
    \end{align}
\end{theorem}
\begin{proof}
    参考李导数定义\eqref{chdm:def_LieD-tensor}可知局部等距同构等价于$\Lie_{\xi} g_{ab} =0$,
    这说明条件(1)正确.由下式可直接得到条件(2)也正确,
    式\eqref{chrg:eqn_killing-2}称为{\bfseries \heiti Killing方程}.
    \begin{equation*}
        0=\Lie_{\xi} g_{ab}\xlongequal{\ref{chccr:eqn_LieD-tensor-Nabla}} \xi^c \nabla_c g_{ab}+
         g_{cb} \nabla_a \xi^c + g_{ac} \nabla_b \xi^c = \nabla_a \xi_b + \nabla_b \xi_a .
    \end{equation*}
    证明过程中注意广义黎曼流形$(M,g)$的Levi-Civita联络是无挠、容许联络.
%        条件(3)只是条件(2)的\S\ref{chmla:sec_tensor}记号表示.
\end{proof}

将Killing方程\eqref{chrg:eqn_killing-2}的指标收缩,可得一个常用公式:
\begin{equation}\label{chrg:eqn_nk=0}
    {\nabla_a} \xi^a = 0 .
\end{equation}

%    如果$\xi^a$是局部开集$U$的Killing矢量场,那么存在局部坐标系$(U;x^i)$使得
%度规场$g_{ab}$的全部分量满足$\frac{\partial g_{ij}}{\partial x^1}=0$.
\subsection{基本性质}
\begin{theorem}\label{chrg:thm_killing-partial-x1}
    非零$\xi^a$是局部开集$U$的Killing矢量场的充分必要条件是:存在局部坐标系$(U;x^i)$使得
    度规场$g_{ab}$的全部分量满足$\frac{\partial g_{ij}}{\partial x^1}=0$,
    并且$\xi^a |_U = (\frac{\partial }{\partial x^1} )^a$.
\end{theorem}
\begin{proof}
    先证“$\Rightarrow$”.
    由定理\ref{chdm:thm_1PDG-ppy}可知,对于任意非零Killing矢量场$\xi^a$都存在
    局部{\kaishu 适配坐标系}$(U;x^i)$使得$\xi^a |_U = (\frac{\partial }{\partial x^1} )^a$.
    因$\xi^a$是Killing矢量场,故 %(参考式\eqref{chdm:eqn_LieD-adaptedX})
    \begin{equation}
        \Lie_{\xi} g_{ij}  \xlongequal{\ref{chdm:eqn_LieD-adaptedX}}
        \bigl(\Lie_{\xi} g_{ab}\bigr)
        \left(\frac{\partial }{\partial x^i} \right)^a \left(\frac{\partial }{\partial x^j} \right)^b=0
         \ \Rightarrow \ 
        0=\Lie_{\xi} g_{ij}= \frac{\partial g_{ij}}{\partial x^1}.
    \end{equation}
    再证“$\Leftarrow$”.
    %根据假设有:存在局部坐标系$(U;x^i)$使得
%    $g_{ab}$的全部分量满足$\frac{\partial g_{ij}}{\partial x^1}=0$.
    $\left(\frac{\partial }{\partial x^1}\right)^a$可以看作是$U$上的一个切矢量场,
    并且$\{x^i\}$就是局部{\kaishu 适配坐标系}.
    根据\S\ref{chdm:sec_One-Parameter-Transformations-Groups}中理论可知,它一定会诱导一个
    局部单参数可微变换群$\phi_t:U\to U$.$\forall f\in C^\infty(U)$,$\phi_t$所诱导出的李导数就
    是$\Lie_{\frac{\partial }{\partial x^1}} f= \frac{\partial f}{\partial x^1}$.
    把它应用到度规场$g_{ab}$上,有
    \begin{equation}
        0= \frac{\partial g_{ij}}{\partial x^1} = \Lie_{\xi} g_{ij}
        \xlongequal{\ref{chdm:eqn_LieD-adaptedX}}
        \bigl(\Lie_{\frac{\partial }{\partial x^1}} g_{ab}\bigr)
        \left(\frac{\partial }{\partial x^i} \right)^a
        \left(\frac{\partial }{\partial x^j} \right)^b .
    \end{equation}
    因局部自然标架场$\{\left(\frac{\partial }{\partial x^i} \right)^a\}$恒不为零,
    故必有$\Lie_{\frac{\partial }{\partial x^1}} g_{ab}=0$;
    所以$\left(\frac{\partial }{\partial x^1}\right)^a$是局部开集$U$上的Killing矢量场.
\end{proof}


%\begin{theorem}\label{chrg:thm_killing-loc-2}
%    如果存在局部坐标系$(U;x^i)$使得度规$g_{ab}$的全部分量满足$\frac{\partial g_{ij}}{\partial x^1}=0$,
%    那么$(\frac{\partial }{\partial x^1} )^a$是坐标域$U$上Killing矢量场.
%\end{theorem}

\begin{example}\label{chrg:exam_killing-local}
	流形$M$上局部等距\ref{chrg:def_isometry-local-isomorphism}的局部坐标系表述.
\end{example}
    与例\ref{chrg:exam_isometry-MNcoord}类似,设有广义黎曼流形$(M,g)$;
    为简单起见,我们只研究单参数微分变换群$\phi_t:M\to M$.
    再设$(U;x)$和$(V;y)$分别是$\forall p\in M$和$q=\phi_t(p)\in M$的局部坐标,即$y=\phi_t(x)$.    
    为了更有区分度,我们混用拉丁、希腊字母.映射$\phi_t$将$q$点的度规拉回$p$点,
    由于两点距离无穷小,可记$y^a=x^a+\epsilon^a(x)$;则
    \begin{align*}
    	\tilde{g}_{ab}(x)\overset{def}{=}&\phi_t^* g_{ab}(q)
    	= \phi_t^* \left[g_{\alpha\beta}(y)({\rm d}y^\alpha)_a ({\rm d}y^\beta)_b\right]
    	\xlongequal{\ref{chdm:eqn_pull-bases}}
    	\frac{\partial y^\alpha}{\partial x^i}\frac{\partial y^\beta}{\partial x^j} 
    	g_{\alpha\beta}(y) ({\rm d}x^i)_a ({\rm d}x^j)_b\\
    	=&\left(\delta^\alpha_i + \partial_i \epsilon^\alpha\right)
    	\left(\delta^\beta_j + \partial_j \epsilon^\beta\right)
    	\left(g_{\alpha\beta}(x)+\epsilon^k\partial_k g_{\alpha\beta}(x) +\cdots \right)
    	({\rm d}x^i)_a ({\rm d}x^j)_b \\
    	\xlongequal[\text{阶小量}]{\text{忽略高}}&
    	\left[g_{ij}(x) + g_{\alpha j}(x) \partial_i \epsilon^\alpha 
    	+ g_{i \beta}(x) \partial_j \epsilon^\beta +\epsilon^c\partial_c g_{ij}(x)
    	\right] ({\rm d}x^i)_a ({\rm d}x^j)_b \\
    	\xlongequal{\ref{chccr:eqn_LieD-tensor-Partial}}& 
    	g_{ab}(x) + \Lie_{\epsilon} g_{ab}(x) .
    \end{align*} %\setlength{\mathindent}{2em}
    现在点$p$(其局部用坐标$\{x\}$表示)处有两个度量,一个是$\tilde{g}_{ab}(x)$,
    另一个是$g_{ab}(x)$;两个度量未必相等.
    上式说明:当$\phi_t$不是局部等距同构变换时(即矢量场$\epsilon^a$不是Killing场,
    $\Lie_{\epsilon} g_{ab} \neq 0$),那么我们无法得到$g_{ab}(x)= \tilde{g}_{ab}(x)$.
    若$\phi_t$是局部等距同构变换,则可得其局部坐标系表达式为(去掉自然坐标基矢量):
    \begin{equation}\label{chdm:eqn_isometry-MMcoord}
    	g_{ij}(x)=\frac{\partial y^\alpha}{\partial x^i}
    	\frac{\partial y^\beta}{\partial x^j} g_{\alpha\beta}(y) .
    \end{equation}
    注意如下区别:由单点张量坐标变换式(见式\eqref{chdm:eqn_tensor-component-trans})
    可以得到$\tilde{g}_{ij}=\frac{\partial y^\alpha}{\partial x^i}
    \frac{\partial y^\beta}{\partial x^j} g_{\alpha\beta}$,等号左端是$\tilde{g}$;
    这个式子等号两端必须是同一点的张量.在变换式\eqref{chdm:eqn_isometry-MMcoord}中,
    我们用坐标$\{x\}$、$\{y\}$来代表两个不同点($p$、$q$两点距离非常近).
    两式的差异是一个是单点张量变换,另一个是两点张量变换;但公式形式完全相同.
    而$\tilde{g}$上是否带有波浪号则是无关紧要的. \qed


\begin{example}\label{chrg:exam_killing-poisson}
    设$\xi^a$和$\eta^a$是广义黎曼流形$(M,g)$上的两个Killing矢量场.
    \begin{equation}
       \Lie_{[\xi,\eta]} g_{ab} \xlongequal{\ref{chccr:eqn_Lie-XYYX=Liexy}}
         \Lie_{\xi} \circ \Lie_{\eta}g_{ab} -\Lie_{\eta}\circ \Lie_{\xi} g_{ab}
       = \Lie_{\xi} (0) -\Lie_{\eta}(0) = 0.
    \end{equation}
    故$[\xi,\eta]^a$也是Killing矢量场.
    本例也可直接把$[\xi,\eta]^a$带入Killing方程去证明,
    但计算过程略繁琐.    \qed
\end{example}

\begin{theorem}\label{chrg:thm_killing-MN}
    设两个广义黎曼流形$(M,g,\nabla_a)$和$(N,h,{\rm D}_a)$间存在局部等距同构
    $\phi:M\to N$,$\xi^a\in \mathfrak{X}(M)$.
    则$\phi_{*}\xi^a$是$N$上Killing矢量场的充要条件是$\xi^a$是$M$上Killing矢量场.
\end{theorem}
\begin{proof}
    由定理\ref{chrg:thm_killing}出发,得
    \begin{equation*}
        \nabla_a \xi_b + \nabla_b \xi_a =0 {\ \color{red}\Leftrightarrow \ }
        \phi^{-1*}\bigl( \nabla_a \xi_b + \nabla_b \xi_a \bigr) =0
        \ \xLeftrightarrow[\ref{chrg:thm_isometry-connection-cov}]{\text{命题}}  \
        {\rm D}_a (\phi^{-1*}\xi_b) + {\rm D}_b (\phi^{-1*}\xi_a)  =0 .
    \end{equation*}
    上式便证明了定理.
\end{proof}



\begin{theorem}\label{chrg:thm_killing-geodisic}
    设有广义黎曼流形$(M,g)$,$\xi^a$是其上Killing矢量场,$T^a$是其
    测地线$\gamma(t)$的切矢量.那么有$\nabla_T(T^a \xi_a)=0$,
    即$T^a \xi_a$沿测地线$\gamma(t)$是常数.
\end{theorem}
\begin{proof}
    测地线定义见\ref{chccr:def_geodesic},即$ \nabla _T T^a=0$.
    \begin{equation*}
        \nabla_T(T^a \xi_a)= T^a \nabla_T \xi_a +\xi_a\nabla_T T^a = T^a T^b \nabla_b \xi_a
        \xlongequal{\ref{chrg:thm_killing}(2)}T^{(a} T^{b)} \nabla_{[b} \xi_{a]}
        \xlongequal{\ref{chmla:eqn_gkd-35}}0.
    \end{equation*}
    在广义相对论中,经常用到这个定理.
\end{proof}


\subsection{Killing矢量场与曲率关系}
Killing矢量场$\xi_a$满足如下对易子式子(见式\eqref{chccr:eqn_Riemannian13-CoVec-commutator})
\begin{equation}\label{chrg:eqn_tmp-kr}
    {\nabla_a}{\nabla_b}{\xi_c} - {\nabla_b}{\nabla_a}{\xi_c} = -R_{cab}^e{\xi_e}.
\end{equation}
由上式和Bianchi第一恒等式\eqref{chccr:eqn_Bianchi-I-global}可得
\begin{align}
    &0=-R_{[cab]}^{e}{\xi_e}= {\nabla_{[a}}{\nabla_b}{\xi_{c]}} - {\nabla_{[b}}{\nabla_a}{\xi_{c]}}
    = {\nabla_{[c}}{\nabla_a}{\xi_{b]}} - {\nabla_{[c}}{\nabla_b}{\xi_{a]}}
    \xlongequal{\ref{chrg:eqn_killing-2}} 2 {\nabla_{[c}}{\nabla_a}{\xi_{b]}} \notag \\
    &\xRightarrow{\ref{chrg:eqn_killing-2}} \quad
    0= {\nabla_{c}}{\nabla_a}{\xi_{b}} + {\nabla_{a}}{\nabla_b}{\xi_{c}} -
    {\nabla_{b}}{\nabla_a}{\xi_{c}} . \label{chrg:eqn_killing-cycle}
\end{align}
将式\eqref{chrg:eqn_killing-cycle}带回\eqref{chrg:eqn_tmp-kr}可得
\begin{equation}\label{chrg:eqn_killing-Riemann-1}
    {\nabla_a}{\nabla_b}{\xi_c} -   {\nabla_{c}}{\nabla_a}{\xi_{b}}
    - {\nabla_{a}}{\nabla_b}{\xi_{c}} = -R_{cab}^e{\xi_e} \quad \Rightarrow \quad
    {\nabla_c}{\nabla_a}{\xi_b} = R_{cab}^e{\xi_e}.
\end{equation}
这是Killing矢量场与黎曼曲率的关系式之一.
将上式中指标收缩可得第二式:
\begin{equation}\label{chrg:eqn_killing-Riemann-2}
    {\nabla_c}{\nabla^c}{\xi_b} = -R_{eb}{\xi^e}.
\end{equation}
用Killing场$\xi^b$缩并约化Bianchi恒等式\eqref{chrg:eqn_Bianchi-contract}两端,
经计算可得
\begin{equation}\label{chrg:eqn_killing-Riemann-3}
    \xi^b\nabla_b R= 0 .
\end{equation}
这是第三个Killing场与曲率间的关系式.
%\begin{align*}
%    \xi^b\nabla_b R =& 2\xi^b \nabla^a R_{ab}= 2 \nabla^a (R_{ab}\xi^b)
%      - 2R_{ab} \nabla^a \xi^b \xlongequal{\ref{chrg:eqn_killing-2}} 2 \nabla^a (R_{ab}\xi^b) \\
%    \xlongequal{\ref{chrg:eqn_killing-Riemann-2}} &
%     -2 \nabla_a ({\nabla_c}{\nabla^c}{\xi^a})
%    \xlongequal{\ref{chccr:eqn_Riemannian13-Tensor-commutator}}
%    -2 {\nabla_c} \nabla_a ({\nabla^c}{\xi^a})
%    -2R_{\cdot eac}^c {\nabla^e}{\xi^a}-2R_{\cdot eac}^a {\nabla^c}{\xi^e} \\
%    =& -2 {\nabla_c} \nabla_a ({\nabla^c}{\xi^a})
%    +2R_{ea} {\nabla^e}{\xi^a}-2R_{ec} {\nabla^c}{\xi^e}
%    \xlongequal{\ref{chrg:eqn_killing-2}} -2 {\nabla^c} (\nabla_a {\nabla_c}{\xi^a}) \\
%    \xlongequal{\ref{chccr:eqn_Riemannian13-Vec-commutator}}  &
%    -2 {\nabla^c} \left( \nabla_c {\nabla_a}{\xi^a} + R_{\cdot eac}^a \xi^e \right)
%    \xlongequal{\ref{chrg:eqn_nk=0}}
%    -2 {\nabla^c} \left(  R_{ec} \xi^e \right) .
%\end{align*}

设$\xi^a$是$M$上的Killing矢量场,则曲率的李导数为:
\begin{equation}\label{chrg:eqn_LieR=0}
    \Lie_{\xi} R_{abcd} = 0;\qquad   \Lie_{\xi} R_{ac} = 0;\qquad  \Lie_{\xi} R = 0 .
\end{equation}
上式的计算过程如下:
\begin{align*}
    \Lie_{\xi} R_{bcd}^a 
    \xlongequal{\ref{chccr:eqn_LieD-tensor-Nabla}}& 
    \nabla_\xi  R_{bcd}^a 
    - R_{bcd}^e \nabla_e \xi^a + R_{ecd}^a \nabla_b \xi^e
    + R_{bed}^a \nabla_c \xi^e + R_{bce}^a \nabla_d \xi^e \\
    \xlongequal{\ref{chccr:eqn_Bianchi-II-global}}& 
    - \nabla_c (\xi^e R_{bde}^a) - \nabla_d (\xi^e R_{bec}^a ) 
    +R_{bde}^a \nabla_c \xi^e +R_{bec}^a \nabla_d \xi^e \\
    &- R_{bcd}^e \nabla_e \xi^a + R_{ecd}^a \nabla_b \xi^e
    + R_{bed}^a \nabla_c \xi^e + R_{bce}^a \nabla_d \xi^e \\
    \xlongequal{\ref{chrg:eqn_killing-Riemann-1}}& 
    - \nabla_c ({\nabla_d}{\nabla_b}{\xi^a}) 
    + \nabla_d ({\nabla_c}{\nabla_b}{\xi^a}) 
    +{\color{red} R_{bde}^a \nabla_c \xi^e }
    +{\color{blue} R_{bec}^a \nabla_d \xi^e }\\
    &- R_{bcd}^e \nabla_e \xi^a + R_{ecd}^a \nabla_b \xi^e
    +{\color{red} R_{bed}^a \nabla_c \xi^e }
    +{\color{blue} R_{bce}^a \nabla_d \xi^e }\\
    \xlongequal{\ref{chccr:eqn_Riemannian13-Tensor-commutator}}&
    - R_{bdc}^e \nabla_e \xi^a + R_{edc}^a \nabla_b \xi^e 
    - R_{bcd}^e \nabla_e \xi^a + R_{ecd}^a \nabla_b \xi^e
    \xlongequal[\text{反对称性}]{\text{利用曲率}}0 .
\end{align*}
其实,Killing切矢量场可诱导出单参数等距变换群,
等距变换自然保黎曼曲率不变(定理\ref{chrg:thm_isometry-Riemann}),
故黎曼曲率沿Killing场的李导数恒为零.

%它保联络不变(定理\ref{chrg:thm_isometry-connection-vector}),
%在适配坐标系有定理\ref{chrg:thm_killing-partial-x1};
%而黎曼曲率完全由度规决定,再利用偏导数可交换次序可得:
%在适配系中黎曼曲率分量沿Killing场李导数为零,
%又因黎曼曲率是张量,进而它在任意坐标系的Killing场李导数都为零.



\begin{theorem}\label{chrg:thm_killing-MAX}
    $m$维广义黎曼流形$M$上最多有$m(m+1)/2$个线性独立Killing场.
\end{theorem}
\begin{proof}
    定理中线性独立是指$\mathbb{R}$-线性独立\ref{chmla:def_linear-dependence_base},
    也就是组合系数必须是常数;
    不是$C^\infty(M)$-线性独立(见第\pageref{chdm:eqn_functimesv}页
    及\pageref{chdm:thm_Tensor-Characterization-Lemma}页).

    在给出点$p$的$\xi_a$和$\nabla_b\xi_a$后,我们可以
    从式\eqref{chrg:eqn_killing-Riemann-1}决定$\xi_a$在$p$点的二阶导数.
    对式\eqref{chrg:eqn_killing-Riemann-1}继续求协变导数则可陆续得出
    $\xi_a$在$p$点的高阶导数,所有高阶导数都可表示成$\xi_a(p)$和$\nabla_b\xi_a(p)$的
    $C^\infty(M)$线性组合.于是,在点$p$的某个小邻域内,
    任意度规$g_{ab}$的Killing矢量场$\xi_a(x)$都可以表示为
    $x^\mu-x^\mu_p$的Taylor级数(如果它存在).
    \begin{align*}
        \xi_a(x)=&\xi_\rho(x) \left({\rm d}x^\rho\right)_a
        = \left({\rm d}x^\rho\right)_a \bigg( \xi_\rho(p)+
         \xi_{\rho;\mu}(p) (x^\mu -x_p^\mu) \\
        &\quad + \frac{1}{2!}\xi_{\rho;\mu;\nu}(p) (x^\mu -x_p^\mu)(x^\nu -x_p^\nu)
        +\cdots         \bigg)
    \end{align*}
    依照前面分析,所有高于一阶的导数都能表示为$\xi_a(p)$和$\nabla_b\xi_a(p)$的
    $C^\infty(M)$线性组合;故将上式重新组合为
    \begin{equation}\label{chrg:eqn_tmpkillnum}
        \xi_a(x)= \left({\rm d}x^\rho\right)_a
        \left(A_\rho^\lambda(x^\mu;p) \xi_\lambda(p)+
        B_\rho^{\lambda\nu}(x^\mu;p) \xi_{\lambda;\nu}(p) \right).
    \end{equation}
    对于$m$维流形,$\xi_\lambda(p)$有$m$个分量指标,$\xi_{\lambda;\nu}(p)$的
    分量指标有$m(m-1)/2$个独立(因反对称关系\eqref{chrg:eqn_killing-2}).

    我们来看看$p$点邻域内共有多少个$\mathbb{R}$-线性独立的$\xi_a^n(x)$,这些$\xi_a^n(x)$
    都可以表示成式\eqref{chrg:eqn_tmpkillnum}的样子,只需要在右上角增加角标$n$.
    我们假设当有$N$个$\xi_a^n(x)$时,存在非零实常数$\{c_n\}$使得下式成立.
    \setlength{\mathindent}{0em}
    \begin{equation}\label{chrg:eqn_tmp-killN}
        0=\sum_{n=1}^{N} c_n \xi_\rho^n(x) =  %\left({\rm d}x^\rho\right)_a
        A_\rho^\lambda(x^\mu;p) \sum_{n=1}^{N} \bigl(c_n\xi_\lambda^n(p)\bigr)+
        B_\rho^{\lambda\nu}(x^\mu;p) \sum_{n=1}^{N}
         \bigl(c_n\xi_{\lambda;\nu}^n(p)\bigr)  .
    \end{equation}  \setlength{\mathindent}{2em}
    如果$N$是个极大的正整数,那么上式必然成立;我们是来求$N$的最小值.
    由于$A_\rho^\lambda(x^\mu;p)$和$B_\rho^{\lambda\nu}(x^\mu;p)$是$C^\infty(M)$函数场,
    它们不会为零;故要使式\eqref{chrg:eqn_tmp-killN}成立,必然等价于
    \begin{equation}\label{chrg:eqn_tmp-killN2}
        \sum_{n=1}^{N} \bigl(c_n\xi_\lambda^n(p)\bigr) =0
        \quad \text{并且}\quad
        \sum_{n=1}^{N} \bigl(c_n\xi_{\lambda;\nu}^n(p)\bigr) =0 .
    \end{equation}
    我们把上式改成一个线性代数方程组
    \begin{equation}\label{chrg:eqn_kvm}
        \begin{pmatrix}
            \xi_1^1(p)& \xi_1^2(p)& \cdots & \xi_1^N(p) \\
            \vdots & \vdots & \cdots & \vdots \\
            \xi_m^1(p)& \xi_m^2(p)& \cdots & \xi_m^N(p) \\
            \xi_{1;2}^1(p)& \xi_{1;2}^2(p)& \cdots & \xi_{1;2}^N(p) \\
            \vdots & \vdots & \cdots & \vdots \\
            \xi_{m-1;m}^1(p)& \xi_{m-1;m}^2(p)& \cdots & \xi_{m-1;m}^N(p)
        \end{pmatrix}
        \begin{pmatrix}
            c_1 \\ c_2 \\ \vdots \\ c_N
        \end{pmatrix} =0 .
    \end{equation}
    显然只有$N>m(m+1)/2$时(列数大于行数),上式才肯定有非零解;
    故点$p\in M$邻域内$\mathbb{R}$-线性独立的Killing矢量场$\xi_a(x)$的
    最大个数是$m(m+1)/2$.需要注意的是,并不是任何情形下流形的
    独立Killing矢量场都能达到$m(m+1)/2$个;很多时候独立场个数都小于这个数.   
\end{proof}

%    从群论角度的证明可参见定理\ref{chlg:thm_killing-MAX}.

\subsection{平直闵氏空间的Killing矢量场}\label{chrg:sec_killing-Minkowski}
本节试着找到$m+1$维平直(黎曼曲率为零)Minkowski空间$\mathbb{R}^{m+1}_1$的全部Killing矢量场.
根据定理\ref{chrg:thm_killing-MAX}可知$\mathbb{R}^{m+1}_1$最多有$(m+2)(m+1)/2$个
$\mathbb{R}$-线性独立的Killing矢量场.
求解Killing方程是寻找Killing矢量场的通用方式;一般说来在弯曲时空这种方式
很难奏效(方程复杂,很难求出精确解),但在平直时空却很好用.
$\mathbb{R}^{m+1}_1$的度规场是
\begin{equation}
    \eta_{ab} = -({\rm d}x^0)_a ({\rm d}x^0)_b+
    \sum_{i=1}^{m} ({\rm d}x^i)_a({\rm d}x^i)_b .
\end{equation}
对于平直空间,黎曼曲率恒为零,协变导数变为偏导数;
自然坐标基矢场$(\frac{\partial}{\partial x^\mu})^a$和
其对偶基矢场$({\rm d}x^\mu)_a$都是常矢量场,不逐点变化.
由式\eqref{chrg:eqn_killing-Riemann-1}可知
\begin{equation}
    {\partial_c}{\partial_a}{\xi_b} = 0 \quad \Rightarrow \quad
    \frac{\partial^2 \xi_\sigma}{\partial x^\mu \partial x^\nu} = 0.
\end{equation}
这说明$\xi_b = \xi_\mu ({\rm d}x^\mu)_b$的系数$\xi_\mu$是坐标$x_\alpha$的线性函数,即
\begin{equation}\label{chrg:eqn_killing-coef}
    \xi_\mu = \sum_{\alpha}c_{\mu \alpha}x_\alpha + a_\mu,\qquad
    c_{\mu \alpha}, a_\mu \text{是实常数}.
\end{equation}


平直闵氏空间的Killing方程\eqref{chrg:eqn_killing-2}是
\begin{equation}\label{chrg:eqn_killing-Minkowski}
    \frac{\partial \xi_\mu}{\partial x^\nu}+\frac{\partial \xi_\nu}{\partial x^\mu} =0 .
\end{equation}

式\eqref{chrg:eqn_killing-coef}最简单情形为$c_{\mu \alpha}=0,\, a_\mu \neq 0$,
它自然满足式\eqref{chrg:eqn_killing-Minkowski},故下式
\begin{equation}\label{chrg:eqn_Killing-Minkowski-1}
    \xi^a= \left(\frac{\partial}{\partial x^\mu }\right)^a ,
    \qquad    0\leqslant \mu \leqslant m,\quad \text{共$m+1$个}
\end{equation}
是平直闵氏空间的Killing矢量场,我们将其标记为第一组.
上式中取$a_\mu=1$;如果$a_\mu$取任何非零实常数都不会导致
新的$\mathbb{R}$-线性独立Killing矢量场.
下面只讨论$a_\mu=0$情形.


将式\eqref{chrg:eqn_killing-coef}带入Killing方程\eqref{chrg:eqn_killing-Minkowski}得
\begin{equation}\label{chrg:eqn_killing-coef-eta}
    0=\sum_{\alpha}\frac{\partial c_{\mu \alpha}x_\alpha}{\partial x^\nu}
    +\sum_{\alpha} \frac{\partial c_{\nu \alpha}x_\alpha}{\partial x^\mu}
    =\sum_{\alpha} c_{\mu \alpha} \eta_{\alpha \nu}+\sum_{\alpha} c_{\nu \alpha} \eta_{\alpha \mu} .
\end{equation}
当$\mu=\nu$时,式\eqref{chrg:eqn_killing-coef-eta}变成了
\begin{equation}
    0=\sum_{\alpha} c_{\mu \alpha} \eta_{\alpha \mu} =c_{\mu \mu} \eta_{\mu \mu} ,
    \qquad \text{重复指标}\mu\text{不求和}.
\end{equation}
$\eta_{\mu \mu}=\pm 1$,这说明$c_{\mu \mu}=0$,也就是Killing场分量$\xi_\mu$不含$x^\mu$(或$x_\mu$).

下面只考虑$\mu\neq\nu$的情形,式\eqref{chrg:eqn_killing-coef-eta}变成了
\begin{equation}\label{chrg:eqn_killing-coef-eta-cc}
    0=c_{\mu \nu} \eta_{\nu \nu}+c_{\nu \mu} \eta_{\mu \mu} , \qquad \text{重复指标不求和}.
\end{equation}
下面针对平直闵氏空间开始讨论式\eqref{chrg:eqn_killing-coef-eta-cc};
因度规$\eta={\rm diag}(-1,1,\cdots,1)$,其中“$-1$”只有一个,所以可以取$\mu=0,\,\nu=i(i>0)$,有
\begin{equation}
    0=c_{0i} \eta_{ii}+c_{i 0} \eta_{00} =c_{0i} -c_{i 0} .
\end{equation}
由上式可以得到平直闵氏空间的第二组Killing矢量场
\begin{equation}\label{chrg:eqn_Killing-Minkowski-2}
    \xi^a=x^0 \left(\frac{\partial}{\partial x^i}\right)^a
    +x^i \left(\frac{\partial}{\partial x^0}\right)^a, \qquad
    1\leqslant i \leqslant m,\quad \text{共$m$个}.
\end{equation}
因为系数$c_{\mu\nu}$是实常数,不妨取其值为$\pm 1$,
这不影响Killing场的线性独立性.

下面考虑$\eta$中“$+1$”部分;可取$\mu=j,\,\nu=i(i\neq j>0)$,
式\eqref{chrg:eqn_killing-coef-eta-cc}变为
\begin{equation}
    0=c_{ji} \eta_{ii}+c_{ij} \eta_{jj} =c_{ji} + c_{ij},  \qquad i\neq j>0 .
\end{equation}
由上式可以得到平直闵氏空间的第三组Killing矢量场
\begin{equation}\label{chrg:eqn_Killing-Minkowski-3}
    \xi^a= -x^j \left(\frac{\partial}{\partial x^i}\right)^a
    +x^i \left(\frac{\partial}{\partial x^j}\right)^a, \
    1\leqslant i < j \leqslant m,\ \text{共$\frac{m(m-1)}{2}$个}.
\end{equation}

依据定理\ref{chrg:thm_killing-MAX}可知,
上述三组Killing矢量场(共有${(m+2)(m+1)}/{2}$个)是
$m+1$维平直闵氏空间的全部Killing矢量场.


我们将在\S\ref{chlg:sec_killing-example}继续探讨这些Killing矢量场的物理意义.

\section{广义黎曼流形上的微分算子}
本节介绍几个广义黎曼流形上的几个常用微分算子,这在数学、物理中有着较为广泛的应用;
这些算子都是适用于整体的,都有局部坐标表示,它们为构造流形不变量提供了有力的工具.
本节中我们假设$(M,g)$是$m$维、已定向的广义黎曼流形,自然选用Levi-Civita联络;
如无特殊声明,假设局部坐标系是$(U;x^i)$.
%为此需先介绍黎曼流形上的体积元.

\index[physwords]{体积元}
\subsection{体积元}\label{chrg:sec_volume}
定向无非是给定一个$m$阶处处非零的微分型式场
(见定义\ref{chdf:def_orientation-dual}).在积分的定义
中(见式\eqref{chdf:eqn_multiple-integral}或\eqref{chdf:eqn_def-integral-omega-partition}),
也是对一个$m$阶微分型式场进行的.
在有了度规后,可以定义一个特殊的$m$阶微分型式场$\Omega_{a_1 \cdots a_m}$,
称之为与度规$g_{ab}$适配的{\heiti 体积元},简称{\heiti 体元};即
\begin{equation}\label{chrg:eqn_volume-g}
    \Omega^{a_1 \cdots a_m}\Omega_{a_1 \cdots a_m}= (-)^s m! ,
     \qquad \text{其中}\ \Omega^{a_1 \cdots a_m} = g^{a_1 b_1}\cdots g^{a_m b_m} \Omega_{b_1 \cdots b_m} .
\end{equation}
其中$s$是度规$g_{ab}$负本征值的个数(即式\eqref{chmla:eqn_gmetric}中“$-1$”的个数),对于
Minkowski时空来说$s=1$.因为它是$m$阶的微分型式场,所以它的基矢量只有一个,
它同构于标量函数场(见\S \ref{chmla:sec_def-exterior-product}).
设流形$M$有局部坐标系$(U;x^i)$,则体元$\Omega_{a_1 \cdots a_m}$可以表示为
\begin{equation}
    \Omega_{a_1 \cdots a_m}= \Omega\, ({\rm d}x^1)_{a_1} \wedge \cdots \wedge({\rm d}x^m)_{a_m} .
\end{equation}
将此式带入到式\eqref{chrg:eqn_volume-g}中,有
\begin{align*}
    (-)^s m! =& \Omega^2  \ g^{a_1 b_1}\cdots g^{a_m b_m}  ({\rm d}x^1)_{a_1} \wedge \cdots
    \wedge({\rm d}x^m)_{a_m} ({\rm d}x^1)_{b_1} \wedge \cdots \wedge({\rm d}x^m)_{b_m} \\
    \xlongequal{\ref{chmla:eqn_ep-base}}& \Omega^2 \delta_{i_1 \cdots i_m}^{1 \cdots m}
     \delta_{j_1 \cdots j_m}^{1 \cdots m} g^{i_1 j_1}\cdots g^{i_m j_m},
     {\qquad \text{将$\delta_{i_1 \cdots i_m}^{1 \cdots m}$按定义展开,得}} \\
    =& \Omega^2 \delta_{j_1 \cdots j_m}^{1 \cdots m} \left(
    (g^{1 j_1}\cdots g^{m j_m}) - (g^{2 j_1}g^{1 j_2} \cdots g^{m j_m})
     + \cdots \right),  {\quad \text{共$m!$项}} \\
    \xlongequal{\tiny \textcircled{4}}& m! \Omega^2 \delta_{j_1 \cdots j_m}^{1 \cdots m} g^{1 j_1}\cdots g^{m j_m}
    \xlongequal{\ref{chmla:eqn_gkd-305}} m! \Omega^2 \det(g^{ij}).
\end{align*}
其中$\xlongequal{\tiny \textcircled{4}}$步是交换$\{j\}$中次序(以第二项为例:
先进行变量代换,将哑标$j_1 \to k, j_2\to j_1$,即
$-\delta_{k j_1 j_3 \cdots j_m}^{1 2 3\cdots m}g^{2 k}g^{1 j_1} \cdots g^{m j_m}$;
然后交换$\delta_{k j_1 j_3\cdots j_m}^{12 3\cdots m}$中的$k j_1$次序,会产生一个“$-$”,
即变为$+\delta_{j_1 kj_3\cdots j_m}^{12 3\cdots m}g^{2 k}g^{1 j_1} \cdots g^{m j_m}$,与前面的负号刚好负负得正;
接着再作变量替换$k \to j_2$,这样第二项便与第一项完全相同;这样的项共有$m!$个.).
由此便可得到$\Omega^2=(-)^s g$,其中$g\equiv \det(g_{ij})$(注$\det(g_{ij})$与$\det(g^{ij})$刚好
互为倒数);便可得常用\uwave{体积元}
\begin{equation}\label{chrg:eqn_volume-element}
    \Omega_{a_1 \cdots a_m}= \pm \sqrt{(-)^s g}\, ({\rm d}x^1)_{a_1} \wedge \cdots \wedge({\rm d}x^m)_{a_m},
    \qquad g\equiv \det(g_{ij}).
\end{equation}
%其中正负号代表左右手定向,通常选为右手,即正号.
虽然是在局部坐标系下得到的表达式,但上式在坐标变换下是形式不变的,
换句话说此式是定义在整个流形$M$上的;验证过程留给读者(即另选一个
局部坐标系$\{y\}$进行坐标变换,$\sqrt{(-)^s g}$会产生一个Jacobi行列式;
$m$阶外积也会产生一个Jacobi行列式;两者刚好互逆,故形式不变).
其实从求解过程便可看出
此点,我们是从式\eqref{chrg:eqn_volume-g}出发的,此式是缩并式,其结果是在$M$上不变的.
虽然体积元$\Omega_{a_1 \cdots a_m}$是整个流形上的坐标变换形式不变量,
但单独的$({\rm d}x^1)_{a_1} \wedge \cdots \wedge({\rm d}x^m)_{a_m}$以及标量函数场$g\equiv \det(g_{ij})$不是.



有了体积元,再重复一下定向问题.  %(参见式\eqref{chdf:eqn_induced-orientation-nt})
给定可定向的广义黎曼流形$(M,g)$,其有
坐标图册$(U_\alpha,\varphi_\alpha;x^j_\alpha)$,坐标图册自身是定向相符合的.
若取式\eqref{chrg:eqn_volume-element}的“$+$”,则称体积元与坐标图册同向;
若取“$-$”,则称体积元与坐标图册反向.体积元本身就是一个定向.


\uwave{如无特殊声明,我们约定取同向,也就是取式\eqref{chrg:eqn_volume-element}的“$+$”.}




以后,我们会经常遇到体积元间的缩并运算,下面给出\uwave{体积元缩并公式}:
\begin{equation}\label{chrg:eqn_VE-contract-VE}
    \boxed{
    \Omega_{a_1 \cdots a_r a_{r+1} \cdots a_m}\Omega^{{b_1 \cdots b_r} a_{r+1} \cdots a_m}
    = (-)^s \, (m-r)! \, r! \, \delta_{\lbrack a_1}^{b_1} \delta_{a_2}^{b_2}\cdots 
    \delta_{a_r \rbrack}^{b_r}   .  }
\end{equation}
其中$\Omega^{{b_1 \cdots b_r} a_{r+1} \cdots a_m}\equiv g^{b_{1}c_{1}} \cdots g^{b_{r}c_{r}}
g^{a_{r+1}c_{r+1}} \cdots g^{a_{m}c_{m}}\Omega_{c_1 \cdots c_r c_{r+1} \cdots c_m}$,
$0\leqslant r \leqslant m$.
直接计算证明上式.
\setlength{\mathindent}{0em}
\begin{align*}
    &\Omega_{a_1 \cdots a_r a_{r+1} \cdots a_m}
    \Omega^{{b_1 \cdots b_r} a_{r+1} \cdots a_m} \\
    &= (-)^s g ({\rm d}x^1)_{a_1} \wedge \cdots \wedge({\rm d}x^m)_{a_m}
    \otimes ({\rm d}x^1)_{c_1} \wedge \cdots \wedge({\rm d}x^m)_{c_m}
    g^{b_{1}c_{1}}\cdots g^{b_{r}c_{r}} g^{a_{r+1}c_{r+1}} \cdots g^{a_{m}c_{m}} \\
    &\xlongequal{\ref{chmla:eqn_ep-base}} (-)^s g
    \delta_{i_1 \cdots i_m}^{1 \cdots m} \delta_{j_1 \cdots j_m}^{1 \cdots m}
    ({\rm d}x^{i_{1}})_{a_{1}} \otimes \cdots \otimes ({\rm d}x^{i_r})_{a_r}  \otimes
    ({\rm d}x^{j_{1}})_{c_{1}} \otimes \cdots \otimes ({\rm d}x^{j_r})_{c_r}  \\
    &\qquad g^{b_{1}c_{1}}\cdots g^{b_{r}c_{r}} \cdot g^{i_{r+1}j_{r+1}} \cdots g^{i_{m}j_{m}} \\
    &= %\xlongequal{\ref{chmla:eqn_gkd-205}}
    (-)^s \delta_{i_1 \cdots i_m}^{1 \cdots m}
    ({\rm d}x^{i_{1}})_{a_{1}}  \cdots  ({\rm d}x^{i_r})_{a_r}
    \left(\frac{\partial }{\partial x^{\mu_1}}\right) ^{b_1} \cdots
    \left(\frac{\partial }{\partial x^{\mu_r}}\right) ^{b_r} \times g \times \\
    & \qquad  \delta_{j_1 \cdots j_m}^{1 \cdots m}
    g^{\mu_{1} j_{1}}\cdots g^{\mu_{r} j_{r}}  g^{i_{r+1}j_{r+1}} \cdots g^{i_{m}j_{m}} .
\end{align*}\setlength{\mathindent}{2em}
上面最后一个等号的第二行,即$\delta_{j_1 \cdots j_m}^{1 \cdots m}
g^{\mu_{1} j_{1}}\cdots g^{\mu_{r} j_{r}}  g^{i_{r+1}j_{r+1}} \cdots g^{i_{m}j_{m}} $,
无非是个行列式
\begin{equation}
    \begin{vmatrix}
        g^{\mu_{1} 1} & \cdots & g^{\mu_{r} 1} & g^{i_{r+1} 1} & \cdots & g^{i_{m} 1} \\
        g^{\mu_{1} 2} & \cdots & g^{\mu_{r} 2} & g^{i_{r+1} 2} & \cdots & g^{i_{m} 2} \\
        \vdots & \vdots & \vdots & \vdots & \vdots & \vdots \\
        g^{\mu_{1} m} & \cdots & g^{\mu_{r} m} & g^{i_{r+1} m} & \cdots & g^{i_{m} m}
    \end{vmatrix}
=g^{-1}\delta^{\mu_1 \cdots\mu_r i_{r+1}\cdots i_m}_{1 \cdots m} .
\end{equation}
带入上上式后继续计算,得
%\begin{equation}\label{chrg:eqn_VE-contract-VE-comp}
%\begin{aligned}
\begin{align*}
    &\Omega_{a_1 \cdots a_r a_{r+1} \cdots a_m}
    \Omega^{{b_1 \cdots b_r} a_{r+1} \cdots a_m} \\
   &= (-)^s \delta_{i_1 \cdots i_m}^{1 \cdots m}
    ({\rm d}x^{i_{1}})_{a_{1}}  \cdots  ({\rm d}x^{i_r})_{a_r}
    \left(\frac{\partial }{\partial x^{\mu_1}}\right) ^{b_1} \cdots
    \left(\frac{\partial }{\partial x^{\mu_r}}\right) ^{b_r}
    g\cdot g^{-1}\delta^{\mu_1 \cdots\mu_r i_{r+1}\cdots i_m}_{1 \cdots m} \\
   &\xlongequal{\ref{chmla:eqn_gkd-205}} (-)^s
     \delta_{i_1 \cdots i_r i_{r+1}\cdots i_m}^{\mu_1 \cdots\mu_r i_{r+1}\cdots i_m}
    ({\rm d}x^{i_{1}})_{a_{1}}  \cdots  ({\rm d}x^{i_r})_{a_r}
    \left(\frac{\partial }{\partial x^{\mu_1}}\right) ^{b_1} \cdots
    \left(\frac{\partial }{\partial x^{\mu_r}}\right) ^{b_r} \\
   &\xlongequal{\ref{chmla:eqn_gkd-120}} (-)^s
   (m-r)! \delta^{\mu_1 \cdots \mu_{r}}_{i_1 \cdots i_{r}}
   ({\rm d}x^{i_{1}})_{a_{1}}  \cdots  ({\rm d}x^{i_r})_{a_r}
   \left(\frac{\partial }{\partial x^{\mu_1}}\right) ^{b_1} \cdots
   \left(\frac{\partial }{\partial x^{\mu_r}}\right) ^{b_r} \\
   &\xlongequal{\ref{chmla:eqn_gkd-15} } (-)^s (m-r)! r!
   \delta_{\lbrack i_1}^{\mu_1} \cdots \delta_{i_r \rbrack}^{\mu_r}
      ({\rm d}x^{i_{1}})_{a_{1}}  \cdots  ({\rm d}x^{i_r})_{a_r}
   \left(\frac{\partial }{\partial x^{\mu_1}}\right) ^{b_1} \cdots
   \left(\frac{\partial }{\partial x^{\mu_r}}\right) ^{b_r} .
\end{align*}
%\end{aligned}
%\end{equation}
%式\eqref{chrg:eqn_VE-contract-VE-comp}
上式最后一个等号是式\eqref{chrg:eqn_VE-contract-VE}的分量
形式.证毕. \qed


$\forall X^a\in \mathfrak{X}(M)$,有$\nabla_X\Omega_{a_1 \cdots a_m}$是$m$次外微分型式场.
由于体元已是$m$维流形$M$中的最高次($m$次)外微分型式场,故此空间是一维空间,
必然有$\nabla_X\Omega_{a_1 \cdots a_m} = h\cdot \Omega_{a_1 \cdots a_m}$,其中$h\in C^\infty(M)$.
由式\eqref{chrg:eqn_volume-g}可知
\begin{equation}
    0= \Omega^{a_1 \cdots a_m} \nabla_X\Omega_{a_1 \cdots a_m}
    = h \cdot \Omega^{a_1 \cdots a_m} \Omega_{a_1 \cdots a_m}
    = h\cdot (-)^s m! {\ \color{red}\Rightarrow\  } h=0.
\end{equation}
再因$X^a$的任意性,可得到一个常用公式
\begin{equation}\label{chrg:eqn_dVE=0}
    \nabla_X\Omega_{a_1 \cdots a_m} = 0 \quad \Leftrightarrow \quad
    \nabla_b\Omega_{a_1 \cdots a_m} = 0 .
\end{equation}

虽然体积元的协变导数恒为零,但是它的李导数不是零.
\begin{equation}\label{chrg:eqn_LieDVE}
    \Lie_{X} \Omega_{a_1 \cdots a_m}
    \xlongequal[\ref{chrg:eqn_dVE=0}]{\ref{chccr:eqn_Lie-i(X)-d}}
    {\rm d}_{a_1}\bigl(X^b \Omega_{ba_2 \cdots a_m} \bigr)
    = (\nabla_{b}X^b) \Omega_{a_1 \cdots a_m}  .
\end{equation}
此式的证明过程与体积元协变导数证明过程类似;
同样由于体元已是$m$维流形$M$中的最高次外微分型式场,所以
有${\rm d}_{a_1}\bigl(X^b \Omega_{ba_2 \cdots a_m} \bigr) = h\cdot \Omega_{a_1 \cdots a_m}$,
其中$h\in C^\infty(M)$;用$\Omega^{a_1 \cdots a_m}$缩并此式的等号两边,可得
\setlength{\mathindent}{0em}
\begin{align*}
   & h\cdot \Omega^{a_1 \cdots a_m} \Omega_{a_1 \cdots a_m}
   =\Omega^{a_1 \cdots a_m} {\rm d}_{a_1}\bigl(X^b \Omega_{ba_2 \cdots a_m} \bigr)
    \xRightarrow[\ref{chrg:eqn_dVE=0}]{\ref{chrg:eqn_VE-contract-VE}}
    (-)^s m! \cdot h \\
    &=  m \Omega^{a_1 \cdots a_m} \nabla_{a_1}\bigl(X^b \Omega_{ba_2 \cdots a_m} \bigr)
    =m \Omega^{a_1 \cdots a_m} \Omega_{ba_2 \cdots a_m} \nabla_{a_1}X^b
    =(-)^s m (m-1)! \delta_{b}^{a_1} \nabla_{a_1}X^b \\
    &{\color{red}\Rightarrow \quad} h = \nabla_{b}X^b .
\end{align*}\setlength{\mathindent}{2em}
将$h = \nabla_{b}X^b$带回即可得到体积元的李导数公式.


\index[physwords]{Hodge星对偶}
\subsection{Hodge星对偶算子}\label{chrg:sec_Hodge}
设$m$维黎曼流形$(M,g)$有局部坐标系$(U;x^i)$,$\omega_{a_1 \cdots a_r}\in A^r(M)$的局部坐标表示为
\begin{equation}
    \omega_{a_1 \cdots a_r}= \frac{1}{r!}\omega_{i_1\cdots i_r} ({\rm d}x^{i_1})_{a_1} \wedge \cdots
      \wedge({\rm d}x^{i_r})_{a_r}, \qquad \omega_{i_1\cdots i_r} \in C^\infty(U) .
\end{equation}
借助体积元(见式\eqref{chrg:eqn_volume-element}),可定义开邻域$U$上的Hodge星对偶算子:
\begin{subequations}\label{chrg:eqn_Hodge-star}
\begin{align}
    &*\omega _{ a_{r+1}  \cdots a_m} \overset{def}{=} 
    \frac{1}{r!} \omega^{a_1 \cdots a_r} \Omega_{a_1 \cdots a_r a_{r+1} \cdots a_m},
      \label{chrg:eqn_Hodge-star-1} \\
    =& \frac{\sqrt{(-)^s g}}{r! (m-r)!} \delta_{i_{1} \cdots i_m}^{1 \cdots m}
      \cdot \omega^{i_1\cdots i_r}\cdot   ({\rm d}x^{i_{r+1}})_{a_{r+1}} \wedge
     \cdots \wedge({\rm d}x^{i_m})_{a_m} . \label{chrg:eqn_Hodge-star-2}
\end{align}
\end{subequations}
上面两式都能当成定义.
由式\eqref{chrg:eqn_Hodge-star-1}到\eqref{chrg:eqn_Hodge-star-2}的推导并不困难;
直接计算,得
\setlength{\mathindent}{0em}
\begin{align*}
    *\omega _{a_{r+1} \cdots a_m} {=}  &
    \frac{\sqrt{(-)^s g}}{r! r!} \omega_{i_1\cdots i_r} g^{a_1 b_1}\cdots g^{a_r b_r}
    ({\rm d}x^{i_1})_{b_1} \wedge \cdots \wedge({\rm d}x^{i_r})_{b_r}
    ({\rm d}x^1)_{a_1} \wedge \cdots \wedge({\rm d}x^m)_{a_m} \\
    \xlongequal{\ref{chmla:eqn_ep-base}}& \frac{\sqrt{(-)^s g}}{r! r!} \omega_{i_1\cdots i_r}
    \delta^{i_1 \cdots i_r}_{k_1 \cdots k_r}  \delta_{j_1 \cdots j_m}^{1 \cdots m}
     g^{k_1 j_1}\cdots g^{k_r j_r}
     ({\rm d}x^{j_{r+1}})_{a_{r+1}} \otimes \cdots \otimes ({\rm d}x^{j_m})_{a_m}  \\
     \xlongequal{\ref{chmla:eqn_gkd-25}}& \frac{\sqrt{(-)^s g}}{r!} \omega^{j_1\cdots j_r}
      \delta_{j_1 \cdots j_m}^{1 \cdots m}
     ({\rm d}x^{j_{r+1}})_{a_{r+1}} \otimes \cdots \otimes ({\rm d}x^{j_m})_{a_m}  \\
     \xlongequal{\ref{chmla:eqn_gkd-210}}&
     \frac{\sqrt{(-)^s g}}{r! (m-r)!} \omega^{j_1\cdots j_r}
     \delta_{j_1 \cdots j_r k_{r+1}\cdots k_m}^{1 \cdots r {r+1}\cdots m}
     \cdot \delta^{k_{r+1} \cdots k_m}_{j_{r+1} \cdots j_m}
     ({\rm d}x^{j_{r+1}})_{a_{r+1}} \otimes \cdots \otimes ({\rm d}x^{j_m})_{a_m}  \\
     =&\frac{\sqrt{(-)^s g}}{r! (m-r)!} \omega^{j_1\cdots j_r}
     \delta_{j_1 \cdots j_r k_{r+1}\cdots k_m}^{1 \cdots r {r+1}\cdots m}
     ({\rm d}x^{k_{r+1}})_{a_{r+1}} \wedge \cdots \wedge ({\rm d}x^{k_m})_{a_m}  .
\end{align*}\setlength{\mathindent}{2em}
这便证明了\eqref{chrg:eqn_Hodge-star}.
大致重复上面推导过程(令$\omega_{i_1\cdots i_r}=1$)可得:
\setlength{\mathindent}{0em}
\begin{align}
    &*\bigl( ({\rm d}x^{i_{1}})_{a_{1}} \wedge \cdots \wedge ({\rm d}x^{i_r})_{a_r}\bigr) =\notag \\
%    &=\frac{\sqrt{(-)^s g}}{r!}  g^{a_1 b_1}\cdots g^{a_r b_r}
%    ({\rm d}x^{i_1})_{b_1} \wedge \cdots \wedge({\rm d}x^{i_r})_{b_r}
%    ({\rm d}x^1)_{a_1} \wedge \cdots \wedge({\rm d}x^m)_{a_m} \\
%    &= \frac{\sqrt{(-)^s g}}{r!} \delta^{i_1 \cdots i_r}_{k_1 \cdots k_r}  \delta_{j_1 \cdots j_m}^{1 \cdots m}
%       g^{k_1 j_1}\cdots g^{k_r j_r} ({\rm d}x^{j_{r+1}})_{a_{r+1}} \otimes \cdots \otimes ({\rm d}x^{j_m})_{a_m}  \\
%    &= \frac{\sqrt{(-)^s g}}{r!} \delta_{j_1 \cdots j_m}^{1 \cdots m}
%    \begin{vmatrix}
%        g^{i_1 j_1} & \cdots & g^{i_1 j_r}  \\
%        \vdots & \ddots & \vdots \\
%        g^{i_r j_1} & \cdots & g^{i_r j_r}  \\
%    \end{vmatrix}
%    ({\rm d}x^{j_{r+1}})_{a_{r+1}} \otimes \cdots \otimes ({\rm d}x^{j_m})_{a_m}  \\
%    &=     \frac{\sqrt{(-)^s g}}{r! (m-r)!}
%    \delta_{j_1 \cdots j_r k_{r+1}\cdots k_m}^{1 \cdots r {r+1}\cdots m}
%    \cdot \delta^{k_{r+1} \cdots k_m}_{j_{r+1} \cdots j_m}
%    ({\rm d}x^{j_{r+1}})_{a_{r+1}} \otimes \cdots \otimes ({\rm d}x^{j_m})_{a_m}
%    \begin{vmatrix}
%        g^{i_1 j_1} & \cdots & g^{i_1 j_r}  \\
%        \vdots & \ddots & \vdots \\
%        g^{i_r j_1} & \cdots & g^{i_r j_r}  \\
%    \end{vmatrix} \notag \\
%    &= \frac{\sqrt{(-)^s g}}{r! (m-r)!}
%    \delta_{j_1 \cdots j_m}^{1 \cdots m}
%    ({\rm d}x^{j_{r+1}})_{a_{r+1}} \wedge \cdots \wedge ({\rm d}x^{j_m})_{a_m}
%    \begin{vmatrix}
%        g^{i_1 j_1} & \cdots & g^{i_1 j_r}  \\
%        \vdots & \ddots & \vdots \\
%        g^{i_r j_1} & \cdots & g^{i_r j_r}  \\
%    \end{vmatrix} \\
    & \sqrt{(-)^s g} \sum_{\substack {j_1<\cdots<j_r \\ j_{r+1}<\cdots<j_m}}
    \delta_{j_1 \cdots j_m}^{1 \cdots m}
    ({\rm d}x^{j_{r+1}})_{a_{r+1}} \wedge \cdots \wedge ({\rm d}x^{j_m})_{a_m}
    \begin{vmatrix}
        g^{i_1 j_1} & \cdots & g^{i_1 j_r}  \\
        \vdots & \ddots & \vdots \\
        g^{i_r j_1} & \cdots & g^{i_r j_r}  \\
    \end{vmatrix}    \label{chrg:eqn_dx1r-Hodge}
\end{align}\setlength{\mathindent}{2em}
式\eqref{chrg:eqn_dx1r-Hodge}中的指标$i_1,\cdots,i_r$是自由指标,不参与求和;
在具体应用时最好按顺序排列.
下面给出两次Hodge星算子的公式,直接对\eqref{chrg:eqn_Hodge-star-1}取星算子,得
\begin{equation}\label{chrg:eqn_double-star-Hodge}
\begin{aligned}
    *(*\omega)_{b_1 \cdots b_r} =&  \frac{(-)^{r(m-r)}}{r! (m-r)!} \omega_{a_1 \cdots a_r}
     \Omega^{a_1 \cdots a_r a_{r+1} \cdots a_m}
     \Omega_{{b_1 \cdots b_r} a_{r+1} \cdots a_m} \\
     \xlongequal{\ref{chrg:eqn_VE-contract-VE}} &
     (-)^{s+r(m-r)} \delta^{\lbrack a_1}_{b_1} \cdots \delta^{a_r \rbrack}_{b_r}
     \omega_{a_1 \cdots a_r}
     =(-)^{s+r(m-r)} \omega_{b_1 \cdots b_r} .
\end{aligned}
\end{equation}

\paragraph{闵氏空间对偶} %\label{chrg:sec_Minkowski-Hodge-1}
闵氏时空$(\mathbb{R}^4,\eta)$的坐标系为$\{t,x,y,z\}$;
因标架场正交归一,度规$\eta={\rm diag}(-1,1,1,1)$,
式\eqref{chrg:eqn_dx1r-Hodge}的形式简单;
表示如下(省略抽象指标)
%\begin{equation}\label{chrg:eqn_Minkowski-Hodge-1}
\begin{align*}
    * {\rm d}t =& -{\rm d}x \wedge {\rm d}y \wedge {\rm d}z , \qquad
    * {\rm d}x =  -{\rm d}t \wedge {\rm d}y \wedge {\rm d}z , \\
    * {\rm d}y =& +{\rm d}t \wedge {\rm d}x \wedge {\rm d}z , \qquad
    * {\rm d}z =  -{\rm d}t \wedge {\rm d}x \wedge {\rm d}y . \\
    * ({\rm d}t \wedge {\rm d}x) =&- {\rm d}y \wedge {\rm d}z, \quad
    * ({\rm d}t \wedge {\rm d}y) = + {\rm d}x \wedge {\rm d}z, \quad
    * ({\rm d}t \wedge {\rm d}z) = - {\rm d}x \wedge {\rm d}y. \\
    * ({\rm d}x \wedge {\rm d}y) =&+ {\rm d}t \wedge {\rm d}z, \quad
    * ({\rm d}x \wedge {\rm d}z) = - {\rm d}t \wedge {\rm d}y, \quad
    * ({\rm d}y \wedge {\rm d}z) = + {\rm d}t \wedge {\rm d}x.
\end{align*}
%\end{equation}


\paragraph{三维空间的对偶}
三维笛卡尔空间$(\mathbb{R}^3,\delta)$的坐标系为$\{x,y,z\}$;
因标架场正交归一,$\delta={\rm diag}(1,1,1)$,
我们给出式\eqref{chrg:eqn_dx1r-Hodge}的具体表示(省略抽象指标)
\begin{equation}\label{chrg:eqn_Cartesian-Hodge}
    * {\rm d}x = {\rm d}y \wedge {\rm d}z, \quad
    * {\rm d}y = {\rm d}z \wedge {\rm d}x, \quad
    * {\rm d}z = {\rm d}x \wedge {\rm d}z .
\end{equation}

叉乘、散度、旋度等公式都可用Hodge星对偶表示,请读者自行写出.





\index[physwords]{散度算子}
\subsection{散度及相关}
先计算克氏符$\Gamma^k_{kj}$;
为此,需用到一个行列式中的公式,设$g=\det(g_{ij})$.
\setlength{\mathindent}{0em}
\begin{align*}
    &\frac{\partial g}{\partial x^k}= \frac{\partial }{\partial x^k}
    \left( \sum_{j_1 \cdots j_n} (-)^{\tau(j_1 \cdots j_n)}
    g_{1 j_1} g_{2 j_2}\cdots g_{n j_n} \right) \\ %\qquad i\ \text{取固定值} \\
    =&\left( \sum_{j_1 \cdots j_n} (-)^{\tau(j_1 \cdots j_n)}
    \frac{\partial g_{1 j_1}}{\partial x^k} g_{2 j_2}\cdots g_{n j_n} \right)
    +\cdots + \left( \sum_{j_1 \cdots j_n} (-)^{\tau(j_1 \cdots j_n)}
    g_{1 j_1} \cdots g_{n-1, j_{n-1}} \frac{\partial g_{n j_n}}{\partial x^k}  \right) \\
    =& \sum_{i=1}^{n}
    \begin{vmatrix}
        g_{1 1} & \cdots & g_{1 j} & \cdots & g_{1 n} \\
        \vdots & \ddots & \vdots & \ddots & \vdots \\
        \frac{\partial g_{i 1}}{\partial x^k} & \cdots &
           \frac{\partial g_{i j}}{\partial x^k} & \cdots &
           \frac{\partial g_{i n}}{\partial x^k} \\
        \vdots & \ddots & \vdots & \ddots & \vdots \\
        g_{n 1} & \cdots & g_{n j} & \cdots & g_{nn}
    \end{vmatrix} .
\end{align*}\setlength{\mathindent}{2em}
这是行列式的求导公式,将上式按有导数那一行展开,有
\begin{equation}\label{chrg:eqn_detga}
    \frac{\partial g}{\partial x^k}= \sum_{i=1}^{n} \sum_{j=1}^{n}
    \frac{\partial g_{ij}}{\partial x^k} A^{ij}, \qquad
    \text{其中}A^{ij}\text{是}\frac{\partial g_{ij}}{\partial x^k}\text{或}g_{ij}\text{的代数余子式}.
\end{equation}
在行列式理论中,元素$g_{il}$与代数余子式$A^{lj}$有正交关系,即
\begin{equation}
    \sum_{l=1}^{n}g_{il} A^{lj} = \delta_i^j \cdot g .
\end{equation}
又因度规$g_{il}$及其共轭量$g^{lj}$互逆,即$g_{il}g^{lj}=\delta_i^j$;由此不难得出
\begin{equation}
    A^{ij}=g\cdot g^{ij} .
\end{equation}
将上式带入式\eqref{chrg:eqn_detga},并整理得(考虑到$g$可正可负,故加上绝对值)
\begin{equation}\label{chrg:eqn_detgijij}
    \frac{1}{g}\frac{\partial g}{\partial x^k}= g^{ij} \frac{\partial g_{ij}}{\partial x^k}
    \quad \Rightarrow \quad
    \frac{1}{\sqrt{|g|}}\frac{\partial \sqrt{|g|}}{\partial x^k}=
     \frac{1}{2}g^{ij} \frac{\partial g_{ij}}{\partial x^k} .
\end{equation}
利用上式容易得到{\heiti 收缩第二类克氏符}为
\begin{equation}\label{chrg:eqn_Gamma-KKJ}
    \Gamma^k_{kj} = \frac{1}{2}{g^{kl}}\left( \frac{\partial g_{kl}}{\partial x^j}
        + \frac{\partial g_{lj}}{\partial x^k}  - \frac{\partial g_{kj}} {\partial x^l} \right)
    = \frac{1}{2}{g^{kl}}\frac{\partial g_{kl}}{\partial x^j}
    =\frac{1}{\sqrt{|g|}}\frac{\partial \sqrt{|g|}}{\partial x^j} .
\end{equation}



我们已知切矢量场$v^a\in \mathfrak{X}(M)$的协变导数,将两者缩并可得{\heiti 散度算子}:
\begin{equation}\label{chrg:eqn_divergence}
    \nabla_a v^a = \frac{\partial v^j}{\partial x^j} +v^k \Gamma_{kj}^{j}(x)
    = \frac{\partial v^j}{\partial x^j} +v^k \frac{1}{\sqrt{|g|}}\frac{\partial \sqrt{|g|}}{\partial x^k}
    =\frac{1}{\sqrt{|g|}}\frac{ \partial \left(v^k\sqrt{|g|}\right)}{\partial x^k} 
\end{equation}

微积分中的Laplace算子,在微分流形中已被推广为{\bfseries Beltrami--Laplace 算子},
只需将式\eqref{chrg:eqn_divergence}中$v^k$换成$g^{kj} \frac{ \partial f}{\partial x^j}$即可
\begin{equation}\label{chrg:eqn_Beltrami-Laplace}
    \square f \equiv \nabla_a \nabla^a f = \frac{1}{\sqrt{|g|}}\frac{ \partial }{\partial x^k}
    \left( \sqrt{|g|}g^{kj} \frac{ \partial f}{\partial x^j}  \right) ,
    \qquad \forall f \in C^\infty(M) .
\end{equation}
纯数学上更习惯使用“$\Delta$”来替换上式中的“$\square$”;但在物理学的四维闵氏时空中
都使用这个方块算符来代表d'Alembert算符,所以在此我们使用方块算符.



\begin{exercise}
	我们使用度规定义了矢量场的散度公式.
	实际上,矢量场散度无需度规;现有公式$\nabla_a v^b$,
	令其上下指标缩并即可;试给出无度规时的矢量场散度公式(用克氏符表示).
\end{exercise}

\begin{exercise}
	用Hodge星算子表示三维欧式空间的叉乘、散度、旋度公式.
\end{exercise}




\index[physwords]{黎曼曲率!型式}
\section{黎曼曲率型式}\label{chrg:sec_form2}
本节介绍广义黎曼流形中的曲率型式\cite[\S 1.7]{chandrasekhar-1983},
\S\ref{chccr:sec_form1}中的理论自然适用于本节,
这里主要讨论度规场对型式理论的影响.
设有$m$维广义黎曼流形$(M,g)$,以及无挠的Levi-Civita联络$\nabla_a$.


\subsection{黎曼度规下的型式理论}
在标架场$\{(e_\mu)^a\}$上,度规表示为
\begin{equation}
g_{ab}=g_{\mu\nu}(e^\mu)_a (e^\nu)_b, \qquad
g^{ab}=g^{\mu\nu}(e_\mu)^a (e_\nu)^b.
\end{equation}
再引入两个记号
\begin{align}
(e_\rho)_a &\overset{def}{=} g_{ab}(e_\rho)^b \
= g_{\mu\nu}(e^\mu)_a (e^\nu)_b (e_\rho)^b
= g_{\mu\rho}(e^\mu)_a, \\
(e^\rho)^a &\overset{def}{=} g^{ab}(e^\rho)_b \
= g^{\mu\nu}(e_\mu)^a (e_\nu)^b(e^\rho)_b
= g^{\mu\rho}(e_\mu)^a .
\end{align}
上面两式说明,基矢的内外指标均可用度规相应指标进行升降.
需要说明,如果没有度规,无法定义$(e_\rho)_a$和$(e^\rho)^a$.
由此可得:
\begin{equation}
  g_{ab}=(e_\mu)_a (e^\mu)_b, \  g^{ab}=(e^\mu)^a (e_\mu)^b ; \
  g_{\mu\nu}=(e_\mu)_a (e_\nu)^a, \  g^{\mu\nu}=(e^\mu)^a (e^\nu)_a .
\end{equation}
可以再定义
\begin{equation}
    (\omega _{\mu\nu})_{a} \overset{def}{=}
      g_{\mu\sigma} (\omega ^{\sigma}_{\cdot\nu})_{a}
    = g_{\mu\sigma} (e^\sigma)_c \nabla _a (e_\nu)^c
    = (e_\mu)_c \nabla _a (e_\nu)^c
    = g_{\mu\sigma}\Gamma^{\sigma}_{\nu\tau} (e^\tau)_a .
\end{equation}
上式计算用到了式\eqref{chccr:def_1form}.同样,上式必须在引入度规之后才能定义.

\index[physwords]{黎曼曲率!型式!相容性条件}
\paragraph{相容性条件}
自然坐标下的相容性条件\eqref{chrg:eqn_Dg=0}在一般标架场下变为
\setlength{\mathindent}{0em}
\begin{align*}
    &0=\nabla_a g_{bc} \quad {\color{red}\Rightarrow}\quad 0=
    (\nabla_a g_{\mu\nu}) (e^\mu)_b (e^\nu)_c
    +g_{\mu\nu} \bigl(\nabla_a (e^\mu)_b\bigr) (e^\nu)_c
    +g_{\mu\nu}(e^\mu)_b \bigl(\nabla_a (e^\nu)_c \bigr) \\
    {\color{red}\Rightarrow} &0=
    ({\rm d}_a g_{\mu\nu}) (e^\mu)_b (e^\nu)_c
    -g_{\mu\nu} \bigl( (e^j)_b (\omega_{\cdot j}^{\mu})_a \bigr) (e^\nu)_c
    -g_{\mu\nu}(e^\mu)_b \bigl((e^j)_c (\omega_{\cdot j}^{\nu})_a \bigr) .
\end{align*}\setlength{\mathindent}{2em}
上式应用了式\eqref{chccr:eqn_D-omega-form}($\nabla _a (e^\mu)_b=- (e^j)_b (\omega_{\cdot j}^{\mu})_a$).
消去基矢,得到度规在一般标架场下的相容性条件,即$\nabla_a g_{bc}=0$的具体表达式
\begin{equation}\label{chrg:eqn_Dg=0-onframes}
    {\rm d}_a g_{\mu\nu} 
     = (\omega_{\nu \mu})_a + (\omega_{\mu \nu})_a
     = \left(g_{\mu\sigma}\Gamma^{\sigma}_{\nu\tau}
      + g_{\nu\sigma}\Gamma^{\sigma}_{\mu\tau}\right) (e^\tau)_a .
\end{equation}

\index[physwords]{黎曼曲率!型式!Cartan结构方程}
\paragraph{Cartan结构方程}
这里采用Levi-Civita联络,挠率恒为零;
原形式的Cartan结构方程见下节的式\eqref{chrg:eqn_CST-Rieman-Conn};
曲率型式\eqref{chccr:eqn_Cform}中的内指标降下来的公式为:
\begin{equation}\label{chrg:eqn_Cform-down}
    (\Omega_{\mu\nu} )_{ab} 
    = g_{\mu\sigma}(\Omega_{\cdot \nu} ^{\sigma})_{ab}
    = \frac{1}{2} R_{\mu \nu \rho\sigma}  (e^\rho)_a \wedge (e^\sigma)_b
    = R_{\mu \nu \rho\sigma} (e^\rho)_a (e^\sigma)_b .
\end{equation}
应用式\eqref{chrg:eqn_Dg=0-onframes}容易得到内指标降下来后的结构方程具体形式
\begin{subequations}\label{chrg:eqn_CST-down}
    \begin{align}
        {\rm d}_a (e_\mu)_b &=- (e^\sigma)_a \wedge (\omega_{\sigma \mu})_b  , \label{chrg:eqn_CST-down-I} \\
        {\rm d}_a (\omega _{\mu\nu})_b  &= (\omega_{\sigma \mu})_a \wedge (\omega _{\cdot \nu} ^\sigma  )_b
          + (\Omega_{\mu \nu})_{ab}  . \label{chrg:eqn_CST-down-II}
    \end{align}
\end{subequations}

%\begin{align*}
%    {\rm d}_a (e_\mu)_b =& {\rm d}_a\bigl(g_{\mu\sigma}(e^\sigma)_b\bigr)
%    ={\rm d}_a (g_{\mu\sigma}) \wedge (e^\sigma)_b +g_{\mu\sigma} {\rm d}_a\bigl((e^\sigma)_b\bigr) \\
%    =& \bigl((\omega_{\sigma \mu})_a + (\omega_{\mu \sigma})_a \bigr) \wedge (e^\sigma)_b
%     + g_{\mu\sigma}(e^\nu)_a \wedge (\omega _{\cdot \nu}^\sigma)_b
%    = - (e^\sigma)_a \wedge (\omega_{\sigma \mu})_b
%\end{align*}
%\begin{align*}
%    {\rm d}_a (\omega _{\mu\nu})_b = &{\rm d}_a \bigl( g_{\mu\sigma}(\omega _{\cdot \nu} ^\sigma  )_b \bigr)
%    = {\rm d}_a ( g_{\mu\sigma}) \wedge (\omega _{\cdot \nu} ^\sigma  )_b
%     +g_{\mu\sigma} {\rm d}_a \bigl( (\omega _{\cdot \nu} ^\sigma  )_b \bigr) \\
%    =& \bigl( (\omega_{\sigma \mu})_a + (\omega_{\mu \sigma})_a \bigr) \wedge (\omega _{\cdot \nu} ^\sigma  )_b
%    +g_{\mu\sigma} ( \omega _{\cdot \nu}^\pi )_a
%    \wedge ( \omega _{\cdot \pi}^\sigma )_b + g_{\mu\sigma}(\Omega_{\cdot \nu} ^{\sigma})_{ab} \\
%    =& (\omega_{\sigma \mu})_a \wedge (\omega _{\cdot \nu} ^\sigma  )_b
%     + (\omega_{\mu \sigma})_a \wedge (\omega _{\cdot \nu} ^\sigma  )_b
%     + (\omega_{\cdot \nu}^\pi )_a  \wedge ( \omega _{\mu \pi})_b + (\Omega_{\mu \nu})_{ab} \\
%    =& (\omega_{\sigma \mu})_a \wedge (\omega _{\cdot \nu} ^\sigma  )_b + (\Omega_{\mu \nu})_{ab}
%\end{align*}

\index[physwords]{黎曼曲率!型式!Bianchi恒等式}
\paragraph{Bianchi恒等式}
内指标降下来后的第一Bianchi恒等式\eqref{chccr:eqn_Bianchi-CI}变为
\begin{equation}\label{chrg:eqn_Bianchi-CI-bak}
    (e^\sigma)_c \wedge (\Omega_{\cdot \sigma} ^{\mu})_{ab} = 0 =
    (e^\sigma)_c \wedge (\Omega_{\mu \sigma} )_{ab}  .
\end{equation}
应用式\eqref{chrg:eqn_Dg=0-onframes}容易得到内指标降下来后的第二Bianchi恒等式 %\eqref{chccr:eqn_Bianchi-CII}
\begin{align}
    {\rm d}_c (\Omega_{\mu\nu} )_{ab} = &
       ( \omega _{\cdot \nu}^\sigma )_c \wedge (\Omega_{\mu \sigma} )_{ab}
      +(\Omega_{\cdot \nu} ^{\sigma})_{ca} \wedge (\omega_{\sigma \mu})_{b}
       . \label{chrg:eqn_Bianchi-CII-down} \\
    {\rm d}_c (\Omega_{\cdot \nu} ^{\mu})_{ab}  =&
       ( \omega _{\cdot \nu}^\sigma )_c  \wedge  (\Omega_{\cdot \sigma} ^{\mu})_{ab}
     - (\Omega_{\cdot \nu} ^{\sigma})_{ca}\wedge ( \omega _{\cdot \sigma}^\mu )_b
       . \tag{\ref{chccr:eqn_Bianchi-CII}}
\end{align}
为了比对,保留了式\eqref{chccr:eqn_Bianchi-CII}.

%推导过程
%\begin{align*}
%    {\rm d}_c (\Omega_{\mu\nu} )_{ab} =& {\rm d}_c \bigl( g_{\mu\sigma}(\Omega_{\cdot \nu} ^{\sigma})_{ab} \bigr)
%    = {\rm d}_c ( g_{\mu\sigma} ) \wedge(\Omega_{\cdot \nu} ^{\sigma})_{ab}
%    + g_{\mu\sigma} {\rm d}_c \bigl( (\Omega_{\cdot \nu} ^{\sigma})_{ab} \bigr) \\
%    \xlongequal[\ref{chccr:eqn_Bianchi-CII}]{\ref{chrg:eqn_Dg=0-onframes}}&
%    \bigl((\omega_{\sigma \mu})_c + (\omega_{\mu \sigma})_c \bigr)\wedge (\Omega_{\cdot \nu} ^{\sigma})_{ab}
%    + g_{\mu\sigma} \Bigl ( ( \omega _{\cdot \nu}^\pi )_c
%    \wedge (\Omega_{\cdot \pi} ^{\sigma})_{ab} - (\Omega_{\cdot \nu} ^{\pi})_{ca}\wedge
%    ( \omega _{\cdot \pi}^\sigma )_b  \Bigr) \\
%    =&(\omega_{\sigma \mu})_c\wedge (\Omega_{\cdot \nu} ^{\sigma})_{ab}
%     + (\omega_{\mu \sigma})_c \wedge (\Omega_{\cdot \nu} ^{\sigma})_{ab}
%     + ( \omega _{\cdot \nu}^\pi )_c \wedge (\Omega_{\mu \pi} )_{ab}
%     - (\Omega_{\cdot \nu} ^{\pi})_{ca}\wedge  ( \omega _{\mu \pi} )_b \\
%    =&(\omega_{\sigma \mu})_c\wedge (\Omega_{\cdot \nu} ^{\sigma})_{ab}
%    + ( \omega _{\cdot \nu}^\pi )_c \wedge (\Omega_{\mu \pi} )_{ab}
%    + (\omega_{\mu \sigma})_c \wedge (\Omega_{\cdot \nu} ^{\sigma})_{ab}
%    - (\omega _{\mu \pi})_{c}\wedge  ( \Omega_{\cdot \nu} ^{\pi} )_{ab}  \\
%    =&(\omega_{\sigma \mu})_c\wedge (\Omega_{\cdot \nu} ^{\sigma})_{ab}
%    + ( \omega _{\cdot \nu}^\sigma )_c \wedge (\Omega_{\mu \sigma} )_{ab}
%    = ( \omega _{\cdot \nu}^\sigma )_c \wedge (\Omega_{\mu \sigma} )_{ab}
%    + (\Omega_{\cdot \nu} ^{\sigma})_{ca} \wedge (\omega_{\sigma \mu})_{b}
%\end{align*}








\index[physwords]{欧几里得空间!局部}
\subsection{局部广义欧几里得空间}\label{chrg:sec_local-EuclideanSpace}

\index[physwords]{平直}
\index[physwords]{弯曲}

\begin{definition}\label{chrg:def_flat-curved}
    若广义黎曼流形$(M,g)$上由Levi-Civita联络决定的黎曼曲率恒为零,
    则称$M$为{\heiti 平直的}(或平坦的,flat);  否则称$M$为{\heiti 弯曲的}(curved).
\end{definition}

\index[physwords]{局部欧氏空间}

\begin{definition}\label{chrg:def_local-EuclideanSpace}
    设有$m$维广义黎曼流形$(M,g)$,若$\forall p \in M$都存在局部坐标系使得
    \begin{equation}\label{chrg:eqn_euclid-local}
        g_{ab}|_U = \eta_{\alpha\beta} ({\rm d}x^\alpha)_a ({\rm d}x^\beta)_b ,
        \quad     \text{在整个局部坐标系}\ (U;x^\alpha) \ \text{内成立}
    \end{equation}
    其中$\eta_{\alpha\beta}={\rm diag}(-1,\cdots,-1,+1,\cdots,+1)$,且$\pm 1$个数总和是$m$;
    则广义黎曼流形$(M,g)$称为{\heiti 局部广义欧几里得空间}.
\end{definition}
本节目的就是要证明下述定理\cite[\S 4.2]{chen-li-2023-2ed-v1}.
\begin{theorem}\label{chrg:thm_local-EuclideanSpace}
    设有$m$维广义黎曼流形$(M,g)$,其黎曼曲率张量为零的充分必要条件是$(M,g)$为局部广义欧几里得空间.
\end{theorem}

充分性:若$M$是局部广义欧氏空间,经计算可得其黎曼曲率恒为零.

下面证明必要性.因Levi-Civita联络无挠,故方程\eqref{chccr:eqn_CST}变为
\begin{subequations}\label{chrg:eqn_CST-Rieman-Conn}
    \begin{align}
        {\rm d}_a (e^\mu)_b &=(e^\nu)_a \wedge (\omega _{\cdot \nu}^\mu)_b , \label{chrg:eqn_CST-Rieman-Conn-I} \\
        {\rm d}_a ( \omega _{\cdot \nu} ^\mu  )_b  &=  ( \omega _{\cdot \nu}^\sigma )_a
        \wedge ( \omega _{\cdot \sigma}^\mu )_b + (\Omega_{\cdot \nu} ^{\mu})_{ab}. \label{chrg:eqn_CST-Rieman-Conn-II}
    \end{align}
\end{subequations}
上式的基矢量是$(e_\mu)^a$(称为旧的).
%其中式\eqref{chrg:eqn_CST-Rieman-Conn-III}已由式\eqref{chrg:eqn_omega-anti}证明;
%        (\omega _{\cdot \nu} ^\mu)_a &=- (\omega _{\cdot \mu} ^\nu)_a . \label{chrg:eqn_CST-Rieman-Conn-III}

现另选一套新基矢$(\tilde{e}_i)^a$,
在标架变换\eqref{chccr:base_change}下(即$(\tilde{e} _i)^a = (e_\mu)^{a} A_{\cdot i}^{\mu},\
(\tilde{e} ^j)_a = (e^\nu)_{a} B^{\cdot j}_{\nu}$,其中$A\cdot B^T=I$),
新的联络系数及联络一次微分型式场为式\eqref{chccr:eqn_connection-form-tilde}:
\begin{equation}
    (\tilde{e}_i)^b \nabla _b (\tilde{e}_j)^a  =
    \tilde{\Gamma}^{k}_{ji} (\tilde{e}_k)^a,   \qquad
    (\tilde{\omega} _{\cdot j}^{k})_a  \equiv
    \tilde{\Gamma}^{k}_{j i} (\tilde{e}^i)_a .
    \tag{\ref{chccr:eqn_connection-form-tilde}}
\end{equation}
新旧一次微分型式场基本变换关系为式\eqref{chccr:eqn_1form-transformation}:
\begin{equation}
    {\rm d}_a A_{\cdot j}^{\sigma} + (\omega ^{\sigma}_{\cdot\nu})_a  A_{\cdot j}^{\nu}
    =  A_{\cdot k}^{\sigma} (\tilde{\omega} _{\cdot j} ^{k})_a .
    \tag{\ref{chccr:eqn_1form-transformation}}
\end{equation}
我们的目的是在$p$点附近寻找一组适当的光滑函数$A_{\cdot k}^{\sigma}\in C^\infty(U)$使
得$\det(A_{\cdot k}^{\sigma})\neq 0$,并且在新标架场$(\tilde{e}_i)^a$下
\uwave{新一次联络型式场}$(\tilde{\omega} _{\cdot j} ^{k})_a $恒为零,也就是新的联络
系数$\tilde{\Gamma}^{k}_{j i}$恒为零;由上面的式子\eqref{chccr:eqn_connection-form-tilde}可
知$(\tilde{e}_i)^b \nabla _b (\tilde{e}_j)^a=0$,也就是新标架场$(\tilde{e}_i)^a$式平行标架场.
由上面的式\eqref{chccr:eqn_1form-transformation}可知,
这个问题转化为如下Pfaff方程(未知量是$A_{\cdot j}^{\sigma}$)的求解.
\begin{equation}\label{chrg:eqn_tmp109}
    (\pi _{\cdot j} ^{\sigma})_a \overset{def}{=}
    {\rm d}_a A_{\cdot j}^{\sigma} + (\omega ^{\sigma}_{\cdot\nu})_a  A_{\cdot j}^{\nu}=0,
    \quad \det(A_{\cdot k}^{\sigma})\neq 0,\quad 1 \leqslant j,k,\sigma,\nu \leqslant m .
\end{equation}
此Pfaff方程是由$m^2$个一次微分式组成的,它的秩是$m^2$.
我们对此式再次求外微分,并利用\uwave{黎曼曲率为零}的条件(即$(\Omega_{\cdot \nu} ^{\mu})_{ab}=0$)
\begin{align*}
    {\rm d}_b(\pi _{\cdot j} ^{\sigma})_a =&
       \bigl({\rm d}_b (\omega ^{\sigma}_{\cdot\nu})_a\bigr)  A_{\cdot j}^{\nu}
      + ({\rm d}_b  A_{\cdot j}^{\nu}) \wedge (\omega ^{\sigma}_{\cdot\nu})_a   \\
    \xlongequal[\ref{chrg:eqn_tmp109}]{\ref{chrg:eqn_CST-Rieman-Conn-II}}&
    ( \omega _{\cdot \nu}^\xi )_b \wedge ( \omega _{\cdot \xi}^{\sigma} )_a  A_{\cdot j}^{\nu}
    + \bigl((\pi _{\cdot j} ^{\nu})_b-(\omega ^{\nu}_{\cdot\xi})_b  A_{\cdot j}^{\xi}\bigr)
     \wedge (\omega ^{\sigma}_{\cdot\nu})_a
     =(\pi _{\cdot j} ^{\nu})_b \wedge (\omega ^{\sigma}_{\cdot\nu})_a  .
\end{align*}
于是Pfaff方程组\eqref{chrg:eqn_tmp109}满足Frobenius条件\eqref{chdf:eqn_frobenius-Diff-Form};
由Frobenius定理\ref{chdf:thm_frobenius-Diff-Form}可知此方程组完全可积.
Pfaff方程组\eqref{chrg:eqn_tmp109}没有给定初始条件,
我们将$p$点初始条件设为(下式包括正定、不定度规情形)
\begin{equation}\label{chrg:eqn_tmp109-init}
    A_{\cdot \alpha}^{i}(p)\, A_{\cdot \beta}^{j}(p) \, g_{ij} (p) = \eta_{\alpha\beta} .
    \quad \eta_{\alpha\beta}\text{表达式见定义\ref{chrg:def_local-EuclideanSpace}中描述}
\end{equation}
此时的$g_{ij}(x)$是一般的度规,非对角元可能非零.现在可假设满足Pfaff方程组\eqref{chrg:eqn_tmp109}和
其初始条件\eqref{chrg:eqn_tmp109-init}的解为$a_{\cdot \alpha}^{i}(x)$,令
\begin{equation}\label{chrg:eqn_tmp109-fij}
    f_{\alpha\beta}(x) \equiv a_{\cdot \alpha}^{i}(x) a_{\cdot \beta}^{j}(x) g_{ij} (x)-\eta_{\alpha\beta} ,
\end{equation}
那么在$p$点有$f_{\alpha\beta}(p)=0$(初始条件);对式\eqref{chrg:eqn_tmp109-fij}求外微分,
得(需注意,当$a_{\cdot \alpha}^{i}(x)$是解时,由式\eqref{chrg:eqn_tmp109}可知
${\rm d}_b a_{\cdot j}^{\sigma} =- (\omega ^{\sigma}_{\cdot\nu})_b  a_{\cdot j}^{\nu}$)
\begin{align*}
    {\rm d}_b f_{\alpha\beta}(x) =&  ({\rm d}_b a_{\cdot \alpha}^{i}) a_{\cdot \beta}^{j} g_{ij}
    + a_{\cdot \alpha}^{i} ({\rm d}_b a_{\cdot \beta}^{j}) g_{ij}
    + a_{\cdot \alpha}^{i} a_{\cdot \beta}^{j} {\rm d}_b g_{ij}  \\
    =&  -(\omega ^{i}_{\cdot k })_b a_{\cdot \alpha}^{k} a_{\cdot \beta}^{j} g_{ij}
    - a_{\cdot \alpha}^{i}(\omega ^{j}_{\cdot k})_b  a_{\cdot \beta}^{k} g_{ij}
    + a_{\cdot \alpha}^{i} a_{\cdot \beta}^{j} {\rm d}_b g_{ij}  \\
    =&  a_{\cdot \alpha}^{k} a_{\cdot \beta}^{j}  \left( -(\omega ^{i}_{\cdot k })_b g_{ij}
     - (\omega ^{l}_{\cdot j})_b  g_{kl} + {\rm d}_b g_{kj}  \right)
     \xlongequal{\ref{chrg:eqn_Dg=0-onframes}}  0 .
\end{align*}
既然在整个邻域$U$内都有${\rm d}_b f_{\alpha\beta}(x)=0$,那么必然可以
得出$f_{\alpha\beta}(x)$在整个邻域$U$内为常数,所以有$f_{\alpha\beta}(x)=f_{\alpha\beta}(p)=0$.
由此可见在邻域$U$内,有
\begin{equation}\label{chrg:eqn_gij=deltaij}
    a_{\cdot \alpha}^{i}(x) a_{\cdot \beta}^{j}(x) g_{ij} (x) 
    \xlongequal{\ref{chrg:eqn_tmp109-fij}} \eta_{\alpha\beta} ,
\end{equation}
而度规$g_{ab}$在新标架场$(\tilde{e}_i)^a$的分量为(要用上式)
\begin{equation}
    \tilde{g}_{ij}(x)=g_{bc} (\tilde{e}_i)^b(\tilde{e}_j)^c
    = g_{bc} (e_\mu)^{b} a_{\cdot i}^{\mu} (e_\nu)^{c} a_{\cdot j}^{\nu}
    = g_{\mu\nu}(x) a_{\cdot i}^{\mu}(x)  a_{\cdot j}^{\nu}(x) = \eta_{ij}.
\end{equation}
此式在整个邻域$U$上成立.

在上面论述过程中,已经指出新标架场$(\tilde{e}_i)^a$是平行的,即有$(\tilde{\omega} _{\cdot \nu}^\mu)_b$恒为零,
将其带入Cartan第一结构方程\eqref{chrg:eqn_CST-Rieman-Conn-I},可以得到,
\begin{equation}
    {\rm d}_a (\tilde{e}^\mu)_b =(\tilde{e}^\nu)_a \wedge (\tilde{\omega} _{\cdot \nu}^\mu)_b =0 .
\end{equation}
也就是每一个$(\tilde{e}^\mu)_b$都是闭的,由Poincar\'{e}引理\ref{chdf:thm_poincare-lemma}可知
存在$p\in M$的局部坐标邻域$U$以及其上的光滑函数族$\{x^\mu\}$使得
$ (\tilde{e}^\mu)_a = {\rm d}_a x^\mu = ({\rm d} x^\mu)_a, \ (1\leqslant \mu \leqslant m) $;
这说明在$U$内存在自然标架场$\{(\frac{\partial }{\partial x^\mu})^a\}$等于$(\tilde{e}_\mu)^{a}$.
故定理\ref{chrg:thm_local-EuclideanSpace}得证.\qed



\index[physwords]{黎曼曲率!型式!刚性标架}
\subsection{刚性标架}\label{chrg:sec_rigid-frame}
标架场$\{(e^\mu)_{a}\}$下的度规分量$g_{\mu\nu}$是流形上的标量场,一般不是常数.
\begin{definition}\label{chrg:def_rigid-frame}
    若标架场$\{(e^\mu)_{a}\}$下度规分量$g_{\mu\nu}$是常数($\nabla_a g_{\mu\nu}={\rm d}_a g_{\mu\nu}=0$),
    则我们称标架场$\{(e^\mu)_{a}\}$为{\heiti 刚性标架}(rigid frame).
\end{definition}
注意,对于一般标架场,虽然度规和联络
相容,即$\nabla_a g_{bc}\equiv0$;但$\nabla_a g_{\mu\nu}$未必等于零.
对于刚性标架可以得到联络型式场$(\omega _{\mu\nu})_{a}$对指标反对称性:
\begin{equation}\label{chrg:eqn_omega-anti}
    {\rm d}_a g_{\mu\nu}=0 \quad \xRightarrow{\ref{chrg:eqn_Dg=0-onframes}}\quad
    (\omega _{\mu\nu})_{a} = -(\omega _{\nu\mu})_{a} .
\end{equation}
对于非刚性标架,因$\nabla_a g_{\mu\nu}\neq 0$,故$(\omega _{\mu\nu})_{a}$不具有上述的反对称性.
%\begin{equation}
%    \begin{aligned}
%        (\omega _{\mu\nu})_{a} & = (e_\mu)_c \nabla _a (e_\nu)^c
%        = \nabla _a \bigl((e_\mu)^c (e_\nu)_c\bigr)
%        - (e_\nu)^c \nabla _a (e_\mu)_c  \\
%        & =\nabla _a(g_{\mu\nu}) -(e_\nu)^c \nabla _a (e_\mu)_c
%        =-(e_\nu)_c \nabla _a (e_\mu)^c =-(\omega _{\nu\mu})_{a}.
%    \end{aligned}
%\end{equation}


定义{\heiti Ricci旋转系数}(Ricci rotation coefficient)为(类似第一类克氏符)
\begin{equation}\label{chrg:eqn_ricci0}
    \omega_{\mu\nu\rho} \overset{def}{=} (\omega_{\mu\nu})_a (e_\rho)^{a}
    =g_{\mu\sigma}\Gamma^{\sigma}_{\nu\rho}
    {\quad \color{red} \Leftrightarrow \quad}
    (\omega_{\mu\nu})_a  = \omega_{\mu\nu\rho} (e^\rho)_{a} .
\end{equation}
将联络系数公式\eqref{chccr:eqn_concoef}中的联络系数换成Ricci旋转系数
\begin{align}
    (e_\rho)^b \nabla _b (e_\mu)^a & = \Gamma^{\sigma}_{\mu\rho} (e_\sigma)^a
    = g^{\nu\sigma}\omega_{\nu\mu\rho} (e_\sigma)^a =-\omega_{\mu\nu\rho} (e^\nu)^a
    {\quad \color{red} \Leftrightarrow }  \label{chrg:eqn_ricci1} \\
    \nabla _b (e_\mu)^a & = -\omega_{\mu\nu\rho} (e^\rho)_b (e^\nu)^a
    {\ \color{red} \Leftrightarrow \ }
    \nabla _b (e_\mu)_a   = -\omega_{\mu\nu\rho}(e^\rho)_b (e^\nu)_a
    {\ \color{red} \Leftrightarrow }  \label{chrg:eqn_ricci3}  \\
    \nabla _b (e^\mu)_a & = -g^{\mu\sigma}\omega_{\sigma\nu\rho}(e^\rho)_b (e^\nu)_a
    {\quad \color{red} \Leftrightarrow }  \label{chrg:eqn_ricci4}  \\
    \omega_{\mu\nu\rho} & = -(e_\nu)_a (e_\rho)^b \nabla _b (e_\mu)^a
    = (e_\mu)^a  (e_\rho)^b \nabla _b (e_\nu)_a . \label{chrg:eqn_ricci5}
\end{align}
只有在刚性标架下,Ricci旋转系数才能简便计算,内外指标才能方便升降.


%虽然自然坐标基底的无挠联络系数$\oversetmy{x}{\Gamma}_{ij}^k$($x$表示自然坐标基底)关于两个下标具有对称性,
%但一般的基底场下的联络系数$\Gamma_{ij}^k$通常情况下没有这种对称性;
由式\eqref{chccr:eqn_XYcommutator}及\eqref{chrg:eqn_ricci1}可得如下对易关系:
\begin{align}
    \left[ {X,Y} \right]^a =&\left(e_\nu (Y^\sigma)X^\nu  -  e_\nu (X^\sigma) Y^\nu
    + X^\mu Y^\nu ( \Gamma^{\sigma}_{\nu\mu} - \Gamma^{\sigma}_{\mu\nu}) \right)(e_\sigma)^a .
      \label{chrg:eqn_XYcommutator-Ebase} \\
    \bigl[(e_\mu), (e_\nu)\bigr]^a
    =& \bigl(\Gamma^{\sigma}_{\nu\mu} - \Gamma^{\sigma}_{\mu\nu}\bigr) (e_\sigma)^a
    = \bigl(-\omega_{\nu\sigma\mu} + \omega_{\mu\sigma\nu}\bigr) (e^\sigma)^a .
      \label{chrg:eqn_EmuEnucommutator}
\end{align}
式\eqref{chrg:eqn_EmuEnucommutator}第一个等号适用于一般标架,第二个等号只适用于刚性标架. %(定义见\ref{chrg:def_rigid-frame})

%\begin{align*}
%    \left[ {X,Y} \right]^a =& X^b\nabla_b Y^a - Y^b\nabla_b X^a \\
%    =& X^b \left[e_j (Y^i) + Y^l \Gamma^{i}_{lj} \right] (e^j)_b (e_i)^a
%     - Y^b \left[e_j (X^i) + X^l \Gamma^{i}_{lj} \right] (e^j)_b (e_i)^a   \\
%    =& X^j \left[e_j (Y^i) + Y^l \Gamma^{i}_{lj} \right] (e_i)^a
%     - Y^j \left[e_j (X^i) + X^l \Gamma^{i}_{lj} \right] (e_i)^a  \\
%    =&  e_j (Y^i)X^j(e_i)^a  -  e_j (X^i) Y^j(e_i)^a
%     + X^l Y^j ( \Gamma^{i}_{jl} - \Gamma^{i}_{lj}) (e_i)^a      \\
%    =&  \left(e_\nu (Y^\sigma)X^\nu  -  e_\nu (X^\sigma) Y^\nu
%     + X^\mu Y^\nu ( \Gamma^{\sigma}_{\nu\mu} - \Gamma^{\sigma}_{\mu\nu}) \right)(e_\sigma)^a
%\end{align*}

%\begin{align}
%    \nabla_a u^c &= \left[e_j (u^\alpha) + u^l \Gamma^{i}_{lj} \right] (e^j)_a (e_i)^c
%    =  \left[{\rm d}_a (u^i) + u^l (\omega_{\cdot l}^{i})_a \right] (e_i)^c .   \\
%    {e} _i (f)&= A_{\cdot i}^{j}(x) \frac{\partial f}{\partial x^j}, \\
%    {\rm d}_a (f) &=   [{e} _\nu (f)] (e^\nu)_a .
%\end{align}




\subsection{曲率张量的计算}
    广义黎曼流形中,挠率为零,
    曲率张量计算可分三个步骤\cite[\S 1.7]{chandrasekhar-1983}:
    \textcircled{1}选一{\kaishu 刚性}标架,
    \textcircled{2}计算Ricci旋转系数,
    \textcircled{3}利用Cartan第二结构方程计算曲率张量.

\subsubsection{计算Ricci旋转系数}
     需要先选定一局部坐标系$\{U;x^\mu\}$
    {\footnote{读者需要认识到,微分流形局部上同胚于$\mathbb{R}^m$开子集,所以任何流形
            自然存在局部坐标系.}},
刚性标架基矢场$(e_\mu)_{a}$、$(e_\mu)^{a}$在自然坐标
基矢场$(\frac{\partial}{\partial x^k})^{a}$、$({\rm d}x^k)_{a}$下的分量是
\begin{equation}\label{chrg:eqn_component-base}
(e_\nu)_{k} \equiv (e_\nu)_{a} \left(\frac{\partial}{\partial x^k}\right)^{a} , \quad
(e_\mu)^{k} \equiv (e_\mu)^{a} ({\rm d}x^k)_{a}.
\end{equation}
计算Ricci旋转系数并不需要基矢场$(e^\nu)_{a}$,所以上式中未给出其分量表达式.
从Ricci旋转系数定义可得
\begin{align*}
  \omega_{\mu\nu\rho} &= (e_\mu)^a  (e_\rho)^b \nabla _b (e_\nu)_a
  = (e_\mu)^{k} \left(\frac{\partial}{\partial x^k}\right)^{a}
   (e_\rho)^{l} \left(\frac{\partial}{\partial x^l}\right)^{b}
    \nabla _b \bigl[ (e_\nu)_{n}({\rm d}x^n)_{a} \bigr]  \\
  &=(e_\mu)^{k}(e_\rho)^{l} (e_\nu)_{k,l}
   -(e_\mu)^{k}(e_\rho)^{l}\Gamma^n _{kl}(e_\nu)_{n} .
\end{align*}
交换指标$\rho$和$\mu$,有
$ \omega_{\rho\nu\mu} =(e_\rho)^{k}(e_\mu)^{l} (e_\nu)_{k,l}
-(e_\rho)^{k}(e_\mu)^{l}\Gamma^n _{kl}(e_\nu)_{n}$;
将这两式相减,并利用坐标基底克氏符$\Gamma^n _{kl}$关于下标的对称性,得
\begin{align}
  \omega_{\mu\nu\rho}  -  \omega_{\rho\nu\mu}=
  \left[ (e_\mu)^{k}(e_\rho)^{l}
     - (e_\rho)^{k}(e_\mu)^{l}  \right]
     \frac{\partial (e_\nu)_{k}}{\partial x^l}
  \ \overset{def}{=} \Lambda_{\mu\nu\rho}. \label{chrg:eqn_Coef-Lambda1}
\end{align}
上式最后一步定义了一个中间步符号$\Lambda_{\mu\nu\rho}$,它还可以表示为
\begin{align}
    \Lambda_{\mu\nu\rho} \equiv \sum_{k,l }
    \left[ \frac{\partial (e_\nu)_{k}}{\partial x^l} -\frac{\partial (e_\nu)_{l}}{\partial x^k} \right]
    (e_\mu)^{k} (e_\rho)^{l} .  \label{chrg:eqn_Coef-Lambda2}
\end{align}
易见$\Lambda$具有反对称性$\Lambda_{\mu\nu\rho}=-\Lambda_{\rho\nu\mu}$.
由式\eqref{chrg:eqn_Coef-Lambda1}出发,进行指标轮换后相加减,
不难证明Ricci旋转系数可由这个中间步符号$\Lambda_{\mu\nu\rho}$表示为
\begin{equation}\label{chrg:eqn_Ricci-Coef}
\omega_{\mu\nu\rho} = \frac{1}{2}\left(\Lambda_{\mu\nu\rho}
+ \Lambda_{\rho\mu\nu} - \Lambda_{\nu\rho\mu} \right) .
\end{equation}
这样,先选定局部自然坐标基底场$(\frac{\partial}{\partial x^k})^{a}$,
再选择一个刚性标架场$(e_\mu)^a$,
通过式\eqref{chrg:eqn_Coef-Lambda1}或者\eqref{chrg:eqn_Coef-Lambda2}算出
中间步符号$\Lambda_{\mu\nu\rho}$,利用上式便可计算$\omega_{\mu\nu\rho}$了.

\begin{remark}\label{chrg:remark_ricci-coef-num}
    注解\ref{chccr:remark_connection-coef-num}指出
    无挠联络系数$\Gamma^{i}_{jk}$有$\frac{1}{2}m^2(m+1)$个.
    因Ricci旋转系数$\omega_{\mu\nu\rho}$关于前两个指标反对称,
    故共有$\frac{1}{2}m^2(m-1)$个;比联络系数少$m^2$个.
\end{remark}


\subsubsection{黎曼张量的分量表示}\label{chrg:sec_Riemann_rr}
现在可以导出Cartan第二结构方程的具体表达式.由于这里采用的是{\kaishu 刚性}标架,
度规分量为常数,所以式\eqref{chccr:eqn_Cartan-Structure-II-comp}可重写为
\setlength{\mathindent}{0em}
\begin{align*}
    &R_{\rho\sigma\mu\nu} = g_{\rho\pi} \Gamma _{\sigma \nu}^\tau \Gamma _{\tau\mu}^\pi
      - g_{\rho\pi}\Gamma _{\sigma \mu}^\tau \Gamma_{\tau\nu}^\pi +
      \bigl[{\rm d}_a (\omega_{\rho \sigma} )_b \bigr] (e_\mu)^a (e_\nu)^b     \\
     &\xlongequal{\ref{chrg:eqn_ricci0}} 
      (e_\mu)^a (e_\nu)^b \nabla_a\bigl(\omega_{\rho\sigma\pi} (e^\pi)_{b}\bigr) -
      (e_\mu)^a (e_\nu)^b \nabla_b \bigl(\omega_{\rho\sigma\pi} (e^\pi)_{a}  \bigr)
      + g^{\xi\zeta} \omega _{\zeta\sigma\nu} \omega _{\rho\xi\mu}
      -g^{\xi\zeta} \omega _{\zeta\sigma\mu} \omega _{\rho\xi\nu}  \\
     &\xlongequal{\ref{chrg:eqn_ricci4}} 
       g^{\xi\zeta}\omega_{\nu\xi\mu}\omega_{\rho\sigma\zeta}
      +(e_\mu)^a \nabla_a\bigl(\omega_{\rho\sigma\nu} \bigr)
      - g^{\xi\zeta}\omega_{\mu\xi\nu}\omega_{\rho\sigma\zeta}
      -(e_\nu)^b \nabla_b \bigl(\omega_{\rho\sigma\mu} \bigr)   \\
    &{\qquad} + g^{\xi\zeta} (\omega _{\zeta\sigma\nu} \omega _{\rho\xi\mu}
      - \omega _{\zeta\sigma\mu} \omega _{\rho\xi\nu}) .
\end{align*}\setlength{\mathindent}{2em}
利用式\eqref{chccr:eqn_natural-base_change}可得\uwave{刚性标架}$(e_\mu)^a$下黎曼曲率分量表达式
\begin{equation}\label{chrg:eqn_Riemann_rr}
\begin{aligned}
   R_{\rho\sigma\mu\nu} =&
    C_{\hphantom{\pi} \mu}^{\pi}(x) \frac{\partial \omega_{\rho\sigma\nu}}{\partial x^\pi}
   -C_{\hphantom{\pi} \nu}^{\pi}(x) \frac{\partial \omega_{\rho\sigma\mu}}{\partial x^\pi} \\
   & + g^{\xi\zeta} \bigl[ \omega_{\rho\sigma\zeta} (\omega_{\nu\xi\mu}
   -\omega_{\mu\xi\nu}  ) + \omega _{\zeta\sigma\nu} \omega _{\rho\xi\mu}
   - \omega _{\zeta\sigma\mu} \omega _{\rho\xi\nu} \bigr] .
\end{aligned}
\end{equation}
可以看出在{\kaishu 刚性}标架场$(e_\mu)^{a}$中计算黎曼张量,
不需要计算自然坐标基矢场的克氏符$\Gamma^n _{kl}$.
标架法中的Ricci旋转系数$\omega _{\mu\zeta\sigma}$与自然坐标基底法
中的克氏符$\Gamma^n _{kl}$地位相当,但Ricci旋转系数计算量较小.


表面上来看,式\eqref{chrg:eqn_Riemann_rr}不具有对称性$R_{\rho\sigma\mu\nu}=R_{\mu\nu\rho\sigma}$;
然而,我们可以证明一般标架场下的黎曼曲率分量也具有这种对称性:
\begin{align*}
    R_{\rho\sigma\mu\nu}&= R_{abcd}(e_\rho)^a(e_\sigma)^b(e_\mu)^c(e_\nu)^d
    \xlongequal[\text{指标记号}]{\text{交换抽象}}
    R_{cdab}(e_\rho)^c(e_\sigma)^d(e_\mu)^a(e_\nu)^b \\
    &\xlongequal{\ref{chrg:eqn_Riemann-sym-abcd=cdab}}
    R_{abcd}(e_\rho)^c(e_\sigma)^d(e_\mu)^a(e_\nu)^b =
    R_{\mu\nu\rho\sigma}.
\end{align*}


\subsubsection{例题}
在这里给出几个例题,以便读者更好地理解标架理论.
\begin{example}
计算度规${\rm d}s^2= \Omega^{2}(t,x) \bigl(-{\rm d}t^2 + {\rm d}x^2\bigr) $的黎曼曲率.
\end{example}

\noindent \fbox{甲}
由于$\Omega^{2}(t,x)$是自然坐标$\{t,x\}$的函数,所以
自然坐标基矢不是刚性标架,需另选基矢.我们选成如下正交归一的切基矢及对偶基矢
\begin{align}
(e^0)_{a} & = \Omega ({\rm d}t)_{a},    & (e^1)_{a} & = \Omega ({\rm d}x)_{a}.    \\
(e_0)^{a} & = \Omega^{-1} (\partial_t)^{a}, & (e_1)^{a} & = \Omega^{-1}(\partial_x)^{a}.
\end{align}
由此容易写出度规张量
\begin{align}
g_{ab}  =  - \Omega^2 ({\rm d}t)_{a} ({\rm d}t)_{b}
+ \Omega^{2} ({\rm d}x)_{a} ({\rm d}x)_{b}
=    - (e^0)_{a}(e^0)_{b} + (e^1)_{a}(e^1)_{b}
\end{align}
显然,在新选的标架场$(e_\mu)^a$下,度规系数为常数,此标架是刚性的.
用上式容易得到将指标升降后的基矢表达式
\begin{align}
(e_0)_{a} =  -\Omega ({\rm d}t)_{a}, \qquad
(e_1)_{a} =  \Omega ({\rm d}x)_{a}.
\end{align}
由式\eqref{chrg:eqn_component-base}可得全部坐标分量
\begin{align}
(e_0)_{0} &= -\Omega, &(e_0)_{1} &= 0; & (e_1)_{0} &=0, & (e_1)_{1} &= \Omega.   \\
(e_0)^{0} &= \Omega^{-1}, &(e_0)^{1} &= 0; &(e_1)^{0} &=0,  & (e_1)^{1} &= \Omega^{-1}.
\end{align}
协变基矢导数不为零的分量为,“点”代表对$t$求导,“撇”代表对$x$求导.
\begin{align}
(e_0)_{0,0} = -\dot{\Omega}, \quad (e_0)_{0,1} = -\Omega' , \quad
(e_1)_{1,0} = \dot{\Omega}, \quad (e_1)_{1,1} = \Omega' .
\end{align}


\noindent \fbox{乙}
由式\eqref{chrg:eqn_Coef-Lambda1}或\eqref{chrg:eqn_Coef-Lambda2}计算
中间步符号$\Lambda$,全部不为零的式子为
\begin{align}
\Lambda_{001 } &= -\Lambda_{100}
= \bigl( (e_0)_{0,1}-(e_0)_{1,0} \bigr) (e_0)^{0} (e_1)^{1}
=  -\Omega^{-2} \Omega' , \\
\Lambda_{011 } &= -\Lambda_{110}
= \bigl( (e_1)_{0,1}-(e_1)_{1,0} \bigr) (e_0)^{0} (e_1)^{1}
=  -\Omega^{-2} \dot{\Omega} .
\end{align}
由式\eqref{chrg:eqn_Ricci-Coef}可得两个
独立的Ricci旋转系数$\omega_{\mu\nu\rho}$,分别是
\begin{align}
\omega_{010} &= \frac{1}{2}\left(\Lambda_{010} + \Lambda_{001} - \Lambda_{100} \right)
= \Lambda_{001} = -\Omega^{-2} \Omega' = -\omega_{100} \\
\omega_{011} &= \frac{1}{2}\left(\Lambda_{011} + \Lambda_{101} - \Lambda_{110} \right)
= \Lambda_{011} = -\Omega^{-2} \dot{\Omega}  = -\omega_{101}
\end{align}

\noindent \fbox{丙}
由\eqref{chrg:eqn_Riemann_rr}式可得唯一独立的黎曼分量
\begin{align*}
R_{0101} &= (e_0)^a\nabla_a \omega_{011} - (e_1)^a\nabla_a \omega_{010}  \\
&{\quad} + g^{\xi\zeta} \left(
 \omega_{01\zeta} ( \omega_{1\xi 0} - \omega_{0\xi 1} )
+ \omega _{\zeta 11} \omega _{0\xi 0} -\omega _{\zeta 10} \omega _{0\xi 1} \right)   \\
&= \Omega^{-1}\partial_t \left(-\Omega^{-2} \dot{\Omega} \right)
  -\Omega^{-1}\partial_x \left(-\Omega^{-2} \Omega' \right)  \\
&{\quad} + g^{00} ( \omega_{010} (\omega_{100} - \omega_{001} )
+ \omega _{011} \omega _{000} -\omega _{010} \omega _{001} )   \\
&{\quad} + g^{11} ( \omega_{011} ( \omega_{110} - \omega_{011} )
+ \omega _{111} \omega _{010} -\omega _{110} \omega _{011} )   \\
&= -\Omega^{-3} \ddot{\Omega} +2 \Omega^{-4}  \bigl(\dot{\Omega}\bigr)^2
+\Omega^{-3} \Omega'' -2 \Omega^{-4}  (\Omega')^2
 + \left( \Omega^{-2} \Omega' \right)^2
- \left( \Omega^{-2} \dot{\Omega} \right)^2     \\
&= \Omega^{-3} \bigl(\Omega'' -\ddot{\Omega} \bigr)
   +\Omega^{-4} \bigl(\dot{\Omega}^2 - \Omega'^2 \bigr)
\end{align*}
这是在基矢$\{(e^\mu)_{a}\}$下的黎曼曲率分量.
计算过程中注意$g^{00}=-1,g^{11}=+1$,其它度规分量为零;
全部系数为常数是刚性标架的体现.
\qed


\begin{example}\label{chrg:exm_Kruskal}
给定Kruskal度规
\begin{equation}
{\rm d}s^2= -4 e^{2A} {\rm d}u {\rm d}v + Z^{2}({\rm d}\theta^2 + (\sin\theta \, {\rm d}\phi)^2)
\end{equation}
局部坐标是$\{u,v,\theta,\phi \}$,系数$A,Z$只是$u,v$的函数.
\end{example}
\noindent \fbox{甲}
局部坐标基矢不是刚性标架,另选如下正交归一协变基矢
\begin{equation}
\begin{aligned}
  (e^1)_{a} & = \sqrt{2}e^A ({\rm d}u)_{a},
& (e^2)_{a} & = \sqrt{2}e^A ({\rm d}v)_{a},    \\
  (e^3)_{a} & = Z ({\rm d}\theta)_{a},
& (e^4)_{a} & = Z \sin\theta ({\rm d}\phi)_{a}.
\end{aligned}
\end{equation}
由此容易写出度规张量
\begin{align}
  g_{ab}  = - (e^1)_{a}(e^2)_{b} - (e^2)_{a}(e^1)_{b}
    + (e^3)_{a}(e^3)_{b}  + (e^4)_{a}(e^4)_{b}
\end{align}
显然,在新选的标架场$(e^\mu)_a$下,度规系数为常数,
即非零分量是$g_{12}=g_{21}=-1,\ g_{33}=g_{44}=1$;此标架是刚性的.
用上式容易得到逆变基矢
\begin{equation}
\begin{aligned}
(e_1)^{a} & = \frac{e^{-A}}{\sqrt{2}} (\partial_u)^{a},
   & (e_2)^{a} & = \frac{e^{-A}}{\sqrt{2}} (\partial_v)^{a},    \\
(e_3)^{a} & = Z^{-1} (\partial_\theta)^{a},
& (e_4)^{a} & = (Z \sin\theta)^{-1} (\partial_\phi)^{a}.
\end{aligned}
\end{equation}
将外指标降下来,有
\begin{equation}
\begin{aligned}
  (e_1)_{a}&=g_{ab} (e_1)^{b}= -(e^2)_{a} =-\sqrt{2}e^A ({\rm d}v)_{a} \\
  (e_2)_{a}&=g_{ab} (e_2)^{b}= -(e^1)_{a} =-\sqrt{2}e^A ({\rm d}u)_{a} \\
  (e_3)_{a}&=g_{ab} (e_3)^{b}= +(e^3)_{a} =+Z ({\rm d}\theta)_{a} \\
  (e_4)_{a}&=g_{ab} (e_4)^{b}= +(e^4)_{a} =+Z \sin\theta ({\rm d}\phi)_{a}
\end{aligned}
\end{equation}
\noindent \fbox{乙}
矢量$(e_\mu)_{a}$在自然坐标基矢$({\rm d}x^j)_{a}$上的非零分量为
\begin{equation}
  (e_1)_{v}=-\sqrt{2}e^A, \quad
  (e_2)_{u}=-\sqrt{2}e^A, \quad
  (e_3)_{\theta}=Z, \quad
  (e_4)_{\phi}=Z \sin\theta
\end{equation}
刚性标架基矢$(e_\mu)^{a}$在自然坐标基矢$(\partial _j)^{a}$上的非零分量为
\begin{equation}
  (e_1)^{u}=\frac{e^{-A}}{\sqrt{2}}, \quad
  (e_2)^{v}=\frac{e^{-A}}{\sqrt{2}}, \quad
  (e_3)^{\theta}=Z^{-1}, \quad
  (e_4)^{\phi}=(Z\sin\theta)^{-1}
\end{equation}
基矢非零分量$(e_\mu)_{k}$的非零导数是
\begin{equation}
\begin{aligned}
  (e_1)_{v,u}&=-\sqrt{2}e^A \partial_u A,  & (e_1)_{v,v}&=-\sqrt{2}e^A \partial_v A  \\
  (e_2)_{u,u}&=-\sqrt{2} e^A \partial_u A, & (e_2)_{u,v}&=-\sqrt{2} e^A \partial_v A  \\
  (e_3)_{\theta,u}&= \partial_u Z, & (e_3)_{\theta,v}&= \partial_v Z \\
  (e_4)_{\phi,u}&= \sin\theta \, \partial_u Z, &
  (e_4)_{\phi,v}&= \sin\theta \, \partial_v Z, &
  (e_4)_{\phi,\theta}=Z \cos\theta
\end{aligned}
\end{equation}
利用式\eqref{chrg:eqn_Coef-Lambda1}或\eqref{chrg:eqn_Coef-Lambda2},
可得独立非零中间步符号
\begin{equation}
\begin{aligned}
  \Lambda_{112}&= \frac{e^{-A}}{\sqrt{2}} \partial_u A, &
  \Lambda_{221}&= \frac{e^{-A}}{\sqrt{2}} \partial_v A, \\
  \Lambda_{331}&= \frac{e^{-A}}{\sqrt{2}\, Z} \partial_u Z, &
  \Lambda_{332}&= \frac{e^{-A}}{\sqrt{2}\, Z} \partial_v Z, \\
  \Lambda_{441}&= \frac{e^{-A}}{\sqrt{2}\, Z} \partial_u Z, &
  \Lambda_{442}&= \frac{e^{-A}}{\sqrt{2}\, Z} \partial_v Z, &
  \Lambda_{443} = \frac{\cot\theta}{Z}
\end{aligned}
\end{equation}
进而利用式\eqref{chrg:eqn_Ricci-Coef},可得独立非零Ricci旋转系数
\begin{equation}
\begin{aligned}
  \omega_{121} &= \frac{e^{-A}}{\sqrt{2}}  \partial_u A, &
  \omega_{212} &= \frac{e^{-A}}{\sqrt{2}}  \partial_v A, \\
  \omega_{313} &= \frac{e^{-A}}{\sqrt{2}\, Z}  \partial_u Z, &
  \omega_{323} &= \frac{e^{-A}}{\sqrt{2}\, Z}  \partial_v Z, \\
  \omega_{414} &= \frac{e^{-A}}{\sqrt{2}\, Z}  \partial_u Z, &
  \omega_{424} &= \frac{e^{-A}}{\sqrt{2}\, Z}  \partial_v Z, &
  \omega_{434}  = \frac{\cot\theta}{Z}
\end{aligned}
\end{equation}

\noindent \fbox{丙}
由\eqref{chrg:eqn_Riemann_rr}式可得在基矢$(e^\mu)_a$下
独立非零黎曼张量分量(注意求和过程中应用刚性标架的度规系数,见\fbox{甲}部分)
\begin{equation}
\begin{aligned}
  R_{1221} =& -e^{-2 A} \partial_u\partial_v A, \qquad
  R_{1332} = R_{1442} = \frac{e^{-2A} \partial_u \partial_v Z }{{2}\, Z}, \\
  R_{1313} =& R_{1414}= \frac{e^{-2A}}{{2}\, Z} (2\, \partial_u A \,
    \partial_u Z  - \partial_u \partial_u Z ),  \\
  R_{2323} =& R_{2424} = \frac{e^{-2A}}{{2}\, Z}
    (2\, \partial_v A \, \partial_v Z - \partial_v \partial_v Z ),  \\
  R_{3434} =& Z^{-2} (1+ e^{-2A} \partial_u Z \,\partial_v Z )
\end{aligned}
\end{equation}
进而易得在基矢$(e^\mu)_a$下独立非零Ricci张量分量
\begin{equation}
\begin{aligned}
  R_{11} &= e^{-2A} Z^{-1} ( 2\,\partial_u A \,\partial_u Z - \partial_u \partial_u Z)  \\
  R_{12} &=-e^{-2A} Z^{-1} ( Z\partial_u\partial_v A + \partial_u \partial_v Z )  \\
  R_{22} &= e^{-2A} Z^{-1} ( 2\,\partial_v A \,\partial_v Z - \partial_v \partial_v Z)  \\
  R_{33} &= R_{44}=e^{-2A}Z^{-2}(e^{2A}+\partial_u Z \,\partial_v Z + Z\partial_u \partial_v Z)
\end{aligned}
\end{equation}
需要注意,以上给出的分量是在$(e^\mu)_a$下的分量,不是在$({\rm d}x^\mu)_a$下的分量.
\qed

\begin{example}\label{chrg:exm_SWPlane}
	平面引力波(Sachs--Wu).
\end{example}

设有局部坐标系$\{t, x, y, z\}$;
令 $u \equiv t-z$,$f(u)$ 和 $g(u)$ 是 $u$ 的两个任意的光滑函数,并且 $f^2+g^2$ 不恒为零.
设 $F$ 是坐标 $x, y$ 和 $u$ 的如下函数:
\begin{equation}\label{chrg:eqn_SWP-F}
	F(x, y, u)=\frac{1}{2} f(u)\left(x^2-y^2\right)+g(u) x y.
\end{equation}
度规为
\begin{equation}\label{chrg:eqn_SWP-g}
	{\rm d}s^2= {\rm d}x^2 + {\rm d}y^2
	+(2F-1) {\rm d}t^2  + (2F+1){\rm d}z^2
	- 2F {\rm d}t{\rm d}z - 2F {\rm d}z{\rm d}t.
\end{equation}
很明显,式\eqref{chrg:eqn_SWP-g}是对称的.
我们先验证它是Lorentz型的.取
\begin{equation}\label{chrg:eqn_SWP-g-frame}
	\begin{aligned}
	(e^1)_a =& ({\rm d}x)_a,\quad (e^2)_a = ({\rm d}y)_a,\quad 
	(e^4)_a = ({\rm d}t)_a -({\rm d}z)_a, \\
	(e^3)_a =& \left(\frac{1}{2}-F(x, y, u)\right)({\rm d}t)_a 
	+\left(\frac{1}{2}+F(x, y, u)\right)({\rm d}z)_a .
	\end{aligned}
\end{equation}
则在上述基矢$\{(e^1)_a,(e^2)_a,(e^3)_a,(e^4)_a\}$下,式\eqref{chrg:eqn_SWP-g}的分量为
\begin{equation}
	g_{\mu \nu}=\begin{pmatrix}
		1 & 0 & 0 & 0 \\
		0 & 1 & 0 & 0 \\
		0 & 0 & 0 & -1 \\
		0 & 0 & -1 & 0
	\end{pmatrix} \quad  = g^{\mu \nu} .
\end{equation}
上式中$(g_{\mu \nu})$的逆矩阵等于自身.
容易看出它是Lorentz型的,并且非退化.
上式同时表明它是刚性标架.
容易得到式\eqref{chrg:eqn_SWP-g-frame}的对偶基矢:
\begin{equation}
	\begin{aligned}
		(e_1)^{a} =& (\partial_x)^{a},\quad
		(e_2)^{a} = (\partial_y)^{a},    \quad
		(e_3)^{a} = (\partial_t)^{a}+(\partial_z)^{a}, \\
		(e_4)^{a} =& \left(\frac{1}{2}+F(x, y, u)\right) (\partial_t)^{a}
		+\left(-\frac{1}{2}+F(x, y, u)\right)(\partial_z)^{a}.
	\end{aligned}
\end{equation}
将外指标降下来,有
\begin{equation}
	\begin{aligned}
		(e_1)_{a}=& ({\rm d}x)_{a}, \quad		(e_2)_{a}=({\rm d}y)_{a},\quad 
		(e_3)_{a}=-({\rm d}t)_{a} +({\rm d}z)_{a}, \\
		(e_4)_{a}=&\left(F(x, y, u)-\frac{1}{2}\right) ({\rm d}t)_{a}
		- \left(F(x, y, u)+\frac{1}{2}\right) ({\rm d}z)_{a} .
	\end{aligned}
\end{equation}
直接计算得非零量为(注意$\partial_t F(x,y,t-z)=-\partial_z F(x,y,t-z)$):
\begin{align*}
&\Lambda_{144} = -\partial_x F,\quad
\Lambda_{244} = -\partial_y F.\quad
\omega_{144} = -\partial_x F,\quad
\omega_{244} = -\partial_y F. \\
&R_{1414} = -\partial^{2}_{xx} F,\quad
R_{1424} = -\partial^2_{xy}F,\quad 
R_{2424} = -\partial^2_{yy}F. \quad
R_{44} = -\partial^2_{xx}F -\partial^2_{yy}F .
\end{align*}
将$F$的表达式\eqref{chrg:eqn_SWP-F}带入,便可给出基矢$\{(e^\mu)_a\}$下的黎曼曲率非零分量:
\begin{align*}
	R_{1441}= f(u) = R_{2424},\quad R_{1442} = g(u) .
	\qquad R_{\mu\nu}\equiv 0 .
\end{align*}
由于我们要求$f^2+g^2 \neq 0$,故可知黎曼曲率分量不全为零;故此时空为弯曲的.
进而,可得Ricci曲率所有分量都为零(符合真空爱氏场方程)!
\qed

\begin{example}\label{chrg:exm_BondiPlane}
	平面引力波(Bondi--Ehlers--Kundt).
\end{example}

设有局部坐标系$\{u,v, x, y\}$.
$L(u)$、$\beta(u)$是$u$ 的两个任意的光滑函数.
我们直接使用$\{u, v, x, y\}$的切标架场,不再另选.
度规场为
\begin{equation*}
	{\rm d}s^2 = L^2\left( e^{2\beta}{\rm d}x^2 +e^{-2\beta}{\rm d}y^2 \right) 
	- \frac{1}{2}{\rm d}u{\rm d}v - \frac{1}{2}{\rm d}v{\rm d}u.
\end{equation*}
由此可得非零克氏符:
\begin{align*}
	\Gamma^v_{xx} =& 2 L e^{2 \beta } \left(L'+L \beta'\right),\quad
	\Gamma^v_{yy} = -2 L e^{-2\beta } \left(L\beta'-L'\right) ,\\
	\Gamma^x_{ux} =& \Gamma^x_{xu} =  \frac{L'}{L}+\beta', \qquad
	\Gamma^y_{uy} =  \Gamma^y_{yu} = \frac{L'}{L}-\beta' .
\end{align*}
非零黎曼曲率:
\begin{align*}
	R_{uxux} =& -e^{ 2 \beta} L \left(L''+ L\beta'^2 + L\beta'' +2 L' \beta'\right),\\
	R_{uyuy} =& -e^{-2 \beta} L \left(L''+ L\beta'^2 - L \beta''-2 L' \beta'\right).
\end{align*}
非零Ricci曲率:
\begin{align*}
	R_{uu} = -\frac{2}{L} \left(L''+L \beta'^2\right) .
\end{align*}
由Ricci曲率可见只要$L(u)$、$\beta(u)$满足方程$L''+L\beta'^2=0$,
那么上面给出的度规场便是真空爱氏场方程的解;并且黎曼曲率可以非零.
\qed

\begin{exercise}
	证明式\eqref{chrg:eqn_CST-down-I}、\eqref{chrg:eqn_CST-down-II}、\eqref{chrg:eqn_Bianchi-CII-down}.
\end{exercise}

\begin{exercise}
	请自己编写符号演算程序来计算式\eqref{chrg:eqn_Riemann_rr}.
\end{exercise}

\begin{exercise}
	补全例题\ref{chrg:exm_SWPlane}、\ref{chrg:exm_BondiPlane}中遗漏步骤.
\end{exercise}




\section{$E^3$中的活动标架法}\label{chrg:sec_E3MF}
\S \ref{chccr:sec_form1}和\S\ref{chrg:sec_form2}的工作
起源于$E^3$中的活动标架法(参见\parencite[\S 6.3]{cc2001-zh});
我们沿着历史发展的时间顺序,介绍一下$E^3$中的活动标架法.
我们把\S\ref{chcdg:sec_E3}中的活动标架用抽象指标重新表述一遍.
$E^3$中的坐标原点是$O$,再任意指定一点$P\in E^3$和其伴随矢量空间$\mathbb{R}^3$中的
一组基底$\{(e_1)^a,(e_2)^a,(e_3)^a\}$.度规是正定的单位矩阵.

在$E^3$中先选定固定的、正交归一的切标架场$\{O;(\delta_i)^a\},\, 1\leqslant i \leqslant 3$;
对应坐标直线是$\{x^i\}$,即$(\delta_i)^a= (\frac{\partial}{\partial x^i})^a$;
这套基矢在$E^3$中每点都相同.
则任意活动标架场$\{P;(e_\alpha)^a\}$(标架可能非正交归一,可能逐点不同)可表示为:
\begin{equation}
    (OP)^a = x^i (\delta_i)^a,\qquad
    (e_\beta)^a = (\delta_j)^a A^j_{\cdot \beta} .
\end{equation}
上式中为了让抽象指标匹配,已令$(OP)^a=\overrightarrow{OP}$.
需要注意切矢量场的外微分运算是没有定义的;
由正定欧几里得度规场$g_{ab}=\delta_{ij}(\delta^i)_a (\delta^j)_b
=g_{\alpha\beta}(e^\alpha)_a(e^\beta)_b$可以
很简单地把切矢量变成余切矢量:
\begin{equation}
    (OP)_a = x^i (\delta_i)_a,\qquad
    (e_\alpha)_a = (\delta_j)_a A^j_{\cdot \alpha} .
\end{equation}
有了余切矢量场,便可以用外微分算符了(注意常矢量$(\delta_i)_a$的微分恒为零).
\begin{equation}
    {\rm d}_b (OP)_a = ({\rm d}_b x^i) \wedge (\delta_i)_a,\quad
    {\rm d}_b (e_\beta)_a = ({\rm d}_b A^j_{\cdot \beta}) \wedge  (\delta_j)_a .
\end{equation}
把上式中的固定基矢场$(\delta_i)_a$消去,换为活动基矢场$(e_\alpha)_a$,有
\begin{equation}\label{chrg:eqn_tmpopeBA}
    {\rm d}_b (OP)_a = B^{\alpha}_{\cdot i} {\rm d}_b x^i \wedge (e_\alpha)_a  ,\quad
    {\rm d}_b (e_\beta)_a = B^{\alpha}_{\cdot j}{\rm d}_b A^j_{\cdot \beta} \wedge  (e_\alpha)_a .
\end{equation}
其中,变换矩阵$A$和$B$是互逆关系,即$B^{-1} = A$.令
\begin{equation}\label{chrg:eqn_om324}
    (\omega^\alpha)_b = B^{\alpha}_{\cdot i} {\rm d}_b x^i,\qquad
    (\omega^{\alpha}_{\cdot \beta})_b=B^{\alpha}_{\cdot j}{\rm d}_b A^j_{\cdot \beta} .
\end{equation}
将上式带回\eqref{chrg:eqn_tmpopeBA},
并对该式再次求取外微分,利用${\rm d}\circ {\rm d}=0$可得结构方程:
\setlength{\mathindent}{0em}
\begin{align}
    0=&\bigl({\rm d} (\omega^\alpha)\bigr)_{cb} \wedge (e_\alpha)_a - (\omega^\alpha)_c \wedge {\rm d}_b(e_\alpha)_a
    {\color{red}\ \Rightarrow \ }
    {\rm d}_c (\omega^\alpha)_b = (\omega^\sigma)_c\wedge (\omega_{\cdot \sigma}^\alpha)_b . \label{chrg:eqn_jgfc1} \\
    0=&\bigl({\rm d} (\omega^\alpha_{\cdot \beta})\bigr)_{cb} \wedge  (e_\alpha)_a 
    - (\omega^\alpha_{\cdot \beta})_c \wedge {\rm d}_b (e_\alpha)_a {\color{red}\ \Rightarrow \ }
    {\rm d}_c (\omega^\alpha_{\cdot \beta})_b = (\omega^{\sigma}_{\cdot \beta})_c 
    \wedge (\omega^{\alpha}_{\cdot \sigma})_b .    \label{chrg:eqn_jgfc2}
\end{align}\setlength{\mathindent}{2em}



我们取活动标架场为$P$点的自然切标架,令$(e_\alpha)^a= (\frac{\partial}{\partial u^\alpha})^a$,
$\{u^\alpha\}$是活动标架(例如球坐标)所对应的坐标曲线.
此时(下式的记号略有些乱,省略了位矢$\boldsymbol{r}$的抽象指标)
\begin{equation}
    \boldsymbol{r}=\overrightarrow{OP}= (OP)^a = x^i (\delta_i)^a 
    = x^i(u^1,u^2,u^3) \left(\frac{\partial}{\partial x^i}\right)^a .
\end{equation}
点$P$的活动标架场是该点的自然标架$\{P;(e_\alpha)^a\}$:
\begin{equation}
    (e_\alpha)^a = \left.\left(\frac{\partial}{\partial u^\alpha}\right)^a\right|_{P}
    =\frac{\partial x^j}{\partial u^\alpha}
    \left.\left(\frac{\partial}{\partial x^j}\right)^a \right|_{P}
    =A^j_{\cdot \alpha} \left.\left(\frac{\partial}{\partial x^j}\right)^a\right|_{P} .
\end{equation}
在$E^3$($g_{ab}=\delta_{ij}({\rm d}x^i)_a ({\rm d}x^j)_b$)中,
活动标架场$\{P;(e_\alpha)^a\}$下的度规分量是:
\begin{equation}\label{chrg:eqn_gabEuclid}
    g_{\alpha\beta}= g_{ab}(e_\alpha)^a (e_\beta)^b
    =g_{ab} \frac{\partial x^i}{\partial u^\alpha} \left(\frac{\partial}{\partial x^i}\right)^a
    \frac{\partial x^j}{\partial u^\beta} \left(\frac{\partial}{\partial x^j}\right)^b
    = \sum_{i=1}^{3}\frac{\partial x^i}{\partial u^\alpha}\frac{\partial x^i}{\partial u^\beta} .
\end{equation}
在欧几里得空间中,任一点切空间仍是欧氏空间,与自身完全重合;
这使得欧氏空间的基矢的导数可以在自身展开,即
\begin{equation}\label{chrg:eqn_tmpr23}
    \frac{\partial (e_\alpha)^a }{\partial u^\beta} = \Gamma^\sigma_{\alpha\beta} (e_\sigma)^a  
    \quad \Leftrightarrow \quad
    \frac{\partial }{\partial u^\beta} \left.\left(\frac{\partial}{\partial u^\alpha}\right)^a\right|_{P}
    = \Gamma^\sigma_{\alpha\beta} \left.\left(\frac{\partial}{\partial u^\sigma}\right)^a \right|_{P}.
\end{equation}
欧氏联络(即$\partial$)自然与欧氏度规相互容许,
故联络系数满足$\Gamma^\sigma_{\alpha\beta}=\Gamma^\sigma_{\beta\alpha}$;
由偏导数求导次序无关也可得到这个对称性,即
\begin{equation}\label{chrg:eqn_tmpra32}
    \frac{\partial (e_\alpha)^a }{\partial u^\beta}
    =\frac{\partial^2 \boldsymbol{r}}{\partial u^\alpha \partial u^\beta}
    =\frac{\partial (e_\beta)^a }{\partial u^\alpha}
    =\frac{\partial^2 x^j}{\partial u^\alpha\partial u^\beta}
    \left. \left(\frac{\partial}{\partial x^j}\right)^a \right|_{P}.
\end{equation}
将式\eqref{chrg:eqn_tmpr23}与$(e_\nu)^a$作内积,得
\begin{equation}
    g_{ab} \left[\frac{\partial }{\partial u^\beta} \left(\frac{\partial}{\partial u^\alpha}\right)^a\right]_P
    \left. \left(\frac{\partial}{\partial u^\nu}\right)^b\right|_{P}
    = \sum_{j=1}^{3}\frac{\partial^2 x^j}{\partial u^\alpha\partial u^\beta}
    \frac{\partial x^j}{\partial u^\nu}
    = \Gamma^\sigma_{\alpha\beta} g_{\sigma\nu}  .
\end{equation}
对式\eqref{chrg:eqn_gabEuclid}求偏导数,有(结合上式)
\begin{align}
    \frac{\partial g_{\alpha\beta}}{\partial u^\mu} = 
    \sum_{i=1}^{3}\left[\frac{\partial^2 x^i}{\partial u^\alpha\partial u^\mu}
    \frac{\partial x^i}{\partial u^\beta}+ \frac{\partial x^i}{\partial u^\alpha}
    \frac{\partial^2 x^i}{\partial u^\beta\partial u^\mu}   \right] 
    = \Gamma^\sigma_{\alpha\mu} g_{\sigma\beta} + \Gamma^\sigma_{\beta\mu} g_{\sigma\alpha} .
\end{align}
由上式不难得到联络系数表达式:
\begin{equation}\label{chrg:eqn_Gmf}
    \Gamma^\mu_{\alpha\beta}= \frac{1}{2} g^{\mu\sigma}\left(
    \frac{\partial g_{\alpha\sigma}}{\partial u^\beta}+
    \frac{\partial g_{\beta\sigma}}{\partial u^\alpha}-
    \frac{\partial g_{\alpha\beta}}{\partial u^\sigma}\right) .
\end{equation}
很明显,与式\eqref{chrg:eqn_Christoffel-naturalbases}相同;不相同才奇怪!
虽然欧氏空间是平直的,当选用活动标架时,其克氏符\eqref{chrg:eqn_Gmf}未必都是零;
但由这组克氏符计算得到的黎曼曲率一定恒为零.



继续讨论结构方程\eqref{chrg:eqn_jgfc1}、\eqref{chrg:eqn_jgfc2}.
在活动标架场$\{P;(\frac{\partial}{\partial u^\alpha})^a\}$下,
我们计算式\eqref{chrg:eqn_om324}:
\begin{align}
    (\omega^\alpha)_a =& B^{\alpha}_{\cdot i} {\rm d}_a x^i=\frac{\partial u^\alpha}{\partial x^i}
    \frac{\partial x^i}{\partial u^\sigma}({\rm d}u^\sigma)_a=({\rm d}u^\alpha)_a, \label{chrg:eqn_tmpe45} \\
    (\omega^{\alpha}_{\cdot \beta})_a=&B^{\alpha}_{\cdot j}{\rm d}_a A^j_{\cdot \beta}
    =\frac{\partial u^\alpha}{\partial x^j} \frac{\partial^2 x^j}
    {\partial u^\sigma\partial u^\beta}({\rm d}u^\sigma)_a
    \xlongequal[\ref{chrg:eqn_tmpra32}]{\ref{chrg:eqn_tmpr23}}
    \Gamma^\alpha_{\beta \sigma}({\rm d}u^\sigma)_a. \label{chrg:eqn_tmpomg5}
\end{align}
由式\eqref{chrg:eqn_tmpe45}可知$(\omega^\alpha)_a$就是自然坐标
基矢$(e_\alpha)^a=(\frac{\partial}{\partial u^\alpha})^a$的对偶基矢,
可以把它记为$(\omega^\alpha)_a=(e^\alpha)_a$.
由式\eqref{chrg:eqn_tmpomg5}可知它与式\eqref{chccr:def_1form}完全相同.
故可得结论:结构方程\eqref{chrg:eqn_jgfc1}、\eqref{chrg:eqn_jgfc2}与
结构方程\eqref{chccr:eqn_CST}(欧氏空间黎曼曲率恒为零)本质相同;不相同就奇怪了.
历史上,Cartan先在欧氏空间发展了活动标架法,得到了结构方程\eqref{chrg:eqn_jgfc1}、\eqref{chrg:eqn_jgfc2};
之后把它们推广成为式\eqref{chccr:eqn_CST}的形式.


{\kaishu 
    陈省身(\parencite{cc2001-zh}附录二)说(大意):
    (1) 现在,活动标架法(\S \ref{chccr:sec_form1}、\S\ref{chrg:sec_form2}、
    \S\ref{chnull:sec_NP1}和\S\ref{chnull:sec_NP2}内容)是微分几何中极为重要的方法.
    (2) 张量分析法(利用局部坐标切矢量作为标架)弊多利少,但简单明了,在微分几何的初等问题中功不可没.
}


%\index[physwords]{Schmidt正交化}
%
%\begin{example}
%    Schmidt正交化.
%\end{example}
%上面给出的自然标架场$(\frac{\partial}{\partial u^\alpha})^a$可能是非正交的,
%我们可以通过Schmidt过程得到正交归一化标架场.度规$g_{\alpha\beta}$见式\eqref{chrg:eqn_gabEuclid}.
%\begin{align*}
%    (e_1)^a =& \frac{1}{\sqrt{g_{11}}}\left(\frac{\partial}{\partial u^1}\right)^a,\\
%    (e_2)^a =& 
%\end{align*}


\subsection{正交归一标架场}\label{chrg:sec_E3-Orthogonal-Normalization}
本小节重复抽象指标求和;重复分量指标不求和,求和分量指标将显示标识.

我们总可以将一个标架场正交化(参见\pageref{chmla:thm_obm}页定理\ref{chmla:thm_obm}中论证).
故可以假设活动标架场$(\frac{\partial}{\partial u^\alpha})^a$是相互正交的,但未必归一;归一化关系为:
\begin{equation}
    ({E}_\alpha)^a = L_{\alpha}^{-1} \left(\frac{\partial}{\partial u^\alpha}\right)^a ,
    \quad \text{重复指标$\alpha$不求和,} L_{\alpha} \ \text{是Lam\'e系数}
\end{equation}
$({E}_\alpha)^a$是归一化后的矢量.带入度规,有
\begin{align*}
    \delta_{\alpha \beta} =& g_{ab} ({E}_\alpha)^a ({E}_\beta)^b = g_{ab}
    L_{\alpha}^{-1} \left(\frac{\partial}{\partial u^\alpha}\right)^a 
    L_{\beta}^{-1}  \left(\frac{\partial}{\partial u^\beta}\right)^b 
    =g_{\alpha\beta} L_{\alpha}^{-1} L_{\beta}^{-1} \\
    \xlongequal{\ref{chrg:eqn_gabEuclid}} &    L_{\alpha}^{-1} L_{\beta}^{-1}
    \sum_{l=1}^{3}\frac{\partial x^l}{\partial u^\alpha}\frac{\partial x^l}{\partial u^\beta},
    \qquad \text{重复指标}\alpha,\beta \text{不求和} 
\end{align*}
因此可得Lam\'e系数为:
\begin{equation}
    L_{\alpha} = \sqrt{ \left(\frac{\partial x}{\partial u^\alpha}\right)^2 
        + \left(\frac{\partial y}{\partial u^\alpha}\right)^2
        + \left(\frac{\partial z}{\partial u^\alpha}\right)^2 } \  ;
    \qquad \alpha = 1,2,3
\end{equation}
对于正交的活动标架场$(\frac{\partial}{\partial u^\alpha})^a$,其度规可以表示为
\begin{equation}\label{chrg:eqn_E3gab}
    \begin{aligned}
        g_{ab}=& (L_1)^2({\rm d}u^1)_a ({\rm d}u^1)_b
        +(L_2)^2({\rm d}u^2)_a ({\rm d}u^2)_b
        +(L_3)^2({\rm d}u^3)_a ({\rm d}u^3)_b \\
        =& (E^1)_a (E^1)_b + (E^2)_a (E^2)_b + (E^3)_a (E^3)_b .
    \end{aligned}    
\end{equation}
由式\eqref{chrg:eqn_Gmf}可得坐标$\{u^\alpha\}$的联络系数$\oversetmy{u}{\Gamma}$(下式重复指标不求和)
\setlength{\mathindent}{0em}
\begin{equation}\label{chrg:eqn_E3gamma}
    \oversetmy{u}{\Gamma}^\mu_{\alpha\beta}= 0, \ (\alpha\neq \beta \neq \mu) ; \quad
    \oversetmy{u}{\Gamma}^\mu_{\alpha \alpha}= - \frac{L_\alpha}{L_\mu^2} 
    \frac{\partial L_\alpha}{\partial u^\mu} , \ (\alpha \neq \mu) ; \quad
    \oversetmy{u}{\Gamma}^\mu_{\mu \beta}= \frac{1}{L_\mu} \frac{\partial L_\mu}{\partial u^\beta} .
\end{equation}\setlength{\mathindent}{2em}
式\eqref{chrg:eqn_E3gamma}中最后一式中的$\beta$可以等于$\mu$、也可以不等于$\mu$.

%对于复杂的标架场,我们可以通过Cartan结构方程或者式\eqref{chccr:eqn_D-omega-form}来求解标架运动.
对于上述正交归一标架场可通过如下直接计算方式求得基矢导数.
\begin{align*}
    \nabla_{\frac{\partial }{\partial u^j}} (E_k)^b = & \nabla_{\frac{\partial }{\partial u^j}}
    \left[ L_{k}^{-1} \left(\frac{\partial}{\partial u^k}\right)^b \right]
    =-L_{k}^{-2} \frac{\partial L_{k}}{\partial u^j}\left(\frac{\partial}{\partial u^k}\right)^b
    +L_{k}^{-1} \nabla_{\frac{\partial }{\partial u^j}} \left(\frac{\partial}{\partial u^k}\right)^b \\
    =&-L_{k}^{-1} \frac{\partial L_{k}}{\partial u^j} (E_k)^b
    +L_{k}^{-1}  \sum_{l=1}^{3} \oversetmy{u}{\Gamma}^l_{kj} L_{l} (E_l)^b  .
\end{align*}
利用式\eqref{chrg:eqn_E3gamma},可将上式分成如下两种形式(下式中重复指标不求和):
\begin{subequations}\label{chrg:eqn_E3-Triad}
    \begin{align}    
        \nabla_{\frac{\partial }{\partial u^j}} (E_k)^b  = &
        \frac{1}{L_k} \frac{\partial L_j}{\partial u^k} (E_j)^b  ,
        \qquad k\neq j ; \label{chrg:eqn_E3-Triad-1}\\
        \nabla _{\frac{\partial }{\partial u^j}} (E_j)^b =& 
        - (E_k)^b \frac{1}{L_k} \frac{\partial L_j}{\partial u^k} 
        - (E_i)^b \frac{1}{L_i} \frac{\partial L_j}{\partial u^i},
        \qquad j\neq k \neq i . \label{chrg:eqn_E3-Triad-2}
    \end{align}    
\end{subequations}
式\eqref{chrg:eqn_E3-Triad-2}中$1\leqslant i,j,k \leqslant 3$,且互不相等.
共有三组,分别是:$j=1$、$k=2$、$i=3$;$j=2$、$k=3$、$i=1$;$j=3$、$k=1$、$i=2$.


式\eqref{chrg:eqn_E3-Triad}是力学张量分析中的常用公式,
由该式可以得到球坐标等正交归一基矢量的偏导数关系,请读者试着推演.
因$\{u^i\}$是活动标架,故不建议把这两式中的
协变导数$\nabla _{\frac{\partial }{\partial u^i}} (E_j)^b$记
成偏导数$\frac{\partial }{\partial u^i} (E_j)^b$.
需要强调的是:$\{u^i\}$不是正交归一基矢$\{(E_i)^a\}$的坐标线,
$\{u^i\}$是正交、可能非归一基矢$(\frac{\partial }{\partial u^j})^a$的坐标线.


%\subsection{$E^3$中的二维曲面论}\label{chrg:sec_E3-2surface}


\begin{exercise}
	将式\eqref{chrg:eqn_E3-Triad-1}、\eqref{chrg:eqn_E3-Triad-2}写成分量形式.
\end{exercise}


\index[physwords]{截面曲率}

\section{截面曲率}\label{chrg:sec_sectional-curvature}
截面曲率是二维曲面论中高斯曲率在高维空间的推广,本节来讨论这个概念以及与之有关的一些定理.
设有$m(\geqslant 2)$维广义黎曼流形$(M,g)$,在$p\in M$点切空间$T_pM$中
有两个不共线的、非零模、非零切矢量$u^a$和$v^a$,
将它们进行非退化线性变换
\begin{equation}\label{chrg:eqn_tmp3221}
    \tilde{u}^a = C_1^1 u^a +C_1^2 v^a, \ \tilde{v}^a = C_2^1 u^a +C_2^2 v^a;
    \ \det(C) = C_1^1C_2^2 - C_1^2 C^1_2\neq 0 .
\end{equation}
得到的$\tilde{u}^a$和$\tilde{v}^a$同样是$T_pM$中两个不共线切矢量.令
\begin{equation}\label{chrg:eqn_tmp3222}
    Q(u,v)\equiv {u}_a{u}^a \cdot {v}_b{v}^b - ({u}_a{v}^a)^2 .
\end{equation}
通过直接计算可证明,在非退化变换\eqref{chrg:eqn_tmp3221}下,有
\begin{equation}\label{chrg:eqn_tmp3223}
    Q(\tilde{u},\tilde{v}) = \bigl(\det(C)\bigr)^2 Q(u,v) .
\end{equation}
计算过程只是繁琐但不困难,请读者补齐.
因$u^a,v^a$不共线,它们共同确定了一个平面,
这个二维子空间称为切空间$T_pM$的{\heiti 二维截面},记作$[u\wedge v]$.
当度规正定时,恒为正的$Q(u,v)$恰好是两个矢量所张成平行四边形的面积平方;
当度规不定时,$Q(u,v)$可能为负,此时取绝对值以保证$Q(u,v)>0$.

同样通过直接计算可证明下式(给出计算过程)
\begin{align*}
    R_{abcd}\tilde{u}^a\tilde{v}^b\tilde{u}^c\tilde{v}^d = & R_{abcd}
    (C_1^1 u^a +C_1^2 v^a) (C_2^1 u^b +C_2^2 v^b)(C_1^1 u^c +C_1^2 v^c) (C_2^1 u^d +C_2^2 v^d) \\
    =&R_{abcd}(C_1^1 C_2^1 u^a u^b -C_1^2 C_2^1 v^b u^a+ C_1^1 C_2^2 u^a v^b +C_1^2 C_2^2 v^a v^b) \\
    &\phantom{R_{a}}\times (C_1^1 C_2^1 u^c u^d -C_1^2 C_2^1 v^d u^c+ C_1^1 C_2^2 u^c v^d +C_1^2 C_2^2 v^c v^d)
\end{align*}
考虑到黎曼曲率关于下标的反对称性($R_{abcd}=-R_{bacd}=-R_{abdc}$),上式最终为
\begin{equation}\label{chrg:eqn_tmp3224}
    R_{abcd}\tilde{u}^a\tilde{v}^b\tilde{u}^c\tilde{v}^d = \bigl(\det(C)\bigr)^2 R_{abcd} u^a v^b u^c v^d .
\end{equation}
由此可以定义二维截面$[u\wedge v]$的{\heiti 截面曲率}为
\begin{equation}\label{chrg:eqn_sectional-curvature}
    K([u\wedge v]) \overset{def}{=}  \frac{R_{abcd} u^a v^b u^c v^d}{Q(u,v)} 
    = \frac{R_{abcd} u^a v^b u^c v^d}{{u}_a{u}^a \cdot {v}_b{v}^b - ({u}_a{v}^a)^2 } .
\end{equation}
其中$K$的参数是由$u^a,v^a$所确定的截面,与$u^a,v^a$本身没有关系,这由
式\eqref{chrg:eqn_tmp3223}以及\eqref{chrg:eqn_tmp3224}一望而知.
由定义可以看出$K$是一个实数.

\begin{example}\label{chrg:exam_sc-gauss}
    当$m=2$时,来计算一下$K$值.设$u^a=(\frac{\partial}{\partial x})^a$及
    $v^a=(\frac{\partial}{\partial y})^a$,有
    \begin{equation}
        K([u\wedge v]) = \frac{R_{abcd} (\frac{\partial}{\partial x})^a (\frac{\partial}{\partial y})^b
            (\frac{\partial}{\partial x})^c (\frac{\partial}{\partial y})^d}
          {Q((\frac{\partial}{\partial x}),(\frac{\partial}{\partial y}))}
        =\frac{R_{1212}}{g_{11}g_{22}-g_{12}g_{21}} .
    \end{equation}
    这正是二维曲面$p$点高斯曲率,见式\eqref{chcdg:eqn_RK2}.
\end{example}

\index[physwords]{常曲率空间}

\subsection{常曲率空间}\label{chrg:sec_const-curvature}
\begin{definition}
        若广义黎曼流形$(M,g)$上所有点的二维截面曲率都是同一常数,则$M$为{\heiti 常曲率空间}.
\end{definition}

\begin{theorem}\label{chrg:thm_const-curvature}
    $M$是常曲率空间的充要条件是:
    $\forall p\in M,\ \forall u^a,v^a,w^a,z^a\in T_pM$,有
    \begin{equation}\label{chrg:eqn_const-curvature}
        R_{abcd} u^a v^b w^c z^d = K_0 ( {u}_a {w}^a \cdot {v}_b{z}^b - {u}_a{z}^a \cdot {v}_b{w}^b ).
    \end{equation}
\end{theorem}
\begin{proof}
    “$\Leftarrow$”部分是显然的.

    下面证明“$\Rightarrow$”.
    点$p$所有二维截面的截面曲率都是实常数$K_0$,即
    \setlength{\mathindent}{0em}
    \begin{align*}
        0=& R_{abcd} u^a v^b u^c v^d -K_0 ({u}_a{u}^a \cdot {v}_b{v}^b - {u}_a{v}^a {u}^b{v}_b ) \\
        \xLongrightarrow{v\to v+z}
        0 = & R_{abcd} u^a v^b u^c v^d + R_{abcd} u^a v^b u^c z^d+R_{abcd} u^a z^b u^c v^d+R_{abcd} u^a z^b u^c z^d \\
         -& K_0 \bigl(u_a u^a (v_b v^b+v_b z^b + z_b v^b + z_b z^b ) - u_a u^b (v^a v_b + v^a z_b + z^a v_b + z^a z_b )\bigr) \\
        \xLongrightarrow[\text{恒零项}]{\text{约去}}
        0 = &   R_{abcd} u^a z^b u^c v^d - K_0 \bigl( u_a u^a z_b v^b  - u_a u^b v^a z_b \bigr) \\
        \xLongrightarrow{u\to u+w}
        0 = &   R_{abcd} (u^au^c+w^au^c+u^aw^c+ w^aw^c) z^b v^d \\
         -& K_0 \bigl( (u^a u_a+ w^au_a+u^aw_a+ w^aw_a) z_b v^b  - (u^b u_a+ w^bu_a+u^bw_a+ w^bw_a) v^a z_b \bigr) \\
       \xLongrightarrow[\text{恒零项}]{\text{约去}}   0 = &  R_{abcd} (w^a u^c+u^aw^c ) z^b v^d
         - K_0 \bigl( ( w^au_a+u^aw_a) z_b v^b  - ( w^bu_a+u^bw_a) v^a z_b \bigr) \\
        \Rightarrow 0=&
            R_{abcd}u^a v^b w^c z^d  - K_0 \bigl(  w^au_a z_b v^b  - u^bz_b w_a v^a \bigr) \\
         +& R_{abcd}u^a z^b w^c v^d  - K_0 \bigl(  u^aw_a z_b v^b  - u_av^a z_b w^b \bigr)
    \end{align*}  \setlength{\mathindent}{2em}
    上式倒数第一行中的$T(u,z,w,v)\equiv -K_0 ( u^aw_a z_b v^b  - u_av^a z_b w^b ) $满足第一Bianchi恒等式(循环恒等式),
    即在代换$v\to w, w\to z, z\to v$作用下有$T(u,v,w,z)+T(u,w,z,v)+T(u,z,v,w)=0$;直接计算便可验证这个恒等式.
    同时不难验证它还满足$T(u,z,w,v)=-T(u,z,v,w)=-T(z,u,w,v)$和$T(u,z,w,v)=T(w,v,u,z)$;
    黎曼曲率本身当然也满足上述这些对称性.
    上式最后一步表明,当所有二维截面曲率为常数时,具有如下对称性:
    \begin{equation}\label{chrg:eqn_tmpcc10}
        R_{abcd}u^a v^b w^c z^d  + T(u,v,w,z) = R_{abcd}u^a z^b v^c w^d  + T(u,z,v,w).
    \end{equation}
    这相当于进行如下轮换时具有不变性,即$v\to z, w\to v, z\to w$;很显然,上式再次轮换后,还有下述等式
    \begin{equation}\label{chrg:eqn_tmpcc20}
        R_{abcd}u^a v^b w^c z^d  + T(u,v,w,z) = R_{abcd}u^a w^b z^c v^d  + T(u,w,z,v).
    \end{equation}
    将式\eqref{chrg:eqn_tmpcc10}和\eqref{chrg:eqn_tmpcc20}中三个不同项加在一起,有
    \begin{equation}\label{chrg:eqn_tmpcc30}
      \begin{aligned}
         &\bigl(R_{abcd}u^a v^b w^c z^d  + T(u,v,w,z)\bigr)+  \bigl(R_{abcd}u^a z^b v^c w^d  + T(u,z,v,w)\bigr) \\
         +& \bigl( R_{abcd}u^a w^b z^c v^d  + T(u,w,z,v)\bigr) .
      \end{aligned}
    \end{equation}
    因$T$和曲率$R$都满足第一Bianchi恒等式,所以式\eqref{chrg:eqn_tmpcc30}恒为零.
    而式\eqref{chrg:eqn_tmpcc30}本身等于$3\bigl(R_{abcd}u^a v^b w^c z^d  + T(u,v,w,z)\bigr)$,
    最终有
    \begin{equation}
        3\cdot \bigl(R_{abcd} u^a v^b w^c z^d  - K_0 ( u_a w^a\cdot  z_b v^b   - u^bz_b \cdot w_a v^a  ) \bigr)= 0.
    \end{equation}
    这便是定理中的式\eqref{chrg:eqn_const-curvature}.
\end{proof}

%由定理\ref{chrg:thm_const-curvature}可以直接得到如下定理,
\begin{corollary}\label{chrg:thm_const-curvature-component}
    设$(M,g)$是常曲率空间($K_0$是曲率),则在局部标架场$\{(e_i)^a\}$下有
    \begin{equation}\label{chrg:eqn_const-curvature-component}
        R_{ijkl}= R_{abcd} (e_i)^a (e_j)^b (e_k)^c (e_l)^d
        = K_0 ( g_{ik}g_{jl}- g_{il}g_{jk}  ).
    \end{equation}
\end{corollary}

在常曲率空间中,曲率型式有较为简洁的表达式,由表达式\eqref{chrg:eqn_Cform-down}得
\begin{equation}\label{chrg:eqn_const-curvature-inFORM}
    (\Omega_{ij})_{ab} = \frac{1}{2}R_{ijkl} (e^k)_a \wedge (e^l)_b
    \xlongequal{\ref{chrg:eqn_const-curvature-component}}
    K_0 (e_i)_a \wedge (e_j)_b .
\end{equation}
需注意,上式所有指标都在下面.
利用上式,F. Schur证明了如下重要定理:
\begin{theorem}\label{chrg:thm_Schur-lemma}
    设有$m(\geqslant 3)$维连通广义黎曼流形$(M,g)$,
    若$\forall p\in M$,此点的任意二维截面曲率是常数$K(p)$.
    那么,$K(p)$必为常数,即$M$是常曲率空间.
\end{theorem}
\begin{proof}
    定理中$K(p)$可能是逐点不同的,也就是$K(p)$是$M$上光滑函数场;
    我们要证明的是:在整个流行$M$上,$K(p)$只能取常数值,不能逐点不同.

    由式\eqref{chrg:eqn_const-curvature-inFORM}可知:
    $(\Omega_{ij})_{ab}(p) = K(p) (e_i)_a \wedge (e_j)_b , \quad 1\leqslant i,j \leqslant m$.
    对此式求外微分(注意$K(p)\in C^\infty(M)$),得
    \begin{align*}
        &{\rm d}_c(\Omega_{ij})_{ab} = {\rm d}_c (K)\wedge (e_i)_a \wedge (e_j)_b
        + K \bigl({\rm d}_c(e_i)_a \bigr) \wedge (e_j)_b - K (e_i)_c \wedge \bigl({\rm d}_a (e_j)_b \bigr) \\
        &\xlongequal{\ref{chrg:eqn_CST-down-I}} {\rm d}_c (K)\wedge (e_i)_a \wedge (e_j)_b
        - K (e^k)_c \wedge (\omega_{k i})_a \wedge (e_j)_b
        + K (e_i)_c \wedge (e^k)_a \wedge (\omega_{k j})_b .
    \end{align*}
    同时,由式\eqref{chrg:eqn_Bianchi-CII-down}可知联络型式还可以表示为
    \begin{align*}
      {\rm d}_c (\Omega_{ij})_{ab}  =&  ( \omega _{\cdot j}^k )_c
      \wedge (\Omega_{i k} )_{ab} + ( \omega _{\cdot i}^{k} )_c \wedge (\Omega_{k j})_{ab}   \\
      \xlongequal{\ref{chrg:eqn_const-curvature-inFORM}}&
      K( \omega _{\cdot j}^k )_c \wedge (e_i)_a \wedge (e_k)_b
      +K ( \omega _{\cdot i}^{k} )_c \wedge (e_k)_a \wedge (e_j)_b .
    \end{align*}
    对比以上两式可知,有
    \begin{equation}
        {\rm d}_c (K)\wedge (e_i)_a \wedge (e_j)_b = 0.
    \end{equation}
    在标架场$\{(e^i)_a\}$中,可令${\rm d}_c (K) = K_l (e^l)_c$;带入上式,有
    \begin{equation}\label{chrg:eqn_tmpschur}
       \sum_{l } K_l (e^l)_c \wedge (e_i)_a \wedge (e_j)_b = 0, \qquad \forall i,j .
    \end{equation}
    取$i=1,j=2$,由上式得$K_l=0(l \geqslant 3)$.再取$i=3,j=1$和$i=3,j=2$,由式\eqref{chrg:eqn_tmpschur}可
    得$K_1=K_2=0$.因此有${\rm d}_c (K) = 0$,进而可知$K(p)$对于整个流形只能取常数,不能是逐点变化的.
\end{proof}

%很明显,上面给出的常曲率空间满足爱因斯坦流形的定义\ref{chrg:def_Einstein-manifold},因此有
%\begin{corollary}\label{chrg:thm_conformal-Einstein-manifold}
%    常曲率广义黎曼流形是爱因斯坦流形.
%\end{corollary}
%有关爱因斯坦流形更多内容请参阅文献\parencite{besse-2008-EM}.

\begin{exercise}
	在不同文献中,式\eqref{chrg:eqn_sectional-curvature}的定义会有正负号差异,缘由大体如此:
	二维曲面高斯曲率\eqref{chcdg:eqn_RK2}定义是固定不变的(指$K=\kappa_1 \kappa_2$不变),
	不会有人在这个式子上加个负号(即不会定义成$K=-\kappa_1 \kappa_2$).
	而不同文献中$\binom{0}{4}$型黎曼曲率的定义\eqref{chrg:eqn_RiemannianCurvature-d4}有可能相差一个负号;
	为了调整这个符号差异,不同文献在式\eqref{chrg:eqn_sectional-curvature}的定义会有正负号差异;
	例题\ref{chrg:exam_sc-gauss}恰好可以验证这种差异.	如读者有意愿,请翻阅其它书籍,
	查看其曲率正负号(指截面曲率、$\binom{0}{4}$型黎曼曲率和二维曲面高斯曲率)是否前后一致.
\end{exercise}


\index[physwords]{共形变换}
\section{共形变换}\label{chrg:sec_comformal-transformation}
有$m$维广义黎曼流形$(M,g)$,设$\Omega \in C^\infty(M)$是处处恒正函数场,于是
\begin{equation}\label{chrg:eqn_comformal-transformation}
    \tilde{g}_{ab} = \Omega^2 g_{ab} \quad \Leftrightarrow \quad
    \tilde{g}^{ab} = \Omega^{-2} g^{ab},
\end{equation}
在流形$M$上定义了一个\uwave{新}度规场$\tilde{g}_{ab}$;
式\eqref{chrg:eqn_comformal-transformation}称为流形$M$上的{\heiti 共形变换};
特别的,如果$\Omega$是正的实常数,则称为称为{\heiti 相似变换}.

与$g_{ab}$相容的联络记为$\nabla_a$,克氏符记为$\Gamma_{ij}^k$.
与$\tilde{g}_{ab}$相容的联络记为$\tilde{\nabla}_a$,克氏符记为$\tilde{\Gamma}_{ij}^k$.
一般说来,带波纹的量与不带波纹的量不再相同,本节主要来计算它们间的差异.
需要注意的是,带波纹张量需要用带波纹度规$\tilde{g}$升降指标;
不带波纹张量需要用不带波纹度规$g$升降指标.为了避免不必要的误解,
我们将显示写出这些度规.


\paragraph{克氏符差异}
设有局部坐标系$\{x^i\}$,
仿照式\eqref{chrg:eqn_Christoffel-2-naturalbases}可知
\begin{align}
    \tilde{\Gamma}_{ij}^k =& \frac{1}{2} \tilde{g}^{kl} \left( \frac{\partial \tilde{g}_{il}  }{\partial x^j}
        + \frac{\partial \tilde{g}_{lj} } {\partial x^i}  - \frac{\partial \tilde{g}_{ij}} {\partial x^l} \right)
        =\frac{1}{2} \Omega^{-2}{g}^{kl} \left( \frac{\partial \Omega^{2}{g}_{il}  }{\partial x^j}
        + \frac{\partial \Omega^{2}{g}_{lj} } {\partial x^i}  
        - \frac{\partial \Omega^{2}{g}_{ij}} {\partial x^l} \right) \notag \\
%      =&  \frac{1}{2}{g}^{kl}\left( \frac{\partial {g}_{il}  }{\partial x^j}
%      + \frac{\partial {g}_{lj} } {\partial x^i}  - \frac{\partial {g}_{ij}} {\partial x^l} \right)
%      +\frac{1}{2} \Omega^{-2}{g}^{kl} \left( {g}_{il}\frac{\partial \Omega^{2}  }{\partial x^j}
%      +{g}_{lj} \frac{\partial \Omega^{2}} {\partial x^i}  
%      - {g}_{ij}\frac{\partial \Omega^{2}} {\partial x^l} \right) \notag \\
      =&\Gamma_{ij}^k+ \delta_i^k \frac{\partial \ln \Omega  }{\partial x^j}
      + \delta_j^k \frac{\partial \ln\Omega} {\partial x^i}
      - {g}_{ij} {g}^{kl}\frac{\partial \ln\Omega} {\partial x^l}
      \equiv \Gamma_{ij}^k+ \Xi_{ij}^k. \label{chrg:eqn_comformal-Gamma}
\end{align}
上式最后一步定义了
\begin{equation}\label{chrg:eqn_comformal-Xi}
    \Xi_{ij}^k \equiv \delta_i^k \frac{\partial \ln \Omega  }{\partial x^j}
    + \delta_j^k \frac{\partial \ln\Omega} {\partial x^i}
    - {g}_{ij} {g}^{kl}\frac{\partial \ln\Omega} {\partial x^l} .
\end{equation}
虽然克氏符不是张量,但两者的差$\Xi_{ab}^c\equiv \Xi_{ij}^k ({\rm d}x^i)_a ({\rm d}x^j)_b
(\frac{\partial}{\partial x^k} )^c$是张量;
很明显有$\Xi_{ab}^c=\Xi_{ba}^c$.
因$\ln \Omega$是标量函数场,在局部上可以将偏导数换成协变导数,
即$\partial_a \ln \Omega = \nabla_a\ln \Omega$;所以上式还可以表示成
\begin{equation}\label{chrg:eqn_comformal-Xi-nabla}
    \Xi_{ab}^c = \delta_a^c \nabla_b\ln \Omega + \delta_b^c \nabla_a\ln \Omega
    - {g}_{ab} {g}^{ce}\nabla_e\ln \Omega .
\end{equation}

有了克氏符的差异,那联络差异便容易求得了,$\forall X^a\in \mathfrak{X}(M),\ \omega_a \in \mathfrak{X}^*(M)$
\begin{align}
    (\tilde{\nabla}_a-\nabla_a) X^b & = \Xi_{ca}^b X^c,    \label{chrg:eqn_comformal-X-D} \\
    (\tilde{\nabla}_a-\nabla_a) \omega_b & = -\Xi_{ba}^c \omega_c.    \label{chrg:eqn_comformal-w-D}
\end{align}
由上两式容易求得高阶张量场的协变导数差异.





\paragraph{曲率差异}
首先计算黎曼曲率.从式\eqref{chccr:eqn_Riemannian13-Vec-commutator}出发,
先计算下式($\forall Z^a\in \mathfrak{X}(M)$)
\begin{align*}
    \tilde{\nabla}_a \tilde{\nabla}_b Z^d %= \tilde{\nabla}_a (\nabla_b Z^d + \Xi_{fb}^d Z^f )
    =&{\nabla}_a (\nabla_b Z^d + \Xi_{fb}^d Z^f ) + \Xi_{ea}^d (\nabla_b Z^e + \Xi_{fb}^e Z^f )
    - \Xi_{ba}^e (\nabla_e Z^d + \Xi_{fe}^d Z^f ) \\
    =&{\nabla}_a \nabla_b Z^d + Z^f {\nabla}_a \Xi_{fb}^d   +\Xi_{fb}^d {\nabla}_a  Z^f +
       \Xi_{ea}^d (\nabla_b Z^e + \Xi_{fb}^e Z^f ) - \Xi_{ba}^e \tilde{\nabla}_e Z^d
\end{align*}
对上式中的下标$ab$取反对称(注意利用$\Xi_{ab}^c=\Xi_{ba}^c$),有
%\setlength{\mathindent}{0em}
\begin{align*}
    \tilde{R}_{cab}^d Z^c =& 2 \tilde{\nabla}_{[a} \tilde{\nabla}_{b]} Z^d
    =2 {\nabla}_{[a} \nabla_{b]} Z^d + 2Z^f {\nabla}_{[a} \Xi_{b]f}^d  
    +2\Xi_{f[b}^d {\nabla}_{a]} Z^f \\
    +&2\Xi_{e[a}^d \nabla_{b]} Z^e + 2\Xi_{e[a}^d\Xi_{b]f}^e Z^f  
    \ =R_{cab}^d Z^c + 2Z^f {\nabla}_{[a} \Xi_{b]f}^d  + 2\Xi_{e[a}^d\Xi_{b]f}^e Z^f .
\end{align*} %\setlength{\mathindent}{2em}
因$\forall Z^a\in \mathfrak{X}(M)$,故有
\begin{subequations}\label{chrg:eqn_comformal-Riemann}
\begin{align}
    \tilde{R}_{cab}^d=& R_{cab}^d + 2 {\nabla}_{[a} \Xi_{b]c}^d  + 2\Xi_{e[a}^d\Xi_{b]c}^e
    \label{chrg:eqn_comformal-Riemann-a}\\
%    \xlongequal{\ref{chrg:eqn_comformal-Xi}}&R_{cab}^d
%    -2\delta_{[a}^d{\nabla}_{b]}  \partial_c \ln \Omega + 2{g}^{df} {g}_{c[a} {\nabla}_{b]} \partial_f \ln\Omega
%    +2 (\partial_c \ln\Omega) \delta_{[a}^d\partial_{b]} \ln\Omega  \notag \\
%    &- 2 {g}_{c[a} (\partial_{b]} \ln \Omega)  {g}^{df}\partial_f \ln\Omega
%    - 2\delta_{[a}^d {g}_{b]c} {g}^{ef}(\partial_e \ln\Omega)\partial_f \ln\Omega
%    \label{chrg:eqn_comformal-Riemann-b} \\
    \xlongequal{\ref{chrg:eqn_comformal-Xi-nabla}}& R_{cab}^d -2\delta_{[a}^d{\nabla}_{b]}  \nabla_c \ln \Omega
    + 2{g}^{df} {g}_{c[a} {\nabla}_{b]} \nabla_f \ln\Omega
    + 2(\nabla_c \ln\Omega) \delta_{[a}^d\nabla_{b]} \ln\Omega  \notag \\
    &- 2 {g}_{c[a} (\nabla_{b]} \ln \Omega)  {g}^{df}\nabla_f \ln\Omega
    - 2\delta_{[a}^d {g}_{b]c} {g}^{ef}(\nabla_e \ln\Omega)\nabla_f \ln\Omega .
    \label{chrg:eqn_comformal-Riemann-c}
\end{align}
\end{subequations}
上式计算过程并不复杂,请读者补齐计算.
%式\eqref{chrg:eqn_comformal-Riemann-a}到\eqref{chrg:eqn_comformal-Riemann-b}的计算过程
%\begin{align*}
%    2 {\nabla}_{[a} \Xi_{b]c}^d  =&  2\delta_{[b}^d{\nabla}_{a]}  \partial_c \ln \Omega +
%     2\delta_c^d {\nabla}_{[a} \partial_{b]} \ln\Omega - 2{g}^{df} {g}_{c[b} {\nabla}_{a]} \partial_f \ln\Omega
%  \\=& -2\delta_{[a}^d{\nabla}_{b]}  \partial_c \ln \Omega + 2{g}^{df} {g}_{c[a} {\nabla}_{b]} \partial_f \ln\Omega
%  \\  2\Xi_{e[a}^d\Xi_{b]c}^e = &
%      \delta_b^d (\partial_a \ln \Omega)  \partial_c \ln \Omega
%      + \delta_a^d (\partial_b \ln\Omega )\partial_c \ln \Omega
%      - {g}_{ba} {g}^{df} (\partial_f \ln\Omega)  \partial_c \ln \Omega
% \\& +\delta_c^d (\partial_a \ln \Omega ) \partial_b \ln\Omega
% + \delta_a^d (\partial_c \ln\Omega) \partial_b \ln\Omega
%  - {g}_{ca} {g}^{df}(\partial_f \ln\Omega ) \partial_b \ln\Omega
% \\& - (\partial_a \ln \Omega) {g}_{bc} {g}^{dh}\partial_h \ln\Omega
%  - \delta_a^d {g}_{bc} {g}^{eh}(\partial_e \ln\Omega)\partial_h \ln\Omega
%  +  {g}^{df}(\partial_f \ln\Omega) {g}_{bc} \partial_a \ln\Omega
%\\ \xlongequal{\text{加上[]}}&
%   2 (\partial_c \ln\Omega) \delta_{[a}^d\partial_{b]} \ln\Omega
%  + 2(\partial_{[a} \ln \Omega) {g}_{b]c} {g}^{dh}\partial_h \ln\Omega
%  - 2\delta_{[a}^d {g}_{b]c} {g}^{eh}(\partial_e \ln\Omega)\partial_h \ln\Omega
%\end{align*}
%\begin{align*}
%    & \Xi_{bc}^d \equiv \delta_b^d \partial_c \ln \Omega + \delta_c^d \partial_b \ln\Omega - {g}_{bc} {g}^{df}\partial_f \ln\Omega
%  \\& \Xi_{bc}^e \equiv \delta_b^e \partial_c \ln \Omega + \delta_c^e \partial_b \ln\Omega - {g}_{bc} {g}^{eh}\partial_h \ln\Omega
%  \\& \Xi_{ea}^d \equiv \delta_e^d \partial_a \ln \Omega + \delta_a^d \partial_e \ln\Omega - {g}_{ea} {g}^{df}\partial_f \ln\Omega
%  \\& \Xi_{ab}^c \equiv \delta_a^c \partial_b \ln \Omega + \delta_b^c \partial_a \ln\Omega - {g}_{ab} {g}^{ce}\partial_e \ln\Omega
%\end{align*}

式\eqref{chrg:eqn_comformal-Riemann-c}收缩后得Ricci曲率间的差异是:
\begin{equation}\label{chrg:eqn_comformal-Ricci}
    \begin{aligned}
        \tilde{R}_{bc}=& R_{bc} - {g}_{bc} {g}^{ef}{\nabla}_{e} {\nabla}_f \ln\Omega
        -(m-2) {\nabla}_{b}  {\nabla}_c \ln \Omega \\
        & + (m-2) ({\nabla}_b \ln\Omega) {\nabla}_{c} \ln\Omega
        - (m-2) {g}_{bc} {g}^{ef}({\nabla}_e \ln\Omega){\nabla}_f \ln\Omega .
    \end{aligned}
\end{equation}
%Ricci曲率的具体计算
%\begin{align*}
%    \tilde{R}_{cab}^d=& R_{cab}^d
%    -\delta_{a}^d{\nabla}_{b}  \partial_c \ln \Omega + \delta_{b}^d{\nabla}_{a}  \partial_c \ln \Omega
%    + {g}^{df} {g}_{ca} {\nabla}_{b} \partial_f \ln\Omega -{g}^{df} {g}_{cb} {\nabla}_{a} \partial_f \ln\Omega \\
%    & + (\partial_c \ln\Omega) \delta_{a}^d\partial_{b} \ln\Omega -(\partial_c \ln\Omega) \delta_{b}^d\partial_{a} \ln\Omega   \\
%    &- {g}_{ca} (\partial_{b} \ln \Omega)  {g}^{df}\partial_f \ln\Omega
%      +{g}_{cb} (\partial_{a} \ln \Omega)  {g}^{df}\partial_f \ln\Omega \\
%    &- \delta_{a}^d {g}_{bc} {g}^{ef}(\partial_e \ln\Omega)\partial_f \ln\Omega
%    +\delta_{b}^d {g}_{ac} {g}^{ef}(\partial_e \ln\Omega)\partial_f \ln\Omega \\
%    \xLongrightarrow{\text{收缩}}
%        \tilde{R}_{cab}^a=& R_{cab}^a
%    -m {\nabla}_{b}  \partial_c \ln \Omega + {\nabla}_{b}  \partial_c \ln \Omega
%    + {\nabla}_{b} \partial_c \ln\Omega - {g}_{cb} {\nabla}^{f} \partial_f \ln\Omega \\
%    & + m (\partial_c \ln\Omega) \partial_{b} \ln\Omega -(\partial_c \ln\Omega) \partial_{b} \ln\Omega   \\
%    &-  (\partial_{b} \ln \Omega)  \partial_c \ln\Omega
%    +{g}_{cb} (\partial_{a} \ln \Omega)  {g}^{af}\partial_f \ln\Omega \\
%    &- m {g}_{bc} {g}^{ef}(\partial_e \ln\Omega)\partial_f \ln\Omega
%    + {g}_{bc} {g}^{ef}(\partial_e \ln\Omega)\partial_f \ln\Omega   \\
%    =& R_{bc}-(m-2) {\nabla}_{b}  \partial_c \ln \Omega
%     - {g}_{bc} {g}^{ef}{\nabla}_{e} \partial_f \ln\Omega \\
%    & + (m-2) (\partial_c \ln\Omega) \partial_{b} \ln\Omega
%    - (m-2) {g}_{bc} {g}^{ef}(\partial_e \ln\Omega)\partial_f \ln\Omega
%\end{align*}
再次收缩得标量曲率间的差异是(需用度规,所以有系数$\Omega^2$)
\begin{equation}\label{chrg:eqn_comformal-R-scalar}
   \Omega^2\tilde{R}= R -2(m-1) g^{bc} {\nabla}_{b}  {\nabla}_c \ln \Omega
        -(m-1)(m-2)  {g}^{bc}({\nabla}_b \ln\Omega){\nabla}_c \ln\Omega .
\end{equation}
%标量曲率的具体计算
%\begin{align*}
%    \tilde{R}_{bc}=& R_{bc} - {g}_{bc} {g}^{ef}{\nabla}_{e} \partial_f \ln\Omega
%    -(m-2) {\nabla}_{b}  \partial_c \ln \Omega \\
%    & + (m-2) (\partial_b \ln\Omega) \partial_{c} \ln\Omega
%    - (m-2) {g}_{bc} {g}^{ef}(\partial_e \ln\Omega)\partial_f \ln\Omega
%    \\ \Rightarrow
%    \Omega^2\tilde{R}=& R -2(m-1) g^{bc} {\nabla}_{b}  \partial_c \ln \Omega
%    -(m-1)(m-2)  {g}^{ef}(\partial_e \ln\Omega)\partial_f \ln\Omega
%\end{align*}
$\Tpq{1}{3}$型Weyl张量(可由式\eqref{chrg:eqn_WeylConform-d4}将指标升上来得到)间的差异是
\begin{equation}\label{chrg:eqn_comformal-Weyl}
    \tilde{C}_{\hphantom{a} bcd}^a = C_{\hphantom{a} bcd}^a .
\end{equation}
上式计算较为繁琐,但无需任何技巧,直接把式\eqref{chrg:eqn_comformal-Riemann}、
\eqref{chrg:eqn_comformal-Ricci}和\eqref{chrg:eqn_comformal-R-scalar}带入
式\eqref{chrg:eqn_WeylConform-d4}硬算便是;
注意利用:因联络无挠,故$\nabla_b \nabla_c \ln \Omega =\nabla_c \nabla_b \ln \Omega$.
在共形映射下,$\Tpq{1}{3}$型Weyl张量是不变的,所它也被称为{\heiti 共形张量}.
\index[physwords]{共形变换!共形张量|see{Weyl张量}}
需要注意其它类型Weyl张量未必是共形不变的,比如$\Tpq{0}{4}$型的就不是,
即$\tilde{C}_{abcd} =\Omega^2 C_{abcd}$.
%具体计算过程是
%\setlength{\mathindent}{0em}
%\begin{align*}
%    \tilde{C}_{bcd}^a-&C_{bcd}^a{=}
%    \tilde{R}_{bcd}^a- \frac{1}{m-2}\left( \delta_c^a \tilde{R}_{bd} - \delta_d^a \tilde{R}_{bc}
%    + \tilde{g}_{bd}\tilde{R}_{ce}\tilde{g}^{ea} - \tilde{g}_{bc}\tilde{R}_{de}\tilde{g}^{ea}\right)
%    + \frac{1}{(m-1)(m-2)}\left(\delta_c^a \tilde{g}_{bd}-\delta_d^a \tilde{g}_{bc}\right) \tilde{R} \\
%    & -R_{bcd}^a + \frac{1}{m-2}\left( \delta_c^a R_{bd} - \delta_d^a R_{bc}+ g_{bd}R_{ce}{g}^{ea}
%        - g_{bc}R_{de}{g}^{ea}\right)  + \frac{1}{(m-1)(m-2)}\left(\delta_c^a g_{bd}-\delta_d^a g_{bc}\right) R \\
%    =& -2\delta_{[c}^a{\nabla}_{d]}  \partial_b \ln \Omega + 2{g}^{af} {g}_{b[c} {\nabla}_{d]} \partial_f \ln\Omega
%    +2 (\partial_b \ln\Omega) \delta_{[c}^a\partial_{d]} \ln\Omega   \\
%     &\quad- 2 {g}_{b[c} (\partial_{d]} \ln \Omega)  {g}^{af}\partial_f \ln\Omega
%    - 2\delta_{[c}^a {g}_{d]b} {g}^{ef}(\partial_e \ln\Omega)\partial_f \ln\Omega \\
%    &- \frac{1}{m-2} \bigg[\delta_c^a \Bigl( - {g}_{bd} {g}^{ef}{\nabla}_{e} \partial_f \ln\Omega
%    -(m-2) {\nabla}_{b}  \partial_d \ln \Omega \\
%    & \quad + (m-2) (\partial_b \ln\Omega) \partial_{d} \ln\Omega
%    - (m-2) {g}_{bd} {g}^{ef}(\partial_e \ln\Omega)\partial_f \ln\Omega \Bigr) \\
%    &-\delta_d^a \Bigl(- {g}_{bc} {g}^{ef}{\nabla}_{e} \partial_f \ln\Omega
%    -(m-2) {\nabla}_{b}  \partial_c \ln \Omega \\
%    & \quad + (m-2) (\partial_b \ln\Omega) \partial_{c} \ln\Omega
%    - (m-2) {g}_{bc} {g}^{ef}(\partial_e \ln\Omega)\partial_f \ln\Omega\Bigr) \\
%    &+ g_{bd}{g}^{ha} \Bigl(- {g}_{hc} {g}^{ef}{\nabla}_{e} \partial_f \ln\Omega
%    -(m-2) {\nabla}_{h}  \partial_c \ln \Omega \\
%    & \quad + (m-2) (\partial_h \ln\Omega) \partial_{c} \ln\Omega
%    - (m-2) {g}_{hc} {g}^{ef}(\partial_e \ln\Omega)\partial_f \ln\Omega\Bigr) \\
%    &- g_{bc}{g}^{ha} \Bigl(- {g}_{dh} {g}^{ef}{\nabla}_{e} \partial_f \ln\Omega
%    -(m-2) {\nabla}_{d}  \partial_h \ln \Omega \\
%    & \quad + (m-2) (\partial_d \ln\Omega) \partial_{h} \ln\Omega
%    - (m-2) {g}_{dh} {g}^{ef}(\partial_e \ln\Omega)\partial_f \ln\Omega\Bigr)     \bigg] \\
%    &+ \frac{1}{(m-1)(m-2)} \bigg[(\delta_c^a g_{bd}-\delta_d^a g_{bc})
%    \Bigl(-2(m-1) g^{ef} {\nabla}_{e}  \partial_f \ln \Omega
%    -(m-1)(m-2)  {g}^{ef}(\partial_e \ln\Omega)\partial_f \ln\Omega\Bigr)\bigg] \\
%    =& -{\color{orange} \delta_{c}^a{\nabla}_{d}  \partial_b \ln \Omega }
%       +{\color{yellow} \delta_{d}^a{\nabla}_{c}  \partial_b \ln \Omega }
%       +{\color{pink} {g}^{af} {g}_{bc} {\nabla}_{d} \partial_f \ln\Omega }
%       -{\color{magenta} {g}^{af} {g}_{bd} {\nabla}_{c} \partial_f \ln\Omega }\\
%    &+ {\color{blue} (\partial_b \ln\Omega) \delta_{c}^a\partial_{d} \ln\Omega  }
%     - {\color{green} (\partial_b \ln\Omega) \delta_{d}^a\partial_{c} \ln\Omega }  \\
%    &- {\color{violet}{g}_{bc} (\partial_{d} \ln \Omega)  {g}^{af}\partial_f \ln\Omega }
%     + {g}_{bd} (\partial_{c} \ln \Omega)  {g}^{af}\partial_f \ln\Omega \\
%    &- {\color{cyan}\delta_{c}^a {g}_{db} {g}^{ef}(\partial_e \ln\Omega)\partial_f \ln\Omega
%     +  \delta_{d}^a {g}_{cb} {g}^{ef}(\partial_e \ln\Omega)\partial_f \ln\Omega } \\
%    & + { \color{red} \frac{1}{m-2} \delta_c^a{g}_{bd} {g}^{ef}{\nabla}_{e} \partial_f \ln\Omega }
%    + {\color{orange} \delta_c^a {\nabla}_{b}  \partial_d \ln \Omega }
%    - {\color{blue} \delta_c^a (\partial_b \ln\Omega) \partial_{d} \ln\Omega }
%    + { \color{cyan} \delta_c^a{g}_{bd} {g}^{ef}(\partial_e \ln\Omega)\partial_f \ln\Omega } \\
%    & + {\color{red} \frac{1}{m-2}g_{bd}\delta_c^a {g}^{ef}{\nabla}_{e} \partial_f \ln\Omega }
%     + {\color{magenta} g_{bd}{g}^{ha} {\nabla}_{h}  \partial_c \ln \Omega }
%     - g_{bd}{g}^{ha} (\partial_h \ln\Omega) \partial_{c} \ln\Omega
%     +{\color{cyan} \delta_c^a g_{bd}{g}^{ef}(\partial_e \ln\Omega)\partial_f \ln\Omega} \\
%    & -{ \color{red} \frac{1}{m-2}\delta_d^a{g}_{bc} {g}^{ef}{\nabla}_{e} \partial_f \ln\Omega }
%    - {\color{yellow} \delta_d^a{\nabla}_{b}  \partial_c \ln \Omega }
%    + {\color{green}\delta_d^a(\partial_b \ln\Omega) \partial_{c} \ln\Omega }
%    -  {\color{cyan} \delta_d^a{g}_{bc} {g}^{ef}(\partial_e \ln\Omega)\partial_f \ln\Omega }\\
%    & -{\color{red}  \frac{1}{m-2}\delta_d^a g_{bc}{g}^{ef}{\nabla}_{e} \partial_f \ln\Omega }
%      -{\color{pink} g_{bc}{g}^{ha} {\nabla}_{d}  \partial_h \ln \Omega }
%      +{\color{violet} g_{bc} {g}^{ha} (\partial_d \ln\Omega) \partial_{h} \ln\Omega }
%    - { \color{cyan}\delta_d^a g_{bc}{g}^{ef}(\partial_e \ln\Omega)\partial_f \ln\Omega   } \\
%    &- {\color{red} 2\frac{1}{m-2} (\delta_c^a g_{bd}-\delta_d^a g_{bc}) g^{ef} {\nabla}_{e}  \partial_f \ln \Omega }
%     - {\color{cyan} (\delta_c^a g_{bd}-\delta_d^a g_{bc}) {g}^{ef}(\partial_e \ln\Omega)\partial_f \ln\Omega } \\
%    =&0
%\end{align*}\setlength{\mathindent}{2em}
%需要注意:因$\ln \Omega$是标量函数场,在局部$\partial_b \ln \Omega = \nabla_b\ln \Omega$.
%但联络是无挠的,所以${\nabla}_{d}  \nabla_b \ln \Omega =\nabla_{b}  \nabla_d \ln \Omega $.

\index[physwords]{共形变换!平坦}

%\paragraph{共形平坦}
\begin{definition}
    如果$m$维广义黎曼流形$(M,g)$上对于每一点$p$,都存在包含$p$的开邻域$U$以及$U$上的平坦
    度量$\tilde{g}$(即它的黎曼曲率$\tilde{R}^a_{bcd}$恒为零),使得在$U$上$g$和$\tilde{g}$是共形的,
    则称$(M,g)$是局部{\heiti 共形平坦}的黎曼流形.
\end{definition}

\begin{theorem}\label{chrg:thm_conformal-flat}
    $m(>3)$维广义黎曼流形局部共形平坦的充要条件是$C^a_{\hphantom{a} bcd}$恒为零.
\end{theorem}
\begin{proof}
    参见文献\parencite[\S 3.3]{baizg-2004-irg};$m= 3$的情形也可参见上述文献.
\end{proof}

二维流形则一定局部共形平坦,我们将其分为两个定理.

\begin{proposition}\label{chrg:thm_exist-oth-coord}
	对于度规正定的、可嵌入$E^3$中的二维流形来说,一定存在局部坐标系$\{x,y\}$使得
	度规满足${\rm d}s^2= \Omega(x,y)({\rm d}x^2 + {\rm d}y^2)$,其中$\Omega(x,y)$恒正.
\end{proposition}
\begin{proof}
	这是古典微分几何中的一个经典定理,在很多文献中都有证明.
	这里给出一个物理类文献\parencite[\S 11]{chandrasekhar-1983}.
	需注意:当流形维数大于$2$时,此定理未必成立.
	此定理只在{\kaishu 局部}成立,未必在整个流形上正确.   
\end{proof}

\begin{proposition}\label{chrg:thm_exist-oth-coord-lorentz}
	对于度规场是Lorentz型的二维广义黎曼流形来说,一定存在局部坐标系$\{t,x\}$使得
	度规满足${\rm d}s^2= \Omega(t,x)(-{\rm d}t^2 + {\rm d}x^2)$,其中$\Omega(t,x)$恒正.
\end{proposition}
\begin{proof}
	因流形是二维的,设其局部坐标为$\{u^1,u^2\}$,在此坐标系下度规场为
	\begin{equation*}
		{\rm d}s^2 = g_{11} ({\rm d}u^1)^2 + g_{12} {\rm d}u^1 {\rm d}u^2
		+g_{21} {\rm d}u^2 {\rm d}u^1 + g_{22} ({\rm d}u^2)^2 .
	\end{equation*}
	又因度规是Lorentz型的,故流形上一定存在两组类光测地线,
	我们设它们的仿射参数分别为$\lambda^1$、$\lambda^2$;则有
	\begin{align*}
		0=& g_{ab} \left(\frac{\partial }{\partial \lambda^1}\right)^a\left(\frac{\partial }{\partial \lambda^1}\right)^b
		=g_{ab} \left(\frac{\partial }{\partial u^i}\right)^a \left(\frac{\partial }{\partial u^j}\right)^b
		\frac{\partial u^i}{\partial \lambda^1} \frac{\partial u^j}{\partial \lambda^1}
		=g_{ij} \frac{\partial u^i}{\partial \lambda^1} \frac{\partial u^j}{\partial \lambda^1} , \\
		0=& g_{ab} \left(\frac{\partial }{\partial \lambda^2}\right)^a\left(\frac{\partial }{\partial \lambda^2}\right)^b
		=g_{ab} \left(\frac{\partial }{\partial u^i}\right)^a \left(\frac{\partial }{\partial u^j}\right)^b
		\frac{\partial u^i}{\partial \lambda^2} \frac{\partial u^j}{\partial \lambda^2}
		=g_{ij} \frac{\partial u^i}{\partial \lambda^2} \frac{\partial u^j}{\partial \lambda^2} .
	\end{align*}
	由上式可知,在参数$\{\lambda^1,\lambda^2\}$下,新的度规系数$g'_{11}=0=g'_{22}$,$g'_{12}$非零,即
	\begin{equation*}
		g'_{11}=g_{ij}\frac{\partial u^i}{\partial \lambda^1}\frac{\partial u^j}{\partial \lambda^1}=0,\qquad
		g'_{22}=g_{ij}\frac{\partial u^i}{\partial \lambda^2}\frac{\partial u^j}{\partial \lambda^2}=0.
	\end{equation*}
	在新参数$\{\lambda^1,\lambda^2\}$下的度规线元为
	\begin{equation*}
		{\rm d}s^2 = g'_{12} {\rm d}\lambda^1 {\rm d}\lambda^2	+g'_{21} {\rm d}\lambda^2 {\rm d}\lambda^1,
		\qquad \text{且}\  g'_{12} = g'_{21} .
	\end{equation*}
	在任意一点$p$,我们可以认为$g'_{12}$是正的,否则我们取$\lambda'^1=-\lambda^1$即可变其为正.
	由于度规是连续的,一定存在$p$点的一个小邻域$p\in U$使得$g'_{12}$恒正.
	既然在邻域$U$中$g'_{12}$恒正,可令$\mu = \frac{1}{2}  \ln (2 g'_{12})$,
	则有${\rm d}s^2 = e^{2\mu} {\rm d}\lambda^1 {\rm d}\lambda^2$.
	再作变换
	\begin{equation*}
		t = \frac{1}{2}\left(\lambda^1 - \lambda^2 \right),\
		x = \frac{1}{2}\left(\lambda^1 + \lambda^2 \right) ;
		\quad \Leftrightarrow \quad
		\lambda^1 = x+t,\ \lambda^2=x-t.
	\end{equation*}
	最终线元变为${\rm d}s^2 = e^{2\mu} (-{\rm d}t^2 +{\rm d}x^2)$.%这便证明了命题.
\end{proof}





设$m(>3)$维广义黎曼流形$(M,g)$是曲率为$K_0$的常曲率空间,
那么,由式\eqref{chrg:eqn_const-curvature-component}可得
其各种曲率分量表达式
\begin{align}
    R_{ijkl}=&  K_0 ( g_{ik}g_{jl}- g_{il}g_{jk}  ), \\
    R_{jl}=&  K_0 ( m - 1  ) g_{jl}, \\
    R=& K_0 ( m - 1  ) m , \\
    C_{ijkl}=& 0 .
\end{align}
上面已算出常曲率空间的Weyl张量恒为零,由定理\ref{chrg:thm_conformal-flat}可得如下推论:
\begin{corollary}\label{chrg:thm_conformal-flat-const-curvature}
    常曲率广义黎曼流形(维数大于3)必是局部共形平坦的.
\end{corollary}




\begin{exercise}
	证明式\eqref{chrg:eqn_comformal-Weyl}.
\end{exercise}


\section*{小结}
本章主要参考了\parencite{cc2001-zh}、\parencite{chen-li-2023-2ed-v1}相应章节.

本章内容(除\S\ref{chrg:sec_E3MF}外)适用于正定、不定度规.


\printbibliography[heading=subbibliography,title=第\ref{chrg}章参考文献]

\endinput
