% !TeX spellcheck = <none>
% !TeX encoding = UTF-8
% 此文件从2022.3.04开始写作

\chapter{后牛顿近似} \label{chppn}
爱因斯坦引力场方程是非线性的,很难严格求解.
由爱因斯坦本人提出并发展的后牛顿近似方法是处理引力束缚系统
的常用方法.
为了便于比较量级以及使用,本章以及下一章使用国际单位制;
并且基本用分量语言来描述;坐标系则需要根据要处理的问题
进行选取,如无特殊要求,我们选取固有坐标系(见\S\ref{chfd:sec_proper-coord}).

需要强调:所有张量场(除$g_{\mu\nu},\, g^{\mu\nu}$外)都
需用洛伦兹度规$\eta_{\mu\nu},\, \eta^{\mu\nu}$来升降指标.
由于我们采用的度规号差是$(-+++)$,故在三维纯空间部分,
当用$\eta_{\mu\nu}$(或$\delta_{ij}$)升降指标时,只有象征意义,
不会产生额外的正负号.正是这个原因,在纯空间部分本章很多公式等号两端
的上下指标不匹配.

本章主要参考了文献\parencite[Ch. 9]{weinberg_grav-1972}.
%\parencite[Ch. 7-8]{xu-wu-1999},
%\parencite{poisson-will-2014}.

\section{量级估算}
只有清楚了各物理量的数量级,才能作可靠的近似.
考虑一个由引力束缚在一起的质点系统(如太阳系)
$M$、$r$、$v$分别代表质点的质量、平均距离和速度.
根据位力定理
\begin{equation}
    v^2 \approx \frac{GM}{r} \equiv U .
\end{equation}
其中$U$是引力势函数,
\begin{equation}
    U = -\phi = +\int \frac{G \rho(\boldsymbol{y})}{|\boldsymbol{x}-\boldsymbol{y}|} {\rm d}^3y
      = +\frac{G M }{r} .
\end{equation}
后牛顿近似中更喜欢用$U$,而不是$\phi$.
对于太阳表面而言
\begin{equation}
    \frac{v^2}{c^2}=\frac{G M_{\odot} }{c^2 R_{\odot}} \approx 2 \times 10^{-6} .
\end{equation}
对于地球表面而言
\begin{equation}
    \frac{v^2}{c^2}=\frac{G M_{\oplus} }{c^2 R_{\oplus}} \approx 7 \times 10^{-10} .
\end{equation}
可见不论太阳还是地球表面都可看成弱引力场,
在此区域飞行的质点速度$v/c$都不高.令
\begin{equation}\label{chppn:eqn_vdc}
    \bar{v}\equiv \frac{v}{c}.
\end{equation}
我们将借助参量$\bar{v}$将各种物理量展开.

物质内能可通过压强来表述,单位质量内能与引力势处于同一量级,
\begin{equation}
    \frac{E_{\text{内能}}}{c^2} \sim \frac{p}{c^2 \rho}\sim
    \frac{\phi}{c^2} \sim \bar{v}^2
\end{equation}

系统中的距离和时间尺度分别$r$和$r/v$,故导数量级大约是
\begin{equation}
    \frac{\partial}{\partial x^i}\sim \frac{1}{r},\qquad
    \frac{\partial}{\partial x^0}\sim \frac{\partial}{c \partial t}
    \sim \frac{v}{cr} \sim  \bar{v} \frac{\partial}{\partial x^i}.
\end{equation}


后牛顿近似本质上是对$\bar{v}$(式\eqref{chppn:eqn_vdc})的近似展开,
也经常说成是对光速$c^{-1}$展开.展开后精确到不同量级,
比如$O(\bar{v}),\, O(\bar{v}^2),\, O(\bar{v}^4),\cdots$,
这些量级会在公式中有显示表示;通常我们会把上述量级简记
为$O(1),\, O(2),\, O(4),\cdots$.


\section{基本方法}
基本出发点是引力场方程和自由质点运动方程(即测地线方程).
先讨论测地线方程
\begin{equation}
    \frac{{\rm d}^2 x^\mu}{{\rm d}\tau ^2}+ {\Gamma}_{\nu\lambda}^{\mu}
      \frac{{\rm d} x^\nu}{{\rm d}\tau} \frac{{\rm d} x^\lambda}{{\rm d}\tau} =0,
\end{equation}
其中$\tau$是固有时.由此出发可以算出空间的三维加速度是(注$x^0=ct$)
\begin{align}
    \frac{{\rm d}^2 x^i}{{\rm d} t^2} =& \frac{{\rm d} \tau}{{\rm d}t}
       \frac{{\rm d} }{{\rm d}\tau} \left( \frac{{\rm d} \tau}{{\rm d}t}
       \frac{{\rm d} x^i}{{\rm d} \tau} \right)
    =\left(\frac{{\rm d} t}{{\rm d}\tau} \right)^{-2} \frac{{\rm d}^2 x^i}{{\rm d} \tau^2}
     -\left(\frac{{\rm d} t}{{\rm d}\tau} \right)^{-3}
     \frac{{\rm d}^2 x^0}{c\, {\rm d} \tau^2} \frac{{\rm d} x^i}{{\rm d} \tau}  \notag \\
    =&-{\Gamma}_{\nu\lambda}^{i} \frac{{\rm d} x^\nu}{{\rm d}t} \frac{{\rm d} x^\lambda}{{\rm d}t}
      +\frac{1}{c} {\Gamma}_{\nu\lambda}^{0} \frac{{\rm d} x^i}{{\rm d}t}
       \frac{{\rm d} x^\nu}{{\rm d}t} \frac{{\rm d} x^\lambda}{{\rm d}t} \notag \\
    =& -c^2 \Gamma_{00}^{i} - 2c \Gamma_{0j}^{i}\frac{{\rm d} x^j}{{\rm d}t}
       -\Gamma_{jk}^{i} \frac{{\rm d} x^j}{{\rm d}t} \frac{{\rm d} x^k}{{\rm d}t} \notag \\
      & +\frac{1}{c}\frac{{\rm d} x^i}{{\rm d}t} \left(
        c^2 \Gamma_{00}^{0} + 2c \Gamma_{0j}^{0}\frac{{\rm d} x^j}{{\rm d}t}
       +\Gamma_{jk}^{0} \frac{{\rm d} x^j}{{\rm d}t}
       \frac{{\rm d} x^k}{{\rm d}t}  \right) . \label{chppn:eqn_PN}
\end{align}
回顾\S\ref{chle:sec_Newton-limit},当$\bar{v} \ll 1$时,我们
只保留一阶项,有$g_{00}=-(1+ 2\phi/c^2)$和$g_{0i}=0,\, g_{ij}=\delta_{ij}$,
那么牛顿第二定律和万有引力公式结合之后是
\begin{equation}
    \frac{{\rm d}^2 x^i}{{\rm d}t^2}=-c^2{\Gamma}_{00}^i
    =\frac{c^2}{2} \frac{\partial g_{00}}{\partial x^i}
    =-\frac{\partial \phi}{\partial x^i} .
\end{equation}
由此可见牛顿近似中$\frac{{\rm d}^2 x^i}{{\rm d}t^2}$准确到量级
$\frac{G M}{r^2} \sim \frac{v^2}{r}$.
也就是$\Gamma_{00}^i\sim \frac{G M}{c^2 r^2} \sim \frac{\bar{v}^2}{r} $.
在我们这个后牛顿近似的初步介绍中,
只要求式\eqref{chppn:eqn_PN}中其它各项
准确到$\frac{\bar{v}^4}{r}$量级(忽略更高量级的项),那么
对Christoffel符号诸分量的要求是
\begin{subequations}
\begin{align}
    & \Gamma_{00}^i {\quad \text{须准确到量级}\quad } \frac{\bar{v}^4}{r},  \\
    & \Gamma_{0j}^i {\quad \text{须准确到量级}\quad } \frac{\bar{v}^3}{r},  \\
    & \Gamma_{jk}^i {\quad \text{须准确到量级}\quad } \frac{\bar{v}^2}{r},  \\
    & \Gamma_{00}^0 {\quad \text{须准确到量级}\quad } \frac{\bar{v}^3}{r},  \\
    & \Gamma_{0j}^0 {\quad \text{须准确到量级}\quad } \frac{\bar{v}^2}{r},  \\
    & \Gamma_{jk}^0 {\quad \text{须准确到量级}\quad } \frac{\bar{v}}{r}.
\end{align}
\end{subequations}
以上只是对有质量粒子的要求,对于光子的后牛顿近似另行讨论.

\subsection{度规场近似}
因为克氏符是由度规来表述的,下面我们来看度规的近似展开.
\begin{equation}
    \frac{{\rm d} s^2}{{\rm d}t^2} = c^2 g_{00} + 2 c g_{0i} \frac{{\rm d} x^i}{{\rm d}t}
     + g_{ij} \frac{{\rm d} x^i}{{\rm d}t}  \frac{{\rm d} x^j}{{\rm d}t} .
\end{equation}
后牛顿方法是弱场、低速近似,度规场不会偏离洛伦兹度规太远,故假设度规可以展开为
\begin{subequations}\label{chppn:eqn_gdab}
\begin{align}
    g_{00} =& -1 + \oversetmy{2}{g}_{00}+\oversetmy{4}{g}_{00}+ \cdots,  \\
    g_{ij} =& \delta_{ij} + \oversetmy{2}{g}_{ij}+\oversetmy{4}{g}_{ij}+ \cdots,  \\
    g_{i0} =& \oversetmy{3}{g}_{i0}+\oversetmy{5}{g}_{i0}+ \cdots
\end{align}
\end{subequations}
{\kaishu 其中$\oversetmy{N}{g}_{\mu\nu}$表示$g_{\mu\nu}$中
    量级为$\bar{v}^N$(见式\eqref{chppn:eqn_vdc})的近似项;
其它量也采取类似的表述方式.} 因为在时间反演下${\rm d} s^2$必然是个不变量,
而速度$v^i=\frac{{\rm d} x^i}{{\rm d}t}$是要改变符号的,
故式\eqref{chppn:eqn_gdab}中$g_{0i}$必须是$\bar{v}$的奇数次幂;
而$g_{00},\, g_{ij}$只能包含$\bar{v}$的偶数次幂项.
如果有辐射阻尼之类的作用,那么可能打破上述奇偶规律;遇到该问题时再行讨论.

我们知道度规场有关系式:$g^{\mu\rho}g_{\rho\nu}=\delta^\mu_\nu$;把它展开
\begin{subequations}
\begin{align}
    g^{0\rho}g_{\rho 0}=& g^{00}g_{0 0}+g^{0k}g_{k 0} = 1, \qquad \text{一个方程} \\
    g^{0\rho}g_{\rho j}=& g^{00}g_{0 j}+g^{0k}g_{k j} = 0, \qquad \text{三个方程} \\
    g^{i\rho}g_{\rho j}=& g^{i0}g_{0 j}+g^{ik}g_{k j} =\delta^i_j. \qquad \text{六个方程}
\end{align}
\end{subequations}
我们预估逆变度规有如下形式(为了方便引用,第二个等号是把
式\eqref{chppn:eqn_gugd}带入后的结果)
\begin{subequations}\label{chppn:eqn_guab}
\begin{align}
    g^{00} =& -1 + \oversetmy{2}{g}^{00}+\oversetmy{4}{g}^{00}+ \cdots
      &=& -1 -\oversetmy{2}{g}_{00} -\oversetmy{4}{g}_{00}-(\oversetmy{2}{g}_{00})^2 +\cdots  \\
    g^{ij} =& \delta^{ij} + \oversetmy{2}{g}^{ij}+ \cdots
      &=&  \delta_{ij}  -\oversetmy{2}{g}_{ij} + \cdots \\
    g^{i0} =& \oversetmy{3}{g}^{i0}+ \cdots
      &=& \oversetmy{3}{g}_{0i} + \cdots
\end{align}
\end{subequations}
利用互逆关系($g^{\mu\rho}g_{\rho\nu}=\delta^\mu_\nu$),把式\eqref{chppn:eqn_guab}第二个等号前的
表达式代入式\eqref{chppn:eqn_gdab},有
\begin{equation}\label{chppn:eqn_gugd}
    \oversetmy{2}{g}^{00} = -\oversetmy{2}{g}_{00}, \quad
    \oversetmy{4}{g}^{00} = -\oversetmy{4}{g}_{00}-(\oversetmy{2}{g}_{00})^2, \quad
    \oversetmy{2}{g}^{ij} = -\oversetmy{2}{g}_{ij}, \quad
%    \oversetmy{4}{g}^{ij} = -\oversetmy{4}{g}_{ij}+ \oversetmy{2}{g}_{ik}\oversetmy{2}{g}_{kj}, \quad
    \oversetmy{3}{g}^{0i} = +\oversetmy{3}{g}_{0i}, \quad    \cdots
\end{equation}
注意:得到上式后,再代入式\eqref{chppn:eqn_guab}才得到第二个等号后面的公式.

\subsection{Christoffel符号近似}
克氏符定义为
%\begin{equation}
$    \Gamma^\mu_{\lambda\nu}=\frac{1}{2}g^{\mu\rho}(
      \frac{\partial g_{\rho\nu}}{\partial x^\lambda}
     +\frac{\partial g_{\lambda\rho}}{\partial x^\nu}
     -\frac{\partial g_{\lambda\nu}}{\partial x^\rho} ) $.
%\end{equation}
从而可以求出
\begin{align*}
    \Gamma^i_{00}=&\frac{1}{2}g^{i0}\left(
     \frac{\partial g_{0 0}}{\partial x^0}
    +\frac{\partial g_{0 0}}{\partial x^0}
    -\frac{\partial g_{00}}{\partial x^0} \right)
    +\frac{1}{2}g^{ij}\left(
    \frac{\partial g_{j 0}}{\partial x^0}
    +\frac{\partial g_{0 j}}{\partial x^0}
    -\frac{\partial g_{00}}{\partial x^j} \right) \\
%    =&\frac{1}{2} \oversetmy{3}{g}^{i0} \left(
%     \frac{\partial (-1+\oversetmy{2}{g}_{00})}{\partial x^0}
%    +\frac{\partial (-1+\oversetmy{2}{g}_{00})}{\partial x^0}
%    -\frac{\partial (-1+\oversetmy{2}{g}_{00})}{\partial x^0} \right)
%    +\frac{1}{2}(\delta_{ij}+\oversetmy{2}{g}^{ij})\left(
%     \frac{\partial \oversetmy{3}{g}_{j0}}{\partial x^0}
%    +\frac{\partial \oversetmy{3}{g}_{j0}}{\partial x^0}
%    -\frac{\partial (-1+\oversetmy{2}{g}_{00}+\oversetmy{4}{g}_{00})}
%       {\partial x^j} \right) \\
    =& \frac{1}{2}(\delta_{ij}+\oversetmy{2}{g}^{ij})\left(
    2\frac{\partial \oversetmy{3}{g}_{j0}}{\partial x^0}
    -\frac{\partial (\oversetmy{2}{g}_{00}+\oversetmy{4}{g}_{00})}
      {\partial x^j} \right) + O(5)\\
    =& \frac{1}{2} \left(
    2\frac{\partial \oversetmy{3}{g}_{i0}}{\partial x^0}
    -\frac{\partial (\oversetmy{2}{g}_{00}+\oversetmy{4}{g}_{00})} {\partial x^i}
    -\oversetmy{2}{g}^{ij} \frac{\partial \oversetmy{2}{g}_{00}}{\partial x^j}
      \right) + O(5)
\end{align*}
利用式\eqref{chppn:eqn_gugd},由上式可得
\begin{equation}
    \Gamma^i_{00}= \frac{1}{2} \left(
    -\frac{\partial \oversetmy{2}{g}_{00}} {\partial x^i}
    +2\frac{\partial \oversetmy{3}{g}_{i0}}{\partial x^0}
    -\frac{\partial \oversetmy{4}{g}_{00}} {\partial x^i}
    +\oversetmy{2}{g}_{ij} \frac{\partial \oversetmy{2}{g}_{00}}{\partial x^j}
    \right) + O(5)
\end{equation}
与上述过程类似,可以得到如下各阶克氏符近似表达式
\begin{subequations}\label{chppn:eqn_Gammag}
\begin{align}
    \oversetmy{2}{\Gamma}^i_{00} = & -\frac{1}{2}\frac{\partial \oversetmy{2}{g}_{00}} {\partial x^i} ;\\
    \oversetmy{4}{\Gamma}^i_{00} = & -\frac{1}{2}\frac{\partial \oversetmy{4}{g}_{00}} {\partial x^i}
       + \frac{\partial \oversetmy{3}{g}_{0i}} {\partial x^0}
       + \frac{1}{2}\oversetmy{2}{g}_{ij}\frac{\partial \oversetmy{2}{g}_{00}} {\partial x^i} ;  \\
    \oversetmy{3}{\Gamma}^i_{0j} = & \frac{1}{2}\left(
         \frac{\partial \oversetmy{3}{g}_{0i}} {\partial x^j}
       + \frac{\partial \oversetmy{2}{g}_{ij}} {\partial x^0}
       - \frac{\partial \oversetmy{3}{g}_{0j}} {\partial x^i} \right) ; \\
    \oversetmy{2}{\Gamma}^i_{jk} = & \frac{1}{2}\left(
      \frac{\partial \oversetmy{2}{g}_{ij}} {\partial x^k}
    + \frac{\partial \oversetmy{2}{g}_{ik}} {\partial x^j}
    - \frac{\partial \oversetmy{2}{g}_{jk}} {\partial x^i} \right) ; \\
    \oversetmy{3}{\Gamma}^0_{00} = & -\frac{1}{2}\frac{\partial \oversetmy{2}{g}_{00}} {\partial x^0} ; \\
    \oversetmy{2}{\Gamma}^0_{0j} = & -\frac{1}{2}\frac{\partial \oversetmy{2}{g}_{00}} {\partial x^j} ; \\
    \oversetmy{1}{\Gamma}^0_{ij} = & 0 =\oversetmy{2}{\Gamma}^0_{ij}, \qquad
    \oversetmy{3}{\Gamma}^0_{ij} = -\frac{1}{2} \left(
    \frac{\partial \oversetmy{3}{g}_{0j} }{\partial x^k}
    +\frac{\partial \oversetmy{3}{g}_{0k}}{\partial x^j}
    -\frac{\partial \oversetmy{2}{g}_{jk} }{\partial x^0} \right) .
\end{align}
\end{subequations}

显然,我们必须知道度规分量$g_{ij}$准确到量级$\bar{v}^2$,
$g_{i0}$准确到量级$\bar{v}^3$,$g_{00}$准确到量级$\bar{v}^4$.
应当把这一点同纯牛顿力学对照一下,在那里我们
需要$g_{00}$精确到量级$\bar{v}^2$;而$g_{i0}=0$,$g_{ij}=\delta_{ij}$,
即只需精确到零级.


%其它各项推导过程
%\begin{align*}
%    \Gamma^i_{0j}=&\frac{1}{2}g^{i0}\left(
%     \frac{\partial g_{0 j}}{\partial x^0}
%    +\frac{\partial g_{0 0}}{\partial x^j}
%    -\frac{\partial g_{0j}}{\partial x^0} \right)
%    +\frac{1}{2}g^{il}\left(
%     \frac{\partial g_{l 0}}{\partial x^j}
%    +\frac{\partial g_{j l}}{\partial x^0}
%    -\frac{\partial g_{0 j}}{\partial x^l} \right) \\
%    =&\frac{1}{2}\oversetmy{3}{g}_{0i} \left(
%    \frac{\partial \oversetmy{3}{g}_{0j} }{\partial x^0}
%    +\frac{\partial \oversetmy{2}{g}_{00} }{\partial x^j}
%    -\frac{\partial \oversetmy{3}{g}_{0j} }{\partial x^0} \right)
%    +\frac{1}{2}(\delta_{il}  -\oversetmy{2}{g}_{il})\left(
%    \frac{\partial \oversetmy{3}{g}_{0l} }{\partial x^j}
%    +\frac{\partial  \oversetmy{2}{g}_{lj} }{\partial x^0}
%    -\frac{\partial \oversetmy{3}{g}_{0j} }{\partial x^l} \right) \\
%    =& \frac{1}{2}  \left(
%    \frac{\partial \oversetmy{3}{g}_{0i} }{\partial x^j}
%    +\frac{\partial  \oversetmy{2}{g}_{ij} }{\partial x^0}
%    -\frac{\partial \oversetmy{3}{g}_{0j} }{\partial x^i} \right)
%\end{align*}
%\begin{align*}
%    \Gamma^i_{jk}=&\frac{1}{2}g^{i0}\left(
%    \frac{\partial g_{0 j}}{\partial x^k}
%    +\frac{\partial g_{0 k}}{\partial x^j}
%    -\frac{\partial g_{0j}}{\partial x^0} \right)
%    +\frac{1}{2}g^{il}\left(
%     \frac{\partial g_{l k}}{\partial x^j}
%    +\frac{\partial g_{j l}}{\partial x^k}
%    -\frac{\partial g_{k j}}{\partial x^l} \right) \\
%    =&\frac{1}{2}\oversetmy{3}{g}_{0i} \left(
%    \frac{\partial \oversetmy{3}{g}_{0j} }{\partial x^k}
%    +\frac{\partial \oversetmy{3}{g}_{0k} }{\partial x^j}
%    -\frac{\partial \oversetmy{3}{g}_{0j}}{\partial x^0} \right)
%    +\frac{1}{2}(\delta_{il}  -\oversetmy{2}{g}_{il})\left(
%    \frac{\partial \oversetmy{2}{g}_{lk} }{\partial x^j}
%    +\frac{\partial \oversetmy{2}{g}_{jl} }{\partial x^k}
%    -\frac{\partial \oversetmy{2}{g}_{jk} }{\partial x^l} \right) \\
%    =&\frac{1}{2}\left(
%    \frac{\partial \oversetmy{2}{g}_{ik} }{\partial x^j}
%    +\frac{\partial \oversetmy{2}{g}_{ji} }{\partial x^k}
%    -\frac{\partial \oversetmy{2}{g}_{jk} }{\partial x^i} \right)
%\end{align*}
%\begin{align*}
%    \Gamma^0_{00}=&\frac{1}{2}g^{00}\left(
%    \frac{\partial g_{0 0}}{\partial x^0}
%    +\frac{\partial g_{0 0}}{\partial x^0}
%    -\frac{\partial g_{00}}{\partial x^0} \right)
%    +\frac{1}{2}g^{0j}\left(
%    \frac{\partial g_{j 0}}{\partial x^0}
%    +\frac{\partial g_{0 j}}{\partial x^0}
%    -\frac{\partial g_{00}}{\partial x^j} \right) \\
%    =&\frac{1}{2}(-1 -\oversetmy{2}{g}_{00})
%    \frac{\partial \oversetmy{2}{g}_{00}}{\partial x^0}
%    +\frac{1}{2}\oversetmy{3}{g}_{0j}\left(
%    \frac{\partial \oversetmy{3}{g}_{j0} }{\partial x^0}
%    +\frac{\partial \oversetmy{3}{g}_{j0}}{\partial x^0}
%    -\frac{\partial \oversetmy{2}{g}_{00}}{\partial x^j} \right)
%    =-\frac{1}{2}\frac{\partial \oversetmy{2}{g}_{00}}{\partial x^0}
%\end{align*}
%\begin{align*}
%    \Gamma^0_{0j}=&\frac{1}{2}g^{00}\left(
%    \frac{\partial g_{0 j}}{\partial x^0}
%    +\frac{\partial g_{0 0}}{\partial x^j}
%    -\frac{\partial g_{0j}}{\partial x^0} \right)
%    +\frac{1}{2}g^{0l}\left(
%    \frac{\partial g_{l 0}}{\partial x^j}
%    +\frac{\partial g_{j l}}{\partial x^0}
%    -\frac{\partial g_{0 j}}{\partial x^l} \right) \\
%    =&\frac{1}{2}(-1 -\oversetmy{2}{g}_{00})
%    \frac{\partial \oversetmy{2}{g}_{00}}{\partial x^j}
%    +\frac{1}{2}\oversetmy{3}{g}_{0l}\left(
%    \frac{\partial \oversetmy{3}{g}_{l0} }{\partial x^j}
%    +\frac{\partial \oversetmy{2}{g}_{lj} }{\partial x^0}
%    -\frac{\partial \oversetmy{3}{g}_{j0} }{\partial x^l} \right)
%    =-\frac{1}{2}\frac{\partial \oversetmy{2}{g}_{00}}{\partial x^j}
%\end{align*}
%\begin{align*}
%    \Gamma^0_{jk}=&\frac{1}{2}g^{00}\left(
%    \frac{\partial g_{0 j}}{\partial x^k}
%    +\frac{\partial g_{0 k}}{\partial x^j}
%    -\frac{\partial g_{jk}}{\partial x^0} \right)
%    +\frac{1}{2}g^{0l}\left(
%    \frac{\partial g_{l k}}{\partial x^j}
%    +\frac{\partial g_{j l}}{\partial x^k}
%    -\frac{\partial g_{k j}}{\partial x^l} \right) \\
%    =&\frac{1}{2}(-1-\oversetmy{2}{g}_{00})\left(
%    \frac{\partial \oversetmy{3}{g}_{0j} }{\partial x^k}
%    +\frac{\partial \oversetmy{3}{g}_{0k}}{\partial x^j}
%    -\frac{\partial \oversetmy{2}{g}_{jk} }{\partial x^0} \right)
%    +\frac{1}{2} \oversetmy{3}{g}_{0l} \left(
%    \frac{\partial \oversetmy{2}{g}_{lk} }{\partial x^j}
%    +\frac{\partial \oversetmy{2}{g}_{jl} }{\partial x^k}
%    -\frac{\partial \oversetmy{2}{g}_{jk} }{\partial x^l} \right)\\
%    =&-\frac{1}{2} \left(
%    \frac{\partial \oversetmy{3}{g}_{0j} }{\partial x^k}
%    +\frac{\partial \oversetmy{3}{g}_{0k}}{\partial x^j}
%    -\frac{\partial \oversetmy{2}{g}_{jk} }{\partial x^0} \right)
%    \sim O(3)
%\end{align*}





\subsection{曲率近似}
下面我们来近似Ricci曲率
($ R_{\mu\beta} = \partial_\nu \Gamma_{\mu\beta}^{\nu} -\partial_\beta \Gamma_{\mu\nu}^{\nu}
 + \Gamma_{\mu\beta}^{\pi} \Gamma_{\pi\nu}^{\nu} - \Gamma_{\mu\nu}^{\pi} \Gamma_{\pi\beta}^{\nu} $).
由\eqref{chppn:eqn_Gammag}式可知$R_{\mu\nu}$分量有如下展开式,
\begin{subequations}
\begin{align}
    R_{00} =& \oversetmy{2}{R}_{00} + \oversetmy{4}{R}_{00} + \cdots \\
    R_{0i} =& \oversetmy{3}{R}_{0i} + \cdots \\
    R_{ij} =& \oversetmy{2}{R}_{ij} + \cdots
\end{align}
\end{subequations}
%$\oversetmy{4}{R}_{ij}=\oversetmy{2}{\Gamma}^k_{ij}\oversetmy{2}{\Gamma}^l_{kl}
%+\oversetmy{2}{\Gamma}^k_{ij}\oversetmy{2}{\Gamma}^0_{k0}
%-\oversetmy{2}{\Gamma}^0_{i0}\oversetmy{2}{\Gamma}^0_{j0}
%-\oversetmy{2}{\Gamma}^l_{ik}\oversetmy{2}{\Gamma}^k_{lj}$ 四阶项不为零,为何忽略?
其中
\begin{subequations}\label{chppn:eqn_ricciijk}
\begin{align}
    \oversetmy{2}{R}_{00} =& \frac{\partial \oversetmy{2}{\Gamma}^i_{00}}{\partial x^i}
      = -\frac{1}{2}\frac{\partial^2 \oversetmy{2}{g}_{00}} {\partial x^i \partial x^i}
      = -\frac{1}{2} \nabla^2 \oversetmy{2}{g}_{00} \\
    \oversetmy{4}{R}_{00} =& \frac{\partial \oversetmy{4}{\Gamma}^i_{00}}{\partial x^i}
       -\frac{\partial \oversetmy{3}{\Gamma}^i_{0i}}{\partial x^0}
       + \oversetmy{2}{\Gamma}^i_{00} \oversetmy{2}{\Gamma}^j_{ij}
       - \oversetmy{2}{\Gamma}^i_{00} \oversetmy{2}{\Gamma}^0_{0i}   \notag\\
%       =& \frac{\partial }{\partial x^i} \left(
%       -\frac{1}{2}\frac{\partial \oversetmy{4}{g}_{00}} {\partial x^i}
%       + \frac{\partial \oversetmy{3}{g}_{0i}} {\partial x^0}
%       + \frac{1}{2}\oversetmy{2}{g}_{ki}\frac{\partial \oversetmy{2}{g}_{00}} {\partial x^k}\right)
%       -\frac{1}{2} \frac{\partial^2 \sum_{i} \oversetmy{2}{g}_{ii}  }{\partial x^0\partial x^0}
%       -\frac{1}{4}\frac{\partial \oversetmy{2}{g}_{00}} {\partial x^i}
%        \left(  \frac{\partial \sum_{j}\oversetmy{2}{g}_{jj}} {\partial x^i}
%        + \frac{\partial \oversetmy{2}{g}_{00}} {\partial x^i} \right)  \\
       =& -\frac{1}{2} \nabla^2 \oversetmy{4}{g}_{00}
        +\frac{\partial^2 \oversetmy{3}{g}_{0i}} {\partial x^i \partial x^0}
        -\frac{1}{2} \frac{\partial^2 \sum_{i} \oversetmy{2}{g}_{ii}  }{\partial x^0\partial x^0}
        +\frac{1}{2}\oversetmy{2}{g}_{ki}\frac{\partial^2 \oversetmy{2}{g}_{00}}
         {\partial x^k\partial x^i} \notag \\
        &+\frac{1}{2} \frac{\partial \oversetmy{2}{g}_{00}} {\partial x^k}
         \left(\frac{\partial\oversetmy{2}{g}_{ki}}{\partial x^i}
        -\frac{1}{2}\frac{\partial \sum_{j}\oversetmy{2}{g}_{jj}} {\partial x^k} \right)
        -\frac{1}{4} \left|\nabla \oversetmy{2}{g}_{00} \right|^2          \\
    \oversetmy{3}{R}_{0i} =& \frac{\partial \oversetmy{2}{\Gamma}^0_{0i}}{\partial x^0}
       +\frac{\partial \oversetmy{3}{\Gamma}^j_{0i}}{\partial x^j}
       -\frac{\partial \oversetmy{3}{\Gamma}^0_{00}}{\partial x^i}
       -\frac{\partial \oversetmy{3}{\Gamma}^j_{0j}}{\partial x^i} \notag \\
%       =& -\cancel{\frac{1}{2} \frac{\partial^2 \oversetmy{2}{g}_{00} }{\partial x^i \partial x^0}}
%       +\frac{1}{2}\frac{\partial }{\partial x^j}\left(
%       \frac{\partial \oversetmy{3}{g}_{0j}} {\partial x^i}
%       + \frac{\partial \oversetmy{2}{g}_{ij}} {\partial x^0}
%       - \frac{\partial \oversetmy{3}{g}_{0i}} {\partial x^j} \right)
%       +\cancel{\frac{1}{2}\frac{\partial^2 \oversetmy{2}{g}_{00}}{\partial x^0\partial x^i}}
%       -\frac{1}{2}\frac{\partial^2 \sum_{j}\oversetmy{2}{g}_{jj} }{\partial x^0\partial x^i} \\
       =&\frac{1}{2}\frac{\partial^2 \oversetmy{3}{g}_{0j}} {\partial x^j\partial x^i}
       + \frac{1}{2}\frac{\partial^2 \oversetmy{2}{g}_{ij}} {\partial x^j\partial x^0}
       - \frac{1}{2} \nabla^2 \oversetmy{3}{g}_{0i}
       -\frac{1}{2}\frac{\partial^2 \sum_{j}\oversetmy{2}{g}_{jj} }{\partial x^0\partial x^i} \\
    \oversetmy{2}{R}_{ij} =& \frac{\partial \oversetmy{2}{\Gamma}^k_{ij}}{\partial x^k}
       -\frac{\partial \oversetmy{2}{\Gamma}^0_{0i}}{\partial x^j}
       -\frac{\partial \oversetmy{2}{\Gamma}^k_{ik}}{\partial x^j} \notag \\
       =&\frac{1}{2}\frac{\partial^2 \oversetmy{2}{g}_{kj}} {\partial x^k \partial x^i}
       + \frac{1}{2}\frac{\partial^2 \oversetmy{2}{g}_{ik}} {\partial x^k \partial x^j}
       - \frac{1}{2}\nabla^2 \oversetmy{2}{g}_{ij}
       +\frac{1}{2}\frac{\partial^2 \oversetmy{2}{g}_{00}} {\partial x^i\partial x^j}
       -\frac{1}{2}\frac{\partial^2 \sum_{k}\oversetmy{2}{g}_{kk} }{\partial x^i\partial x^j}
\end{align}
\end{subequations}


上述方程很复杂,为了简化方程需引入规范条件;
规范条件有很多种,我们使用最常用的谐和坐标.
把式\eqref{chppn:eqn_guab}、\eqref{chppn:eqn_Gammag}代
入$g^{\mu\nu}\Gamma^\lambda_{\mu\nu}=0$可得近似后的公式.
先取$\lambda=0$,得
\begin{equation}\label{chppn:eqn_gauge03}
    \frac{\partial \oversetmy{2}{g}_{00}} {\partial x^0}
    -2\frac{\partial \oversetmy{3}{g}_{0j} }{\partial x^j}
    +\frac{\partial \sum_{j}\oversetmy{2}{g}_{jj} }{\partial x^0} =0 .
\end{equation}
%\begin{align*}
%    0  %=g^{\mu\nu}\Gamma^\lambda_{\mu\nu}
%    =& g^{0 0}\Gamma^0_{0 0} + 2g^{j 0}\Gamma^0_{j 0} + g^{j k}\Gamma^0_{j k} \\
%    =& (-1 -\oversetmy{2}{g}_{00} )
%    (-\frac{1}{2}\frac{\partial \oversetmy{2}{g}_{00}} {\partial x^0} )
%    -\oversetmy{3}{g}_{0j} \frac{\partial \oversetmy{2}{g}_{00}} {\partial x^j}
%    -(\delta_{kj}  -\oversetmy{2}{g}_{jk}) \times \frac{1}{2} \left(
%    \frac{\partial \oversetmy{3}{g}_{0j} }{\partial x^k}
%    +\frac{\partial \oversetmy{3}{g}_{0k}}{\partial x^j}
%    -\frac{\partial \oversetmy{2}{g}_{jk} }{\partial x^0} \right) \\
%    =&\frac{1}{2}\left( \frac{\partial \oversetmy{2}{g}_{00}} {\partial x^0}
%    -\frac{\partial \oversetmy{3}{g}_{0j} }{\partial x^j}
%    -\frac{\partial \oversetmy{3}{g}_{0j}}{\partial x^j}
%    +\frac{\partial \oversetmy{2}{g}_{jj} }{\partial x^0} \right) +O(4)
%\end{align*}
再取$\lambda=i \, (1\leqslant i \leqslant 3)$,得
\begin{equation}\label{chppn:eqn_gaugei2}
    \frac{\partial \oversetmy{2}{g}_{00}} {\partial x^i}
    + 2\frac{\partial \oversetmy{2}{g}_{ij}} {\partial x^j}
    - \frac{\partial \sum_{j}\oversetmy{2}{g}_{jj}} {\partial x^i} =0 .
\end{equation}
%\begin{align*}
%    0=&g^{0 0}\Gamma^i_{0 0} +2 g^{j 0}\Gamma^i_{j 0} + g^{j k}\Gamma^i_{j k} \\
%    =&(-1 -\oversetmy{2}{g}_{00} )
%    \left(-\frac{1}{2}\frac{\partial \oversetmy{2}{g}_{00}} {\partial x^i}
%    -\frac{1}{2}\frac{\partial \oversetmy{4}{g}_{00}} {\partial x^i}
%    + \frac{\partial \oversetmy{3}{g}_{0i}} {\partial x^0}
%    + \frac{1}{2}\oversetmy{2}{g}_{ij}\frac{\partial \oversetmy{2}{g}_{00}} {\partial x^i}\right) \\
%    &+\oversetmy{3}{g}_{0j}\left(
%    \frac{\partial \oversetmy{3}{g}_{0i}} {\partial x^j}
%    + \frac{\partial \oversetmy{2}{g}_{ij}} {\partial x^0}
%    - \frac{\partial \oversetmy{3}{g}_{0j}} {\partial x^i} \right)
%    +(\delta_{kj}  -\oversetmy{2}{g}_{jk})\frac{1}{2}\left(
%    \frac{\partial \oversetmy{2}{g}_{ij}} {\partial x^k}
%    + \frac{\partial \oversetmy{2}{g}_{ik}} {\partial x^j}
%    - \frac{\partial \oversetmy{2}{g}_{jk}} {\partial x^i} \right) \\
%    =&\frac{1}{2}\left(\frac{\partial \oversetmy{2}{g}_{00}} {\partial x^i}
%    + \frac{\partial \oversetmy{2}{g}_{ij}} {\partial x^j}
%    + \frac{\partial \oversetmy{2}{g}_{ij}} {\partial x^j}
%    - \frac{\partial \oversetmy{2}{g}_{jj}} {\partial x^i} \right) +O(3)
%\end{align*}
利用谐和坐标\eqref{chppn:eqn_gauge03}、\eqref{chppn:eqn_gaugei2},
经过一些计算,Ricci曲率\eqref{chppn:eqn_ricciijk}简化为
\begin{subequations}\label{chppn:eqn_ricci-gauge}
    \begin{align}
        \oversetmy{2}{R}_{00} =& -\frac{1}{2} \nabla^2 \oversetmy{2}{g}_{00} \\
        \oversetmy{4}{R}_{00} %=&
%         -\frac{1}{2} \nabla^2 \oversetmy{4}{g}_{00}
%        +\frac{1}{2}\frac{\partial^2 \oversetmy{2}{g}_{00}} {\partial x^0\partial x^0}
%        +\frac{1}{2}\frac{\partial^2 \sum_{j}\oversetmy{2}{g}_{jj} }{\partial x^0\partial x^0}
%        -\frac{1}{2} \frac{\partial^2 \sum_{i} \oversetmy{2}{g}_{ii}  }{\partial x^0\partial x^0} \\
%        &+\frac{1}{2}\oversetmy{2}{g}_{ki}\frac{\partial^2 \oversetmy{2}{g}_{00}}
%        {\partial x^k\partial x^i}
%        -\frac{1}{4} \frac{\partial \oversetmy{2}{g}_{00}} {\partial x^k}
%        \frac{\partial \oversetmy{2}{g}_{00}} {\partial x^k}
%        -\frac{1}{4} \left|\nabla \oversetmy{2}{g}_{00} \right|^2          \\
        =& -\frac{1}{2} \nabla^2 \oversetmy{4}{g}_{00}
        +\frac{1}{2}\frac{\partial^2 \oversetmy{2}{g}_{00}} {\partial x^0\partial x^0}
        -\frac{1}{2} \left|\nabla \oversetmy{2}{g}_{00} \right|^2 +\frac{1}{2}\oversetmy{2}{g}_{ki}\frac{\partial^2 \oversetmy{2}{g}_{00}}
        {\partial x^k\partial x^i}  \\
        \oversetmy{3}{R}_{0i} =& - \frac{1}{2} \nabla^2 \oversetmy{3}{g}_{0i} \\
        \oversetmy{2}{R}_{ij} =& - \frac{1}{2}\nabla^2 \oversetmy{2}{g}_{ij}
    \end{align}
\end{subequations}







\section{基本方程}
本节继续叙述上节未完成的内容.

\subsection{能动张量近似}
能量--动量张量$T^{ab}$中,$T^{00}$是质量密度,最大量级$\sim  c^2$;
$T^{0i}$是动量密度,最大量级$\sim  c^1$;
$T^{ij}$是动量流密度,最大量级$\sim  c^0$.
在爱因斯坦引力场方程中,能动张量前还有系数$\frac{8\pi G}{c^4}$;
考虑此点后上述三个物理量的最大量级分别是:
$O(\bar{v}^2)$,$O(\bar{v}^3)$,$O(\bar{v}^4)$.
下面我们写出能动张量的展开表达式,
为了便于使用我们写出的是$\frac{8\pi G}{c^4}T^{ab}$的分量
表达式;但如果包含系数$\frac{8\pi G}{c^4}$,那公式
变得十分臃肿、笨重,故我们省略这个系数.
虽然省略了该系数,但在量级标注上是把它乘上之后的,
比如$\oversetmy{2}{T}^{00}$表示$O(\bar{v}^2)$量.

参考上节内容,我们预估能动张量可以展开成
\begin{subequations}
    \begin{align}
        T^{00} =& \oversetmy{2}{T}^{00} + \oversetmy{4}{T}^{00} + \cdots \\
        T^{0i} =& \oversetmy{3}{T}^{0i} + \cdots \\
        T^{ij} =& \oversetmy{4}{T}^{ij} + \cdots
    \end{align}
\end{subequations}
能动张量的迹$T=g_{\mu\nu}T^{\mu\nu}$的展开式为
\begin{align*}
    T=& g_{\mu\nu}T^{\mu\nu}=
    g_{00}T^{00}+2g_{0i}T^{0i}+g_{ij}T^{ij} \\
    =&(-1 +\oversetmy{2}{g}_{00} ) (\oversetmy{2}{T}^{00} + \oversetmy{4}{T}^{00})
    +2 \oversetmy{3}{g}_{0i} \oversetmy{3}{T}^{0i}
    +(\delta_{ij}  +\oversetmy{2}{g}_{ij}) \oversetmy{4}{T}^{ij} \\
    =&-\oversetmy{2}{T}^{00} -\oversetmy{4}{T}^{00} +\oversetmy{2}{g}_{00}\oversetmy{2}{T}^{00}
     +\sum_{k}\oversetmy{4}{T}^{kk}
\end{align*}
由此可得
\begin{equation}
    \oversetmy{2}{T} = -\oversetmy{2}{T}^{00}; \qquad
    \oversetmy{4}{T} = -\oversetmy{4}{T}^{00}
      +\oversetmy{2}{g}_{00}\oversetmy{2}{T}^{00}
      +\sum_{k}\oversetmy{4}{T}^{kk}  .
\end{equation}

借助上述展开式,求取反迹能动张量
$\overline{T}^{ab}=T^{ab} - \frac{1}{2}g^{ab}T$
(见式\eqref{chfd:eqn_Einstein-B-form})展开式
\begin{subequations}
\begin{align}
    \oversetmy{2}{\overline{T}}^{00} =& \oversetmy{2}{T}^{00}
      - \frac{1}{2} \oversetmy{0}{g}^{00} \oversetmy{2}{T}
      = \frac{1}{2}\oversetmy{2}{T}^{00}  \\
    \oversetmy{4}{\overline{T}}^{00} =& \oversetmy{4}{T}^{00}
      - \frac{1}{2} \oversetmy{0}{g}^{00} \oversetmy{4}{T}
      - \frac{1}{2} \oversetmy{2}{g}^{00} \oversetmy{2}{T}
      = \frac{1}{2}\Bigl( \oversetmy{4}{T}^{00} +\sum_{k=1}^3\oversetmy{4}{T}^{kk} \Bigr) \\
    \oversetmy{3}{\overline{T}}^{0i} =& \oversetmy{3}{T}^{0i}  \\
    \oversetmy{2}{\overline{T}}^{ij} =& - \frac{1}{2}\oversetmy{0}{g}^{ij}
      \oversetmy{2}{T} =  \frac{1}{2} \delta^{ij} \oversetmy{2}{T}^{00}
\end{align}
\end{subequations}

爱氏场方程中需要的是协变张量,把反迹能动张量指标降下来,有
\begin{equation}
    \overline{T}_{\mu\nu}= g_{\mu\rho}g_{\nu\sigma}\overline{T}^{\rho\sigma}=
       g_{\mu 0}g_{\nu 0}\overline{T}^{00}+2g_{\mu 0}g_{\nu i} \overline{T}^{0i}
      +g_{\mu i}g_{\nu j} \overline{T}^{ij} .
\end{equation}
把度规展开式和逆变反迹能动张量展开式代入,经过略显啰嗦的推导可得
\begin{subequations}\label{chppn:eqn_EMT}
    \begin{align}
        \oversetmy{2}{\overline{T}}_{00} =& \frac{1}{2}\oversetmy{2}{T}^{00}  \\
        \oversetmy{4}{\overline{T}}_{00} =& \frac{1}{2}\Bigl( \oversetmy{4}{T}^{00} +\sum_{k=1}^3\oversetmy{4}{T}^{kk}
        -2\oversetmy{2}{g}_{00} \oversetmy{2}{T}^{00} \Bigr) \\
        \oversetmy{3}{\overline{T}}_{0i} =& - \oversetmy{3}{T}^{0i}  \\
        \oversetmy{2}{\overline{T}}_{ij} =& \frac{1}{2} \delta_{ij} \oversetmy{2}{T}^{00}
    \end{align}
\end{subequations}

我们以理想流体(见式\eqref{chlh:eqn_perfect-fluid-Tab})为例,给出上式的具体表达式.
\begin{equation}
    T^{ab}=\left(\rho \Bigl(1+ \frac{\Pi}{c^2} \Bigr) + \frac{p}{c^2}  \right) U^a U^b +p g^{ab} .
    \tag{\ref{chlh:eqn_perfect-fluid-Tab}}
\end{equation}
$\rho$是质量密度,$\Pi$是单位质量内能,$p$是压强,$U^a$是质点四速度

\subsection{后牛顿引力场方程式}
利用Ricci曲率展开式\eqref{chppn:eqn_ricci-gauge}和
能动张量展开式\eqref{chppn:eqn_EMT},
再由爱因斯坦引力场方程\eqref{chfd:eqn_Einstein-B-form}可以
得到谐和坐标(\eqref{chppn:eqn_gauge03}、\eqref{chppn:eqn_gaugei2})下
的后牛顿方程式:
\begin{subequations}\label{chppn:eqn_Post-Newtonian}
    \begin{align}
        \nabla^2 \oversetmy{2}{g}_{00} =& -\frac{8\pi G}{c^4} \oversetmy{2}{T}^{00}  \\
        \nabla^2 \oversetmy{4}{g}_{00} =&
        \frac{\partial^2 \oversetmy{2}{g}_{00}} {\partial x^0\partial x^0}
        - \left|\nabla \oversetmy{2}{g}_{00} \right|^2 +\oversetmy{2}{g}_{ki}\frac{\partial^2 \oversetmy{2}{g}_{00}}  {\partial x^k\partial x^i}
        +\frac{8\pi G}{c^4}\left(2\oversetmy{2}{g}_{00} \oversetmy{2}{T}^{00}
        -\oversetmy{4}{T}^{00} -\sum_{k=1}^3\oversetmy{4}{T}^{kk}  \right)     \\
        \nabla^2 \oversetmy{3}{g}_{0i} =& 2\times \frac{8\pi G}{c^4}\oversetmy{3}{T}^{0i}  \\
        \nabla^2 \oversetmy{2}{g}_{ij} =& - \delta_{ij} \frac{8\pi G}{c^4}\oversetmy{2}{T}^{00}
    \end{align}
\end{subequations}
我们已将能动张量前的系数$\frac{8\pi G}{c^4}$显示写出;
$\oversetmy{2}{T}^{00}$上面的量级阶数“2”是指乘上系数$\frac{8\pi G}{c^4}$之后
的量级为$O(\bar{v}^2)$,其它能动张量项也作同样理解.







%%%%%%%%%%%%%%%%%%%%%%%%%%%%%%%%%%%%%%%%%%%%%%%%%%%%%%%%%%%%%%%%%%%%%%%%%%%%%%%%%%%%%
\printbibliography[heading=subbibliography,title=第\ref{chppn}章参考文献]

\endinput
