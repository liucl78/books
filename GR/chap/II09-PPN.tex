% !TeX spellcheck = <none>
% !TeX encoding = UTF-8
% 此文件从2022.3.04开始写作

\chapter{参数化后牛顿近似} \label{chppn}
爱因斯坦引力场方程是非线性的,很难严格求解.
由爱因斯坦本人参与并发展的后牛顿近似方法是处理引力束缚系统的常用方法.
为了便于比较量级以及使用,本章使用国际单位制;
并且基本用分量语言来描述;坐标系则需要根据要处理的问题进行选取.
%如无特殊要求,我们选取固有坐标系(见\S\ref{chfd:sec_proper-coord}).



\section{量级估算}\label{chppn:sec_OM}

只有清楚了各物理量的数量级,才能作可靠的近似.
考虑一个由引力束缚在一起的质点系统(如太阳系),
$M$、$r$、$v$分别代表质点的质量、平均距离和速度.
根据位力定理,有
\begin{equation}
    v^2 \approx \frac{GM}{r} \equiv U .
\end{equation}
其中$U$是引力势函数:
\begin{equation}
    U = -\phi = +\int \frac{G \rho(\boldsymbol{y})}{|\boldsymbol{x}-\boldsymbol{y}|} {\rm d}^3y
      = +\frac{G M }{r} .
\end{equation}
后牛顿近似中更喜欢用$U$,而不是$\phi$.
对于太阳表面、地球表面,有
\begin{equation}
    \frac{v^2}{c^2}\approx\frac{G M_{\odot} }{c^2 R_{\odot}} \approx 2 \times 10^{-6} ;\qquad
    \frac{v^2}{c^2}\approx\frac{G M_{\oplus} }{c^2 R_{\oplus}} \approx 7 \times 10^{-10} .
\end{equation}
可见不论太阳附近还是地球附近都可看成弱引力场,
在此区域飞行的质点速度$v/c$都不高.
为了令后面公式简洁一些,我们命
\begin{equation}\label{chppn:eqn_vdc}
    \varepsilon\equiv \frac{v}{c}.
\end{equation}
我们将借助参量$\varepsilon$将各种物理量展开.

在天体物理中,通常用理想流体力学来近似各种天文系统;
在此种近似方法中单位质量内能$\Pi$与引力势$U$处于同一量级,
而单位质量内能可通过压强与质量密度比值来表述,故有
\begin{equation}
    \frac{\Pi}{c^2} \sim \frac{p}{c^2 \rho}\sim
    \frac{U}{c^2} \sim \varepsilon^2 .
\end{equation}

系统中的距离和时间尺度分别$r$和$r/v$,故导数量级大约是:
\begin{equation}
    \frac{\partial}{\partial x^i}\sim \frac{1}{r},\qquad
    \frac{\partial}{\partial x^0}\sim \frac{\partial}{c \partial t}
    \sim \frac{v}{cr} \sim  \varepsilon \frac{\partial}{\partial x^i}.
\end{equation}


后牛顿近似本质上是对光速的逆(即$c^{-1}$)的展开,
也说成对$\varepsilon$(\eqref{chppn:eqn_vdc}式)的展开.精确到不同量级,
如$O(c^{-1})$(或$O(\varepsilon)$)、$O(c^{-2})$(或$O(\varepsilon^2)$)等.


\section{后牛顿近似}
后牛顿近似的基本出发点是爱因斯坦引力场方程和自由质点运动方程(即测地线方程).
我们先讨论有质量自由质点的测地线方程
\begin{equation}
    \frac{{\rm d}^2 x^\mu}{{\rm d}\tau ^2}+ {\Gamma}_{\nu\lambda}^{\mu}
      \frac{{\rm d} x^\nu}{{\rm d}\tau} \frac{{\rm d} x^\lambda}{{\rm d}\tau} =0,
\end{equation}
其中$\tau$是固有时.由此出发可以算出空间的三维加速度是(注$x^0=ct$):
\begin{align}
    \frac{{\rm d}^2 x^i}{{\rm d} t^2} =& \frac{{\rm d} \tau}{{\rm d}t}
       \frac{{\rm d} }{{\rm d}\tau} \left( \frac{{\rm d} \tau}{{\rm d}t}
       \frac{{\rm d} x^i}{{\rm d} \tau} \right)
    =\left(\frac{{\rm d} \tau}{{\rm d}t} \right)^{2} \frac{{\rm d}^2 x^i}{{\rm d} \tau^2}
     -\left(\frac{{\rm d} \tau}{{\rm d}t} \right)^{3}
     \frac{{\rm d}^2 x^0}{c\, {\rm d} \tau^2} \frac{{\rm d} x^i}{{\rm d} \tau}  \notag \\
    =&-{\Gamma}_{\nu\lambda}^{i} \frac{{\rm d} x^\nu}{{\rm d}t} \frac{{\rm d} x^\lambda}{{\rm d}t}
      +\frac{1}{c} {\Gamma}_{\nu\lambda}^{0} \frac{{\rm d} x^i}{{\rm d}t}
       \frac{{\rm d} x^\nu}{{\rm d}t} \frac{{\rm d} x^\lambda}{{\rm d}t} \notag \\
    =& -c^2 \Gamma_{00}^{i} - 2c \Gamma_{j0}^{i}\frac{{\rm d} x^j}{{\rm d}t}
       -\Gamma_{jk}^{i} \frac{{\rm d} x^j}{{\rm d}t} \frac{{\rm d} x^k}{{\rm d}t} \notag \\
      & +\frac{1}{c}\frac{{\rm d} x^i}{{\rm d}t} \left(
        c^2 \Gamma_{00}^{0} + 2c \Gamma_{j0}^{0}\frac{{\rm d} x^j}{{\rm d}t}
       +\Gamma_{jk}^{0} \frac{{\rm d} x^j}{{\rm d}t}
       \frac{{\rm d} x^k}{{\rm d}t}  \right) . \label{chppn:eqn_PN}
\end{align}
回顾\S\ref{chle:sec_Newton-limit},当$\varepsilon \ll 1$时,
我们只保留最低阶项,有$g_{00}=-1 + 2U/c^2$和$g_{0i}=0$、$g_{ij}=\delta_{ij}$,
那么牛顿第二定律和万有引力公式结合之后是
\begin{equation}
    \frac{{\rm d}^2 x^i}{{\rm d}t^2}=-c^2{\Gamma}_{00}^i
    =\frac{c^2}{2} \frac{\partial g_{00}}{\partial x^i}
    =+\frac{\partial U}{\partial x^i} .
\end{equation}
由此可见牛顿近似中$\frac{{\rm d}^2 x^i}{{\rm d}t^2}$准确到量级$\frac{G M}{r^2} \sim \frac{v^2}{r}$.
也就是$\Gamma_{00}^i\sim \frac{G M}{c^2 r^2} \sim \frac{\varepsilon^2}{r} $.


在我们这个后牛顿近似的初步介绍中,
只要求加速度$\frac{{\rm d}^2 x^i}{{\rm d}t^2}$准确到$\frac{v^2}{r}\varepsilon^2$量级,
那么对Christoffel符号诸分量的要求是(个别项会保留更高阶小量)
\begin{subequations}
\begin{align}
    & \Gamma_{00}^i {\quad \text{须准确到量级}\quad } \frac{\varepsilon^4}{r},  \\
    & \Gamma_{j0}^i {\quad \text{须准确到量级}\quad } \frac{\varepsilon^3}{r},  \\
    & \Gamma_{jk}^i {\quad \text{须准确到量级}\quad } \frac{\varepsilon^2}{r},  \\
    & \Gamma_{00}^0 {\quad \text{须准确到量级}\quad } \frac{\varepsilon^3}{r},  \\
    & \Gamma_{j0}^0 {\quad \text{须准确到量级}\quad } \frac{\varepsilon^2}{r},  \\
    & \Gamma_{jk}^0 {\quad \text{须准确到量级}\quad } \frac{\varepsilon}{r}.
\end{align}
\end{subequations}
%以上只是对有质量粒子的要求,对于电磁波的后牛顿近似另行讨论.

\subsection{度规场近似}
因为克氏符是由度规来表述的,下面我们来看度规的近似展开.
\begin{equation}
    \frac{{\rm d} s^2}{{\rm d}t^2} = c^2 g_{00} + 2 c g_{0i} \frac{{\rm d} x^i}{{\rm d}t}
     + g_{ij} \frac{{\rm d} x^i}{{\rm d}t}  \frac{{\rm d} x^j}{{\rm d}t} .
\end{equation}
后牛顿方法是弱场、低速近似,度规场不会偏离Lorentz度规太远,故可假设:
\begin{subequations}\label{chppn:eqn_gdab}
\begin{align}
    g_{00} =& -1 + \oversetmy{2}{g}_{00}+\oversetmy{4}{g}_{00}+ \cdots,  \\
    g_{ij} =& \delta_{ij} + \oversetmy{2}{g}_{ij}+ \cdots,  \\
    g_{i0} =& \oversetmy{3}{g}_{i0}+ \cdots
\end{align}
\end{subequations}
{\kaishu 其中$\oversetmy{N}{g}_{\mu\nu}$表示$g_{\mu\nu}$中
    量级为$\varepsilon^N$(见式\eqref{chppn:eqn_vdc})的近似项;
其它量也采取类似的表述方式.} 因为在时间反演下${\rm d} s^2$必然是个不变量,
而速度$v^i=\frac{{\rm d} x^i}{{\rm d}t}$是要改变符号的,
故式\eqref{chppn:eqn_gdab}中$g_{i0}$必须是$\varepsilon$的奇数次幂;
而$g_{00}$、$g_{ij}$只能包含$\varepsilon$的偶数次幂项.
$g_{i0}$中没有一阶项的原因见\S\ref{chppn:sec_PNFE}.
如果有辐射阻尼之类的作用,那么可能打破上述奇偶规律;遇到该问题时再行讨论.

我们知道度规场有关系式:$g^{\mu\rho}g_{\rho\nu}=\delta^\mu_\nu$;把它展开
\begin{subequations}\label{chppn:eqn_gg1}
\begin{align}
    g^{0\rho}g_{\rho 0}=& g^{00}g_{0 0}+g^{0k}g_{k 0} = 1, \qquad \text{一个方程} \\
    g^{0\rho}g_{\rho j}=& g^{00}g_{0 j}+g^{0k}g_{k j} = 0, \qquad \text{三个方程} \\
    g^{i\rho}g_{\rho j}=& g^{i0}g_{0 j}+g^{ik}g_{k j} =\delta^i_j. \qquad \text{六个方程}
\end{align}
\end{subequations}
我们预估逆变度规有如下形式(为了方便引用,
第二个等号是把式\eqref{chppn:eqn_gugd}带入后的结果,预先写在这里)
\begin{subequations}\label{chppn:eqn_guab}
\begin{align}
    g^{00} =& -1 + \oversetmy{2}{g}^{00}+\oversetmy{4}{g}^{00}+ \cdots
      &=& -1 -\oversetmy{2}{g}_{00} -\oversetmy{4}{g}_{00}
      -\left(\oversetmy{2}{g}_{00}\right)^2 +\cdots  \\
    g^{ij} =& \delta^{ij} + \oversetmy{2}{g}^{ij}+ \cdots
      &=&  \delta_{ij}  -\oversetmy{2}{g}_{ij} + \cdots \\
    g^{i0} =& \oversetmy{3}{g}^{i0}+ \cdots
      &=& \oversetmy{3}{g}_{i0} + \cdots
\end{align}
\end{subequations}
把式\eqref{chppn:eqn_gdab}、\eqref{chppn:eqn_guab}第二个等号前的表达式代入式\eqref{chppn:eqn_gg1},有
\begin{equation}\label{chppn:eqn_gugd}
    \oversetmy{2}{g}^{00} = -\oversetmy{2}{g}_{00}, \quad
    \oversetmy{4}{g}^{00} = -\oversetmy{4}{g}_{00}-(\oversetmy{2}{g}_{00})^2, \quad
    \oversetmy{2}{g}^{ij} = -\oversetmy{2}{g}_{ij}, \quad
    \oversetmy{3}{g}^{i0} = +\oversetmy{3}{g}_{i0}, \    \cdots
\end{equation}
注意:得到上式后,再代入式\eqref{chppn:eqn_guab}才得到第二个等号后面的公式.


\begin{remark}
	张量的指标升降需要用式\eqref{chppn:eqn_gdab}以及\eqref{chppn:eqn_guab}.
	显示标明的除外.
\end{remark}

\subsection{Christoffel符号近似}\label{chppn:eqn_CPN}
克氏符定义为
$    \Gamma^\mu_{\lambda\nu}=\frac{1}{2}g^{\mu\rho}(
      \frac{\partial g_{\rho\nu}}{\partial x^\lambda}
     +\frac{\partial g_{\lambda\rho}}{\partial x^\nu}
     -\frac{\partial g_{\lambda\nu}}{\partial x^\rho} ) $.
我们以$\Gamma^i_{00}$为例,可求出
\begin{align*}
    \Gamma^i_{00}&=\frac{1}{2}g^{i0}\left(
     \frac{\partial g_{0 0}}{\partial x^0}
    +\frac{\partial g_{0 0}}{\partial x^0}
    -\frac{\partial g_{00}}{\partial x^0} \right)
    +\frac{1}{2}g^{ij}\left(
    \frac{\partial g_{j 0}}{\partial x^0}
    +\frac{\partial g_{0 j}}{\partial x^0}
    -\frac{\partial g_{00}}{\partial x^j} \right) \\
%    &=\frac{1}{2} \oversetmy{3}{g}^{i0} \left(
%     \frac{\partial (-1+\oversetmy{2}{g}_{00})}{\partial x^0}
%    +\frac{\partial (-1+\oversetmy{2}{g}_{00})}{\partial x^0}
%    -\frac{\partial (-1+\oversetmy{2}{g}_{00})}{\partial x^0} \right)
%    +\frac{1}{2}(\delta_{ij}+\oversetmy{2}{g}^{ij})\left(
%     \frac{\partial \oversetmy{3}{g}_{j0}}{\partial x^0}
%    +\frac{\partial \oversetmy{3}{g}_{j0}}{\partial x^0}
%    -\frac{\partial (-1+\oversetmy{2}{g}_{00}+\oversetmy{4}{g}_{00})}
%       {\partial x^j} \right) \\
    &= \frac{1}{2}(\delta_{ij}+\oversetmy{2}{g}^{ij})\left(
    2\frac{\partial \oversetmy{3}{g}_{j0}}{\partial x^0}
    -\frac{\partial (\oversetmy{2}{g}_{00}+\oversetmy{4}{g}_{00})}
      {\partial x^j} \right) + O(c^{-5})\\
    &= \frac{1}{2} \left(
    2\frac{\partial \oversetmy{3}{g}_{i0}}{\partial x^0}
    -\frac{\partial (\oversetmy{2}{g}_{00}+\oversetmy{4}{g}_{00})} {\partial x^i}
    -\oversetmy{2}{g}^{ij} \frac{\partial \oversetmy{2}{g}_{00}}{\partial x^j} \right) + O(c^{-5}) \\
    &\xlongequal{\ref{chppn:eqn_gugd}}\frac{1}{2} \left(
    -\frac{\partial \oversetmy{2}{g}_{00}} {\partial x^i}
    +2\frac{\partial \oversetmy{3}{g}_{i0}}{\partial x^0}
    -\frac{\partial \oversetmy{4}{g}_{00}} {\partial x^i}
    +\oversetmy{2}{g}_{ij} \frac{\partial \oversetmy{2}{g}_{00}}{\partial x^j} \right) + O(c^{-5}) .
\end{align*}
与上述过程类似,可以得到如下各阶克氏符近似表达式
\begin{subequations}\label{chppn:eqn_Gammag}
\begin{align}
    \oversetmy{2}{\Gamma}^i_{00} = & -\frac{1}{2}\frac{\partial \oversetmy{2}{g}_{00}} {\partial x^i} ,\quad
    \oversetmy{4}{\Gamma}^i_{00} =  -\frac{1}{2}\frac{\partial \oversetmy{4}{g}_{00}} {\partial x^i}
       + \frac{\partial \oversetmy{3}{g}_{0i}} {\partial x^0}
       + \frac{1}{2}\oversetmy{2}{g}_{il}\frac{\partial \oversetmy{2}{g}_{00}} {\partial x^l} ;  \\
    \oversetmy{3}{\Gamma}^i_{j0} = & \frac{1}{2}\left(
         \frac{\partial \oversetmy{3}{g}_{i0}} {\partial x^j}
       + \frac{\partial \oversetmy{2}{g}_{ij}} {\partial x^0}
       - \frac{\partial \oversetmy{3}{g}_{0j}} {\partial x^i} \right) ; \\
    \oversetmy{2}{\Gamma}^i_{jk} = & \frac{1}{2}\left(
      \frac{\partial \oversetmy{2}{g}_{ij}} {\partial x^k}
    + \frac{\partial \oversetmy{2}{g}_{ik}} {\partial x^j}
    - \frac{\partial \oversetmy{2}{g}_{jk}} {\partial x^i} \right); \\
    \oversetmy{3}{\Gamma}^0_{00} = & -\frac{1}{2}\frac{\partial \oversetmy{2}{g}_{00}} {\partial x^0},\quad
    \oversetmy{5}{\Gamma}^0_{00} =
    -\frac{1}{2}\left(\frac{\partial \oversetmy{4}{g}_{00}}{\partial x^0}
    +\oversetmy{2}{g}_{00}\frac{\partial \oversetmy{2}{g}_{00}}{\partial x^0}
    +\oversetmy{3}{g}_{0l}\frac{\partial \oversetmy{2}{g}_{00}}{\partial x^l} \right); \\
    \oversetmy{2}{\Gamma}^0_{j0} = & -\frac{1}{2}\frac{\partial \oversetmy{2}{g}_{00}} {\partial x^j},\quad
    \oversetmy{4}{\Gamma}^0_{j0} = -\frac{1}{2} \oversetmy{2}{g}_{00}\frac{\partial \oversetmy{2}{g}_{00}}{\partial x^j}
    -\frac{1}{2}\frac{\partial \oversetmy{4}{g}_{00}}{\partial x^j}; \\
    \oversetmy{1}{\Gamma}^0_{jk} = & 0 =\oversetmy{2}{\Gamma}^0_{jk}, \quad
    \oversetmy{3}{\Gamma}^0_{jk} =  -\frac{1}{2} \left(
    \frac{\partial \oversetmy{3}{g}_{0j} }{\partial x^k}
    +\frac{\partial \oversetmy{3}{g}_{0k}}{\partial x^j}
    -\frac{\partial \oversetmy{2}{g}_{jk} }{\partial x^0} \right) .
\end{align}
\end{subequations}

由上述克氏符可知,度规分量$g_{ij}$至少要准确到量级$\varepsilon^2$,
$g_{i0}$至少要准确到量级$\varepsilon^3$,$g_{00}$至少要准确到量级$\varepsilon^4$.
应当把这一点同纯牛顿力学对照一下,
在那里我们需要$g_{00}$精确到量级$\varepsilon^2$;
而$g_{i0}=0$,$g_{ij}=\delta_{ij}$,即只需精确到零级.
式\eqref{chppn:eqn_Gammag}中的最后三式保留了高一阶的小量,以备后用.


%其它各项推导过程
%\begin{align*}
%    \Gamma^i_{0j}=&\frac{1}{2}g^{i0}\left(
%     \frac{\partial g_{0 j}}{\partial x^0}
%    +\frac{\partial g_{0 0}}{\partial x^j}
%    -\frac{\partial g_{0j}}{\partial x^0} \right)
%    +\frac{1}{2}g^{il}\left(
%     \frac{\partial g_{l 0}}{\partial x^j}
%    +\frac{\partial g_{j l}}{\partial x^0}
%    -\frac{\partial g_{0 j}}{\partial x^l} \right) \\
%    =&\frac{1}{2}\oversetmy{3}{g}_{0i} \left(
%    \frac{\partial \oversetmy{3}{g}_{0j} }{\partial x^0}
%    +\frac{\partial \oversetmy{2}{g}_{00} }{\partial x^j}
%    -\frac{\partial \oversetmy{3}{g}_{0j} }{\partial x^0} \right)
%    +\frac{1}{2}(\delta_{il}  -\oversetmy{2}{g}_{il})\left(
%    \frac{\partial \oversetmy{3}{g}_{0l} }{\partial x^j}
%    +\frac{\partial  \oversetmy{2}{g}_{lj} }{\partial x^0}
%    -\frac{\partial \oversetmy{3}{g}_{0j} }{\partial x^l} \right) \\
%    =& \frac{1}{2}  \left(
%    \frac{\partial \oversetmy{3}{g}_{0i} }{\partial x^j}
%    +\frac{\partial  \oversetmy{2}{g}_{ij} }{\partial x^0}
%    -\frac{\partial \oversetmy{3}{g}_{0j} }{\partial x^i} \right)
%\end{align*}
%\begin{align*}
%    \Gamma^i_{jk}=&\frac{1}{2}g^{i0}\left(
%    \frac{\partial g_{0 j}}{\partial x^k}
%    +\frac{\partial g_{0 k}}{\partial x^j}
%    -\frac{\partial g_{0j}}{\partial x^0} \right)
%    +\frac{1}{2}g^{il}\left(
%     \frac{\partial g_{l k}}{\partial x^j}
%    +\frac{\partial g_{j l}}{\partial x^k}
%    -\frac{\partial g_{k j}}{\partial x^l} \right) \\
%    =&\frac{1}{2}\oversetmy{3}{g}_{0i} \left(
%    \frac{\partial \oversetmy{3}{g}_{0j} }{\partial x^k}
%    +\frac{\partial \oversetmy{3}{g}_{0k} }{\partial x^j}
%    -\frac{\partial \oversetmy{3}{g}_{0j}}{\partial x^0} \right)
%    +\frac{1}{2}(\delta_{il}  -\oversetmy{2}{g}_{il})\left(
%    \frac{\partial \oversetmy{2}{g}_{lk} }{\partial x^j}
%    +\frac{\partial \oversetmy{2}{g}_{jl} }{\partial x^k}
%    -\frac{\partial \oversetmy{2}{g}_{jk} }{\partial x^l} \right) \\
%    =&\frac{1}{2}\left(
%    \frac{\partial \oversetmy{2}{g}_{ik} }{\partial x^j}
%    +\frac{\partial \oversetmy{2}{g}_{ji} }{\partial x^k}
%    -\frac{\partial \oversetmy{2}{g}_{jk} }{\partial x^i} \right)
%    -\frac{1}{2}\oversetmy{2}{g}_{il} \left(
%    \frac{\partial \oversetmy{2}{g}_{lk} }{\partial x^j}
%    +\frac{\partial \oversetmy{2}{g}_{jl} }{\partial x^k}
%    -\frac{\partial \oversetmy{2}{g}_{jk} }{\partial x^l} \right)
%\end{align*}
%\begin{align*}
%    \Gamma^0_{00}=&\frac{1}{2}g^{00}\left(
%    \frac{\partial g_{0 0}}{\partial x^0}
%    +\frac{\partial g_{0 0}}{\partial x^0}
%    -\frac{\partial g_{00}}{\partial x^0} \right)
%    +\frac{1}{2}g^{0j}\left(
%    \frac{\partial g_{j 0}}{\partial x^0}
%    +\frac{\partial g_{0 j}}{\partial x^0}
%    -\frac{\partial g_{00}}{\partial x^j} \right) \\
%    =&\frac{1}{2}\left( -1 -\oversetmy{2}{g}_{00}  \right)
%    \frac{\partial }{\partial x^0} (\oversetmy{2}{g}_{00}+\oversetmy{4}{g}_{00})
%    +\frac{1}{2}\oversetmy{3}{g}_{0j}\left(
%    \frac{\partial \oversetmy{3}{g}_{j0} }{\partial x^0}
%    +\frac{\partial \oversetmy{3}{g}_{j0}}{\partial x^0}
%    -\frac{\partial \oversetmy{2}{g}_{00}}{\partial x^j} \right) \\
%    =&-\frac{1}{2}\frac{\partial \oversetmy{2}{g}_{00}}{\partial x^0}
%    -\frac{1}{2}\frac{\partial \oversetmy{4}{g}_{00}}{\partial x^0}
%    -\frac{1}{2}\oversetmy{2}{g}_{00}\frac{\partial \oversetmy{2}{g}_{00}}{\partial x^0}
%    -\frac{1}{2}\oversetmy{3}{g}_{0j}\frac{\partial \oversetmy{2}{g}_{00}}{\partial x^j}
%\end{align*}
%\begin{align*}
%    \Gamma^0_{0j}=&\frac{1}{2}g^{00}\left(
%    \frac{\partial g_{0 j}}{\partial x^0}
%    +\frac{\partial g_{0 0}}{\partial x^j}
%    -\frac{\partial g_{0j}}{\partial x^0} \right)
%    +\frac{1}{2}g^{0l}\left(
%    \frac{\partial g_{l 0}}{\partial x^j}
%    +\frac{\partial g_{j l}}{\partial x^0}
%    -\frac{\partial g_{0 j}}{\partial x^l} \right) \\
%    =&\frac{1}{2}\left(-1 -\oversetmy{2}{g}_{00} -\oversetmy{4}{g}_{00}
%    -\left(\oversetmy{2}{g}_{00}\right)^2 \right) 
%    \frac{\partial }{\partial x^j} \left(-1 + \oversetmy{2}{g}_{00}+\oversetmy{4}{g}_{00} \right)  \\
%    &+\frac{1}{2}\oversetmy{3}{g}_{0l}\left(
%    \frac{\partial \oversetmy{3}{g}_{l0} }{\partial x^j}
%    +\frac{\partial \oversetmy{2}{g}_{lj} }{\partial x^0}
%    -\frac{\partial \oversetmy{3}{g}_{j0} }{\partial x^l} \right)\\
%    =&-\frac{1}{2}\frac{\partial \oversetmy{2}{g}_{00}}{\partial x^j}
%    -\frac{1}{2} \oversetmy{2}{g}_{00}\frac{\partial \oversetmy{2}{g}_{00}}{\partial x^j}
%    -\frac{1}{2}\frac{\partial \oversetmy{4}{g}_{00}}{\partial x^j}
%\end{align*}
%\begin{align*}
%    \Gamma^0_{jk}=&\frac{1}{2}g^{00}\left(
%    \frac{\partial g_{0 j}}{\partial x^k}
%    +\frac{\partial g_{0 k}}{\partial x^j}
%    -\frac{\partial g_{jk}}{\partial x^0} \right)
%    +\frac{1}{2}g^{0l}\left(
%    \frac{\partial g_{l k}}{\partial x^j}
%    +\frac{\partial g_{j l}}{\partial x^k}
%    -\frac{\partial g_{k j}}{\partial x^l} \right) \\
%    =&\frac{1}{2}(-1-\oversetmy{2}{g}_{00})\left(
%    \frac{\partial \oversetmy{3}{g}_{0j} }{\partial x^k}
%    +\frac{\partial \oversetmy{3}{g}_{0k}}{\partial x^j}
%    -\frac{\partial \oversetmy{2}{g}_{jk} }{\partial x^0} \right)
%    +\frac{1}{2} \oversetmy{3}{g}_{0l} \left(
%    \frac{\partial \oversetmy{2}{g}_{lk} }{\partial x^j}
%    +\frac{\partial \oversetmy{2}{g}_{jl} }{\partial x^k}
%    -\frac{\partial \oversetmy{2}{g}_{jk} }{\partial x^l} \right)\\
%    =&-\frac{1}{2} \left(
%    \frac{\partial \oversetmy{3}{g}_{0j} }{\partial x^k}
%    +\frac{\partial \oversetmy{3}{g}_{0k}}{\partial x^j}
%    -\frac{\partial \oversetmy{2}{g}_{jk} }{\partial x^0} \right)
%    \sim O(3)
%\end{align*}





\subsection{Ricci曲率近似}
下面我们来近似Ricci曲率
($ R_{\mu\beta} = \partial_\nu \Gamma_{\mu\beta}^{\nu} -\partial_\beta \Gamma_{\mu\nu}^{\nu}
 + \Gamma_{\mu\beta}^{\pi} \Gamma_{\pi\nu}^{\nu} - \Gamma_{\mu\nu}^{\pi} \Gamma_{\pi\beta}^{\nu} $).
由\eqref{chppn:eqn_Gammag}式可知$R_{\mu\nu}$分量有如下展开式,
\begin{subequations}
\begin{align}
    R_{00} =& \oversetmy{2}{R}_{00} + \oversetmy{4}{R}_{00} + \cdots \\
    R_{0i} =& \oversetmy{3}{R}_{0i} + \cdots \\
    R_{ij} =& \oversetmy{2}{R}_{ij} + \cdots
\end{align}
\end{subequations}
将式\eqref{chppn:eqn_Gammag}带入后,有
\begin{subequations}\label{chppn:eqn_ricciijk}
\begin{align}
    \oversetmy{2}{R}_{00} =& \frac{\partial \oversetmy{2}{\Gamma}^i_{00}}{\partial x^i}
      = -\frac{1}{2}\frac{\partial^2 \oversetmy{2}{g}_{00}} {\partial x^i \partial x^i}
      = -\frac{1}{2} \nabla^2 \oversetmy{2}{g}_{00} , \\
    \oversetmy{4}{R}_{00} %=& \frac{\partial \oversetmy{4}{\Gamma}^i_{00}}{\partial x^i}
%       -\frac{\partial \oversetmy{3}{\Gamma}^i_{0i}}{\partial x^0}
%       + \oversetmy{2}{\Gamma}^i_{00} \oversetmy{2}{\Gamma}^j_{ij}
%       - \oversetmy{2}{\Gamma}^i_{00} \oversetmy{2}{\Gamma}^0_{0i}   \notag\\
%       =& \frac{\partial }{\partial x^i} \left(
%       -\frac{1}{2}\frac{\partial \oversetmy{4}{g}_{00}} {\partial x^i}
%       + \frac{\partial \oversetmy{3}{g}_{0i}} {\partial x^0}
%       + \frac{1}{2}\oversetmy{2}{g}_{ki}\frac{\partial \oversetmy{2}{g}_{00}} {\partial x^k}\right)
%       -\frac{1}{2} \frac{\partial^2 \sum_{l} \oversetmy{2}{g}_{ll}  }{\partial x^0\partial x^0}
%       -\frac{1}{4}\frac{\partial \oversetmy{2}{g}_{00}} {\partial x^i}
%        \left(  \frac{\partial \sum_{l}\oversetmy{2}{g}_{ll}} {\partial x^i}
%        + \frac{\partial \oversetmy{2}{g}_{00}} {\partial x^i} \right)  \\
       =& -\frac{1}{2} \nabla^2 \oversetmy{4}{g}_{00}
        +\frac{\partial^2 \oversetmy{3}{g}_{0i}} {\partial x^i \partial x^0}
        -\frac{1}{2} \frac{\partial^2 \sum_{l} \oversetmy{2}{g}_{ll}  }{\partial x^0\partial x^0}
        +\frac{1}{2}\oversetmy{2}{g}_{ki}\frac{\partial^2 \oversetmy{2}{g}_{00}}
         {\partial x^k\partial x^i} \notag \\
        &+\frac{1}{2} \frac{\partial \oversetmy{2}{g}_{00}} {\partial x^k}
         \left(\frac{\partial\oversetmy{2}{g}_{ki}}{\partial x^i}
        -\frac{1}{2}\frac{\partial \sum_{l}\oversetmy{2}{g}_{ll}} {\partial x^k} \right)
        -\frac{1}{4} \left|\nabla \oversetmy{2}{g}_{00} \right|^2    ,      \\
    \oversetmy{3}{R}_{i0} %=& \frac{\partial \oversetmy{2}{\Gamma}^0_{0i}}{\partial x^0}
%       +\frac{\partial \oversetmy{3}{\Gamma}^j_{0i}}{\partial x^j}
%       -\frac{\partial \oversetmy{3}{\Gamma}^0_{00}}{\partial x^i}
%       -\frac{\partial \oversetmy{3}{\Gamma}^j_{0j}}{\partial x^i} \notag \\
%       =& -\cancel{\frac{1}{2} \frac{\partial^2 \oversetmy{2}{g}_{00} }{\partial x^i \partial x^0}}
%       +\frac{1}{2}\frac{\partial }{\partial x^j}\left(
%       \frac{\partial \oversetmy{3}{g}_{0j}} {\partial x^i}
%       + \frac{\partial \oversetmy{2}{g}_{ij}} {\partial x^0}
%       - \frac{\partial \oversetmy{3}{g}_{0i}} {\partial x^j} \right)
%       +\cancel{\frac{1}{2}\frac{\partial^2 \oversetmy{2}{g}_{00}}{\partial x^0\partial x^i}}
%       -\frac{1}{2}\frac{\partial^2 \sum_{j}\oversetmy{2}{g}_{jj} }{\partial x^0\partial x^i} \\
       =&\frac{1}{2}\frac{\partial^2 \oversetmy{3}{g}_{0j}} {\partial x^j\partial x^i}
       + \frac{1}{2}\frac{\partial^2 \oversetmy{2}{g}_{ij}} {\partial x^j\partial x^0}
       - \frac{1}{2} \nabla^2 \oversetmy{3}{g}_{i0}
       -\frac{1}{2}\frac{\partial^2 \sum_{l}\oversetmy{2}{g}_{ll} }{\partial x^0\partial x^i}, \\
    \oversetmy{2}{R}_{ij} %=& \frac{\partial \oversetmy{2}{\Gamma}^k_{ij}}{\partial x^k}
%       -\frac{\partial \oversetmy{2}{\Gamma}^0_{0i}}{\partial x^j}
%       -\frac{\partial \oversetmy{2}{\Gamma}^k_{ik}}{\partial x^j} \notag \\
       =&\frac{1}{2}\frac{\partial^2 \oversetmy{2}{g}_{kj}} {\partial x^k \partial x^i}
       + \frac{1}{2}\frac{\partial^2 \oversetmy{2}{g}_{ik}} {\partial x^k \partial x^j}
       - \frac{1}{2}\nabla^2 \oversetmy{2}{g}_{ij}
       +\frac{1}{2}\frac{\partial^2 \oversetmy{2}{g}_{00}} {\partial x^i\partial x^j}
       -\frac{1}{2}\frac{\partial^2 \sum_{l}\oversetmy{2}{g}_{ll} }{\partial x^i\partial x^j} .
\end{align}
\end{subequations}


上述方程很复杂,为了简化方程需引入规范条件;
规范条件有很多种,我们使用最常用的谐和坐标.
把式\eqref{chppn:eqn_guab}、\eqref{chppn:eqn_Gammag}代
入$g^{\mu\nu}\Gamma^\lambda_{\mu\nu}=0$可得近似后的公式.
先取$\lambda=0$,得
\begin{equation}\label{chppn:eqn_gauge03}
    \frac{\partial \oversetmy{2}{g}_{00}} {\partial x^0}
    -2\frac{\partial \oversetmy{3}{g}_{0j} }{\partial x^j}
    +\frac{\partial \sum_{l}\oversetmy{2}{g}_{ll} }{\partial x^0} =0 .
\end{equation}
%\begin{align*}
%    0  %=g^{\mu\nu}\Gamma^\lambda_{\mu\nu}
%    =& g^{0 0}\Gamma^0_{0 0} + 2g^{j 0}\Gamma^0_{j 0} + g^{j k}\Gamma^0_{j k} \\
%    =& (-1 -\oversetmy{2}{g}_{00} )
%    (-\frac{1}{2}\frac{\partial \oversetmy{2}{g}_{00}} {\partial x^0} )
%    -\oversetmy{3}{g}_{0j} \frac{\partial \oversetmy{2}{g}_{00}} {\partial x^j}
%    -(\delta_{kj}  -\oversetmy{2}{g}_{jk}) \times \frac{1}{2} \left(
%    \frac{\partial \oversetmy{3}{g}_{0j} }{\partial x^k}
%    +\frac{\partial \oversetmy{3}{g}_{0k}}{\partial x^j}
%    -\frac{\partial \oversetmy{2}{g}_{jk} }{\partial x^0} \right) \\
%    =&\frac{1}{2}\left( \frac{\partial \oversetmy{2}{g}_{00}} {\partial x^0}
%    -\frac{\partial \oversetmy{3}{g}_{0j} }{\partial x^j}
%    -\frac{\partial \oversetmy{3}{g}_{0j}}{\partial x^j}
%    +\frac{\partial \oversetmy{2}{g}_{jj} }{\partial x^0} \right) +O(4)
%\end{align*}
再取$\lambda=i \, (1\leqslant i \leqslant 3)$,得
\begin{equation}\label{chppn:eqn_gaugei2}
    \frac{\partial \oversetmy{2}{g}_{00}} {\partial x^i}
    + 2\frac{\partial \oversetmy{2}{g}_{ij}} {\partial x^j}
    - \frac{\partial \sum_{l}\oversetmy{2}{g}_{ll}} {\partial x^i} =0 .
\end{equation}
%\begin{align*}
%    0=&g^{0 0}\Gamma^i_{0 0} +2 g^{j 0}\Gamma^i_{j 0} + g^{j k}\Gamma^i_{j k} \\
%    =&(-1 -\oversetmy{2}{g}_{00} )
%    \left(-\frac{1}{2}\frac{\partial \oversetmy{2}{g}_{00}} {\partial x^i}
%    -\frac{1}{2}\frac{\partial \oversetmy{4}{g}_{00}} {\partial x^i}
%    + \frac{\partial \oversetmy{3}{g}_{0i}} {\partial x^0}
%    + \frac{1}{2}\oversetmy{2}{g}_{ij}\frac{\partial \oversetmy{2}{g}_{00}} {\partial x^i}\right) \\
%    &+\oversetmy{3}{g}_{0j}\left(
%    \frac{\partial \oversetmy{3}{g}_{0i}} {\partial x^j}
%    + \frac{\partial \oversetmy{2}{g}_{ij}} {\partial x^0}
%    - \frac{\partial \oversetmy{3}{g}_{0j}} {\partial x^i} \right)
%    +(\delta_{kj}  -\oversetmy{2}{g}_{jk})\frac{1}{2}\left(
%    \frac{\partial \oversetmy{2}{g}_{ij}} {\partial x^k}
%    + \frac{\partial \oversetmy{2}{g}_{ik}} {\partial x^j}
%    - \frac{\partial \oversetmy{2}{g}_{jk}} {\partial x^i} \right) \\
%    =&\frac{1}{2}\left(\frac{\partial \oversetmy{2}{g}_{00}} {\partial x^i}
%    + \frac{\partial \oversetmy{2}{g}_{ij}} {\partial x^j}
%    + \frac{\partial \oversetmy{2}{g}_{ij}} {\partial x^j}
%    - \frac{\partial \oversetmy{2}{g}_{jj}} {\partial x^i} \right) +O(3)
%\end{align*}
利用谐和坐标\eqref{chppn:eqn_gauge03}、\eqref{chppn:eqn_gaugei2},
经过一些计算,Ricci曲率\eqref{chppn:eqn_ricciijk}简化为:
\begin{subequations}\label{chppn:eqn_ricci-gauge}
    \begin{align}
        \oversetmy{2}{R}_{00} =& -\frac{1}{2} \nabla^2 \oversetmy{2}{g}_{00}, \\
        \oversetmy{4}{R}_{00} %=&
%         -\frac{1}{2} \nabla^2 \oversetmy{4}{g}_{00}
%        +\frac{1}{2}\frac{\partial^2 \oversetmy{2}{g}_{00}} {\partial x^0\partial x^0}
%        +\frac{1}{2}\frac{\partial^2 \sum_{j}\oversetmy{2}{g}_{jj} }{\partial x^0\partial x^0}
%        -\frac{1}{2} \frac{\partial^2 \sum_{i} \oversetmy{2}{g}_{ii}  }{\partial x^0\partial x^0} \\
%        &+\frac{1}{2}\oversetmy{2}{g}_{ki}\frac{\partial^2 \oversetmy{2}{g}_{00}}
%        {\partial x^k\partial x^i}
%        -\frac{1}{4} \frac{\partial \oversetmy{2}{g}_{00}} {\partial x^k}
%        \frac{\partial \oversetmy{2}{g}_{00}} {\partial x^k}
%        -\frac{1}{4} \left|\nabla \oversetmy{2}{g}_{00} \right|^2          \\
        =& -\frac{1}{2} \nabla^2 \oversetmy{4}{g}_{00}
        +\frac{1}{2}\frac{\partial^2 \oversetmy{2}{g}_{00}} {\partial x^0\partial x^0}
        -\frac{1}{2} \left|\nabla \oversetmy{2}{g}_{00} \right|^2 
        +\frac{1}{2}\oversetmy{2}{g}_{ki}\frac{\partial^2 \oversetmy{2}{g}_{00}}
        {\partial x^k\partial x^i},  \\
        \oversetmy{3}{R}_{i0} =& - \frac{1}{2} \nabla^2 \oversetmy{3}{g}_{i0}, \\
        \oversetmy{2}{R}_{ij} =& - \frac{1}{2}\nabla^2 \oversetmy{2}{g}_{ij}.
    \end{align}
\end{subequations}


\subsection{能动张量近似}
能量-动量张量$T^{ab}$中,$T^{00}$是质量密度,最大量级约为$c^2$;
$T^{0i}$是动量密度,最大量级约为$c^1$;
$T^{ij}$是动量流密度,最大量级约为$c^0$.
在爱因斯坦引力场方程中,能动张量前还有系数$\frac{8\pi G}{c^4}$,
乘上此系数后上述三个物理量的最大量级分别是:
$O(\varepsilon^2)$,$O(\varepsilon^3)$,$O(\varepsilon^4)$.
下面我们写出能动张量的展开表达式;
为了简洁,我们使用$\mathcal{T}^{ab}$来代替$\frac{8\pi G}{c^4}T^{ab}$;
具体来说,我们将$\frac{8\pi G}{c^4}\oversetmy{+2}{T}^{00}$(上面的“$+2$”代表$O(c^2)$)
简记为$\oversetmy{2}{\mathcal{T}}^{00}$(上面的“$2$”代表它的量级是$O(c^{-2})$.

参考上节内容,我们预估能动张量可以展开成
\begin{subequations}
    \begin{align}
        \mathcal{T}^{00} =& \oversetmy{2}{\mathcal{T}}^{00} + \oversetmy{4}{\mathcal{T}}^{00} + \cdots \\
        \mathcal{T}^{0i} =& \oversetmy{3}{\mathcal{T}}^{0i} + \cdots \\
        \mathcal{T}^{ij} =& \oversetmy{4}{\mathcal{T}}^{ij} + \cdots
    \end{align}
\end{subequations}
能动张量的迹$\mathcal{T}=g_{\mu\nu}\mathcal{T}^{\mu\nu}$的展开式为:
\begin{align*}
    \mathcal{T}= g_{\mu\nu}\mathcal{T}^{\mu\nu}
    = g_{00}\mathcal{T}^{00}+2g_{0i}\mathcal{T}^{0i}+g_{ij}\mathcal{T}^{ij} 
%    =&(-1 +\oversetmy{2}{g}_{00} ) (\oversetmy{2}{\mathcal{T}}^{00} 
%+ \oversetmy{4}{\mathcal{T}}^{00})
%    +2 \oversetmy{3}{g}_{0i} \oversetmy{3}{\mathcal{T}}^{0i}
%    +(\delta_{ij}  +\oversetmy{2}{g}_{ij}) \oversetmy{4}{\mathcal{T}}^{ij} \\
    \approx -\oversetmy{2}{\mathcal{T}}^{00} -\oversetmy{4}{\mathcal{T}}^{00} 
    +\oversetmy{2}{g}_{00}\oversetmy{2}{\mathcal{T}}^{00}
     +\sum\nolimits_{k}\oversetmy{4}{\mathcal{T}}^{kk}.
\end{align*}
由此可得
\begin{equation}
    \oversetmy{2}{\mathcal{T}} = -\oversetmy{2}{\mathcal{T}}^{00}; \qquad
    \oversetmy{4}{\mathcal{T}} = -\oversetmy{4}{\mathcal{T}}^{00}
      +\oversetmy{2}{g}_{00}\oversetmy{2}{\mathcal{T}}^{00}
      +\sum\nolimits_{k=1}^{3}\oversetmy{4}{\mathcal{T}}^{kk}  .
\end{equation}
故可求取反迹能动张量
$\overline{\mathcal{T}}^{ab}=\mathcal{T}^{ab} - \frac{1}{2}g^{ab} \mathcal{T} $
(见式\eqref{chfd:eqn_Einstein-B-form})展开式:
\begin{subequations}
\begin{align}
    \oversetmy{2}{\overline{\mathcal{T}}}^{00} =& \oversetmy{2}{\mathcal{T}}^{00}
      - \frac{1}{2} \oversetmy{0}{g}^{00} \oversetmy{2}{\mathcal{T}}
      = \frac{1}{2}\oversetmy{2}{\mathcal{T}}^{00} , \\
    \oversetmy{4}{\overline{\mathcal{T}}}^{00} =& \oversetmy{4}{\mathcal{T}}^{00}
      - \frac{1}{2} \oversetmy{0}{g}^{00} \oversetmy{4}{\mathcal{T}}
      - \frac{1}{2} \oversetmy{2}{g}^{00} \oversetmy{2}{\mathcal{T}}
      = \frac{1}{2}\left( \oversetmy{4}{\mathcal{T}}^{00} 
      +\sum\nolimits_{k=1}^3\oversetmy{4}{T}^{kk} \right), \\
    \oversetmy{3}{\overline{\mathcal{T}}}^{0i} =& \oversetmy{3}{\mathcal{T}}^{0i}  ,\\
    \oversetmy{2}{\overline{\mathcal{T}}}^{ij} =& - \frac{1}{2}\oversetmy{0}{g}^{ij}
      \oversetmy{2}{\mathcal{T}} =  \frac{1}{2} \delta^{ij} \oversetmy{2}{\mathcal{T}}^{00}.
\end{align}
\end{subequations}

爱氏场方程中同样需要协变张量,把反迹能动张量指标降下来,有
\begin{equation}
    \overline{\mathcal{T}}_{\mu\nu}= g_{\mu\rho}g_{\nu\sigma}\overline{\mathcal{T}}^{\rho\sigma}=
       g_{\mu 0}g_{\nu 0}\overline{\mathcal{T}}^{00}+2g_{\mu 0}g_{\nu i} \overline{\mathcal{T}}^{0i}
      +g_{\mu i}g_{\nu j} \overline{\mathcal{T}}^{ij} .
\end{equation}
把度规展开式和逆变反迹能动张量展开式代入,经过略显啰嗦的推导可得
\begin{subequations}\label{chppn:eqn_EMT}
    \begin{align}
        \oversetmy{2}{\overline{\mathcal{T}}}_{00} =& \frac{1}{2}\oversetmy{2}{\mathcal{T}}^{00} , \\
        \oversetmy{4}{\overline{\mathcal{T}}}_{00} =& \frac{1}{2}\Bigl( \oversetmy{4}{\mathcal{T}}^{00} 
        +\sum\nolimits_{k=1}^3\oversetmy{4}{\mathcal{T}}^{kk}
        -2\oversetmy{2}{g}_{00} \oversetmy{2}{\mathcal{T}}^{00} \Bigr), \\
        \oversetmy{3}{\overline{\mathcal{T}}}_{i0} =& - \oversetmy{3}{\mathcal{T}}^{i0} , \\
        \oversetmy{2}{\overline{\mathcal{T}}}_{ij} =& \frac{1}{2} \delta_{ij} \oversetmy{2}{\mathcal{T}}^{00}.
    \end{align}
\end{subequations}





\section{基本方程}


\subsection{后牛顿引力场方程式}\label{chppn:sec_PNFE}
利用Ricci曲率展开式\eqref{chppn:eqn_ricci-gauge}和能动张量展开式\eqref{chppn:eqn_EMT},
再由爱氏引力场方程\eqref{chfd:eqn_Einstein-B-form}可以
得到谐和坐标(\eqref{chppn:eqn_gauge03}、\eqref{chppn:eqn_gaugei2})下
的后牛顿方程式:
\begin{subequations}\label{chppn:eqn_Post-Newtonian}
    \begin{align}
        \nabla^2 \oversetmy{2}{g}_{00} =& -\frac{8\pi G}{c^4} \oversetmy{+2}{T}^{00} , \label{chppn:eqn_Post-Newtonian-R00} \\
        \nabla^2 \oversetmy{4}{g}_{00} =&
        \frac{\partial^2 \oversetmy{2}{g}_{00}} {\partial (x^0)^2}
        - \left|\nabla \oversetmy{2}{g}_{00} \right|^2 
        +\oversetmy{2}{g}_{ki}\frac{\partial^2 \oversetmy{2}{g}_{00}}  {\partial x^k\partial x^i}
        +\frac{8\pi G}{c^4}\left(2\oversetmy{2}{g}_{00} \oversetmy{+2}{T}^{00}
        -\oversetmy{0}{T}^{00} -\sum\nolimits_{k=1}^3\oversetmy{0}{T}^{kk}  \right)   ,  
           \label{chppn:eqn_Post-Newtonian-R004} \\
        \nabla^2 \oversetmy{3}{g}_{i0} =& 2\times \frac{8\pi G}{c^4}\oversetmy{+1}{T}^{i0}  , 
           \label{chppn:eqn_Post-Newtonian-Ri0}\\
        \nabla^2 \oversetmy{2}{g}_{ij} =& - \delta_{ij} \frac{8\pi G}{c^4}\oversetmy{+2}{T}^{00}. 
           \label{chppn:eqn_Post-Newtonian-Rij}
    \end{align}
\end{subequations}
我们已将能动张量前的系数$\frac{8\pi G}{c^4}$显示写出.注$x^0\equiv ct$.

因为$\frac{8\pi G}{c^4} \oversetmy{+1}{T}^{i0}$的最高阶是$O(c^{-3})$,所以根据爱氏场方程,
度规分量$g_{i0}$的最高阶也是$O(c^{-3})$,没有$O(c^{-1})$阶项;否则不匹配.


为了方便使用,下面我们用一些符号参数来代替$\oversetmy{2}{g}_{00}$、$\oversetmy{4}{g}_{00}$等.
当$|\boldsymbol{x}|\to \infty$时$g_{\mu\nu}$要趋于Lorentz度规$\eta_{\mu\nu}$,
故在无穷远处$\oversetmy{2}{g}_{00}$、$\oversetmy{4}{g}_{00}$、
$\oversetmy{3}{g}_{i0}$、$\oversetmy{2}{g}_{ij}$为零;
我们应用此种边界条件将微分方程\eqref{chppn:eqn_Post-Newtonian}化为积分方程.

在\S\ref{chle:sec_Newton-limit}中,我们知道在牛顿近似中$g_{00}=-1+2 U / c^{2}$ ,即有
\begin{equation}\label{chppn:eqn_gd002}
	\oversetmy{2}{g}_{00}=+\frac{2 U}{c^{2}} = - \oversetmy{2}{g}^{00}.
\end{equation}
其中$U$是牛顿引力势,由式\eqref{chppn:eqn_Post-Newtonian-R00}可知$U$由Poisson方程决定
\begin{equation}\label{chppn:eqn_DDU=T}
	\nabla^{2} U=-\frac{4 \pi G}{c^{2}} \oversetmy{+2}{T} ^{00} .
\end{equation}
我们已知$T^{00}$是质能密度,其对应最大量级为$O\left(c^{2}\right)$.
因$\oversetmy{2}{g}_{00}$(即 $U$ )必须在无穷远处为零,
所以式\eqref{chppn:eqn_DDU=T}的解是
\begin{equation}
	U(t, \boldsymbol{x}) = \frac{G}{c^{2}} \int \mathrm{d}^{3} y 
	\frac{\oversetmy{+2}{T} ^{00}\left(t, \boldsymbol{y}\right)}
	{\left|\boldsymbol{x}-\boldsymbol{y}\right|} .
\end{equation}
对比式\eqref{chppn:eqn_Post-Newtonian-R00}与式\eqref{chppn:eqn_Post-Newtonian-Rij},
由式\eqref{chppn:eqn_DDU=T}可得到$\oversetmy{2}{g}_{ij}$在无限远处为零的解是
\begin{equation}\label{chppn:eqn_gdij2}
	\oversetmy{2}{g}_{ij}= \delta_{ij} \frac{2 U}{c^{2}} . 
\end{equation}
引入一个新的矢量势$\zeta_{i}$,使得
\begin{equation}\label{chppn:eqn_zetag0i}
	\zeta_{i} \equiv \zeta^{i} \overset{def}{=} -\frac{c^{3}}{4}\oversetmy{3}{g}_{i0}  .
\end{equation}
则由方程式\eqref{chppn:eqn_Post-Newtonian-Ri0}可得
\begin{equation}\label{chppn:eqn_zetai}
	\nabla^{2} \zeta_{i}=-\frac{4 \pi G}{c} \oversetmy{+1}{T}^{i0}, 
\end{equation}
当$\zeta_i$在无穷远处为零时,方程式\eqref{chppn:eqn_zetai}的解是
\begin{equation}
	\zeta_{i}(t, \boldsymbol{x})=\frac{G}{c} \int \mathrm{d}^{3} y 
	\frac{\oversetmy{+1}{T}^{i 0}\left(t, \boldsymbol{y}\right)}
	{\left|\boldsymbol{x}-\boldsymbol{y}\right|} .
\end{equation}
将\eqref{chppn:eqn_gd002}、\eqref{chppn:eqn_gdij2}、\eqref{chppn:eqn_zetag0i}代入
谐和规范条件\eqref{chppn:eqn_gauge03}得
\begin{equation}\label{chppn:eqn_4Uzeta}
	\frac{\partial U}{\partial t}+\sum\nolimits_{j=1}^{3}\frac{\partial \zeta_j}{\partial x^j}=
	\frac{\partial U}{\partial t}+\nabla \cdot \boldsymbol{\zeta}=0 .
\end{equation}
同理可知谐和规范条件\eqref{chppn:eqn_gaugei2}自动满足.
%\begin{align*}
%	0=&\frac{\partial \oversetmy{2}{g}_{00}} {\partial x^0}
%	-2\frac{\partial \oversetmy{3}{g}_{0j} }{\partial x^j}
%	+\frac{\partial \sum_{j}\oversetmy{2}{g}_{jj} }{\partial x^0} 
%	= \frac{\partial 2U} {c^3 \partial t}
%	-2\frac{\partial \zeta_{j} }{c^3 \partial x^j}
%	+\frac{\partial 6U }{c^3 \partial t} \quad \Rightarrow \\
%	0=& \frac{\partial U} { \partial t}
%	-\frac{\partial \zeta_{j} }{ \partial x^j}
%	+\frac{\partial 3U }{ \partial t}
%	=4\frac{\partial U} { \partial t} -\frac{\partial \zeta_{j} }{ \partial x^j} .
%\end{align*}


最后,利用式\eqref{chppn:eqn_gd002},\eqref{chppn:eqn_DDU=T}和\eqref{chppn:eqn_gdij2}并借助于恒等式
\begin{equation}\label{chppn:eqn_tmpphi1}
	(\nabla U) \cdot \nabla U=\tfrac{1}{2} \nabla^{2}\left(U^{2}\right)-U \nabla^{2} U 
\end{equation}
可将方程\eqref{chppn:eqn_Post-Newtonian-R004}化简
\begin{align*}
	\nabla^{2} \oversetmy{4}{g}_{00}= & \frac{2}{c^{2}} U_{,00}
	+\frac{4}{c^{4}} U \nabla^{2} U -\frac{4}{c^{4}} (\nabla U) \cdot \nabla U
	-\frac{8 \pi G}{c^{4}} \oversetmy{0}{T}^{00}
	+\frac{32 \pi G}{c^{4}} \frac{U}{c^{2}} \oversetmy{+2}{T}^{00}
	-\frac{8 \pi G}{c^{4}} \oversetmy{0}{T}^{ii} \\
	\xlongequal{\ref{chppn:eqn_tmpphi1}}
	& \frac{2}{c^{2}} U_{, 00}+\frac{8}{c^{4}} U \nabla^{2} U
	-\frac{2}{c^{4}} \nabla^{2}\left(U^{2}\right)-\frac{8 \pi G}{c^{4}} \oversetmy{0}{T}^{00} 
	+\frac{32 \pi G}{c^{4}} \frac{U}{c^{2}} \oversetmy{+2}{T}^{00}
	-\frac{8 \pi G}{c^{4}} \oversetmy{0}{T}^{ii} \\
	\xlongequal{\ref{chppn:eqn_DDU=T}}& 
	\frac{2}{c^{4}} \frac{\partial^{2} U}{\partial t^{2}} 
	-\frac{2}{c^{4}} \nabla^{2}\left(U^{2}\right)
	-\frac{8 \pi G}{c^{4}}\left(\oversetmy{0}{T}^{00}+\oversetmy{0}{T}^{ii}\right) .
\end{align*}
令(即定义一个新的势$\Psi$来代替$\oversetmy{4}{g}_{00}$)
\begin{equation}
	 \oversetmy{4}{g}_{00} +\frac{2 U^{2}}{c^{4}} \equiv \frac{2 \Psi}{c^{4}}.
\end{equation}
则上上式(即方程\eqref{chppn:eqn_Post-Newtonian-R004})可改写为
\begin{equation}
	\nabla^{2} \Psi=\frac{\partial^{2} U}{\partial t^{2}}
	-4 \pi G \oversetmy{0}{T}^{00} 
	-4 \pi G \sum\nolimits_{i=1}^3 \oversetmy{0}{T}^{i i} .
\end{equation}
因$\oversetmy{4}{g}_{00}$(以及$\Psi$)必须在无限远处为零,故上述方程的解为
\begin{equation}\label{chppn:eqn_g004-Psi}
	\Psi(t, \boldsymbol{x})= \int \frac{\mathrm{d}^{3} y}{\left|\boldsymbol{x}-\boldsymbol{y}\right|}
	\left[G \oversetmy{0}{T}^{00} \left(t, \boldsymbol{y}\right)
	+G \oversetmy{0}{T}^{ii} \left(t, \boldsymbol{y}\right)
	-\frac{1}{4 \pi} \frac{\partial^{2} U\left(t, \boldsymbol{y}\right)}{\partial t^{2}}\right] .
\end{equation}


\subsection{参数化的运动方程}

用\S\ref{chppn:sec_PNFE}中的参数场($U$、$\Psi$、$\zeta_i$)表示度规及其逆(参考式\eqref{chppn:eqn_guab}),有
\begin{subequations}\label{chppn:eqn_metric-PN}
\begin{align}
	g_{00} =& -1+\frac{2U}{c^2} +\frac{2}{c^4}\left(\Psi-U^2\right)+O\left(c^{-6}\right), \\
	g_{0j} =& -\frac{4}{c^3} \zeta_j+O\left(c^{-5}\right), \\
	g_{ij} =& \left(1+{2U}/{c^2}\right) \delta_{ij}+O\left(c^{-4}\right). \\
	g^{00} =& -1 -\frac{2U}{c^2} -\frac{2}{c^{4}}\left(\Psi+U^2\right) +O\left(c^{-6}\right)  ,\\
	g^{0j} =& -\frac{4}{c^3} \zeta^j +O\left(c^{-5}\right) , \\
	g^{ij} =& \left(1-{2U}/{c^2}\right)\delta^{ij} +O\left(c^{-4}\right) . 
\end{align}
\end{subequations}
%可以求出度规行列式($g=\det(g_{\mu\nu})$):
%\begin{equation*}
%	g =- \left(1+\frac{2U}{c^2}\right)^2 \left(1-2 c^{-4} \left(3 U^2+\Psi \right)
%	-4 c^{-6} \left(-4 \boldsymbol{\zeta}\cdot\boldsymbol{\zeta}+U^3+U \Psi \right)\right) .
%\end{equation*}
度规行列式($g=\det(g_{\mu\nu})$)的表达式:
\begin{equation}\label{chppn:eqn_sqrtg}
	\sqrt{-g} = 1+ 2U/c^2 + O(c^{-4}).
\end{equation}
将式\eqref{chppn:eqn_metric-PN}代入\S\ref{chppn:eqn_CPN}中各式可求出克氏符的近似表示
\begin{subequations}\label{chppn:eqn_Gamma-PN}
\begin{align}
	\oversetmy{2}{\Gamma}_{00}^{i}=&-\frac{1}{c^{2}} \frac{\partial U}{\partial x^{i}} ,\quad
	\oversetmy{4}{\Gamma}_{00}^{i}=\frac{1}{c^{4}}\left[\frac{\partial}{\partial x^{i}}\left(2 U^{2}-\Psi\right)
	-4\frac{\partial \zeta_{i}}{\partial t}\right] ,\\
	\oversetmy{3}{\Gamma}_{j0}^{i}=&\frac{1}{c^{3}}\left[2\left(\frac{\partial \zeta_{j}}{\partial x^{i}}
	-\frac{\partial \zeta_{i}}{\partial x^{j}}\right)+\delta_{i j} \frac{\partial U}{\partial t}\right], \\
	\oversetmy{2}{\Gamma}_{jk}^{i}=&\frac{1}{c^{2}}\left(\delta_{i j} \frac{\partial U}{\partial x^{k}}
	+\delta_{i k} \frac{\partial U}{\partial x^{j}}-\delta_{k j} \frac{\partial U}{\partial x^{i}}\right), \\
	\oversetmy{3}{\Gamma}_{00}^{0}=&-\frac{1}{c^{3}} \frac{\partial U}{\partial t}, \qquad 
	\oversetmy{5}{\Gamma}_{00}^{0} = \frac{1}{c^5} \left(4\zeta_j \frac{\partial U}{\partial x^j}
	-\frac{\partial \Psi}{\partial t}  \right), \\
	\oversetmy{2}{\Gamma}_{j0}^{0}=&-\frac{1}{c^{2}} \frac{\partial U}{\partial x^{j}},\qquad
	\oversetmy{4}{\Gamma}_{j0}^{0} = - \frac{1}{c^4} \frac{\partial \Psi}{\partial x^j}, \\
	\oversetmy{3}{\Gamma}_{jk}^{0} = & \frac{1}{c^3} \delta_{jk} \frac{\partial U }{\partial t}
	+\frac{2}{c^3} \left( \frac{\partial \zeta_{j} }{\partial x^k}
	+\frac{\partial \zeta_{k} }{\partial x^j}	 \right).
\end{align}
\end{subequations}


%\begin{align*}
%	\oversetmy{5}{\Gamma}^0_{00} =&
%	-\frac{1}{2}\left(\frac{\partial \oversetmy{4}{g}_{00}}{\partial x^0}
%	+\oversetmy{2}{g}_{00}\frac{\partial \oversetmy{2}{g}_{00}}{\partial x^0}
%	+\oversetmy{3}{g}_{0j}\frac{\partial \oversetmy{2}{g}_{00}}{\partial x^j} \right)
%	=-\frac{1}{c^5} \left(\frac{\partial (U^2 + c^4 \oversetmy{4}{g}_{00}/2)}{\partial t} 
%	+\zeta_j \frac{\partial U}{\partial x^j} \right) \\
%	=&\frac{1}{c^5} \left(\frac{\partial \Psi}{\partial t} 
%	-\zeta_j \frac{\partial U}{\partial x^j} \right) \\
%	\oversetmy{4}{\Gamma}^0_{0j} =& -\frac{1}{2} \oversetmy{2}{g}_{00}\frac{\partial \oversetmy{2}{g}_{00}}{\partial x^j}
%	-\frac{1}{2}\frac{\partial \oversetmy{4}{g}_{00}}{\partial x^j}
%	= -\frac{1}{c^4} \left(\frac{\partial U^2 }{\partial x^j}
%	+\frac{\partial c^4 \oversetmy{4}{g}_{00}/2}{\partial x^j} \right)
%\end{align*}

把上述结果带入自由质点测地线方程\eqref{chppn:eqn_PN},
则可以得到\CJKunderwave{有质量质点后牛顿近似的运动方程}
(其中$v_i\equiv v^i \equiv \frac{{\rm d} x^i}{{\rm d} t}$、
$\partial_i \equiv \partial^i \equiv \frac{\partial}{\partial x^i}$、
$v^2 \equiv v_i v^i$):
\begin{align}
	\frac{{\rm d} v^i}{{\rm d} t}=& +\partial_i U
	+\frac{1}{c^2}\Bigl[ \left(v^2-4 U\right) \partial_i U
	-\left(4 v^k \partial_k U+3 \partial_t U\right) v^i \notag \\
	& -4 v^k\left(\partial_i \zeta_k-\partial_k \zeta_i\right)
	+4 \partial_t \zeta_i+\partial_i \Psi\Bigr]+O(c^{-4}) . \label{chppn:eqn_PN-ddxdt2}
\end{align}
很明显,式\eqref{chppn:eqn_PN-ddxdt2}等号右端第一项($\partial_i U$)为牛顿引力势的梯度(即牛顿引力);
这项称为“0PN”项,“PN”是指Post Newton.
剩余项是后牛顿近似中精确到$O(c^{-2})$阶的修正项,此项称为“1PN”项.
如果精确到$O(c^{-3})$阶,则称为“1.5PN”项;
如果精确到$O(c^{-4})$阶,则称为“2PN”项;等等.

%\begin{align*}
%	\frac{{\rm d}^2 x^i}{{\rm d} t^2} =& 
%	-c^2 \Gamma_{00}^{i} - 2c \Gamma_{j0}^{i}\frac{{\rm d} x^j}{{\rm d}t}
%	-\Gamma_{jk}^{i} \frac{{\rm d} x^j}{{\rm d}t} \frac{{\rm d} x^k}{{\rm d}t} 
%	+\frac{1}{c}\frac{{\rm d} x^i}{{\rm d}t} \left(
%	c^2 \Gamma_{00}^{0} + 2c \Gamma_{j0}^{0}\frac{{\rm d} x^j}{{\rm d}t}
%	+\Gamma_{jk}^{0} \frac{{\rm d} x^j}{{\rm d}t}
%	\frac{{\rm d} x^k}{{\rm d}t}  \right) \\
%	=& -c^2  \left\{-\frac{1}{c^{2}} \frac{\partial U}{\partial x^{i}} 
%	+\frac{1}{c^{4}}\left[\frac{\partial}{\partial x^{i}}\left(2 U^{2}-\Psi\right)
%	-4\frac{\partial \zeta_{i}}{\partial t}\right] \right\} \\
%	& - 2c \left\{\frac{1}{c^{3}}\left[2\left(\frac{\partial \zeta_{j}}{\partial x^{i}}
%	-\frac{\partial \zeta_{i}}{\partial x^{j}}\right)+\delta_{i j} \frac{\partial U}{\partial t}\right]\right\} v^j \\
%	&- \left\{ \frac{1}{c^{2}}\left(\delta_{i j} \frac{\partial U}{\partial x^{k}}
%	+\delta_{i k} \frac{\partial U}{\partial x^{j}}-\delta_{k j} \frac{\partial U}{\partial x^{i}}\right)\right\}v^j v^k\\
%	&+ c v^i \left\{ -\frac{1}{c^{3}} \frac{\partial U}{\partial t}
%	+ \frac{1}{c^5} \left(4\zeta_j \frac{\partial U}{\partial x^j}
%	-\frac{\partial \Psi}{\partial t}  \right)\right\} \\
%	& +2v^i v^j \left\{-\frac{1}{c^{2}} \frac{\partial U}{\partial x^{j}}
%	- \frac{1}{c^4} \frac{\partial \Psi}{\partial x^j}\right\}\\
%	&+c^{-1}v^i v^j v^k \left\{\frac{1}{c^3} \delta_{jk} \frac{\partial U }{\partial t}
%	+\frac{2}{c^3} \left( \frac{\partial \zeta_{j} }{\partial x^k}
%	+\frac{\partial \zeta_{k} }{\partial x^j}	 \right)\right\} \\
%	=&  \frac{\partial U}{\partial x^{i}} 
%	-\frac{1}{c^{2}}\left[\frac{\partial}{\partial x^{i}}\left(2 U^{2}-\Psi\right)
%	-4\frac{\partial \zeta_{i}}{\partial t}\right]  \\
%	& - \frac{1}{c^2}\left[4 v^j\left(\frac{\partial \zeta_{j}}{\partial x^{i}}
%	-\frac{\partial \zeta_{i}}{\partial x^{j}}\right)
%	+2 v^i \frac{\partial U}{\partial t}\right] \\
%	&- \frac{1}{c^{2}}\left( v^i v^k \frac{\partial U}{\partial x^{k}}
%	+v^j v^i \frac{\partial U}{\partial x^{j}}
%	-v^2 \frac{\partial U}{\partial x^{i}}\right)\\
%	& -\frac{v^i}{c^2} \frac{\partial U}{\partial t}
%	+ \frac{v^i}{c^4} \left(4\zeta_j \frac{\partial U}{\partial x^j}
%	-\frac{\partial \Psi}{\partial t}  \right) \\
%	&  -\frac{2v^i v^j}{c^{2}} \frac{\partial U}{\partial x^{j}}
%	- \frac{2v^i v^j}{c^4} \frac{\partial \Psi}{\partial x^j}\\
%	&+c^{-4}v^i v^j v^k \left\{\delta_{jk} \frac{\partial U }{\partial t}
%	+2 \left( \frac{\partial \zeta_{j} }{\partial x^k}
%	+\frac{\partial \zeta_{k} }{\partial x^j}	 \right)\right\} \\
%=&  \frac{\partial U}{\partial x^{i}} 
%-\frac{1}{c^{2}} \Bigl[\frac{\partial}{\partial x^{i}}\left(2 U^{2}-\Psi\right)
%-4\frac{\partial \zeta_{i}}{\partial t}  
% +4 v^j\left(\frac{\partial \zeta_{j}}{\partial x^{i}}
%-\frac{\partial \zeta_{i}}{\partial x^{j}}\right)
%+2 v^i \frac{\partial U}{\partial t} \\
%&+ v^i v^k \frac{\partial U}{\partial x^{k}}
%+v^j v^i \frac{\partial U}{\partial x^{j}}
%-v^2 \frac{\partial U}{\partial x^{i}}
% +v^i \frac{\partial U}{\partial t}
%+ 2v^i v^j\frac{\partial U}{\partial x^{j}} \Bigr] \\
%& + \frac{v^i}{c^4} \left\{  4\zeta_j \frac{\partial U}{\partial x^j}
%-\frac{\partial \Psi}{\partial t}
%-  2 v^j \frac{\partial \Psi}{\partial x^j}
%+   v^2  \frac{\partial U }{\partial t}
%+2  v^j v^k \left( \frac{\partial \zeta_{j} }{\partial x^k}
%+\frac{\partial \zeta_{k} }{\partial x^j}	 \right)\right\} \\
%=&  \frac{\partial U}{\partial x^{i}} 
%+\frac{1}{c^{2}} \Bigl[ (v^2-4 U) \partial_i U
%- v^i \left(3\frac{\partial U}{\partial t}+ 4 v^k \frac{\partial U}{\partial x^{k}}\right) \\
%  &-4 v^k \left(\frac{\partial \zeta_{k}}{\partial x^{i}}
%-\frac{\partial \zeta_{i}}{\partial x^{k}}\right)
%+4\frac{\partial \zeta_{i}}{\partial t}  
%+\partial_i\Psi  \Bigr] \\
%& + \frac{v^i}{c^4} \left\{  
%4\zeta_k \frac{\partial U}{\partial x^k} -\frac{\partial \Psi}{\partial t}
%-2 v^k \frac{\partial \Psi}{\partial x^k}+v^2  \frac{\partial U }{\partial t}
%+2v^j v^k \left( \frac{\partial \zeta_{j} }{\partial x^k}
%+\frac{\partial \zeta_{k} }{\partial x^j}	 \right)\right\} 
%\end{align*}

下面导出谐和坐标系$\{t,x^i\}$中时间$t$与固有时$\tau$的关联.
由式\eqref{chppn:eqn_metric-PN}得
\begin{align*}
	\mathrm{d} s^{2} & =-c^{2} \mathrm{d} \tau^{2}=g_{\mu \nu} \mathrm{d} x^{\mu} \mathrm{d} x^{\nu} 
	=c^{2} \mathrm{d} t^{2}\left(g_{00}+g_{i j} \frac{1}{c^{2}} 
	\frac{\mathrm{d} x^{i}}{\mathrm{d} t} \frac{\mathrm{d} x^{j}}{\mathrm{d} t}
	+2 g_{0 i} \frac{1}{c} \frac{\mathrm{d} x^{i}}{\mathrm{d} t}\right) \\
	& =-c^{2} \mathrm{d} t^{2}\left(1-\frac{2 U}{c^{2}}-\frac{2}{c^{4}}\left(\Psi-U^2\right)
	-\frac{v^{2}}{c^{2}}\left(1+\frac{2U}{c^2}\right) 
	+ \frac{8}{c^{4}} \zeta_i  v^i  +O(c^{-6})\right).
\end{align*}
由此可得
\begin{align}
	\frac{\mathrm{d} \tau}{\mathrm{d} t} = & 
	\left(1-\frac{1}{c^{2}}(2 U + v^{2}) + \frac{2}{c^{4}} \left(
	4\zeta_i  v^i -\Psi+U^2 -Uv^{2} \right)	+O(c^{-6})\right)^{1/2} \notag \\
	\approx & 1-\frac{1}{c^{2}} \left(U + \frac{v^{2}}{2}\right)
	+\frac{1}{c^{4}}\left(\frac{U^2}{2} + 4\zeta_i  v^i -\Psi 
	 -\frac{1}{8} v^4 -\frac{3}{2}  U v^2 \right) .
	\label{chppn:eqn_dtaudt}
\end{align}
上式第二步近似用到了$(1+a x^2+b x^4)^{1/2}\approx 1+\frac{a x^2}{2} + (-\frac{a^2}{8}+\frac{b}{2})x^4$.
再根据$(1+a x^2+b x^4)^{-1/2}\approx 1-\frac{a x^2}{2} + (\frac{3 a^2}{8}-\frac{b}{2})x^4$,有
\begin{equation}\label{chppn:eqn_dtdtau}
	\frac{\mathrm{d} t}{\mathrm{d} \tau}
	\approx  1+\frac{1}{c^{2}}\left(U+\frac{v^{2}}{2} \right)
	+\frac{1}{c^{4}}\left( \frac{U^2}{2} -4\zeta_i  v^i +\Psi
	  + \frac{3}{8}v^4 + \frac{5}{2} U v^2 \right).
\end{equation}
式\eqref{chppn:eqn_dtaudt}、\eqref{chppn:eqn_dtdtau}精确到$O(c^{-4})$.


\subsection{电磁波运动方程}

电磁波测地线为(式中 $\lambda$ 为仿射参数):
\begin{equation}\label{chppn:eqn_EM-geo}
	\frac{\mathrm{d}^{2} x^{\mu}}{\mathrm{d} \lambda^{2}}
	+\Gamma_{\nu \sigma}^{\mu} \frac{\mathrm{d} x^{\nu}}{\mathrm{d} \lambda} 
	\frac{\mathrm{d} x^{\sigma}}{\mathrm{d} \lambda}=0. 
\end{equation}
我们用谐和坐标中的时间$t$替换$\lambda$,有
\begin{equation}\label{chppn:eqn_tlambda}
	\frac{\mathrm{d} x^{\mu}}{\mathrm{d} \lambda}=\frac{\mathrm{d} x^{\mu}}{\mathrm{d} t}
	\left(\frac{\mathrm{d} t}{\mathrm{d} \lambda}\right), \quad
	\frac{\mathrm{d}^{2} x^{\mu}}{\mathrm{d} \lambda^{2}}
	=\frac{\mathrm{d}^{2} x^{\mu}}{\mathrm{d} t^{2}}
	\left(\frac{\mathrm{d} t}{\mathrm{d} \lambda}\right)^{2}
	+\left(\frac{\mathrm{d} x^{\mu}}{\mathrm{d} t}\right) 
	\frac{\mathrm{d}^{2} t}{\mathrm{d} \lambda^{2}} .
\end{equation}
把式\eqref{chppn:eqn_tlambda}中的第二式代入式\eqref{chppn:eqn_EM-geo},
并令式\eqref{chppn:eqn_EM-geo}中的$\mu=i$,可得
\begin{equation}\label{chppn:eqn_vi-photon}
	\frac{\mathrm{d}^{2} x^{i}}{\mathrm{d} t^{2}}
	+\left(\Gamma_{\nu \sigma}^{i}-c^{-1} \Gamma_{\nu \sigma}^{0} \frac{\mathrm{d} x^{i}}{\mathrm{d} t}\right) 
	\frac{\mathrm{d} x^{\nu}}{\mathrm{d} t} \frac{\mathrm{d} x^{\sigma}}{\mathrm{d} t}=0 .
\end{equation}
我们将式\eqref{chppn:eqn_Gamma-PN}带入式\eqref{chppn:eqn_vi-photon}可得
(注$x^0=ct$、$v_i=v^i=\frac{\mathrm{d} x^{i}}{\mathrm{d} t}$)
\begin{align*}
	\frac{\mathrm{d}^{2} x^{i}}{\mathrm{d} t^{2}}
%	=& 
%	\left(c^{-1} \Gamma_{\nu \sigma}^{0} \frac{\mathrm{d} x^{i}}{\mathrm{d} t} -\Gamma_{\nu \sigma}^{i}\right) 
%	\frac{\mathrm{d} x^{\nu}}{\mathrm{d} t} \frac{\mathrm{d} x^{\sigma}}{\mathrm{d} t} \\
%	=& \left(c^{-1} \Gamma_{00}^{0} \frac{\mathrm{d} x^{i}}{\mathrm{d} t} -\Gamma_{00}^{i}\right) 
%	\frac{\mathrm{d} x^{0}}{\mathrm{d} t} \frac{\mathrm{d} x^{0}}{\mathrm{d} t} 
%	+2\left(c^{-1} \Gamma_{0 k}^{0} \frac{\mathrm{d} x^{i}}{\mathrm{d} t} -\Gamma_{0 k}^{i}\right) 
%	\frac{\mathrm{d} x^{0}}{\mathrm{d} t} \frac{\mathrm{d} x^{k}}{\mathrm{d} t} \\
%	&+\left(c^{-1} \Gamma_{jk}^{0} \frac{\mathrm{d} x^{i}}{\mathrm{d} t} -\Gamma_{jk}^{i}\right) 
%	\frac{\mathrm{d} x^{j}}{\mathrm{d} t} \frac{\mathrm{d} x^{k}}{\mathrm{d} t} \\
%	=& -\frac{1}{c^{2}} \frac{\partial U}{\partial t} v^i +\frac{\partial U}{\partial x^{i}}
%	+ 2\left( -\frac{1}{c^{2}} \frac{\partial U}{\partial x^{j}} v^i 
%	- \frac{1}{c^{2}}\left[2\left(\frac{\partial \zeta_{j}}{\partial x^{i}}
%	-\frac{\partial \zeta_{i}}{\partial x^{j}}\right)+\delta_{i j} \frac{\partial U}{\partial t}\right] \right) v^j \\
%	&- \frac{1}{c^{2}}\left(\delta_{i j} \frac{\partial U}{\partial x^{k}}
%	+\delta_{i k} \frac{\partial U}{\partial x^{j}}-\delta_{k j} \frac{\partial U}{\partial x^{i}}\right)v^j v^k \\
%	=& \frac{\partial U}{\partial x^{i}} - \frac{1}{c^2} \left[
%	\frac{\partial U}{\partial t} v^i 
%	+2\frac{\partial U}{\partial x^{j}} v^i v^j 
%	+4\left(\frac{\partial \zeta_{j}}{\partial x^{i}}
%	+\frac{\partial \zeta_{i}}{\partial x^{j}}\right)v^j 
%	+2v^i \frac{\partial U}{\partial t}
%	+v^i v^k \frac{\partial U}{\partial x^{k}}
%	+v^i v^j \frac{\partial U}{\partial x^{j}}
%	-v^2 \frac{\partial U}{\partial x^{i}}	\right] \\
	=& \left( 1+ \frac{v^2}{c^2}\right)\frac{\partial U}{\partial x^{i}} - \frac{1}{c^2} \left[
	3\frac{\partial U}{\partial t} v^i 
	+4\frac{\partial U}{\partial x^{j}} v^i v^j 
	+4\left(\frac{\partial \zeta_{j}}{\partial x^{i}}
	+\frac{\partial \zeta_{i}}{\partial x^{j}}\right)v^j 	\right] .
\end{align*}
电磁波飞行速度$v^i$自然接近于$c$.而$\partial_t U$、$\partial_i \zeta_j$相对于光速是高阶小量;
忽略掉这些小量后,上式变为:
\begin{equation}\label{chppn:eqn_EM-a}
	\frac{\mathrm{d}^{2} x^{i}}{\mathrm{d} t^{2}} = 
	\left( 1+ \frac{v^2}{c^2}\right)\frac{\partial U}{\partial x^{i}} 
	- \frac{4v^i v^j}{c^2} \frac{\partial U}{\partial x^{j}}  + O(c^{-3}).
\end{equation}

我们知道电磁波线长恒为零.利用式\eqref{chppn:eqn_tlambda}可将线长公式变为:
\begin{equation}\label{chppn:eqn_EM-len}
	g_{\mu \nu} \frac{\mathrm{d} x^{\mu}}{\mathrm{d} \lambda} \frac{\mathrm{d} x^{\nu}}{\mathrm{d} \lambda}=0
	\xRightarrow[\text{仿射参数}\lambda\text{替换为}t]{\text{用}\eqref{chppn:eqn_tlambda}\text{将}}
	g_{\mu \nu} \frac{\mathrm{d} x^{\mu}}{\mathrm{d} t} \frac{\mathrm{d} x^{\nu}}{\mathrm{d} t}=0 .
\end{equation}
把式\eqref{chppn:eqn_metric-PN}代入后可得
\begin{align*}
	0=&g_{00}+g_{i j} \frac{1}{c^{2}} 
	\frac{\mathrm{d} x^{i}}{\mathrm{d} t} \frac{\mathrm{d} x^{j}}{\mathrm{d} t}
	+2 g_{0 i} \frac{1}{c} \frac{\mathrm{d} x^{i}}{\mathrm{d} t} 
%	=& -1+\frac{2U}{c^2} 
%	+ \left(1+{2U}/{c^2}\right) \delta_{ij}  \frac{1}{c^{2}} 
%	\frac{\mathrm{d} x^{i}}{\mathrm{d} t} \frac{\mathrm{d} x^{j}}{\mathrm{d} t}
%	-\frac{8}{c^4} \zeta_i \frac{\mathrm{d} x^{i}}{\mathrm{d} t} \\
	\approx -1+\frac{2U}{c^2} + \left(1+\frac{2U}{c^2}\right)  \frac{v^2}{c^{2}} .
\end{align*}
上式已丢掉高阶项.再设电磁波传播方向为$\boldsymbol{n}$(单位矢量),则由上式有
\begin{equation}\label{chppn:eqn_EM-v}
	\boldsymbol{v}=c \left(1-\frac{2U}{c^2}\right)\boldsymbol{n} + O(c^{-3}).
\end{equation}
式\eqref{chppn:eqn_EM-v}中的$\boldsymbol{v}$为电磁波传播速度;
很明显,在引力场中此速度小于真空中的光速$c$.
我们对式\eqref{chppn:eqn_EM-v}求时间导数,有
\begin{align*}
	\frac{\mathrm{d} v^{i}}{\mathrm{d} t} = 
	c\left(1-\frac{2U}{c^2}\right)\frac{\mathrm{d} n^{i}}{\mathrm{d} t}
	- \frac{2n^i}{c} \left(\frac{\partial U}{\partial t} 
	+ \frac{\partial U}{\partial x^j} \frac{\mathrm{d} x^j}{\mathrm{d} t}  \right) .
\end{align*}
相对于光速来说$\partial_t U$是小量,忽略之;由上式再结合式\eqref{chppn:eqn_EM-a},有
\begin{align*}
	&c\left(1-\frac{2U}{c^2}\right)\frac{\mathrm{d} n^{i}}{\mathrm{d} t} = 
	\left( 1+ \frac{v^2}{c^2}\right)\frac{\partial U}{\partial x^{i}} 
	- \frac{4v^i v^j}{c^2} \frac{\partial U}{\partial x^{j}} 
	+\frac{2n^i}{c} \frac{\partial U}{\partial x^j} v^j \\
	=& 	\left( 1+ \left(1-\frac{2U}{c^2}\right)^2\right)\frac{\partial U}{\partial x^{i}} 
	- 4\left(1-\frac{2U}{c^2}\right)^2 n^i n^j \frac{\partial U}{\partial x^{j}} 
	+2n^i n^j \frac{\partial U}{\partial x^j} \left(1-\frac{2U}{c^2}\right)  . 
\end{align*}
忽略高阶小量,由此可得关于电磁波传播方向$\boldsymbol{n}$的方程:
\begin{equation}\label{chppn:eqn_EM-n}
	\frac{\mathrm{d} n^{i}}{\mathrm{d} t} = \frac{2}{c} 
	\left(\delta^{ij}  - n^i n^j \right)
	\frac{\partial U}{\partial x^{j}} + O(c^{-3}) .
\end{equation}





\section{相对论流体力学}
\subsection{理想流体后牛顿近似}\label{chppn:sec_PF-PN}

根据式\eqref{chppn:eqn_dtdtau}(精确到$O(c^{-2})$),四维速度的分量近似为
\begin{align}
	U^{0}=&\frac{\mathrm{d} x^{0}}{\mathrm{d} \tau}
	=c \frac{\mathrm{d} t}{\mathrm{d} \tau}
	=c\left(1+\frac{U}{c^{2}}+\frac{1}{2} \frac{v^{2}}{c^{2}}+\cdots\right) ,\label{chppn:eqn_PF-U0} \\
	U^{i}=&\frac{\mathrm{d} x^{i}}{\mathrm{d} \tau}
	=v^{i} \frac{\mathrm{d} t}{\mathrm{d} \tau}
	=v^{i}\left(1+\frac{U}{c^{2}}+\frac{1}{2} \frac{v^{2}}{c^{2}}+\cdots\right) . \label{chppn:eqn_PF-Ui}
\end{align}


现在对理想流体进行后牛顿近似.其能动张量为式\eqref{chlh:eqn_perfect-fluid-Tab},即下式
\begin{equation}\label{chppn:eqn_pfT}
	T^{\mu\nu}=\rho \left( 1+ \frac{\Pi}{c^2}  + \frac{p}{\rho c^2}  \right) U^\mu U^\nu +p g^{\mu\nu} .
\end{equation}
其中$\rho$是静质量密度,$\Pi$是单位质量内能,$p$是压强,$U^\mu$是质点四速度.


参考式\eqref{chppn:eqn_metric-PN},我们将式\eqref{chppn:eqn_pfT}写成分量形式:
\begin{align}
	T^{00}= & p g^{00}+\left[\rho\left(1+\frac{\Pi}{c^{2}}\right)+\frac{p}{c^{2}}\right] U^{0} U^{0} \notag \\
	= & -p\left(1+\frac{2 U}{c^{2}}\right)+\rho c^{2}\left(1+\frac{\Pi}{c^{2}}\right)
	\left(1+\frac{2 U}{c^{2}}+\frac{v^{2}}{c^{2}}\right) 
	+p\left(1+\frac{2 U}{c^{2}}+\frac{v^{2}}{c^{2}}\right)+\cdots \notag  \\
	= & \rho c^{2}\left(1+\frac{\Pi}{c^{2}}+\frac{2 U}{c^{2}}+\frac{v^{2}}{c^{2}}\right)
	+O(\varepsilon^2),	\label{chppn:eqn_pfT00}  \\
	T^{0 i}= & p g^{0 i}+\left[\rho\left(1+\frac{\Pi}{c^{2}}\right)+\frac{p}{c^{2}}\right] U^{0} U^{i} \notag\\
	= & \rho c v^{i}\left(1+\frac{\Pi}{c^{2}}\right)\left(1+\frac{2 U}{c^{2}}+\frac{v^{2}}{c^{2}}\right) 
	+p \frac{v^{i}}{c}\left(1+\frac{2 U}{c^{2}}+\frac{v^{2}}{c^{2}}\right)+\cdots \notag\\
	= & \rho c v^{i}\left(1+\frac{\Pi}{c^{2}}+\frac{2 U}{c^{2}}+\frac{v^{2}}{c^{2}}+\frac{p}{\rho c^{2}}\right) 
	+O(\varepsilon^3),	\label{chppn:eqn_pfT0i}	\\
	T^{i j}= & p g^{i j}+\left[\rho\left(1+\frac{\Pi}{c^{2}}\right)+\frac{p}{c^{2}}\right] U^{i} U^{j} \notag\\
	=&p \delta^{i j}\left(1-\frac{2 U}{c^{2}}\right)+\rho v^{i} v^{j}\left(1+\frac{\Pi}{c^{2}}\right)
	\left(1+\frac{2 U}{c^{2}}+\frac{v^{2}}{c^{2}}\right) +\frac{p}{c^{2}} v^{i} v^{j}+\cdots \notag\\
	=& p \delta^{i j}\left(1-\frac{2 U}{c^{2}}\right)
	+\rho v^{i} v^{j}\left(1+\frac{\Pi}{c^{2}}
	+\frac{2 U}{c^{2}}+\frac{v^{2}}{c^{2}}+\frac{p}{\rho c^{2}}\right)
	+O(\varepsilon^4) \label{chppn:eqn_pfTij}
\end{align}
可见理想流体中,$T^{00}$的最大量级项$\rho c^{2}$是 $O\left(c^{2}\right)$,
$T^{0 i}$的最大量级项$\rho c v^{i}$是$O\left(c^{1}\right)$,
$T^{i j}$的最大量级项$\rho v^{i} v^{j}+p \delta^{i j}$,是$O\left(c^{0}\right)$;
这与式\eqref{chppn:eqn_EMT}量级一致.

%\begin{example}\label{chppn:exam_rrs}
%	$\rho^*$与$\rho$的差异.
%\end{example}
%在式\eqref{chlh:eqn_rho-star}中,我们引入了$\rho^*$,即
%\begin{align}
%	\rho^* =& \sqrt{-g} \gamma \rho= \sqrt{-g} \rho U^0 / c 
%	\xlongequal[\ref{chppn:eqn_PF-U0}]{\ref{chppn:eqn_sqrtg}}
%	\left(1+ 2U/c^2 + O(c^{-4})\right) \left(1+\frac{U}{c^{2}}
%	+\frac{1}{2} \frac{v^{2}}{c^{2}}+ O(c^{-4})\right) \rho \notag \\
%	= & \rho \left(1+\frac{3U}{c^{2}}+\frac{v^{2}}{2c^{2}} + O(c^{-4})\right). \label{chppn:eqn_rrs}
%\end{align}
%由式\eqref{chppn:eqn_rrs}可知$\rho^*$与$\rho$相差$O(c^{-2})$. \qed

\subsection{参数化流体力学方程}

参见\S\ref{chlh:perfect-fluid}的几个参数列在下面:
\begin{align}
	\gamma \equiv & \frac{\mathrm{d} t}{\mathrm{d} \tau} 
	=\frac{1}{c}\frac{\mathrm{d}x^0}{\mathrm{d} \tau}= U^0/c , \tag{\ref{chlh:eqn_gamma}}\\
	u^i \equiv & v^i \equiv \frac{{\rm d} x^i}{{\rm d}t} 
	=\frac{{\rm d} x^i}{{\rm d}\tau} \frac{\mathrm{d} \tau}{\mathrm{d} t}  
	= U^i /\gamma, \tag{\ref{chlh:eqn_ui}} \\
	\rho^* \overset{def}{=} & \sqrt{-g} \gamma \rho
	=\sqrt{-g} \rho U^0 / c . \tag{\ref{chlh:eqn_rho-star}}
\end{align}


由式\eqref{chppn:eqn_dtdtau}和\eqref{chlh:eqn_gamma}可知:精确到$O(c^{-4})$时,有
\begin{equation}\label{chppn:eqn_gamma-c4}
	\gamma \approx  1+\frac{1}{c^{2}}\left(U+\frac{v^{2}}{2} \right)
	+\frac{1}{c^{4}}\left( \frac{U^2}{2} -4\zeta_i  v^i +\Psi
	+ \frac{3}{8}v^4 + \frac{5}{2} U v^2 \right).
\end{equation}
同理有(参考式\eqref{chppn:eqn_sqrtg},即$\sqrt{-g} \approx 1+ 2U/c^2$)
\begin{equation}\label{chppn:eqn_rho-star}
	\rho^* = \sqrt{-g} \gamma \rho 
	\approx \rho \left(1+\frac{3U}{c^2}+\frac{v^{2}}{2 c^2} \right) + O(c^{-4}).
\end{equation}
用$\rho^*$重新表述式\eqref{chppn:eqn_pfT00}、\eqref{chppn:eqn_pfT0i}、\eqref{chppn:eqn_pfTij},有
\begin{subequations}\label{chppn:eqn_pfT-star}
\begin{align}
	T^{00} = & \rho^* c^{2}\left(1+\frac{\Pi}{c^{2}}-\frac{U}{c^{2}}+\frac{v^{2}}{2c^{2}}\right)
	+O(\varepsilon^2),	\label{chppn:eqn_pfT00-star}  \\
	T^{0 i}	= & \rho^* c v^{i}\left(1+\frac{\Pi}{c^{2}}-\frac{U}{c^{2}}+\frac{v^{2}}{2c^{2}}
	+\frac{p}{\rho^* c^{2}}\right) +O(\varepsilon^3),	\label{chppn:eqn_pfT0i-star}	\\
	T^{i j}	=& p \delta^{i j}\left(1-\tfrac{2 U}{c^{2}}\right)
	+\rho^* v^{i} v^{j}\left(1+\frac{\Pi}{c^{2}}
	-\frac{U}{c^{2}}+\frac{v^{2}}{2c^{2}}+\frac{p}{\rho^* c^{2}}\right)
	+O(\varepsilon^4). \label{chppn:eqn_pfTij-star}
\end{align}
\end{subequations}


\subsubsection{用$\rho^*$表述参数}

将式\eqref{chppn:eqn_pfT-star}中相应量带入式\eqref{chppn:eqn_DDU=T},有
\begin{equation}\label{chppn:eqn_DDU=T-star}
	U(t, \boldsymbol{x}) = \frac{G}{c^{2}} \int \mathrm{d}^{3} y 
	\frac{\oversetmy{+2}{T} ^{00}\left(t, \boldsymbol{y}\right)}
	{\left|\boldsymbol{x}-\boldsymbol{y}\right|}
	= G \int \mathrm{d}^{3} y 
	\frac{\rho^* \left(t, \boldsymbol{y}\right)}
	{\left|\boldsymbol{x}-\boldsymbol{y}\right|} .
\end{equation}
将式\eqref{chppn:eqn_pfT-star}中相应量带入式\eqref{chppn:eqn_zetai},有(注$\zeta_{i}=\zeta^{i}$)
\begin{equation}\label{chppn:eqn_zetai-star}
	\zeta_{i}(t, \boldsymbol{x})=\frac{G}{c} \int \mathrm{d}^{3} y 
	\frac{\oversetmy{+1}{T}^{i 0}\left(t, \boldsymbol{y}\right)}
	{\left|\boldsymbol{x}-\boldsymbol{y}\right|}
	=G \int \mathrm{d}^{3} y 
	\frac{v^i\left(t, \boldsymbol{y}\right) \rho^*\left(t, \boldsymbol{y}\right)}
	{\left|\boldsymbol{x}-\boldsymbol{y}\right|} .
\end{equation}
将式\eqref{chppn:eqn_pfT-star}中相应量带入式\eqref{chppn:eqn_g004-Psi},有
\begin{align*}
	\Psi(t, \boldsymbol{x})=& \int \frac{\mathrm{d}^{3} y}{\left|\boldsymbol{x}-\boldsymbol{y}\right|}
	\left[G \oversetmy{0}{T}^{00} \left(t, \boldsymbol{y}\right)
	+G \oversetmy{0}{T}^{ii} \left(t, \boldsymbol{y}\right)
	-\frac{1}{4 \pi} \frac{\partial^{2} U\left(t, \boldsymbol{y}\right)}{\partial t^{2}}\right] \\
	=& \int \frac{\mathrm{d}^{3} y}{\left|\boldsymbol{x}-\boldsymbol{y}\right|}
	\left[G \rho^* \left(\Pi -U +v^{2}/2 \right) 	+G (\rho^* v^2 + 3 p)  
	-\frac{1}{4 \pi} \partial_{tt} U \right] \\
	=& \int \frac{\mathrm{d}^{3} y}{\left|\boldsymbol{x}-\boldsymbol{y}\right|}
	\left[G \rho^* \left(\Pi - U + 3v^{2}/2 + 3 p/\rho^* \right)
	-\frac{1}{4 \pi} \partial_{tt} U \right] \\
\end{align*}






利用的连续性方程\eqref{chlh:eqn_cont-star}(即$\frac{\partial \rho^* }{\partial t}+\frac{\partial  \rho^* v^i }{\partial x^i}=0$),有
\begin{align*}
	& \frac{\partial}{\partial t} \int \rho^* \left(t, \boldsymbol{y}\right) f\left(\boldsymbol{x}, \boldsymbol{y}\right) \mathrm{d}^{3}y
	 =\int\left(\frac{\partial_t\rho^*\left(t, \boldsymbol{y}\right)}{\partial t}\right) 
	 f\left(\boldsymbol{x}, \boldsymbol{y}\right) \mathrm{d}^{3} y \\
	& \quad=- \int \left[\nabla_y \cdot\left(\rho^* \boldsymbol{v} \right) \right] f\left(\boldsymbol{x}, \boldsymbol{y}\right) \mathrm{d}^{3}y \\
	& \quad=-\int \nabla_y \cdot \left[\rho^* \boldsymbol{v} f\left(\boldsymbol{x}, \boldsymbol{y}\right)\right] \mathrm{d}^{3}y
	+\int \rho^* \boldsymbol{v} \cdot \nabla_y f \left(\boldsymbol{x}, \boldsymbol{y}\right) \mathrm{d}^{3} y  \\
	& \quad=-\oint_{\infty} \rho^* \boldsymbol{v} f\left(\boldsymbol{x}, \boldsymbol{y}\right) \cdot \mathrm{d} S
	+\int \rho^* \boldsymbol{v} \cdot \nabla_y f \left(\boldsymbol{x}, \boldsymbol{y}\right) \mathrm{d}^{3} y  .
\end{align*}
因质量$\rho^*$分布在有限区域中,故上式第一项的面积分为零;从而有
\begin{equation}\label{chppn:eqn_dtrsf=Df}
	\frac{\partial}{\partial t} \int \rho^* \left(t, \boldsymbol{y}\right) f\left(\boldsymbol{x}, \boldsymbol{y}\right) \mathrm{d}^{3}y
	 =\int \rho^*\left(t, \boldsymbol{y}\right) \boldsymbol{v}\left(\boldsymbol{y}\right) \cdot 
	 \nabla_y f \left(\boldsymbol{x}, \boldsymbol{y}\right) \mathrm{d}^{3} y .
\end{equation}


我们定义
\begin{equation}
	X \left(t, \boldsymbol{x}\right) \overset{def}{=} G \int \rho^* \left(t, \boldsymbol{y}\right)
	\left|\boldsymbol{x}-\boldsymbol{y}\right| \mathrm{d}^{3} y .
\end{equation}


利用式\eqref{chppn:eqn_dtrsf=Df}可得
\begin{align*}
	\partial_{t}X\left(t, \boldsymbol{x}\right) & = G \frac{\partial}{\partial t} 
	\int \rho^* \left(t, \boldsymbol{y}\right)
	\left|\boldsymbol{x}-\boldsymbol{y}\right| \mathrm{d}^{3} y
	=G \int \rho^{*} \boldsymbol{v}\cdot \nabla_y\left|\boldsymbol{x}-\boldsymbol{y}\right| \mathrm{d}^{3} y \\
	& =-G\int \rho^{*}\left(t, \boldsymbol{y}\right) \frac{\boldsymbol{v}\left(\boldsymbol{y}\right) \cdot\left(\boldsymbol{x}-\boldsymbol{y}\right)}{\left|\boldsymbol{x}-\boldsymbol{y}\right|} 
	\mathrm{d}^{3}y, \\
	\chi_{, 0 i} & =\frac{G}{c} \int \frac{\rho^{\prime} v^{\prime i}}{\left|\boldsymbol{x}-\boldsymbol{x}^{\prime}\right|} \mathrm{d}^{3} x^{\prime}-\frac{G}{c} \int \frac{\rho^{\prime} \boldsymbol{v}^{\prime} \cdot\left(\boldsymbol{x}-\boldsymbol{x}^{\prime}\right)\left(x^{i}-x^{\prime i}\right)}{\left|\boldsymbol{x}-\boldsymbol{x}^{\prime}\right|^{3}} \mathrm{~d}^{3} x^{\prime} .
\end{align*}


令


\begin{align*}
	& V_{i}=G \int \frac{\rho^{\prime} v^{\prime i}}{\left|\boldsymbol{x}-\boldsymbol{x}^{\prime}\right|} \mathrm{d}^{3} x^{\prime},  \tag{7.83}\\
	& W_{i}=G \int \frac{\rho^{\prime} v^{\prime} \cdot\left(\boldsymbol{x}-\boldsymbol{x}^{\prime}\right)\left(\boldsymbol{x}^{i}-x^{\prime i}\right)}{\left|\boldsymbol{x}-\boldsymbol{x}^{\prime}\right|^{3}} \mathrm{~d}^{3} x^{\prime}, \tag{7.84}
\end{align*}


则


\begin{equation*}
	\chi_{, 0 i}=\frac{1}{c}\left(V_{i}-W_{i}\right) \tag{7.85}
\end{equation*}


最后由(7.66)和(7.80)得到


\begin{align*}
	& \zeta_{i}=-4 V_{i}  \tag{7.86}\\
	& \stackrel{(3)}{3}_{g_{0 i}}=-\frac{1}{c^{3}}\left(\frac{7}{2} V_{i}+\frac{1}{2} W_{i}\right) . \tag{7.87}
\end{align*}


\subsubsection{分量表述}
流体能动张量$T^{\alpha \beta}$必然满足方程$\nabla_\beta T^{\alpha \beta}=0$,即
\begin{equation}
	0=\nabla_\beta T^{\alpha \beta}=\partial_\beta T^{\alpha \beta}
	+\Gamma_{\mu \beta}^\alpha T^{\mu \beta}
	+\Gamma_{\mu \beta}^\beta T^{\alpha \mu}.
\end{equation}
我们用$\Gamma_{\mu \beta}^\beta=(-g)^{-1 / 2} \partial_\mu (-g)^{1 / 2}$可以将上式改写为
\begin{equation}\label{chppn:eqn_DT=0}
	\partial_\beta\left(\sqrt{-g} T^{\alpha \beta}\right)
	+\Gamma_{\mu\beta}^\alpha\left(\sqrt{-g} T^{\mu\beta}\right) =0.
\end{equation}
令式\eqref{chppn:eqn_DT=0}中的$\alpha =0$,有
\begin{align*}
	0 =& \partial_\beta\left(\sqrt{-g} T^{0 \beta}\right)
	+\Gamma_{\mu\beta}^{0}\left(\sqrt{-g} T^{\mu\beta}\right)
	 =  \frac{1}{c} \partial_t\left(\sqrt{-g} T^{00}\right)+\partial_j\left(\sqrt{-g} T^{0 j}\right) \\
		& +\Gamma_{00}^0\left(\sqrt{-g} T^{00}\right)+2 \Gamma_{0 j}^0\left(\sqrt{-g} T^{0 j}\right)
		+\Gamma_{j k}^0\left(\sqrt{-g} T^{j k}\right),
\end{align*}
将式\eqref{chppn:eqn_pfT-star}带入上式,有
\begin{align*}
	0 =&  \frac{1}{c} \partial_t\left(\sqrt{-g} T^{00}\right)+\partial_j\left(\sqrt{-g} T^{0 j}\right) 
	 +\Gamma_{00}^0\left(\sqrt{-g} T^{00}\right)+2 \Gamma_{0 j}^0\left(\sqrt{-g} T^{0 j}\right)
	+\Gamma_{j k}^0\left(\sqrt{-g} T^{j k}\right)\\
	=& 
\end{align*}



\section*{小结}

本章主要参考了文献\parencite[Ch. 9]{weinberg_grav-1972}、
\parencite[Ch. 7-8]{xu-wu-1999}、\parencite{poisson-will-2014}.

%%%%%%%%%%%%%%%%%%%%%%%%%%%%%%%%%%%%%%%%%%%%%%%%%%%%%%%%%%%%%%%%%%%%%%%%%%%%%%%%%%%%%
\printbibliography[heading=subbibliography,title=第\ref{chppn}章参考文献]

\endinput
