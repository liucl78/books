% !TeX encoding = UTF-8
% 此文件从2021.8开始

\chapter{测地线}\label{chgd}

我们先讨论了测地线的一般属性;
借助测地线可以定义指数映射,由指数映射自然地引入黎曼法坐标的概念,法坐标系
对于简化计算有特别的帮助.
我们还简要介绍了测地完备性以及弧长变分公式.
本章最后介绍了Jacobi场,这是描述引力论中测地偏离方程的工具.



\index[physwords]{测地线}

\section{测地线概念}\label{chgd:sec_concept}
欧几里得空间$\mathbb{R}^m$的直线在几何图像上是“直”的,
但这种感官性质的“直”是无法向流形上推广的.
然而$\mathbb{R}^m$直线方程的两次导数恒为零的{\kaishu 微积分属性}是可以推广到流形上的,
流形上测地线定义就是依据这个属性进行推广的.
流形上的测地线感官上来看,一般不是“直”的.
测地线是沿自身平行的曲线,可以看作是欧氏空间中直线方向不变的推广.

\S \ref{chccr:sec_pxyd}节中,已经给出仿射空间测地线$\gamma(t)$定义\ref{chccr:def_geodesic};
这个定义也适用于$m$维广义黎曼流形$(M,g)$,此时有度规,内容更丰富.
该定义中称满足$T^b \nabla _b T^a=0$的参数为仿射参数;
有时也把满足$T^b \nabla _b T^a =f T^a$($f$为非零标量场)的曲线称为测地线;
不过,最好称之为{\kaishu 非仿射参数化的测地线(non-affinely parametrized geodesic)},
或者{\kaishu 准测地线}(pre-geodesic).


为了更精准的讨论测地线嵌入问题,需要诱导联络概念,可参考\S\ref{chfb:sec_pull-back-bundle}.
在平行移动定义\ref{chccr:def_px}中所用的联络便是诱导联络的特例,即一条曲线上的诱导联络.
有了诱导联络和度规,可以有如下定理:平行移动保内积不变.
\begin{theorem}\label{chgd:thm_parallel-metric}
    设有$m$维广义黎曼流形$(M,g)$,其上存在光滑曲线$\gamma(t)$.
    如果有沿$\gamma(t)$平行的切矢量场$X^a, Y^a$,那么$g_{ab}X^aY^b$沿$\gamma(t)$是常数.
\end{theorem}
\begin{proof}
    沿$\gamma(t)$平行是指:$\nabla_{\frac{\partial}{\partial t}}X^a=0,
     \nabla_{\frac{\partial}{\partial t}}Y^a=0$;
    其中$(\frac{\partial}{\partial t})^a$是$\gamma(t)$的切线.
    \begin{equation*}
        \nabla_{\frac{\partial}{\partial t}}\bigl( g_{ab}X^aY^b \bigr)
        =\nabla_{\frac{\partial}{\partial t}}( g_{ab} ) X^aY^b
        +g_{ab}\nabla_{\frac{\partial}{\partial t}}( X^a ) Y^b
        +g_{ab}X^a\nabla_{\frac{\partial}{\partial t}}( Y^b )     =0.
    \end{equation*}
    为零原因:度规、联络相互容许,切矢量场$X^a, Y^a$沿$\gamma(t)$平行.
    这说明,沿曲线$\gamma(t)$平行移动保持切矢量长度和夹角不变;换句话说:
    平行移动是线性、局部等距同构映射.
\end{proof}

\index[physwords]{测地线!存在唯一性}
\subsection{测地线的存在唯一性}
设广义黎曼流形$M$的局部坐标系为$(U;x^i)$;
曲线$\gamma(t)$的局部坐标表达式为$x^i\circ\gamma(t)$,
由式\eqref{chccr:eqn_px-formula}可得:
曲线$\gamma(t)$是测地线的充分必要条件是其
分量$x^i\circ\gamma(t)$(为简洁起见,省掉后缀“$\circ\gamma$”)满足如下方程组:
\begin{equation}\label{chgd:eqn_geodesic-formula}
    \frac{{\rm d}^2 x^i(t)}{{\rm d}t^2} + \Gamma_{jk}^i(t)
    \frac{{\rm d}x^j(t)}{{\rm d} t} \frac{{\rm d}x^k(t)}{{\rm d} t} =0,
    \qquad 1 \leqslant i \leqslant m , \quad t\in (-\delta,\delta).
\end{equation}
上式中所有量均沿曲线$\gamma(t)$取值,并且当$t$变动时,$\gamma(t)$没有跑出开子集$U$.

%\begin{equation}
%    \frac{{\rm d}^2 x^\rho(\tau)}{{\rm d}\tau^2} + \Gamma_{\mu\nu}^\rho(\tau)
%    \frac{{\rm d}x^\mu(\tau)}{{\rm d} \tau} \frac{{\rm d}x^\nu(\tau)}{{\rm d} \tau} =0.
%\end{equation}


式\eqref{chgd:eqn_geodesic-formula}是常微分方程(不论线性与否),
根据常微分方程解存在唯一性以及初值依赖性,
当给定初值时(点$p\in U$的坐标$x_p$以及此点的一个切矢量$v^a$)
\begin{equation}\label{chgd:eqn_geodesic-init-data}
    x^i(0) = x^i_p, \quad \frac{{\rm d}x^i(0)}{{\rm d}t}=v^i;
    \qquad 1 \leqslant i \leqslant m.
\end{equation}
常微分方程式\eqref{chgd:eqn_geodesic-formula}在局部开集$U$内存在满足
上述初始条件的唯一解;也就是说开集$U$内存在\uwave{唯一一条测地线}$\gamma(t;p,v^a)$满足
初始条件\eqref{chgd:eqn_geodesic-init-data},
即$\gamma(0;p,v)=p$及$(\frac{\partial}{\partial t})^a|_{\gamma(0)}=v^a$.

\begin{definition}
    设$\gamma(t)$是广义黎曼流形$M$上的一条测地线,用$T^a=(\frac{\partial}{\partial t})^a$表示其切矢量;
    若$T^a$处处非零,则称其为{\heiti 正则测地线};
    若$T_a T^a=1$处处成立,则称其为{\heiti 正规测地线}.
\end{definition}

测地线存在唯一性定理与度规场没有本质联系.

\index[physwords]{测地线!仿射、缩放变换}
\subsection{仿射变换}
设有映射$t(u)$将\uwave{正则}测地线$\gamma(t)$实参数$t$变成了$u$(这称为重参数化),
曲线也变成了$\tilde{\gamma}(u)$;对重参数化后的曲线作如下计算
(注:对单参量$t$(或$u$)的偏导数相当于普通导数,即$\partial \to {\rm d}$)
\setlength{\mathindent}{0em}
\begin{align}
    \nabla_{\frac{\partial}{\partial u}} \left(\frac{\partial}{\partial u}\right)^a =&
    \nabla_{\frac{{\rm d}t}{{\rm d}u}\frac{\partial}{\partial t}}
    \left(\frac{{\rm d}t}{{\rm d}u} \frac{\partial}{\partial t}\right)^a =
    \bigl(t'(u)\bigr)^2 \nabla_{\frac{\partial}{\partial t}} \left(\frac{\partial}{\partial t}\right)^a +
    t'(u)\left(\frac{\partial}{\partial t}\right)^a
    \nabla_{\frac{\partial}{\partial t}} \left( \frac{{\rm d}t}{{\rm d}u}\right) \notag \\
    \xlongequal[\text{测地线}]{\gamma(t)\text{是}} &
    \left(\frac{\partial}{\partial t}\right)^a  \frac{{\rm d}t}{{\rm d}u}
    {\frac{\partial}{\partial t}} \left( \frac{{\rm d}t}{{\rm d}u}\right)
    =t''(u) \left(\frac{\partial}{\partial t}\right)^a . \label{chgd:eqn_gaf}
\end{align}\setlength{\mathindent}{2em}
由此式可以看出当变换$t(u)$是\uwave{仿射变换}时,即
\begin{equation}\label{chgd:eqn_affine-transform}
    t(u) = a u + b, \qquad a,b \text{是实常数,并且} a\neq 0.
\end{equation}
此时$t''(u)=0$,所以变换后的曲线$\tilde{\gamma}(u)$仍是测地线,
即满足$\nabla_{\frac{\partial}{\partial u}} (\frac{\partial}{\partial u})^a=0$.

反之,若要求变换后曲线$\tilde{\gamma}(u)$是测地线,
则由式\eqref{chgd:eqn_gaf}可知必有$t''(u)=0$(因$\gamma(t)$正则,
故在任意点都有$(\frac{\partial}{\partial t})^a\neq 0$),
所以$t(u)$必是变换\eqref{chgd:eqn_affine-transform}.

下面开始讨论一类特殊的仿射变换:缩放变换.
对于测地线$\gamma(t;p,v^a)$,我们来作参数变换$t=\lambda u$(其中$\lambda$是正的实常数),
变换前后的参数取值范围仍在测地线定义域内.
这显然是仿射变换,变换后的曲线$\tilde{\gamma}(u)\equiv\gamma(\lambda u;p,v^a)$是仍是测地线,并且有
\begin{equation}
    \tilde{\gamma}(0)=\gamma(0)=p;\qquad
    \left. \left(\frac{\partial}{\partial u}\right)^a \right|_{\tilde{\gamma}(0)}
    = \lambda \left. \left(\frac{\partial}{\partial t}\right)^a \right|_{{\gamma}(0)}
    = \lambda v^a .
\end{equation}
这说明$\tilde{\gamma}(u)$是经过$p$点以$\lambda\cdot  v^a$为初始切矢量的测地线,
由测地线唯一性得:
\begin{equation}\label{chgd:eqn_geodesic-zoom-transfrom}
    \tilde{\gamma}(u) = \gamma(\lambda u;p, v^a) =\gamma(u;p,\lambda v^a)
    \quad \Leftrightarrow \quad
    \gamma(u;p, v^a) = \gamma(\lambda u;p, \frac{v^a}{\lambda} ).
\end{equation}
仿射变换、缩放变换与度规场没有本质联系.

\subsection{弧长参数}
下面来看测地线$\gamma(t)$的线元长度,计算下式
\begin{equation}
    \nabla_{\frac{\partial}{\partial t}} \left(g_{ab} \left(\frac{\partial}{\partial t}\right)^a
    \left(\frac{\partial}{\partial t}\right)^b    \right) =
    2 g_{ab} \nabla_{\frac{\partial}{\partial t}} \left( \left(\frac{\partial}{\partial t}\right)^a \right)
    \left(\frac{\partial}{\partial t}\right)^b     = 0.
\end{equation}
可见测地线弧长平方求导之后恒为零,这说明测地线线元长度是实常数,
因此必然有(也就是把上式换成下式)
\begin{equation}
    \frac{{\rm d}s}{{\rm d}t} =
    \sqrt{g_{ab} \left(\frac{\partial}{\partial t}\right)^a
        \left(\frac{\partial}{\partial t}\right)^b }
    = a \quad >0.
\end{equation}
于是有常数$b$使得仿射变换$s=at+b$成立,这说明弧长参数$s$是仿射参数;
以线长为参数的测地线,其切矢量恒为$1$(见式\eqref{chrg:eqn_arc-unit}),是正规测地线.



\subsection{例}
寻找测地线的通用方法是求解测地线方程\eqref{chgd:eqn_geodesic-formula},见下例.

\begin{example}\label{chgd:exm_Rmnu}
    $\mathbb{R}^m_\nu$中测地线是直线或其一部分.
\end{example}
$\mathbb{R}^m_\nu$是平直空间,所有克氏符都为零.
由测地线方程\eqref{chgd:eqn_geodesic-formula}可知:
\begin{align}
    \frac{{\rm d}^2 x^i(t)}{{\rm d}t^2} = 0     \ \Rightarrow \
    x^i(t) = a^i t + b^i, \quad  a^i, b^i \in \mathbb{R},
    \quad t\in (-\delta,\delta)  .
\end{align}
很明显这是直线(段),且参数$t$可以延拓至无穷远,
故$\mathbb{R}^m_\nu$是测地完备的(完备性的定义见\ref{chgd:def_complete-geodesic}).
\qed

然而,一般情形下测地线方程\eqref{chgd:eqn_geodesic-formula}难于求解,故这种方法虽然通用但不好用.
我们可以借助流形自身的某些对称性来寻找测地线,为此先介绍一个命题.

\begin{proposition}\label{chgd:thm_iso-geodesic}
    设有广义黎曼流形$(M,g)$,其上有非零模、光滑曲线$\gamma(s)$,$s$是弧长参数.
    如果存在$M$到自身的等距映射$\sigma$使得$\sigma$的不动点集恰好是曲线$\gamma$上的点
    构成的集合,那么$\gamma(s)$是$M$上一条正规测地线.
\end{proposition}
\begin{proof}
    由命题已知条件得:对于定义域内任意$s$有$\sigma\circ \gamma(s) = \gamma(s)$,
    则由式\eqref{chdm:eqn_phiD=Dphi}可知
    \begin{equation}
        \sigma_{*}\left(\left.\frac{{\rm d}  }{{\rm d} s}\right|_{\gamma(s)}\right)^a
        = \left(\left.\frac{{\rm d}  }{{\rm d} s}\right|_{\sigma\circ\gamma(s)}\right)^a
        = \left(\left.\frac{{\rm d}  }{{\rm d} s}\right|_{\gamma(s)}\right)^a .
    \end{equation}
    对于$\gamma$上任意一点$s_0$,设$\beta(s)$是一条测地线,它满足的初始条件是:
    \begin{equation}\label{chgd:eqn_tmp20d}
        \beta(s_0)=\gamma(s_0),\qquad 
        \left(\left.\frac{{\rm d}  }{{\rm d} s}\right|_{\beta(s_0)}\right)^a
        =\left(\left.\frac{{\rm d}  }{{\rm d} s}\right|_{\gamma(s_0)}\right)^a .
    \end{equation}
    由定理\ref{chrg:thm_geodesic-MN}(需用$\sigma$是等距)可
    知:$\sigma\circ \beta(s)$也是测地线,且满足初条件
    \begin{equation*}
        \sigma\circ\beta(s_0)=\beta(s_0)=\gamma(s_0),\
        \sigma_* \left(\left.\frac{{\rm d}  }{{\rm d} s}\right|_{\beta(s_0)}\right)^a
        =\left(\left.\frac{{\rm d}  }{{\rm d} s}\right|_{\beta(s_0)}\right)^a
        =\left(\left.\frac{{\rm d}  }{{\rm d} s}\right|_{\gamma(s_0)}\right)^a .
    \end{equation*}
    根据测地线唯一性,由上式可知:$\sigma\circ \beta(s) = \beta(s)$,
    即$\beta$是$\sigma$的不动点集合.
    命题已知条件中已假设$\sigma$不动点集合都在$\gamma$上,故$\beta(s)$与$\gamma(s)$是
    重合的,只是参数化有所不同.由于$s$是$\beta(s)$与$\gamma(s)$的弧长,
    且有式\eqref{chgd:eqn_tmp20d}成立,那么$\gamma(s)\equiv \beta(s)$是测地线.
\end{proof}


除了上述命题外,\S\ref{chsm:sec_sub-Geodesic}命题\ref{chsm:thm_sub-Geodesic}也是寻找测地线常用工具.


\begin{example}
    二维球面$S^2$上测地线是大圆或其一部分.
\end{example}
大圆是指过球心的二维平面与二维球面的交线.
假设$\sigma$是$\mathbb{R}^3$上关于原点(即$S^2$球心)的反射变换,
它在$S^2$上的限制是$S^2$到其自身的等距映射;$\sigma$的不动点集合是$S^1$.
由命题\ref{chgd:thm_iso-geodesic}可知$S^1$是$S^2$的测地线.
另一方面,在$S^2$上任一点$p$沿任意切方向$v^a$都有一个大圆通过$p$点
并以$v^a$为切线切矢量;根据测地线唯一性,$S^2$上的测地线只能
是大圆或大圆的一部分.
\qed


\index[physwords]{指数映射} \index[physwords]{测地线!指数映射}

\section{指数映射与法坐标系}\label{chgd:sec_exp-RNC}
我们先引入指数映射定义,然后借用指数映射介绍黎曼法坐标系(Riemanian Normal Coordinates).

\subsection{指数映射}\label{chgd:sec_exp}
现有满足初始条件\eqref{chgd:eqn_geodesic-init-data}的广义黎曼流形$(M,g)$上的测地线$\gamma(t;p,v)$,
当参数$t=1$时,赋予它一个新的名称:{\heiti 指数映射},它是从$TM$上开子集到流形$M$开子集的映射:
\begin{equation}\label{chgd:eqn_geodesic-exp}
    \exp_p(v^a)\equiv\exp(p,v^a)\overset{def}{=}\gamma(1;p,v^a)
    \xlongequal{\ref{chgd:eqn_geodesic-zoom-transfrom}}\gamma\left(|v|;p,\frac{v^a}{|v|}\right).
\end{equation}
指数映射的几何意义是沿测地线$\gamma(t)$由点$p\equiv \gamma(0)$到点$\gamma(1)$的测地线弧长
等于$|v|$.需要注意,如果切矢量$v^a$过大,指数映射可能没有定义;但只要有定义则必然唯一.
同理有
\begin{equation}\label{chgd:eqn_geodesic-exptv}
    \exp_p(t v^a)=\gamma(1;p,t v^a)=\gamma(t;p,v^a).
\end{equation}
指数映射\eqref{chgd:eqn_geodesic-exp}只是借助测地线将$T_pM$映射到流形$M$中,
参数$t\equiv 1$;
式\eqref{chgd:eqn_geodesic-exptv}则是测地线本身,参数$t$是变量.


\begin{theorem}\label{chgd:thm_exp-homeomorphism}
    设有$m$维广义黎曼流形$(M,g)$.$\forall p \in M$,在线性空间$T_p M$中存在原点的开邻域$V$,
    使得指数映射$\exp_{p}$是从$V$到$M$中开子集$U=\exp_{p}(V)$上的微分同胚.
\end{theorem}
\begin{proof}
    原点是指矢量空间$T_p M$中矢量为零的点,令
    \begin{equation}
        B_p(\epsilon) \equiv \{v^a\in T_pM \mid |v|<\epsilon \} .
    \end{equation}
    显然$B_p(\epsilon)$是$T_pM$中以原点为中心,半径为$\epsilon$的开球.
    $\forall v\in B_p(\epsilon) \subset T_pM= T_0(T_pM) $
    {\footnote{$T_pM$自然可看作流形;原点$0$是此流形上一点,
            $T_0(T_p M)$是指流形$T_pM$中原点处切空间.}},令$|t|\leqslant 1$,则
    式\eqref{chgd:eqn_geodesic-exptv}肯定有定义;参考式\eqref{chdm:eqn_phiD=Dphi}
    \begin{equation}
        \phi_{*}\left(\left.\frac{{\rm d}  }{{\rm d} t}\right|_{\gamma(0)}\right)^a
        = \left(\left.\frac{{\rm d}  }{{\rm d} t}\right|_{\phi\circ\gamma(0)}\right)^a.
        \tag{\ref{chdm:eqn_phiD=Dphi}}
    \end{equation}
    可得
    \begin{equation}
        (\exp_{p})_{*0} (v^a) = \left(\left.\frac{{\rm d}  }{{\rm d} t}
         \right|_{\gamma(0;p,v^a)}\right)^a = v^a .
    \end{equation}
    此处的$(\exp_{p})_{*}$相当于式\eqref{chdm:eqn_phiD=Dphi}的$\phi_{*}$,
    式\eqref{chdm:eqn_phiD=Dphi}中的$\gamma(t)$就是此处的测地线$\gamma(t;p,v^a)$;
    而$(\exp_{p})_{*}$本质上是测地线的推前映射,所以才有最终上式的结果.
    这说明$(\exp_{p})_{*0}$是恒等映射,那它自然是非退化的(Jacobi矩阵满秩),  %    而$B_p(\epsilon)$的维数与流形$M$的维数相同,
    由定理\ref{chdm:thm_inv-in-M}可知映射$\exp_{p}$在原点${\bf 0}\in T_pM$处是局部微分同胚.
\end{proof}

\begin{example}
    欧几里得空间$\mathbb{R}^m$上的指数映射.
\end{example}
欧氏空间测地线就是所谓的“直线”.原点处的切空间$T_0(\mathbb{R}^m)$就是$\mathbb{R}^m$自身.
指数映射$\exp_{0}$则是恒等映射.一般的,$\forall p\in \mathbb{R}^m$,不难得到
\begin{equation}
    \exp_{p} = p + v, \qquad \forall v^a\in T_p(\mathbb{R}^m)=\mathbb{R}^m.
\end{equation}
我们借用这道例题来解释一下上面提到的指数映射“有无定义”这件事情.
在本例中,我们把定义域和值域都取成了$\mathbb{R}^m$;如果我们把
指数映射的定义域取成半径为$1$的开球,那么当$|v|>1$时就会出现所谓的“无定义”情形,
当$|v|<1$时属于“有定义”情形.


%李群中也可定义指数映射,且在满足一定条件下两者是相同的,见推论\ref{chlg:thm_expiso}.


\index[physwords]{测地线!黎曼法坐标系}

\subsection{黎曼法坐标系}\label{chgd:sec_RNC}
在几何学中,为了能够简单明了的描述问题,选择适当坐标系非常重要;如果选择不当会使计算或证明异常困难.
对于黎曼流形来说,“黎曼法坐标系”是非常合适的坐标系,它是借助指数映射建立起来的.

要严格叙述黎曼法坐标系需要测地凸邻域等概念,请参阅\parencite[\S 3.4-3.5]{chen-li-2023-2ed-v1};
我们只给出描述性的说明.
只要在$T_pM$原点足够小的邻域内,指数映射之后的值域$W\subset M$也足够小,那么在映射后
的($M$中)开集$W$内连接任意点$q\in W$和$p$的测地线全部位于$W$中,并且
在$W$内连接$p$和$q$的测地线是唯一的;满足这些条件的$W$被称为测地凸邻域,
注解\ref{chgd:rmk_geo-sph}给出了测地球概念,这便是一个测地凸邻域.
可以证明$\forall p \in M$都存在包含$p$点的测地凸邻域.
如果测地线不能完全落在$W$中,那么我们无法建立$T_pM$与$W$间的双射,也就无法建立坐标.
如果测地线不唯一,建立后的坐标将是多值的,不符要求.



指数映射将原点附近足够小的开邻域$V\subset T_pM$映射为
\begin{equation}
    \exp_{p} : V \to U \equiv\exp_{p}(V) \subset M .
\end{equation}
设$U$是$\exp_{p}(0)$点(即$p$点)测地凸邻域.
由定理\ref{chgd:thm_exp-homeomorphism}可知指数映射是局部微分同胚的.
因$T_pM \cong \mathbb{R}^m$,其上天然存在正交归一的笛卡尔坐标系;
那么局部恒等映射$\varphi = \exp_{p}^{-1}:U\to V$便将$V\subset \mathbb{R}^m$中
笛卡尔坐标携带到$U$中,从而$U$有一个局部坐标系$(U,\varphi;y^i)$,
称为广义黎曼流形$(M,g)$在$p$点的{\heiti 黎曼法坐标系},相应开集$U$称为{\heiti 黎曼法坐标邻域};
简称为{\heiti 法坐标系(邻域)}.

\index[physwords]{法坐标系}

%(其实由高斯引理的注解\ref{chgd:rmk_Gauss-lemma}可知指数映射是沿径向测地线局部等距同构映射)

\begin{theorem}\label{chgd:thm_RNC}
    设$(U,\varphi;y^i)$是$m$维广义黎曼流形$(M,g)$上任意点$p$的法坐标系,那么在$p$点有
    $\bar{g}_{ij}(p)=\eta_{ij}$,$\bar{\Gamma}_{ij}^k(p)=0$,
    $\frac{\partial \bar{g}_{ij}}{\partial y^k}(p)=0$.
\end{theorem}
\begin{proof}
    我们把法坐标系下的诸多分量上加了一道横杠以区别,比如$\bar{g}_{ij},\bar{\Gamma}_{ij}^k$.

    因点$p$切空间$T_pM$同构于$\mathbb{R}^m$中开子集,
    故我们可以在$T_pM$中取正交归一的标架场$(e_i)^a$;
    即$g_{ab}(e_i)^a(e_j)^a= \eta_{ij}$,
    其中$\eta_{ij}$是指对角元为$\pm 1$,非对角元全为零的矩阵.
    再任取$V$中一个矢量$v^a$,
    在这个正交归一的标架场$(e_i)^a$上分量为$v=(v^1,\cdots,v^m)\in V$,其中$v^i$是实常数.
    依据法坐标系定义,局部恒等映射映射$\varphi = \exp_{p}^{-1}$会把这个坐标带到$U$中,
    矢量$v^a$在$(U;y^i)$中的坐标是
    \begin{equation}\label{chgd:eqn_normal-coord}
        y^i\circ\exp_{p}(v^a) \equiv v^i, \qquad
        \left.\left(\frac{\partial}{\partial y^i}\right)^a\right|_{p} \equiv (e_i)^a;
        \qquad 1\leqslant i\leqslant m.
    \end{equation}
    故$\bar{g}_{ij}(p)=g_{ab} (\frac{\partial}{\partial y^i})^a|_{p} (\frac{\partial}{\partial y^j})^b|_{p}
    =g_{ab}(e_i)^a(e_j)^b=\eta_{ij}$.
%    在离开$p$点后,仍有$(\frac{\partial}{\partial y^i})^a = (e_i)^a$.

    在$V$中测地线是真正的“直线”;局部恒等映射$\varphi = \exp_{p}^{-1}$也
    会把它带入到$U$中的法坐标系,因此从$p$点出发的相应测地线方程$\exp_{p}(sv^a)$的分量表达是
    \begin{equation}\label{chgd:eqn_normal-coord-gedU}
        y^i(s)= y^i\circ \exp_{p}(s v^a) = s v^i,\quad s\text{是弧长},  \qquad 1\leqslant i\leqslant m.
    \end{equation}
    把式\eqref{chgd:eqn_normal-coord-gedU}带入测地线方程式\eqref{chgd:eqn_geodesic-formula},立即可得
    \begin{equation}\label{chgd:eqn_tmp-Gamma}
        v^i v^j \bar{\Gamma}_{ij}^k(s v^a) = 0 ,\qquad 1\leqslant i,j,k\leqslant m .
    \end{equation}
    特别地,当$s=0$时,利用$v^a$的任意性以及$\bar{\Gamma}_{ij}^k$关于下标$i,j$的对称性(如果不对称则没有以下结果),
    可得$\bar{\Gamma}_{ij}^k(0)=0,\ 1\leqslant i,j,k\leqslant m $.
    如果$s\neq 0$,那么$\bar{\Gamma}_{ij}^k(s v^a)$将与$v^a$相关,则式\eqref{chgd:eqn_tmp-Gamma}不是二次型,
    未必能得到$\bar{\Gamma}_{ij}^k(s v^a)=0$.

    由$\bar{g}_{ij}(p)=\eta_{ij}, \bar{\Gamma}_{ij}^k(p)=0$,再从
    式\eqref{chrg:eqn_Dg=0}易得$\frac{\partial \bar{g}_{ij}}{\partial y^k}(p)=0$.


    定理中给出的关系式仅在$p$点成立,哪怕离开一点点也未必再成立.
\end{proof}

%    注意,指数映射本身就是借助测地线来实现的,$\exp_{p}(s v^a)$是把$v^a\in V$映射为$U\subset M$中的
%    测地线$\gamma(s;p,v^a)$.在$V$中,$s v^a$的坐标分量自然是$s v^i$,指数映射把这个
%    坐标携带到$U$中测地线上,所以测地线坐标分量自然是$y^i(t) =s v^i$.
%    定理\ref{chgd:thm_exp-homeomorphism}和注解\ref{chgd:rmk_Gauss-lemma}可知指数映射是沿径向测地线局部等距的;
%    在$V$中,测地线是直线,依据定理\ref{chrg:thm_geodesic-MN}可知
%    它自然也把这个属性搬到了$U$中,即在法坐标系下$U$中的测地线也是直线,
%    见其分量表达式\eqref{chgd:eqn_normal-coord-gedU}.

%    关于式\eqref{chgd:eqn_normal-coord-gedU},需要做点注解.
%    法坐标系不能用来覆盖整个流形,只对测地凸邻域内有效,这个区域并不大.
%    如果在很大范围内,测地线都是直线,那么此区域范围内的黎曼曲率肯定为零.

%    首先,$U$中法坐标系下有局部标架场$(\frac{\partial}{\partial y^i})^a$,在$p$点,
%    因指数映射在原点附近是恒等映射,显然有;

%$V$是切空间$T_pM$的开子集,在其中肯定存在笛卡尔直角坐标系,指数映射将两个相互正交的坐标轴映射为两个
%相互正交的坐标轴(见高斯引理\ref{chgd:thm_Gauss-lemma}),所以在这个

上面用抽象的映射定义了法坐标系,下面我们用分量语言再次看一下这个问题,
并且给出法坐标系下诸多几何量的展开式;为此需要分成几部分来叙述.

\index[physwords]{Christoffel记号!广义克氏符号}
\subsubsection{广义克氏符号}
对广义黎曼流形$(M,g)$上任意点$p$,在$T_pM$中取定理\ref{chgd:thm_RNC}中所述的
正交归一标架$(e_i)^a$.
再取$p$点局部坐标系为$(U;x^i)$,原点就是$p$;在局部坐标系中
取标架场$\{(\frac{\partial}{\partial x^i})^a\}$,
我们要求这个标架场在$p$点
有$(\frac{\partial}{\partial x^i})^a|_p =(e_i)^a
=(\frac{\partial}{\partial y^i})^a|_p $;
但离开$p$点后$\{x\}$、$\{y\}$可能不同.


令任意给定$p$点的切矢量$v^a|_{p}$,那么点$p$和$v^a|_{p}$的局部坐标表达式为
\begin{equation}
    x_p^i\equiv x^i_0, \qquad v^a|_{p} = \left. v^i \right|_{p}
    \left. \left(\frac{\partial}{\partial x^i}\right)^a \right|_{p}.
\end{equation}
则可设满足上式的测地线坐标分量的Taylor展开为(即满足\eqref{chgd:eqn_geodesic-formula}):
\begin{equation}\label{chgd:eqn_tmpgeodesic-expand}
    x^i = x^i_0 + s v^i  + \frac{s^2}{2!} \left(\frac{{\rm d}^2x^i}{{\rm d} s^2}\right)_{p}
     + \frac{s^3}{3!} \left(\frac{{\rm d}^3 x^i}{{\rm d} s^3}\right)_{p}
     + \frac{s^4}{4!} \left(\frac{{\rm d}^4 x^i}{{\rm d} s^4}\right)_{p}  + \cdots
\end{equation}
由于我们不晓得测地线解的确切表达式,我们假设上述多项式(假设收敛)是测地线方程式\eqref{chgd:eqn_geodesic-formula}的
解(为叙述简单起见,需要把式\eqref{chgd:eqn_geodesic-formula}中的仿射参数$t$取为测地线弧长$s$).

我们通过对式\eqref{chgd:eqn_geodesic-formula}再次取$s$的导数来
求得$\frac{{\rm d}^3 x^i}{{\rm d} s^3}$及更高阶导数,
\begin{equation}
    \frac{{\rm d}^3 x^i}{{\rm d} s^3}+\Gamma^{i}_{jkl}\frac{{\rm d} x^j}{{\rm d} s}
    \frac{{\rm d} x^k}{{\rm d} s}\frac{{\rm d} x^l}{{\rm d} s} =0 , \quad
    \frac{{\rm d}^4 x^i}{{\rm d} s^4}+\Gamma^{i}_{jkln}\frac{{\rm d} x^j}{{\rm d} s}
    \frac{{\rm d} x^k}{{\rm d} s}\frac{{\rm d} x^l}{{\rm d} s}\frac{{\rm d} x^n}{{\rm d} s} =0,
    \ \cdots
\end{equation}
其中$\Gamma^{i}_{jkl},\Gamma^{i}_{jkln}$等是广义克氏符号
(参见文献\parencite[\S 17]{eisenhart-1997-rg}式(17.13)-(17.15)):
\begin{equation}\label{chgd:eqn_generalized-Christoffel}
    \Gamma^{j}_{i_1i_2i_3\cdots i_n} =  \Gamma^{j}_{(i_1i_2i_3\cdots i_{n-1},\ i_n)}
    - (n-1) \Gamma^{j}_{k(i_2i_3\cdots i_{n-1}}  \Gamma^{k}_{i_1i_n)},\qquad n>2 .
\end{equation}
先由普通的克氏符定义三阶广义克氏符,然后四阶,……,递归定义到$n$阶.
与三、四阶方程类似,可求得其它高阶方程.
求完之后,带入式\eqref{chgd:eqn_tmpgeodesic-expand}得
\begin{equation}\label{chgd:eqn_geodesic-expand}
    x^i = x^i_0 + s v^i  - \frac{s^2}{2!} \Gamma_{jk}^i(p) v^j v^k
        - \frac{s^3}{3!}  \Gamma^{i}_{jkl}(p) v^j v^k v^l
        - \cdots
\end{equation}
上式所有$v^i$均在$p$点取值,为使表达式简洁一些省略了角标$p$.
%当$|s|\lll1$时,式\eqref{chgd:eqn_geodesic-expand}便退化成式\eqref{chgd:eqn_normal-coord-gedU}.
假设式\eqref{chgd:eqn_geodesic-expand}是收敛的;即便如此,$s$也不能太大,因为
在离$p$点非常远的地方(比如$q\in M$)测地线可能相交,这相当于坐标线相交于两点$p$和$q$;
出现多值性,需要增加一个坐标卡来覆盖这个流形了.



继续式\eqref{chgd:eqn_geodesic-expand}的讨论.
黎曼本人引入了坐标\eqref{chgd:eqn_normal-coord}和测地线方程式\eqref{chgd:eqn_normal-coord-gedU}
(约1920年代Birkhoff将它命名为法坐标),
即$ y^i = s\cdot v^i$.我们把它带入式\eqref{chgd:eqn_geodesic-expand}得
\begin{equation}
    x^i = x^i_0 + y^i  - \frac{1}{2!} \Gamma_{jk}^i(p) y^j y^k
    - \frac{1}{3!}  \Gamma^{i}_{jkl}(p) y^j y^k y^l + \cdots
\end{equation}
这样,在上式中就不再显示出现$v^a|_p$的信息了.
很明显在$p$点处Jacobi矩阵$\frac{\partial x^i}{\partial y^j}$是非退化的,
而且上式对经过$p$点的所有测地线都成立;因此$\{y^i\}$可以看作一组新的坐标(即法坐标),
上式可以看作是两组坐标$\{x\}$和$\{y\}$的变换公式.
%来计算一下在新的坐标$\{y\}$下$p$点切矢量
%\begin{equation}
%    \left.\left(\frac{{\rm d}}{{\rm d}s}\right)^a\right|_p
%    =\left.\left(\frac{\partial}{\partial y^i}\right)^a\right|_p
%        \left.\frac{\partial y^i}{\partial s} \right|_p
%    =v^i|_p \left.\left(\frac{\partial}{\partial y^i}\right)^a\right|_p .
%\end{equation}

%在$p$点标架场$\{\left(\frac{\partial}{\partial x^i}\right)^a\}$、
%$\{\left(\frac{\partial}{\partial y^i}\right)^a\}$和标架场
%标架场$\{(e_i)^a\}$是完全重合的,因而在$p$点它们都是正交归一的.
%坐标$\{y\}$就是式\eqref{chgd:eqn_normal-coord-gedU}.

%式\eqref{chgd:eqn_geodesic-expand}的几何解释:点$p$是坐标原点,
%任意给定点$p$处的切矢量$v^a|_p$.
%在法邻域内(测地凸邻域)通过测地线,该式确定了某$q$点
%(即参数为非零$s$值,比如$s=0.1$)的坐标$x^i(s)$.
%换句话说,连接$p$和$q$的是一条沿$v^a|_p$的测地线,通过此条测地线,把原来没有
%坐标的$q$点赋予了坐标.我们可以进一步缩小法邻域的范围,使得
%式\eqref{chgd:eqn_geodesic-expand}中$s^k(k\geqslant 2)$可忽略,
%所得到的$q$点坐标就线性地依赖于测地线线长$s$了.


我们已约定坐标系$\{y\}$中,度规取为$g_{ab}=\bar{g}_{ij}({\rm d}y^i)_a({\rm d}y^j)_a$,
相应的克氏符号取为$\bar{\Gamma}^k_{ij}$.那么对应的测地线方程变为
\begin{equation}\label{chgd:eqn_geodesic-formula-RNC}
    \frac{{\rm d}^2 y^i(s)}{{\rm d}s^2} + \bar{\Gamma}_{jk}^i(s)
    \frac{{\rm d}y^j(s)}{{\rm d} s} \frac{{\rm d}y^k(s)}{{\rm d} s} =0,
    \qquad 1 \leqslant i \leqslant m , \quad s\in (-\delta,\delta).
\end{equation}
很明显,我们可以重复上面的过程,得到类似于式\eqref{chgd:eqn_geodesic-expand}的方程式
\begin{equation}\label{chgd:eqn_geodesic-expand-RNC}
    y^i =  s v^i  - \frac{s^2}{2!} \bar{\Gamma}_{jk}^i(p) v^j v^k
    - \frac{s^3}{3!}  \bar{\Gamma}^{i}_{jkl}(p) v^j v^k v^l + \cdots
\end{equation}
又因已知$y^i =  s v^i$,故可得所有法坐标系(广义)克氏符在$p$点为零,即
\begin{equation}\label{chgd:eqn_Christoffel=0-RNC}
    \bar{\Gamma}^i_{jk}(p)   =0, \quad
    \bar{\Gamma}^i_{jkl}(p)  =0, \quad
    \bar{\Gamma}^i_{jkln}(p) =0, \quad\cdots
\end{equation}

叙述到此,需要作一个声明:定理\ref{chgd:thm_RNC}中所描述的
点$p$处切空间$T_pM$中正交归一基矢$(e_i)^a$是任取的;
如果我们另取一套正交归一基矢$(e'_i)^a=C_i^j(e_j)^a$,
其中$C_i^j$是常数正交矩阵;那么由$(e'_i)^a$也会诱导出
一套法坐标$\{y'\}$,它与法坐标$\{y\}$相差一个常系数正交变换.

\index[physwords]{测地线!法坐标系}
\subsubsection{法坐标展开}
%在法坐标域内,仅一个坐标域就能覆盖住,所以只需偏导数,无需考虑协变导数.

由黎曼曲率的分量公式\eqref{chccr:eqn_Riemannian13-component}
($    R_{\cdot jln}^i = {\partial_l} \Gamma_{jn}^{i} -\partial_n \Gamma_{jl}^{i}
      + \Gamma_{jn}^{k} \Gamma_{kl}^{i} - \Gamma_{jl}^{k}\Gamma_{kn}^{i} $)
%\end{equation}      \tag{\ref{chccr:eqn_Riemannian13-component}}
出发;在法坐标系下,在$p$点所有的(广义)克氏符都是零,故有
\begin{equation}
    \bar{R}_{\cdot jln}^i(p) = {\partial_l} \bar{\Gamma}_{jn}^{i}(p) -\partial_n \bar{\Gamma}_{jl}^{i}(p) .
\end{equation}
由$p$点$\bar{\Gamma}^{i}_{jkl}(p) = 0$得
$\bar{\Gamma}^{i}_{jk,l}(p)+\bar{\Gamma}^{i}_{lj,k}(p)+\bar{\Gamma}^{i}_{kl,j}(p)=0$;
再由上式可得
\begin{equation}\label{chgd:eqn_dGAMMA}
    \partial_n \bar{\Gamma}_{jl}^{i}(p) = -\frac{1}{3}\left(
      \bar{R}_{\cdot jln}^i(p)+\bar{R}_{\cdot ljn}^i(p) \right).
\end{equation}
%推导过程
%\begin{align*}
%    \bar{R}_{\cdot jln}^i(p) =& {\partial_l} \bar{\Gamma}_{jn}^{i}(p) -\partial_n \bar{\Gamma}_{jl}^{i}(p) \\
%    \bar{R}_{\cdot ljn}^i(p) =& {\partial_j} \bar{\Gamma}_{ln}^{i}(p) -\partial_n \bar{\Gamma}_{jl}^{i}(p) \\
%    \bar{R}_{\cdot jln}^i(p)+\bar{R}_{\cdot ljn}^i(p) =&
%    {\partial_l} \bar{\Gamma}_{jn}^{i}(p) -2\partial_n \bar{\Gamma}_{jl}^{i}(p)
%    +{\partial_j} \bar{\Gamma}_{ln}^{i}(p)
%    =-3\partial_n \bar{\Gamma}_{jl}^{i}(p)
%\end{align*}
利用式\eqref{chgd:eqn_dGAMMA},
由式\eqref{chrg:eqn_Dg=0}($\partial_a g_{bc} = \Gamma_{ba}^e g_{ec} + \Gamma_{ca}^e g_{be}$)得
\begin{equation}\label{chgd:eqn_gijkl=R}
    \frac{\partial^2 \bar{g}_{ij}(p)}{\partial y^n\partial y^l}
    = -\frac{1}{3}\left(\bar{R}_{iljn}(p)+\bar{R}_{injl}(p) \right).
\end{equation}
%推导过程
%\begin{align*}
%    \frac{\partial^2 \bar{g}_{ij}}{\partial y^n\partial y^l}
%    =& \frac{\partial}{\partial y^n}
%     \left(\bar{\Gamma}_{il}^r \bar{g}_{rj} + \bar{\Gamma}_{jl}^r \bar{g}_{ir}\right)
%    =   \bar{g}_{rj} \frac{\partial}{\partial y^n}\bar{\Gamma}_{il}^r
%      + \bar{g}_{ir} \frac{\partial}{\partial y^n}\bar{\Gamma}_{jl}^r \\
%    =& -\frac{1}{3} \bar{g}_{rj} \left(\bar{R}_{\cdot iln}^r+\bar{R}_{\cdot lin}^r \right)
%       -\frac{1}{3} \bar{g}_{ir} \left(\bar{R}_{\cdot jln}^r+\bar{R}_{\cdot ljn}^r \right)
%    =  -\frac{1}{3}\left(\bar{R}_{jlin}+\bar{R}_{iljn} \right)
%\end{align*}

根据式\eqref{chgd:eqn_gijkl=R}和\eqref{chgd:eqn_dGAMMA},我们可以得到度规和联络系数的Taylor展开式
\begin{align}
    \bar{g}_{ij}(y) = & \eta_{ij} - \frac{1}{3}\bar{R}_{iljn}(p) y^l y^n + \cdots \label{chgd:eqn_g-normal} \\
    \bar{g}^{ij}(y) = & \eta^{ij} + \frac{1}{3}\bar{R}_{\cdot l\cdot n}^{i\cdot j}(p) y^l y^n
      + \cdots \label{chgd:eqn_g1-normal} \\
    \bar{\Gamma}_{jl}^{i}(y)=& -\frac{1}{3}\left(
      \bar{R}_{\cdot jln}^i(p)+\bar{R}_{\cdot ljn}^i(p) \right) y^n +\cdots \label{chgd:eqn_Gamma-normal}
\end{align}
只需知道$p$点黎曼曲率$\bar{R}_{iljn}(p)$值就可以计算二阶微扰了.
%更高阶法坐标系Taylor展开可用计算机代码\cite{brewin_2009}给出.


%由式\eqref{chrg:eqn_Riemannian04-component}可得$p$点$\Tpq{0}{4}$型曲率表达式
%\begin{equation}
%    \bar{R}_{ijkn} =\frac{1}{2}\left(
%      \frac{\partial^2 \bar{g}_{in}} {\partial x^j\partial x^k}
%    - \frac{\partial^2 \bar{g}_{jn}} {\partial x^i\partial x^k}
%    - \frac{\partial^2 \bar{g}_{ik}} {\partial x^j\partial x^n}
%    + \frac{\partial^2 \bar{g}_{jk}} {\partial x^i\partial x^n} \right)
%\end{equation}
%
%\begin{align*}
%    \bar{R}_{ijkn} =&\frac{1}{2}\left(
%      \frac{\partial^2 \bar{g}_{in}} {\partial x^j\partial x^k}
%    - \frac{\partial^2 \bar{g}_{jn}} {\partial x^i\partial x^k}
%    - \frac{\partial^2 \bar{g}_{ik}} {\partial x^j\partial x^n}
%    + \frac{\partial^2 \bar{g}_{jk}} {\partial x^i\partial x^n} \right)   \\
%    \bar{R}_{inkj} =&\frac{1}{2}\left(
%      \frac{\partial^2 \bar{g}_{ij}} {\partial x^n\partial x^k}
%    - \frac{\partial^2 \bar{g}_{nj}} {\partial x^i\partial x^k}
%    - \frac{\partial^2 \bar{g}_{ik}} {\partial x^n\partial x^j}
%    + \frac{\partial^2 \bar{g}_{nk}} {\partial x^i\partial x^j} \right)
%\end{align*}
%
%
%
%\begin{align*}
%    R^a_{\cdot(bcd,e)} = \Gamma^a_{d(bc,e)} - \Gamma^a_{(bc,e)d}
%    + \left(\Gamma^a_{i(c}\Gamma^i_{bd}\right)_{,e)}
%    - \left(\Gamma^a_{id}\Gamma^i_{(bc}\right){}_{,e)}
%\end{align*}



%因为$\partial_k \partial_l g_{ij} =\partial_l \partial_k g_{ij} $,
%由式\eqref{chrg:eqn_Dg=0}($\partial_a g_{bc} = \Gamma_{ba}^e g_{ec} + \Gamma_{ca}^e g_{be}$)得
%\begin{equation} \label{chgd:eqn_gijkl=dG}
%    \partial_l \partial_k \bar{g}_{ij}
%    = \partial_{(l}(\bar{\Gamma}_{k)j}^p \bar{g}_{pi}) + \partial_{(l}(\bar{\Gamma}_{k)i}^p \bar{g}_{jp}) .
%\end{equation}
%带入式\eqref{chgd:eqn_gijkl=dG}易得


\begin{example}
    二维半球面的法坐标系.
\end{example}

通过这道例题向大家初步展示如何建立法坐标系,以及初步验证上面几个微扰展开式.
设有三维欧氏空间$(\mathbb{R}^3,\delta_{ij})$,半径为$1$的二维球面切于原点$O$,
切平面是$x-y$平面,球心位于$z$轴上$(0,0,1)$点;我们只取$0\leqslant z <1$的半个球面.
此半球面$S$方程是
\begin{equation}
    x^2+y^2 + (z-1)^2 = 1, \qquad -1<x,y< 1,\quad 0\leqslant z <1 .
\end{equation}
二维球面的测地线是过球心平面所截取的大圆.
切空间是$x-y$平面,此空间上有坐标系$\{x,y\}$;
半球面$S$上的指数映射$\exp_O$将坐标$\{x,y\}$带到$S$上;
为了有显示的区分,我们用$\xi,\eta$两个参量来描述半球面$S$的法坐标系坐标参量.
在$x-y$平面上显然有$\xi=x, \eta=y$,在半球面$S$上对应的点需要求出.
点$(\xi,\eta)$在$x-y$平面上的弧长是$\sqrt{\xi^2 +\eta^2}$,
指数映射$\exp_O$将它映射到大圆上弧长相等的点$p$,
$p$点在$\mathbb{R}^3$中的坐标投影是
\begin{equation}
    \phi(\xi,\eta)=\left(\frac{ \sin \theta}{\theta }\xi,\
      \frac{ \sin \theta}{\theta}\eta,\  1-\cos\theta \right); \quad
      \text{其中}\ \theta = \sqrt{\xi^2 +\eta^2} .
\end{equation}
到此,已建立好法坐标系,我们开始验证式\eqref{chgd:eqn_g-normal}--\eqref{chgd:eqn_Gamma-normal}的正确性.
可借助计算机符号软件完成下面繁琐的计算.
用式\eqref{chsm:eqn_gijhab}来求取半球面上的诱导度规
\begin{align}
     h_{11}=& \frac{- \eta ^2 \cos 2\theta + \eta ^2
         +2 \xi ^2 \left(\eta ^2+\xi ^2\right)}{2 \left(\eta ^2+\xi ^2\right)^2}
         \approx 1-\frac{\eta ^2}{3} +\cdots \\
     h_{12}=& \frac{\eta  \xi  \bigl( \cos 2\theta-1
         +2 (\eta ^2+\xi ^2)\bigr)}{2 \left(\eta ^2+\xi ^2\right)^2}
         \approx\frac{\eta  \xi }{3} +\cdots \\
     h_{22}=& \frac{- \xi ^2 \cos 2\theta + \xi ^2
         +2 \eta ^4+2 \eta ^2 \xi ^2}{2 \left(\eta ^2+\xi ^2\right)^2}
         \approx1-\frac{\xi ^2}{3} +\cdots
\end{align}
由上面诱导度规$h_{ab}$可求得半球面的克氏符(表达式十分复杂),只给一个.
\begin{equation*}
    \Gamma^1_{12}=\frac{\eta  \left(\left( \xi ^2+4 \eta ^4-4 \xi ^4\right)
        \cot\theta -  \xi ^2 \cos3\theta \csc\theta
        -4 \eta ^2 \sqrt{\eta ^2+\xi ^2}\right)}
        {4  \left(\eta ^2+\xi ^2\right)^{5/2}}
       \approx-\frac{\eta}{3}+\cdots
\end{equation*}
由克氏符号可求得非零黎曼曲率 %以及截面曲率(见式\eqref{chrg:eqn_sectional-curvature})
\begin{equation}
    R_{1212}=\left(\frac{\sin \theta}{\theta} \right)^2
     \approx 1- \frac{\eta ^2+\xi ^2}{3}  + \cdots
\end{equation}


容易看出在二维半球面的法坐标系$\{\xi,\eta\}$下:度规、克氏符等符合定理\ref{chgd:thm_RNC}中叙述;
首项非零微扰(二阶)展开式也符合式\eqref{chgd:eqn_g-normal}--\eqref{chgd:eqn_Gamma-normal}.




\index[physwords]{完备!测地}

\section{测地完备性}\label{chgd:sec_compelete}
 


\begin{definition}\label{chgd:def_distance}
    设$(M,g)$是$m$维、连通、正定黎曼流形,$\forall p,q \in M$,
    令$\gamma\in M$是连接$p$、$q$的分段光滑曲线,$L(\gamma)$是两点间的线长;
    我们将$p$、$q$两点间{\heiti 距离$d(p,q)$}定义为$L(\gamma)$的下确界.
\end{definition}

可以证明\cite[\S 3.6]{chen-li-2023-2ed-v1}距离$d$满足正定性、对称性和三角关系不等式.

可以证明\cite[\S 3.6]{chen-li-2023-2ed-v1}$M$上由上述距离$d$诱导的拓扑与微分拓扑是同胚的.

\begin{definition}\label{chgd:def_complete-distance}
    设正定黎曼流形$(M,g)$是连通的,如果作为度量空间$(M,d)$是完备的(见定义\ref{chtop:def_complete-metric}),
    那么称$M$是{\heiti 度量完备}的,一般简称{\heiti 完备}.
\end{definition}

因不定度规下存在零模曲线,其线长恒为零;
故定义\eqref{chgd:def_distance}并不适合不定度规情形;
同理,定义\eqref{chgd:def_complete-distance}也不大适合不定度规情形.

\begin{definition}\label{chgd:def_complete-geodesic}
    设广义黎曼流形$(M,g)$是连通的,如果任何一条测地线的
    {\kaishu 仿射参数}定义域都可延拓至整个实数轴,
    那么称$M$是{\heiti 测地完备}的.
\end{definition}
注意:上述定义中必须是{\kaishu 仿射参数}.
测地完备还可以叙述成:$\forall p\in M$或$\forall v^a\in T_p M$或者两者都任意,指数
映射$\exp_{p}(v^a)$都有定义.


\begin{theorem}\label{chgd:thm_Hopf-Rinow}
    (Hopf--Rinow定理)对于连通、正定黎曼流形$(M,g)$下面几个命题等价:
    {\bfseries (1)} $M$是度量完备的;
    {\bfseries (2)} $M$是测地完备的;
    {\bfseries (3)} $(M,d)$的有界闭子集都是$M$的紧致子集.
    (证明请参考\parencite[\S 3.6]{chen-li-2023-2ed-v1}或类似文献)
\end{theorem}


可以证明\cite[\S 3.6]{chen-li-2023-2ed-v1}完备正定黎曼流形上任意两点都可以用最短测地线相互连接.

等距同胚保紧致性、保子集的(开)闭属性、保子集的有界性,所以
由Hope--Rinow定理中的(3)可知:等距同胚保持正定黎曼流形完备性不变.

\begin{definition}\label{chgd:def_extensible}
    设有连通广义黎曼流形$(M,g)$和$(N,h)$,$(\widetilde{M},g)$是$M$的一个真开子流形;
    若存在等距同胚映射$\phi:N\to \widetilde{M}$,则
    称$(N,h)$是{\heiti 可延拓的};否则称$(N,h)$是{\heiti 不可延拓的}.
\end{definition}

测地完备的正定黎曼流形$M$中已将测地线定义域延拓至最大了,
没法再延拓了,所以完备正定黎曼流形是不可延拓的.


Hopf--Rinow定理貌似还没有推广到不定度规情形.
%不定度规情形下的延拓仍未彻底解决.

\index[physwords]{弧长变分}
\section{弧长变分公式}

%先给出这里用到的变分概念以及记号\cite[\S 3.3]{chen-li-2023-2ed-v1}.

\subsection{变分概念}\label{chgd:sec_arc-variation}
\begin{definition}\label{chgd:def_arc-variation}
    设$C:[r_1,r_2]\to M$是广义黎曼流形$(M,g)$中的一条光滑曲线段,如果有$\epsilon > 0$以及
    光滑映射$\sigma:[r_1,r_2]\times (-\epsilon, \epsilon) \to M$使得
    $\sigma(s,0)=C(s),\ \forall s\in [r_1,r_2]$,则称映射$\sigma$是
    曲线$C$的一个{\heiti 变分}(variantion);$C(s)$称为变分的{\heiti 基准曲线}(base curve).
    对于任意固定的$\tau \in(-\epsilon, \epsilon) $,由$\gamma_\tau(s)\equiv \sigma(s,\tau)$定义
    的参数曲线$\gamma_\tau : [r_1,r_2]\to M$称为{\heiti 变分曲线}(curve in the variation;$s$-曲线).
    如果固定$s$,由$\phi_s(\tau)\equiv \sigma(s,\tau)$定义
    的参数曲线称为{\heiti 横截曲线}(transversal curve;$\tau$-曲线).
\end{definition}

对于任意固定的$\tau$,若变分曲线$\gamma_\tau$都是$M$上测地线,
则称其为{\heiti 测地变分}. \index[physwords]{测地变分}

\begin{remark}\label{chgd:remk_cs}
    我们取基准曲线$C(s)$的参数$s$为曲线的弧长.
    同一个$\tau$值下,变分曲线$\gamma_\tau(s)$的参数$s$和基准曲线的$s$数值是相同的,
    但变分曲线$\gamma_\tau(s)$的参数$s$一般不再是弧长,所以$\gamma_\tau(s)$的
    切线切矢量一般不是单位长.
\end{remark}


如果$C(s)$的变分$\sigma$满足
\begin{equation*}
    \sigma(r_1,\tau)=C(r_1), \quad   \sigma(r_2,\tau)=C(r_2), \qquad \forall \tau \in (-\epsilon, \epsilon),
\end{equation*}
则称$\sigma$是基准曲线$C(s)$的一个{\heiti 具有固定端点的变分}.

对于$C$的每一个变分$\sigma$可以引入沿$\sigma$定义的光滑矢量场
\begin{equation}
    \tilde{S}^a = \sigma_{*(s,\tau)}\left(\frac{\partial}{\partial s}\right)^a, \qquad
    \tilde{T}^a = \sigma_{*(s,\tau)}\left(\frac{\partial}{\partial \tau}\right)^a.
\end{equation}
那么$\tilde{S}^a$是变分曲线族$\{\gamma_\tau\}$的切矢场;
$\tilde{T}^a$是横截曲线族$\{\phi_s\}$的切矢场.
当端点固定时,有$\tilde{T}^a(r_1,\tau)=0=\tilde{T}^a(r_2,\tau),\ \forall \tau \in (-\epsilon,\epsilon)$.

由二维区域$[r_1,r_2]\times (-\epsilon, \epsilon)$在$M$上构成的二维子流形上,$s$和$\tau$是两个
相互独立的坐标,故它们的自然基矢是对易的,即诱导联络无挠
(见\eqref{chfb:eqn_induce-con-NoTorsion})
\begin{equation}\label{chgd:eqn_stts}
    \sigma_{*(s,\tau)}\left[\frac{\partial}{\partial s}, \frac{\partial}{\partial \tau}\right]^a = 0
     {\  \Leftrightarrow \ }
     \nabla_{\frac{\partial}{\partial s}}
       \left(\sigma_{*(s,\tau)}\frac{\partial}{\partial \tau}\right)^a -
     \nabla_{\frac{\partial}{\partial \tau}}
       \left(\sigma_{*(s,\tau)}\frac{\partial}{\partial s}\right)^a = 0 .
\end{equation}


为简单起见,令$S^a(s)\equiv \tilde{S}^a(s,0)$和$T^a(s)\equiv \tilde{T}^a(s,0)$.
$T^a(s)$是沿曲线$C(s)$定义的光滑切矢量场,
称为变分$\sigma$的{\heiti 变分矢量场}.一般说来,它不平行于$C(s)$的切线.

\index[physwords]{弧长第一变分}

\subsection{第一、二变分公式}
下面来计算变分曲线$\gamma_\tau(s)$弧长,由曲线弧长\eqref{chrg:eqn_arc-length}有:
\begin{equation}
    L(\tau)=\int_{r_1}^{r_2} \sqrt{g_{ab} \tilde{S}^a \tilde{S}^b} {\rm d}s,
     \qquad \forall \tau \in (-\epsilon,\epsilon) .
\end{equation}
很明显$L(\tau)$是一个标量函数,对其取$\tau$的导数(需要把联络理解成诱导联络).
\setlength{\mathindent}{0em}
\begin{align}
    &\frac{{\rm d}}{{\rm d} \tau }L(\tau)= \left(\frac{\partial}{\partial \tau}\right)^c \nabla_c
    \int_{r_1}^{r_2} \sqrt{ g_{ab} \tilde{S}^a \tilde{S}^b } {\rm d}s
    =  \int_{r_1}^{r_2} \frac{1}{\sqrt{ \tilde{S}_a \tilde{S}^a } } g_{ab} \tilde{S}^a
     \nabla_{\frac{\partial}{\partial \tau}} \tilde{S}^b {\rm d}s        \label{chgd:eqn_1st-tmp02} \\
    &\xlongequal{\ref{chgd:eqn_stts}} 
      \int_{r_1}^{r_2} \frac{1}{\sqrt{ \tilde{S}_a \tilde{S}^a } }
      \tilde{S}_b \nabla_{\frac{\partial}{\partial s}} \tilde{T}^b {\rm d}s
    =\int_{r_1}^{r_2}\frac{1}{\sqrt{ \tilde{S}_a \tilde{S}^a } }
    \left\{ \nabla_{\frac{\partial}{\partial s}} (\tilde{S}_b  \tilde{T}^b)
     -  \tilde{T}^b \nabla_{\frac{\partial}{\partial s}} \tilde{S}_b \right\} {\rm d}s . \notag
\end{align}\setlength{\mathindent}{2em}
需要注意上式中的曲线$\gamma_\tau(s)$的参数一般不是弧长;
当$\tau=0$时,依约定$\gamma_0(s)$的参数$s$是弧长,即${S}_a {S}^a=1$;
由此可以得到{\heiti 弧长第一变分公式}:  
\begin{equation}\label{chgd:eqn_1st-var-arc}
    L'(0)= \left.\left({S}_b  {T}^b\right)\right|_{r_1}^{r_2} - \int_{r_1}^{r_2}
    {T}_b \nabla_{\frac{\partial}{\partial s}} {S}^b {\rm d}s .
\end{equation}
上式中的$S_b(s)=\tilde{S}_b(s,0)$,是曲线$\gamma_0(s)$的切矢量;而$T^a(s)= \tilde{T}^a(s,0)$.
如果所有变分曲线有相同的端点,上式还可以化简(注$\tilde{T}^a(r_1,\tau)=0=\tilde{T}^a(r_2,\tau)$).
\begin{equation}\label{chgd:eqn_1st-var-arc-fixSE}
    L'(0)= -\int_{r_1}^{r_2} {T}_b  \nabla_{\frac{\partial}{\partial s}} {S}^b {\rm d}s .
\end{equation}
当第一变分公式为零时,又因${T}_b=\sigma_{*(s,0)}\left(\frac{\partial}{\partial \tau}\right)^a$不恒为零,
故必有$\nabla_{\frac{\partial}{\partial s}} {S}^b=0$;这是测地线方程,所以曲线$C(s)$是测地线.
最终,由式\eqref{chgd:eqn_1st-var-arc-fixSE}可得定理
\begin{theorem}\label{chgd:thm_geo-1st-var}
    光滑曲线$C(s)$为测地线的充要条件是:在固定端点情形下,曲线$C(s)$是弧长泛函变分的临界点,
    也就是第一变分公式恒为零.
\end{theorem}


如果$C(s)$是分段光滑的,除了讨论会麻烦一些,所有结果都与上述相同.

\index[physwords]{弧长第二变分}

%\subsection{弧长第二变分公式}
继续求弧长泛函的二阶导数,则可得第二变分公式. %\cite[\S 8.6]{cc2001-zh}
%由于主要目的是判断两固定端点间的弧长是最短线还是最长线,所以
需要在$L'(0)=0$(即$C(s)$是测地线)的前提下计算$L''(0)$. %,否则没有意义
继续式\eqref{chgd:eqn_1st-tmp02}的计算,有
\setlength{\mathindent}{0em}
\begin{align*}
    \frac{{\rm d}^2}{{\rm d} \tau^2 }L(\tau)=&   \frac{{\rm d}}{{\rm d} \tau }
    \int_{r_1}^{r_2} \frac{1}{\sqrt{ \tilde{S}_a \tilde{S}^a } }
    \tilde{S}_b \nabla_{\frac{\partial}{\partial s}} \tilde{T}^b {\rm d}s
    =  \nabla_{\frac{\partial}{\partial \tau}}
    \int_{r_1}^{r_2} \frac{1}{\sqrt{ \tilde{S}_a \tilde{S}^a } }
    \tilde{S}_b \nabla_{\frac{\partial}{\partial s}} \tilde{T}^b {\rm d}s  \\
    =& \int_{r_1}^{r_2} \left\{\frac{(\nabla_{\frac{\partial}{\partial \tau}} \tilde{S}_b)
        (\nabla_{\frac{\partial}{\partial s}} \tilde{T}^b)
    +\tilde{S}_b (\nabla_{\frac{\partial}{\partial \tau}} \nabla_{\frac{\partial}{\partial s}} \tilde{T}^b) }
    {\sqrt{ \tilde{S}_a \tilde{S}^a } } - \frac{ (\tilde{S}_a \nabla_{\frac{\partial}{\partial \tau}} \tilde{S}^a)
     (\tilde{S}_b\nabla_{\frac{\partial}{\partial s}} \tilde{T}^b)} {{( \tilde{S}_a \tilde{S}^a) }^{3/2}}  \right\}{\rm d}s \\
    \xlongequal{\ref{chccr:eqn_RiemannianCurvature-13}} &
    \int_{r_1}^{r_2} \biggl\{
    \frac{(\nabla_{\frac{\partial}{\partial s}} \tilde{T}_b) 
    (\nabla_{\frac{\partial}{\partial s}} \tilde{T}^b)}{\sqrt{ \tilde{S}_a \tilde{S}^a } }
    -\frac{(\tilde{S}_b \nabla_{\frac{\partial}{\partial s}} \tilde{T}^b)^2 }
    {{( \tilde{S}_a \tilde{S}^a) }^{3/2}} \\
    &+ \frac{\tilde{S}_b}{\sqrt{ \tilde{S}_a \tilde{S}^a } }\left[
     \nabla_{\frac{\partial}{\partial s}} \nabla_{\frac{\partial}{\partial \tau}} \tilde{T}^b
    + R_{cad}^b{\left(\frac{\partial}{\partial \tau}\right)^a}
    {\left(\frac{\partial}{\partial s}\right)^d}{\tilde{T}^c}\right] \biggr\} {\rm d}s . 
\end{align*}\setlength{\mathindent}{2em}
在$\tau=0$处取值,此时$\tilde{S}_a \tilde{S}^a =1$;继续计算,有
\setlength{\mathindent}{0em}
\begin{align*}
    \frac{{\rm d}^2}{{\rm d} \tau^2 }L(0)
    =&  \int_{r_1}^{r_2}\left\{ (\nabla_{\frac{\partial}{\partial s}} {T}_b)
      (\nabla_{\frac{\partial}{\partial s}} {T}^b)
      -({S}_b \nabla_{\frac{\partial}{\partial s}} {T}^b)^2
    + {S}_b  R_{cad}^b {\left(\frac{\partial}{\partial \tau}\right)^a}
       {\left(\frac{\partial}{\partial s}\right)^d}{{T}^c}  \right\} {\rm d}s  \\
    & + \int_{r_1}^{r_2}  \frac{\partial}{\partial s} \left( {S}_b
        \nabla_{\frac{\partial}{\partial \tau}} {T}^b    \right) {\rm d}s
      - \int_{r_1}^{r_2} (\nabla_{\frac{\partial}{\partial s}} {S}_b) (
        \nabla_{\frac{\partial}{\partial \tau}} {T}^b) {\rm d}s .
\end{align*}\setlength{\mathindent}{2em}
因$C(s)$是测地线,故$\nabla_{\frac{\partial}{\partial s}} {S}_b=0$,继续计算得到{\heiti 弧长第二变分公式}:
\begin{equation}\label{chgd:eqn_2st-var-arc}
    \begin{aligned}
    &L''(0)= \left.\left( {S}_b  \nabla_{\frac{\partial}{\partial \tau}} {T}^b \right)\right|_{r_1}^{r_2} \\
    & +\int_{r_1}^{r_2} \left( (\nabla_{\frac{\partial}{\partial s}} {T}_b)
       (\nabla_{\frac{\partial}{\partial s}} {T}^b)
       -({S}_b \nabla_{\frac{\partial}{\partial s}} {T}^b)^2
     +  R_{cad}^b {T^a} {S^d}T{^c}{S}_b \right) {\rm d}s .
    \end{aligned}
\end{equation}
上面所有计算都需按诱导联络理解. 

%在判断弧长是极大还是极小时,会用到弧长第二变分公式.

\index[physwords]{Gauss引理}
\subsection{Gauss引理}
\begin{theorem}\label{chgd:thm_Gauss-lemma}
    设有$m$维广义黎曼流形$(M,g)$,$\forall p\in M$,$v^a\in T_p M$.如果指数映射$\exp_p$在$v^a$有定义,
    那么$\forall w^a\in T_p M = T_v(T_p M)$,有
    \begin{equation}\label{chgd:eqn_Gauss-lemma}
        g_{ab} \, \left[(\exp_{p})_{*v}(v^a)\right] \,
        \left[(\exp_{p})_{*v}(w^b)\right] = v_b w^b .
    \end{equation}
\end{theorem}
\begin{proof}
    $T_pM$是线性空间,自然可以看成流形;$v\in T_pM$是这个流形上一点,$T_v(T_p M)$是指流形$T_pM$中点$v$处的切空间.
    定理中已假设$\exp_{p}$在$v^a$处有定义;我们任取$w^a \in T_pM$,存在充分小的$\epsilon >0$使得
    下面的映射有定义(参照定义\ref{chgd:def_arc-variation}):
    \begin{equation}\label{chgd:eqn_gltmp01}
        \sigma(s,\tau)=\exp_{p}\bigl(s (v^a+\tau w^a)\bigr),
        \qquad \forall (s,\tau)\in [0,1]\times (-\epsilon,\epsilon) .
    \end{equation}
    显然,对于每一个固定的$\tau\in  (-\epsilon,\epsilon)$,$\gamma_\tau(s)=\sigma(s,\tau)$是测地线;
    依定义\ref{chgd:def_arc-variation},$\sigma(s,\tau)$是测地线$\gamma_0(s)$的一个测地变分.
    容易求得$\gamma_0(s)$的切线切矢量场为
    \begin{equation}
        S^a(s) = \sigma_{*(s,0)}\left(\frac{\partial}{\partial s}\right)^a
        = (\exp_{p})_{*s v} v^a .
    \end{equation}
    由于式\eqref{chgd:eqn_gltmp01}是指数映射,当$s$固定、$\tau$变化时,得到的
    曲线是测地线,记为$\alpha(\tau;q,s w^a)$,其中$q=\exp_p(s v^a)$.
    很明显$\alpha(\tau;q,s w^a)$是横截曲线,它的切线切矢量$T^a(s)$为
    \begin{equation}
        T^a(s) = \sigma_{*(s,0)}\left(\frac{\partial}{\partial \tau}\right)^a = s\cdot (\exp_{p})_{*s v} w^a .
    \end{equation}
    将以上两式带入弧长第一变分公式\eqref{chgd:eqn_1st-var-arc},并注意$\gamma_{\tau}(s)$是测地线,有
    \begin{equation}\label{chgd:eqn_gltmp03}
        L'(0)= \left.\left({S}_b  {T}^b\right)\right|_{0}^{1}
        =g_{ab} \, (\exp_{p})_{*v}(v^a) \, (\exp_{p})_{*v}(w^b) .
    \end{equation}
    上式中的$g_{ab}$自然是在推前后($(\exp_{p})_{*v}(v^a)$)的点取值.

    我们再以另外一种方法来求$L'(0)$.
    对于任意固定的$\tau$,$\gamma_\tau(s)$的弧长也可按下式计算
    \begin{equation}
        L(\tau)=\int_{0}^{1} \left|\frac{{\rm d}}{{\rm d}s} \gamma_\tau(s;p,v^a+\tau w^a)\right| {\rm d}s
        =\int_{0}^{1} |v+\tau w| {\rm d}s = |v+\tau w|.
    \end{equation}
    对上式求$\tau$的导数,
    \begin{align*}
        \left.\frac{{\rm d}L(\tau)}{{\rm d}\tau}\right|_{\tau=0} =& \left.
        \frac{{\rm d} (v_av^a+2\tau v_aw^a+\tau^2 w_a w^a)^{1/2}}{{\rm d}\tau}\right|_{\tau=0}  \\
        =&\left.\frac{v_aw^a+\tau w_a w^a}{ (v_av^a+2\tau v_aw^a+\tau^2 w_a w^a)^{1/2}}\right|_{\tau=0}
        =\frac{v_aw^a}{|v|} .
    \end{align*}
    因为我们采用的是弧长参数,所以$|v|=1$.
    结合上式与式\eqref{chgd:eqn_gltmp03}便可得式\eqref{chgd:eqn_Gauss-lemma}.
    上式中的$v_aw^a$是切空间$T_p M$上的内积,在$p$点取值.
\end{proof}

\begin{lemma}\label{chgd:thm_Gauss-lemma-old}
    指数映射$\exp_{p}$把$T_pM$中正交的切矢量$v^a$和$w^a$映射到$M$中正交的切矢量.
\end{lemma}
\begin{proof}
    只需将$v_aw^a=0$带入定理\ref{chgd:thm_Gauss-lemma}即可得到本引理.本引理才是
    最早由Gauss证明的“Gauss lemma”,定理\ref{chgd:thm_Gauss-lemma}是推广的“Gauss lemma”.
\end{proof}

\begin{proposition}\label{chgd:thm_Gauss-lemma-baochang}
    指数映射$\exp_{p}$沿射线$tv^a$的切方向是保长的,即对于任意的、平行于$v^a$的
    切矢量$w^a\in T_pM$,都有$|(\exp_{p})_{*tv}(w)|=|w|$.
\end{proposition}
\begin{proof}
    设$w^a= \lambda v^a$,其中$\lambda$是非零实常数,则由高斯引理可得
    \begin{equation*}
        |(\exp_{p})_{*tv}(w)| = g_{ab}(\exp_{p})_{*tv}(\lambda v^a) (\exp_{p})_{*tv}(w^b)
        = \lambda v_a w^a = w_a w^a.
    \end{equation*}
    需要提醒读者$v^a$是测地线的切线方向.
\end{proof}

\begin{remark}\label{chgd:rmk_Gauss-lemma}
    由此可见指数映射既保长又保角,故它是沿径向测地线局部等距同构.
\end{remark}

\begin{remark}\label{chgd:rmk_geo-sph}
设$p \in M$,假定 $\delta>0$ 使得指数映射 $\exp _p$ 在
\begin{equation}\label{chgd:eqn_Ball}
    B_p(\delta)=\left\{v^a \in T_p M \mid  |v|<\delta\right\}
\end{equation}
上有定义.根据 Gauss 引理,
\begin{equation}\label{chgd:eqn_gdBall}
    \mathscr{B}_p(\delta)  =\exp _p\bigl(B_p(\delta)\bigr) .
\end{equation}
$\mathscr{B}_p(\delta)$的几何含义是:它由这样的$q$点组成,$q$和$p$间存在测地线$\gamma$连接它们,  
并且两点间测地线长度 $L(\gamma)<\delta$.
$\mathscr{B}_p(\delta)$称为在$M$中以$p$为中心、以$\delta$为半径的{\heiti 测地球};
其边界$\partial \mathscr{B}_p(\delta)$记为$\mathscr{S}_p(\delta)$,
称为在$M$中以$p$为中心、以$\delta$为半径的{\heiti 测地球面}.
很明显,测地球$\mathscr{B}_p(\delta)$和开球$B_p(\delta) \subset T_p M$ 未必是同胚的.
但从定理\ref{chgd:thm_exp-homeomorphism}可知:当$\delta$充分小时,
指数映射是从 $B_p(\delta)$ 到 $\mathscr{B}_p(\delta)$ 的光滑同胚;
此时,$\mathscr{S}_p(\delta)$与球面$S^{m-1}$是同胚的.
根据Gauss引理\ref{chgd:thm_Gauss-lemma-old}可知:
从点$p$出发的测地线与$\mathscr{S}_p(\delta)$是正交的,
所以通常把这样的测地线称为{\heiti 径向测地线}. \qed
\end{remark}
\index[physwords]{测地球}
\index[physwords]{测地球面}
\index[physwords]{径向测地线}


\index[physwords]{Jacobi场}
\section{Jacobi场}\label{chgd:sec_Jacobi}
采用\S \ref{chgd:sec_arc-variation}记号.
需提醒读者,在考虑曲线时,一般使用的是诱导联络.
设曲线$C(s)$是测地线,它的每一个变分$\sigma(s,\tau)$也是测地线,
即只考虑{\kaishu 测地变分}情形.
$\tilde{S}^a$是测地变分曲线族$\{\gamma_\tau(s)\}$的切矢场;
$\tilde{T}^a$是横截曲线族$\{\phi_s(\tau)\}$的切矢场.
因是测地变分,则$ \nabla_{\frac{\partial}{\partial s}}\tilde{S}^a=0$;
再由\eqref{chgd:eqn_stts}($ \nabla_{\frac{\partial}{\partial s}}\tilde{T}^a
- \nabla_{\frac{\partial}{\partial \tau}}\tilde{S}^a =0$)可得
\begin{equation}
    \nabla_{\frac{\partial}{\partial s}} \nabla_{\frac{\partial}{\partial s}}\tilde{T}^a
    =\nabla_{\frac{\partial}{\partial s}} \nabla_{\frac{\partial}{\partial \tau}}\tilde{S}^a
    =\nabla_{\frac{\partial}{\partial s}} \nabla_{\frac{\partial}{\partial \tau}}\tilde{S}^a
     -\nabla_{\frac{\partial}{\partial \tau}} \nabla_{\frac{\partial}{\partial s}}\tilde{S}^a
    =R^{a}_{\cdot bcd} \tilde{S}^c\tilde{T}^d \tilde{S}^b .
\end{equation}
上式最后一步再次利用式\eqref{chgd:eqn_stts}($\left[\frac{\partial}{\partial s},
\frac{\partial}{\partial \tau}\right]^a = 0$)
以及黎曼曲率定义式\eqref{chccr:eqn_RiemannianCurvature-13}.
令上式中的$\tau=0$,那么依照\S \ref{chgd:sec_arc-variation}中约定
(即$S^a(s)\equiv \tilde{S}^a(s,0)$和$T^a(s)\equiv \tilde{T}^a(s,0)$),
去掉符号上面的波浪线;上面方程变为(同时令$T\to J$)
\begin{equation}\label{chgd:eqn_Jacobi}
    \nabla_{\frac{\partial}{\partial s}} \nabla_{\frac{\partial}{\partial s}}{J}^a =
    R^{a}_{\cdot bcd} {S}^c{J}^d {S}^b.
\end{equation}
此式称为{\bfseries \heiti Jacobi方程};自然是因为Jacobi的原因才将$T$变$J$.

\index[physwords]{Jacobi方程}

\begin{definition}\label{chgd:def_Jacobi-field}
    设$C:[r_1,r_2]\to M$是$m$维广义黎曼流形$(M,g)$上的一条测地线,$J^a(s)$是$M$上
    沿$C(s)$定义的一个光滑矢量场.如果$J^a(s)$满足式\eqref{chgd:eqn_Jacobi},
    则称$J^a(s)$是沿测地线$C(s)$定义的一个{\bfseries \heiti Jacobi矢量场},
    简称为{\bfseries \heiti Jacobi场}.
\end{definition}
$J^a(s)$是定义在$C(s)$上的切矢量场;
$S^a$是$C(s)$的切线切矢量.则有
\begin{proposition}\label{chgd:thm_vvJ}
    设$C:[r_1,r_2]\to M$是$m$维广义黎曼流形$(M,g)$上的一条测地线,
    则$C(s)$测地变分的变分矢量场是沿测地线$C(s)$的Jacobi矢量场.
\end{proposition}
%此定理之逆也成立,见\ref{chgd:thm_vvJ-inv}.
%$C(s)$的变分矢量场的方向一般不平行于$C(s)$的切线切矢量(称为纵向),直观感觉它
%沿横向拉着$C(s)$偏离原来的位置,
%所以Jacobi方程也被称为{\heiti 测地偏离方程}(geodesic deviation).



下面给出Jacobi方程的分量表达式.我们取一个沿测地线$C(s)$平行移动的
正交归一活动标架场$\{(e_i)^a\}$,并取$(e_1)^a=(\frac{{\rm d}}{{\rm d}s})^a|_{C(s)}$
(如果指标$i$是从$0$开始记号的,则取$(e_0)^a=(\frac{{\rm d}}{{\rm d}s})^a|_{C(s)}$).
因标架场是平行移动的,所以有
\begin{equation}
    \nabla_{\frac{\partial}{\partial s}} (e_i)^a=0,\qquad 1\leqslant i \leqslant m.
\end{equation}
因$J^a$是沿$C(s)$定义的光滑切矢量场,可以把它表示为
\begin{equation}
    J^a(s) =  J^i(s) (e_i)^a .
\end{equation}
于是有(我们已约定$(e_1)^a=(\frac{{\rm d}}{{\rm d}s})^a|_{C(s)}=S^a$)
\begin{align}
    &\nabla_{\frac{\partial}{\partial s}} J^a(s) = \frac{\partial J^i(s)}{\partial s} (e_i)^a, \qquad
    \nabla_{\frac{\partial}{\partial s}} \nabla_{\frac{\partial}{\partial s}} J^a(s) =
      \frac{\partial^2 J^i(s)}{\partial s^2} (e_i)^a. \\
    &\nabla_{\frac{\partial}{\partial s}} \nabla_{\frac{\partial}{\partial s}} J^a(s)
     -R^{a}_{\cdot bcd} {S}^c{J}^d {S}^b = \left( \frac{\partial^2 J^i(s)}{\partial s^2}
     -R^i_{\cdot 11j}\bigl(C(s)\bigr) J^j(s) \right)(e_i)^a .
\end{align}
由此可得Jacobi方程的在活动标架场$\{(e_i)^a(s)\}$的分量方程式
\begin{equation}\label{chgd:eqn_Jacobi-componet}
    \frac{{\rm d}^2 J_i(s)}{{\rm d} s^2} =R_{i11j}\bigl(C(s)\bigr) J^j(s) ,
    \qquad  1\leqslant i \leqslant m.
\end{equation}
因标架场$\{(e_i)^a(s)\}$是正交归一的,所以$g_{ij}=g_{ab}(e_i)^a(e_j)^b=\eta_{ij}$,
其中$\eta_{ij}$非对角元都是零,对角元是$+1$或$-1$.我们已用$\eta_{ij}$把上式中的指标降下来了,
$J_i=\eta_{ij} J^j$.
同时考虑到方程只是在测地线$C(s)$上取值,是单参数的,所以作了$\partial \to {\rm d}$的替换.
从式\eqref{chgd:eqn_Jacobi-componet}可看出Jacobi方程是线性、齐次常微分方程组.

\begin{proposition}\label{chgd:thm_Jacobi-unique}
    设$C:[r_1,r_2]\to M$是广义黎曼流形$(M,g)$上一条测地线,那么$\forall v^a,w^a\in T_{C(r_1)}M$,
    存在唯一的一个沿$C(s)$的Jacobi场$J^a(s)$满足$J^a(r_1)=v^a,\frac{{\rm d}J^a}{{\rm d}s}|_{r_1}=w^a$.
\end{proposition}
\begin{proof}
    我们先把方程式\eqref{chgd:eqn_Jacobi-componet}化成一阶常微分方程组,
    令$W_i= \frac{{\rm d}J_i}{{\rm d}s}$,有
    \begin{equation}\label{chgd:eqn_Jacobi-componet-1stOrder}
        \begin{cases}
          \frac{{\rm d}J_i}{{\rm d}s} =W_i(s) \\
          \frac{{\rm d} W_i}{{\rm d} s} =R_{i11j}\bigl(C(s)\bigr) J^j(s)
        \end{cases}, \qquad \text{初条件}\
        \begin{cases}
           J_i(r_1) = \eta_{ij} v^j \\ W_i(r_1) = \eta_{ij}w^j
        \end{cases} .
    \end{equation}
    依据常微分方程组解存在唯一性定理可知 %\cite[\S 31]{arnold-2001-ode}
    上述方程式解存在、唯一.
\end{proof}

我们知道$T_{C(r_1)}M$是$m$维的线性空间,而Jacobi场方程\eqref{chgd:eqn_Jacobi-componet-1stOrder}的解
由初始条件$(v^a,w^a)\in T_{C(r_1)}M\oplus T_{C(r_1)}M$唯一确定;由上述命题可知
这个关系是一个\uwave{双射}.因Jacobi方程是线性的,所有解的集合
按定义\ref{chmla:def_linear-space}构成线性空间,很明显有如下命题:
\begin{proposition}\label{chgd:thm_Jacobi-2mD}
    线性Jacobi场方程\eqref{chgd:eqn_Jacobi-componet}或\eqref{chgd:eqn_Jacobi-componet-1stOrder}的
    解空间是一个$2m$维线性空间.
\end{proposition}


有了这些准备,现在可以证明命题\ref{chgd:thm_vvJ}的逆命题了.
\begin{proposition}\label{chgd:thm_vvJ-inv}
    设$C:[r_1,r_2]\to M$是$m$维完备广义黎曼流形$(M,g)$上的测地线,
    $J^a(s)$是一沿测地线$C(s)$的Jacobi场.
    则$J^a(s)$必是$C(s)$的某个测地变分的变分矢量场.
\end{proposition}
\begin{proof}
    为了叙述简单,令$r_1=0$,并记$p=C(0)$.在流形$M$上任取一条光滑曲线$\alpha(\tau)$,
    其中$\tau\in (-\epsilon,\epsilon)$,使得
    \begin{equation}\label{chgd:eqn_tmpinit}
        \alpha(0)=p,\quad \left.\left(\frac{\rm{d}}{{\rm d} \tau}\right)^a \right|_{\alpha(0)}=v^a;
        \qquad \text{对任意固定的}\  v^a \in T_pM .
    \end{equation}
    再取对于任意固定的$w^a \in T_pM$,把$w^a$和$S^a(0)=(\frac{\rm{d}}{{\rm d} s})^a|_{C(0)}$
    (测地线端点的切线切矢量)沿曲线$\alpha(\tau)$作平行移动,得到两个沿$\alpha(\tau)$平行
    的矢量场$W^a(\tau)$和$S^a(\tau)$,它们满足
    \begin{equation}\label{chgd:eqn_tmppp}
        \nabla_{\frac{\rm{d}}{{\rm d} \tau}} W^a(\tau) =0,\qquad
        \nabla_{\frac{\rm{d}}{{\rm d} \tau}} S^a(\tau) =0.
    \end{equation}
    由此可以构造映射$\sigma(s,\tau)$如下
    \begin{equation}\label{chgd:eqn_sigmaJT}
        \sigma(s,\tau)= \exp_{\alpha(\tau)}\Bigl[s\bigl(S^a(\tau)+\tau W^a(\tau)\bigr)\Bigr],
        \qquad (s,\tau)\in [0,r_2]\times(-\epsilon,\epsilon)  .
    \end{equation}
    对于每一个固定的$\tau$,曲线$\alpha_\tau(\cdot)\equiv \sigma(\cdot,\tau)$是测地线,
    并且$C(s)=\alpha_0(s)$;所以$\sigma(s,\tau)$是测地线$C(s)$的{\kaishu 测地变分}.
    其变分曲线和横截曲线的切矢量场分别是
    \begin{equation}
        \tilde{S}^a = \sigma_{*(s,\tau)}\left(\frac{\partial}{\partial s}\right)^a ,\quad
        \tilde{T}^a = \sigma_{*(s,\tau)}\left(\frac{\partial}{\partial \tau}\right)^a .
    \end{equation}
    其中变分曲线的切矢量场是
    \begin{equation*}
        \left.\tilde{S}^a\right|_{s=0} = \sigma_{*(0,\tau)} \left(\frac{\partial}{\partial s}\right)^a
        =\bigl(\exp_{\alpha(\tau)}\bigr)_{*0} \bigl(S^a(\tau)+ \tau W^a(\tau) \bigr)
        =S^a(\tau)+ \tau W^a(\tau) .
    \end{equation*}
    根据命题\ref{chgd:thm_vvJ}可知$T^a=\tilde{T}|_{\tau=0}$是沿$C(s)$的Jacobi场,并且有
    \begin{equation*}
        T^a(0)=\sigma_{*(0,0)}\left(\frac{\partial}{\partial \tau}\right)^a
        = \left.\left(\frac{\partial}{\partial \tau}\right)^a  \right|_{s=0,\tau=0} \bigl(\sigma(0,\tau)\bigr)
        = \left.\left(\frac{\partial}{\partial \tau}\right)^a  \right|_{\tau=0}\alpha(\tau) = v^a.
    \end{equation*}
    再由式\eqref{chgd:eqn_stts}($ \nabla_{\frac{\partial}{\partial s}}\tilde{T}^a
    - \nabla_{\frac{\partial}{\partial \tau}}\tilde{S}^a =0$)可得
    \begin{equation*}
        \nabla_{\frac{\partial}{\partial s}} \left.\tilde{T}^a \right|_{\substack{s=0\\ \tau=0}} \!
        =\nabla_{\frac{\partial}{\partial \tau}} \left.\tilde{S}^a \right|_{\substack{s=0\\ \tau=0}} \!
        =\left. \nabla_{\frac{\partial}{\partial \tau}}\bigl(S^a(\tau)+ \tau W^a(\tau) \bigr)
         \right|_{\substack{s=0\\ \tau=0}}\! \xlongequal{\ref{chgd:eqn_tmppp}} W^a(0) =w^a .
    \end{equation*}
    上面讨论说明$T(s)$是$C(s)$上满足初始条件\eqref{chgd:eqn_tmpinit}的Jacobi场.

    现在取$v^a=J^a(0)$,$w^a=\frac{{\rm d}J^a(0)}{{\rm d} s}$.
    那么$T^a$和$J^a$都是定义在$C(s)$上的由$v^a,w^a\in T_pM$确定的Jacobi场,
    从定理\ref{chgd:thm_Jacobi-unique}的唯一性可知$J^a(s)=T^a(s)$,
    故$J^a(s)$是测地线$C(s)$的测地变分的变分矢量场.
\end{proof}

上述定理证明过程给出了定理\ref{chgd:thm_Jacobi-unique}存在性部分的几何证明.
同时可得:
\begin{corollary}\label{chgd:thm_Jexp}
    设$C:[0,b]\to M$是$m$维完备广义黎曼流形$(M,g)$上的测地线;
    $J^a(s)$是一沿测地线$C(s)$的Jacobi场,它满足$J^a(0)=0$
    和$J^a(s)$是测地变分(式中$p\equiv C(0)$)
    \begin{equation}\label{chgd:eqn_Jexp-1}
        \sigma(s,\tau)= \exp_{p}\Bigl[s\bigl(v^a+\tau w^a\bigr)\Bigr],
        \quad v^a = \left.\left(\frac{\rm{d}}{{\rm d} s}\right)^a \right|_{p},
        \quad \forall w^a \in T_{p} M 
    \end{equation}
    的变分矢量场.根据变分矢量场定义和式\eqref{chgd:eqn_Jexp-1}可知
    \begin{equation}\label{chgd:eqn_Jexp-2}
        J^a(s)= \sigma_{*(s,0)}\left(\frac{\partial}{\partial \tau}\right)^a
        =  (\exp_{p})_{* s v} (sw^a) =s\cdot (\exp_{p})_{* s v} ( w^a) .
    \end{equation}    
\end{corollary}
\begin{proof}
    式\eqref{chgd:eqn_Jexp-1}只不过是\eqref{chgd:eqn_sigmaJT}的变种;
    其中$w^a$是任取的,经计算可知$ w^a = \frac{{\rm d}J^a(0)}{{\rm d}s}$;
    $v^a$是测地线$C(s)$在初始点$s=0$处的切线切矢量.
    由式\eqref{chgd:eqn_Jexp-2}可见$J^a(0)=0$.
\end{proof}



\index[physwords]{法Jacobi场}

\subsection{法Jacobi场}
%沿用上一小节符号.
\begin{definition}\label{chgd:def_normal-Jacobi}
    设$C(s)$是广义黎曼流形$(M,g)$上测地线,$J^a(s)$是沿$C(s)$的Jacobi场.
    若$J^a(s)$与$C(s)$处处正交,则称$J^a(s)$是沿测地线$C(s)$的{\heiti \bfseries 法Jacobi场}.  
\end{definition}

取Jacobi方程\eqref{chgd:eqn_Jacobi-componet}中指标$i=1$,由黎曼曲率的反对称性
可知$\frac{{\rm d}^2 J_1(s)}{{\rm d} s^2} =0$,所以$J_1(s)$是个线性函数,即
\begin{equation}\label{chgd:eqn_tmp-J1kc}
    J_1(s)= k \cdot s + c_0 . \qquad k, c_0 \in \mathbb{R} .
\end{equation}
由此,易得
\begin{equation}
    g_{ab}(e_1)^b\frac{{\rm d}J^a}{{\rm d}s}=g_{ab}(e_1)^b(e_i)^a\frac{{\rm d}J^i}{{\rm d}s}
    =\eta_{11}\frac{{\rm d}J^1}{{\rm d}s}=k.
\end{equation}
令上两式中的$s=r_1$,有
\begin{equation}
    k=g_{ab}(e_1)^b|_{r_1} \frac{{\rm d}J^a(r_1)}{{\rm d}s},\qquad
    k\cdot r_1 +c_0 = J_1(r_1)=g_{ab}(e_1)^b|_{r_1} J^a(r_1).
\end{equation}
由此可得$c_0$为
\begin{equation}
    c_0= g_{ab}(e_1)^b|_{r_1}\left( J^a(r_1) - r_1 \cdot \frac{{\rm d}J^a(r_1)}{{\rm d}s} \right) .
\end{equation}
将$k,c_0$带回式\eqref{chgd:eqn_tmp-J1kc},有
\begin{equation}\label{chgd:eqn_J1kc}
    J_1(s)= (s-r_1)\cdot g_{ab}(e_1)^b|_{r_1} \frac{{\rm d}J^a(r_1)}{{\rm d}s}
       + g_{ab}(e_1)^b|_{r_1} J^a(r_1) .
\end{equation}
我们注意到$(e_1)^a=(\frac{{\rm d}}{{\rm d}s})^a|_{C(s)}$是测地线$C(s)$的切线切矢量,故上式可改写为
\begin{equation*} %\label{chgd:eqn_J1}
    g_{ab} J^a(s) \left.\left(\frac{{\rm d}}{{\rm d}s}\right)^b\right|_{C(s)} = J_1(s)=
    \left( (s-r_1)\cdot \frac{{\rm d}J^a(r_1)}{{\rm d}s}
    +  J^a(r_1) \right) g_{ab}(e_1)^b|_{r_1}  .
\end{equation*}
进而可得如下命题
\begin{proposition}\label{chgd:thm_normal-Jacobi}
    设$C:[r_1,r_2]\to M$是广义黎曼流形$(M,g)$上一条测地线,$J^a(s)$是沿
    测地线$C(s)$的Jacobi场,那么$J^a(s)$与测地线$C(s)$处处正交的充要条件是
    \begin{equation}\label{chgd:eqn_normal-Jacobi}
        g_{ab}\left.\left(\frac{{\rm d}}{{\rm d}s}\right)^b\right|_{C(r_1)} \frac{{\rm d}J^a(r_1)}{{\rm d}s} =0,
        \quad \text{且} \quad
        g_{ab}\left.\left(\frac{{\rm d}}{{\rm d}s}\right)^b\right|_{C(r_1)} J^a(r_1) =0 .
    \end{equation}
\end{proposition}


\begin{proposition}
    设$C:[r_1,r_2]\to M$是$m$维广义黎曼流形$(M,g)$上的一条测地线,
    设$J^a(s)$沿$C(s)$的Jacobi矢量场.若存在两个不同的
    点$s_1,s_2\in [r_1,r_2]$使得
    \begin{equation}\label{chgd:eqn_tmp-s1s2}
        g_{ab}\left.\left(\frac{{\rm d}}{{\rm d}s}\right)^b\right|_{C(s_1)} J^a(s_1) =0 ,
        \quad \text{且} \quad
        g_{ab}\left.\left(\frac{{\rm d}}{{\rm d}s}\right)^b\right|_{C(s_2)} J^a(s_2) =0 .
    \end{equation}
    那么$J^a(s)$是法Jacobi场.
\end{proposition}
\begin{proof}
    将式\eqref{chgd:eqn_J1kc}带入\eqref{chgd:eqn_tmp-s1s2},有
    \begin{align*}
        0=& (s_1 -r_1)\cdot g_{ab}(e_1)^b|_{r_1} \frac{{\rm d}J^a(r_1)}{{\rm d}s}+ g_{ab}(e_1)^b|_{r_1} J^a(r_1) , \\
        0=& (s_2 -r_1)\cdot g_{ab}(e_1)^b|_{r_1} \frac{{\rm d}J^a(r_1)}{{\rm d}s}+ g_{ab}(e_1)^b|_{r_1} J^a(r_1) .
    \end{align*}
    因$s_1\neq s_2$,由上式必然可以得到式\eqref{chgd:eqn_normal-Jacobi};故$J^a(s)$是法Jacobi场.
\end{proof}


\begin{proposition}
    设$C(s)$是$m$维广义黎曼流形$(M,g)$上一条测地线.
    用$\mathfrak{J}^{\bot}(C)$表示$M$上沿测地线$C(s)$法Jacobi场的集合,
    则$\mathfrak{J}^{\bot}(C)$是一个$2m-2$维线性空间.
\end{proposition}
\begin{proof}
    由命题\ref{chgd:thm_Jacobi-2mD}可知$M$上全体沿测地线$C(s)$的Jacobi场的集合是
    一个$2m$维线性空间.从命题\ref{chgd:thm_Jacobi-2mD}的证明过程可知,
    初始条件$(v^a,w^a)\in T_{C(r_1)}M\oplus T_{C(r_1)}M$中使下式
    \begin{equation}
        g_{ab}\left.\left(\frac{{\rm d}}{{\rm d}s}\right)^b\right|_{C(s_1)} v^a =0 =
        g_{ab}\left.\left(\frac{{\rm d}}{{\rm d}s}\right)^b\right|_{C(s_1)} w^a,
    \end{equation}
    成立的子空间是$2m-2$维的;而这个子空间线性同构于$\mathfrak{J}^{\bot}(C)$,
    故$\mathfrak{J}^{\bot}(C)$是$2m-2$维的线性空间.
\end{proof}

因黎曼曲率的反对称性,有
\begin{equation*}
    R^{a}_{\cdot bcd} \left(\frac{{\rm d}}{{\rm d}s}\right)^c
    \left(\frac{{\rm d}}{{\rm d}s}\right)^d \left(\frac{{\rm d}}{{\rm d}s}\right)^b =0, \quad
    R^{a}_{\cdot bcd} s\left(\frac{{\rm d}}{{\rm d}s}\right)^c
    s\left(\frac{{\rm d}}{{\rm d}s}\right)^d s\left(\frac{{\rm d}}{{\rm d}s}\right)^b =0   .
\end{equation*}
很明显测地线$C(s)$的切线切矢量$(\frac{{\rm d}}{{\rm d}s})^a$满足
\begin{equation*}
    \nabla_{\frac{\partial}{\partial s}} \nabla_{\frac{\partial}{\partial s}}
      \left(\frac{{\rm d}}{{\rm d}s}\right)^a =0, \qquad
    \nabla_{\frac{\partial}{\partial s}} \nabla_{\frac{\partial}{\partial s}}
      \left[s \left(\frac{{\rm d}}{{\rm d}s}\right)^a\right] =0  .
\end{equation*}
由上两式可知集合
\begin{equation}
    \mathfrak{J}^{\top}(C) \equiv {\rm Span}_{\mathbb{R}} 
    \left\{\left(\frac{{\rm d}}{{\rm d}s}\right)^a, \quad
    s\cdot \left(\frac{{\rm d}}{{\rm d}s}\right)^a\right\} .
\end{equation}
满足Jacobi方程\eqref{chgd:eqn_Jacobi},因此
两维空间$\mathfrak{J}^{\top}(C)$也是Jacobi场;称它们为切Jacobi场.
很明显$\mathfrak{J}^{\top}(C) \cap \mathfrak{J}^{\bot}(C) = \{{\bf 0}\}$,
故切Jacobi场与法Jacobi场交集为零.



\subsection{常曲率空间}
本节讨论一下常曲率空间(见\S \ref{chrg:sec_const-curvature})的Jacobi场.
设$C:[r_1,r_2]\to M$是$m$维广义黎曼流形$(M,g)$上的一条测地线,测地线参数取弧长$s$;
我们取一个沿测地线$C(s)$平行移动的
正交归一活动标架场$\{(e_i)^a\}$,并取$(e_1)^a=(\frac{{\rm d}}{{\rm d}s})^a|_{C(s)}$.

设常曲率空间$M$的曲率为$c$.利用$\{(e_i)^a\}$正交归一性,由式\eqref{chrg:eqn_const-curvature}知
\setlength{\mathindent}{0em}
\begin{align*}
    &R_{abcd} (e_i)^a (e_1)^b (e_1)^c (e_j)^d  =
      -c(g_{af}(e_i)^f (e_1)^a  g_{bh}(e_1)^h(e_j)^b
        -g_{bf}(e_i)^f (e_j)^a  g_{bh}(e_1)^h(e_1)^b )  \\
    &\Rightarrow\ R_{i11j} = -c (\eta_{i1}\eta_{1j}- \eta_{ij}\eta_{11} )
       = c \eta_{ij}\eta_{11}   .
\end{align*}\setlength{\mathindent}{2em}
设$J^a(s)$是沿$C(s)$定义的法Jacobi场,其展开式为$J^a=J^i (e_i)^a$.
结合上式,由Jacobi方程\eqref{chgd:eqn_Jacobi-componet}得
\begin{equation}
    \frac{{\rm d}^2 J_i(s)}{{\rm d} s^2} =R_{i11j}\bigl(C(s)\bigr) J^j(s)
      = c \eta_{ij}\eta_{11} J^j(s) =c \eta_{11} J_i(s) ,
      \ 1 < i \leqslant m.
\end{equation}
其中$\eta_{11}$是$+1$或$-1$.
考虑到我们使用的是沿$C(s)$平行移动的标架场,故{\kaishu 法场}$J^a$的第一分量
恒为零,即$J_1(s)\equiv 0$;所以上式中的指标$i$可以从$2$记起.上式的通解是
\begin{equation}\label{chgd:eqn_Jacobi-const-curvature}
    J_i(s) = \begin{cases}
        \lambda_i \sin(\sqrt{-c \eta_{11}} s)  + \mu_i \cos(\sqrt{-c \eta_{11}} s ),
            & \text{若 }\ c \eta_{11} <0 ; \\
        \lambda_i s  + \mu_i , & \text{若 }\ c \eta_{11} = 0; \\
        \lambda_i \sinh(\sqrt{ c \eta_{11}} s) + \mu_i \cosh(\sqrt{c \eta_{11}} s) ,
            & \text{若 }\ c \eta_{11}>0 .
    \end{cases}
\end{equation}
其中$\lambda_i, \mu_i$是积分常数.

%        \frac{\lambda_i}{\sqrt{-c \eta_{11}}} \sin (\sqrt{-c \eta_{11}} s)
%+ \frac{\mu_i}{\sqrt{-c \eta_{11}}} \cos(\sqrt{-c \eta_{11}} s )
%& \text{如果 }\ c \eta_{11} <0 , \\
%\lambda_i s  + \mu_i  & \text{如果 }\ c \eta_{11} = 0, \\
%\frac{\lambda_i}{\sqrt{c \eta_{11}}} \sinh (\sqrt{ c \eta_{11}} s)
%+ \frac{\mu_i}{\sqrt{c \eta_{11}}} \cosh(\sqrt{c \eta_{11}} s)
%& \text{如果 }\ c \eta_{11}>0 .


%\subsection{共轭点}\label{chgd:sec_conjugate-point}
%设广义黎曼流形$(M,g)$是完备的,
%由Hopf--Rinow定理可知指数映射$\exp_p$在切空间$T_pM$上处处由定义,
%再由定理\ref{chgd:thm_exp-homeomorphism}可知指数映射在原点附近是非退化的;
%但指数映射未必是处处非退化的.从而引出如下定义:
%\begin{definition}\label{chgd:def_conjugate-point}
%    设$m$维广义黎曼流形$(M,g)$是完备的.$p\in M,\ v^a\in T_pM$,如果
%    指数映射$\exp_{p}$在$v^a$处是退化的,即存在非零切矢量$w^a\in T_pM = T_v(T_pM)$使得
%    $(\exp_{p})_{*v}(w^a)=0$;那么称$q=\exp_p(v^a)\in M$是点$p$(沿
%    测地线$\exp_{p}(tv^a)$)的{\heiti 共轭点}.
%\end{definition}



\section*{小结}
本章主要参考了\parencite[Ch.3, Ch.5, Ch.6]{chen-li-2023-2ed-v1}.

%\parencite[Ch.4]{baizg-2004-irg}

本章除了\S\ref{chgd:sec_compelete}外,所有内容适用于正定、不定度规.

%%%%%%%%%%%%%%%%%%%%%%%%%%%%%%%%%%%%%%%%%%%%%%%%%%%%%%%%%%%%%%%%%%%%%%%%%%%%%%%%%%%
%\vspace{1em}
%要了解我们未给出的定理证明过程,或需要学习更多内容的读者,


%%%%%%%%%%%%%%%%%%%%%%%%%%%%%%%%%%%%%%%%%%%%%%%%%%%%%%%%%%%%%%%%%%%%%%%%%%%%%%%%%%%
\printbibliography[heading=subbibliography,title=第\ref{chgd}章参考文献]

\endinput
























%\index[physwords]{诱导联络}
%\subsection{诱导联络}\label{chgd:sec_induced-connection}
%为了更精准地讨论,需先引入诱导联络的概念;请先阅读\S\ref{chdm:sec_related}.
%先看一下目前的Levi-Civita联络有何不足.设$\gamma(t)$是流形$M$中的一条曲线,$X^a$是
%沿$\gamma$的一条光滑切矢量场(注意,它未必平行于$\gamma$的切线;$X^a$是切丛$TM$中的矢量,
%只是在$\gamma$上取值而已);$T^a=(\frac{\partial }{\partial t})^a$是$\gamma(t)$的真正的切线切矢量.
%在讨论$\gamma$的平行移动或者$\gamma$是测地线时,我们需要定义$T^b\nabla_b X^a$;
%此时需要将$\gamma$上的$X^a$延拓至整个$M$上.
%如果$\gamma$有自交点,那么在交点处出现多值性;如果$T^a$在某点为零,那么联络定义也有问题.
%而下面定义的诱导联络大体可以避免这种缺陷.
%
%设$N$是$n$维光滑流形;$(M,g)$是$m$维广义黎曼流形,
%$\nabla_a$是其Levi-Civita联络;存在光滑\uwave{浸入}映射$\phi:N\to M$.
%如果$\forall p\in N$,$X^a\in T_{\phi(p)}M$,则称$X^a$是{\heiti $\phi$上的矢量场}.
%
%前面讨论的曲线$\gamma(t)$自然是一条浸入曲线,可看作此定义中的流形$N$.
%如果$u^a$是$N$上的矢量场,一般说来$\phi_{*}u^a$可能不是$M$上的矢量场;
%比如$\phi$不是满的(注:$\phi$是浸入,必是局部单射),那么推前之后,在$M$上不是每一点都有一个矢量场.
%{\kaishu 但$\phi_{*}u^a$是$\phi$上的矢量场,因为映射$\phi:N\to \phi(N)$是(局部)双射.}
%可见有必要引入“$\phi$上的矢量场”这一概念.
%矢量场$u^a$在$N$中原来只有$n(<m)$个分量;在浸入$M$后,矢量场$\phi_{*}u^a$可以假设
%前$n$个分量不变{\footnote{因为是浸入,所以可以认为$N$中的$n$个坐标与流形$M$中的前$n$个坐标
%        完全重合;见定理\ref{chdm:thm_immerse-1}.}},
%后$m-n$个分量为零;这样$\phi_{*}u^a$就有了$m$个分量,便是$TM$中矢量了.
%如果$TM$有矢量$Y^a$在点${\phi(p)}\in M$有定义,但后$m-n$个分量中有非零值,那么$Y^a$也是$\phi$上的矢量场.
%$\phi$上的矢量场不一定都由$TN$推前得到.
%
%设$\{(e_i)^a, 1\leqslant i \leqslant m\}$是在点$\phi(p)\in U \subset M$邻域$U$中的局部标架场.
%为了讨论诱导联络定义与局部标架场无关,我们同时设点$\phi(p)\in V$有
%从$N$带来的局部标架场$\{(\tilde{e}_\alpha)^a\}$,并且$\phi(p)\in U\cap V \neq \varnothing$;
%映射$\phi$只带来$n$个局部标架,我们任意补充$m-n$个与前$n$个正交的标架场;
%这样$p$点局部邻域$V$也是$M$上的一个开邻域.
%再设存在$m^2$个光滑函数$A_\alpha^i\in C^\infty(U\cap V)$使得
%$(\tilde{e}_\alpha)^a= A_\alpha^i (e_i)^a {\color{red}\Leftrightarrow} B^\alpha_i  (\tilde{e}_\alpha)^a= (e_i)^a$.
%那么$\phi$上的矢量场$X^a\in T_{\phi(p)}(U\cap V) \subset T_{\phi(p)}M$可展开为
%\begin{equation}
%    X^a = \xi^i_{\phi(p)} \cdot (e_i)^a|_{\phi(p)}
%    = \eta^\alpha_{\phi(p)} \cdot (\tilde{e}_\alpha)^a|_{\phi(p)}
%    = \bigl(\eta^\alpha_{\phi(p)} \cdot A_\alpha^i \bigr) \cdot (e_i)^a|_{\phi(p)} .
%\end{equation}
%
%定义沿$v^a\in T_pN$的$\phi$上矢量场$X^a\in T_{\phi(p)}M$的{\heiti 诱导联络}
%$\tilde{\nabla}_{v} X^a \in  T_{\phi(p)}M$为
%\begin{equation}\label{chgd:eqn_induced-connection}
%    \tilde{\nabla}_{v} X^a \overset{def}{=} \bigl(\phi_{*}v(\xi^i)\bigr) (e_i)^a|_{\phi(p)}
%    + \xi^i_{\phi(p)} \nabla_{\phi_* v} \bigl( (e_i)^a|_{\phi(p)}\bigr)
%    = {\nabla}_{\phi_{*}v} X^a .
%\end{equation}
%先证明这个定义与局部标架场$\{(e_i)^a\}$选取无关,为简洁起见省略角标${\phi(p)}$.
%\begin{align*}
%    &\bigl(\phi_{*}v(\xi^i)\bigr) (e_i)^a|_{\phi(p)}
%    + \xi^i_{\phi(p)} \nabla_{\phi_* v} \bigl( (e_i)^a|_{\phi(p)}\bigr) \\
%    =& v(\eta^\alpha A_\alpha^i ) (\tilde{e}_\beta)^a B^\beta_i
%    + \eta^\alpha A_\alpha^i \nabla_{\phi_* v} \bigl( (\tilde{e}_\beta)^a B^\beta_i  \bigr) \\
%    =& \eta^\alpha v( A_\alpha^i ) (\tilde{e}_\beta)^a B^\beta_i
%    + A_\alpha^i v(\eta^\alpha  ) (\tilde{e}_\beta)^a B^\beta_i
%    + \eta^\alpha A_\alpha^i B^\beta_i \nabla_{\phi_* v} \bigl( (\tilde{e}_\beta)^a   \bigr)
%    + \eta^\alpha A_\alpha^i (\tilde{e}_\beta)^a \nabla_{\phi_* v} \bigl(  B^\beta_i  \bigr) \\
%    =&  v(\eta^\alpha  ) (\tilde{e}_\alpha)^a
%    + \eta^\alpha  \nabla_{\phi_* v}  (\tilde{e}_\alpha)^a
%    + \eta^\alpha  (\tilde{e}_\beta)^a \left(v( A_\alpha^i )\cdot  B^\beta_i
%    +  A_\alpha^i  v \bigl(  B^\beta_i  \bigr) \right) \\
%    =&  v(\eta^\alpha  ) (\tilde{e}_\alpha)^a
%    + \eta^\alpha  \nabla_{\phi_* v}  (\tilde{e}_\alpha)^a
%    + \eta^\alpha  (\tilde{e}_\beta)^a \left(v( A_\alpha^i B^\beta_i) \right) \\
%    =&  v(\eta^\alpha  ) (\tilde{e}_\alpha)^a
%    + \eta^\alpha  \nabla_{\phi_* v}  (\tilde{e}_\alpha)^a   .
%\end{align*}
%注意$A^T=B^{-1}$.这便证明了定义与标架场选取无关.
%%在上面计算过程中用到了这些关系式,但都省略了这些繁杂的变换关系.
%
%%$\bigl(v(\xi)\bigr)_{\phi(p)}$应理解为${\phi_* v}(\xi)=v(\xi\circ \phi)$,以及
%
%直接带入即可验证式\eqref{chgd:eqn_induced-connection}满足
%定义\ref{chccr:def_connection}中的前三个条件(将$K$和$L$看成矢量场或标量场);
%最后两条看作附加原则;因此式\eqref{chgd:eqn_induced-connection}符合联络定义.
%
%
%
%再列出几条关于诱导联络的性质.
%\begin{align}
%    v(X_a Y^a) =& (\tilde{\nabla}_{v} X_a) Y^a + X_a (\tilde{\nabla}_{v}Y^a) ; \qquad
%    \forall X^a,Y^a \in T_{\phi(p)}M, \quad \forall v^a\in T_pN . \label{chgd:eqn_induce-con-compatibility} \\
%    \phi_{*}[X,Y]^a =& \tilde{\nabla}_{X}(\phi_{*}Y^a)-\tilde{\nabla}_{Y}(\phi_{*}X^a) ; \qquad
%    \forall X^a,Y^a \in T_pN . \label{chgd:eqn_induce-con-NoTorsion}
%\end{align}
%式\eqref{chgd:eqn_induce-con-compatibility}说明度规与诱导联络是相容的,请比对式\eqref{chgd:eqn_connection-compatibility};
%式\eqref{chgd:eqn_induce-con-NoTorsion}说明诱导联络是无挠的,请比对式\eqref{chccr:eqn_Ttorsion}.
%
%
%先证明式\eqref{chgd:eqn_induce-con-compatibility}.
%注意$v(X_a Y^a)$应理解为${\phi_* v}(X_a Y^a)$,故  %因$\phi$是浸入,
%\begin{align*}
%    (\phi_{*}v)(X_a Y^a) =&  \nabla_{\phi_{*}v}(X_a Y^a)
%    = Y^a \nabla_{\phi_{*}v} X_a  + X_a \nabla_{\phi_{*}v} Y^a
%    =(\tilde{\nabla}_{v} X_a) Y^a + X_a (\tilde{\nabla}_{v}Y^a).
%\end{align*}
%再证明式\eqref{chgd:eqn_induce-con-NoTorsion}.
%设$T_pN$处的局部标架场是$\{(E_i)^a\}(1\leqslant i \leqslant n)$;因$\phi$是浸入,故基矢推前后在$T_{\phi(p)}M$处
%基矢场$\{(e_\alpha)^a\}$展开表示为$(E_i)^a=c_i^\alpha (e_\alpha)^a$,其中$c_i^\alpha\in C^\infty(M)$.
%有了这些准备,可得$\phi_{*}X^a = \phi_{*}(X^i (E_i)^a)= X^i c_i^\alpha (e_\alpha)^a$,
%和$\phi_{*}Y^a = \phi_{*}(Y^j (E_j)^a)= Y^j c_j^\beta (e_\beta)^a$.
%\begin{equation}
%    \begin{aligned}
%        &\tilde{\nabla}_{X}(\phi_{*}Y^a)-\tilde{\nabla}_{Y}(\phi_{*}X^a) =
%        \bigl(X^i c_i^\alpha e_\alpha(Y^j c_j^\pi)\bigr) (e_\pi)^a
%        + Y^j c_j^\pi \bigl( X^i c_i^\alpha (e_\alpha)^b \bigr)\nabla_{b} (e_\pi)^a  \\
%        &\quad - \bigl(Y^i c_i^\alpha e_\alpha(X^j c_j^\pi)\bigr) (e_\pi)^a
%        - X^j c_j^\pi \bigl( Y^i c_i^\alpha (e_\alpha)^b \bigr)\nabla_{b} (e_\pi)^a \\
%        &=\Bigl(X^i c_i^\alpha e_\alpha(Y^j c_j^\sigma)
%        + Y^j c_j^\pi  X^i c_i^\alpha \gamma_{\pi\alpha}^\sigma
%        - Y^i c_i^\alpha e_\alpha(X^j c_j^\sigma)
%        - X^j c_j^\pi  Y^i c_i^\alpha \gamma_{\pi\alpha}^\sigma  \Bigr) (e_\sigma)^a  \\
%        &=\Bigl(X^i c_i^\alpha e_\alpha(Y^j c_j^\sigma)
%        - Y^i c_i^\alpha e_\alpha(X^j c_j^\sigma)   + X^i c_i^\alpha Y^j c_j^\pi
%        \bigl(\gamma_{\pi\alpha}^\sigma -  \gamma_{\alpha\pi}^\sigma\bigr)   \Bigr) (e_\sigma)^a  \\
%        &\xlongequal{\ref{chrg:eqn_XYcommutator-Ebase}}
%        \bigl[ \phi_{*}X, \phi_{*}Y \bigr]^a 
%        \xlongequal[\ref{chdm:thm_push-Poisson-related}]{\text{定理}} \phi_{*}[X,Y]^a .
%    \end{aligned}
%\end{equation}
