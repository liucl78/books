% !TeX encoding = UTF-8
% 2023.08.06

\chapter{零模}\label{chnull}

在微分几何部分,其它各章节几乎不涉及零模问题;
我们单列一章来讨论这个问题.
首先,讲述零模线性子空间,以及零模超曲面.
其次,介绍了Newman--Penrose型式,这是处理零模问题的有效方法.
接着,给出了Weyl张量的Petrov分类,以及类光曲线.
最后,证明了Goldberg--Sachs定理.

\index[physwords]{固有广义欧氏空间}
\index[physwords]{线性函数!不定对称双线性函数}

\section{广义欧氏子空间}\label{chnull:sec_indef-algebra}


我们将继续\S\ref{chmla:sec_indefbif}和\S\ref{chdm:sec_Euclidean-space}的探讨.
当广义欧几里得空间$V$的度规特征值至少含有一个$-1$,同时至少含有
一个$+1$时,称之为{\heiti 固有广义欧几里得空间}.




与非退化度规定义\ref{chmla:def_nondegenerate-metric}相反,
$m$维实线性空间$V$上的退化(degenerate)、对称双线性函数$g$是指:
存在非零矢量$\xi \in V$,使得$g(\xi,v)=0$,$\forall v \in V$.
与之相随的是退化空间的定义,见\ref{chmla:def_degenerate}.

零模流形主要研究退化对称双线性型式.当度规退化时,我们将尽量避免使用度规这个术语,
而使用退化对称双线性型式(degenerate symmetric bilinear form),简称{\heiti 退型}.

\index[physwords]{退型}

值得注意的是,$\mathbb{R}^m$上的非退化对称度规可以在$\mathbb{R}^m$的子空间上诱导出
非退化度规或者{\kaishu 退型}.
我们将会处理这两类子空间,故当不知道度规是否退化时,使用$g$;当度规非退化时,则尽量用$\eta$.

\begin{definition}
    设有$m$维实线性空间$(W,g)$,$g$可能退化,也可能非退化;它有如下子空间:
    \begin{equation}
        {\rm Rad} W= \{\xi^a \in W \ |\  g_{ab} \xi^a v^b=0,\ \forall v^b\in W \} .
    \end{equation}
    我们称${\rm Rad} W$是$W$在$g$下的{\heiti 根空间}(Radical Space);
    空间${\rm Rad} W$的维数称为度规$g$的{\heiti 零度}(nullity degree).
\end{definition}

\index[physwords]{根空间}

当空间$W$的零度是〇时,空间$W$是非退化的;当零度大于〇时,空间$W$是退化的;反之亦然.
$g$是否退化就是看式\eqref{chmla:eqn_gmetric}中是否有零特征值.


设$(W,g)$是一个实$m$维线性空间,${\rm Rad} W$是它的根空间;
那么,$W$的子空间可能不退化.为了支持这一论断,我们证明了下面的一般结果.

\begin{proposition}\label{chnull:thm_screen-radical}
    设有$m$维线性空间$(W,g)$,$W$的零度是$r$,并且$r<m$.
    那么根空间${\rm Rad} W$的正交补空间是非退化的.
\end{proposition}
\begin{proof}
    令$SW$是根空间${\rm Rad} W$在$W$中的正交补空间,即我们作了如下分解:
    \begin{equation}\label{chnull:eqn_RWSW}
        W = {\rm Rad} W \dotplus SW .
    \end{equation}
    其中$\dotplus$是正交直和,即有${\rm Rad} W \cap SW=\{0\}$,
    并且${\rm Rad} W $和$SW$中元素相互正交.
    用反证法证明$SW$是非退化的.
    假设存在非零元素$\zeta \in SW$使得$g_{ab} \zeta^a v^b=0$,$\forall v^b\in SW$;
    式\eqref{chnull:eqn_RWSW}意味着$g_{ab}\zeta^a \xi^b=0$,$\forall \xi^b \in {\rm Rad}W$;
    因此,$\zeta^a \in {\rm Rad}W$.这违背了${\rm Rad} W $、$SW$互补关系;
    最终,$SW$是非退化的.
\end{proof}

\begin{definition}
    实线性空间$(W,g)$的根空间${\rm Rad} W$的正交补空间$SW$称为$W$的{\heiti 根补空间}(Screen Space).
\end{definition}

\index[physwords]{根补空间}

由于$SW$对于$g$来说是非退化的,故它是广义欧几里得空间;
它的正交基是记为$\{(e_{r+1})^a,\cdots,(e_m)^a\}$.
我们再设${\rm Rad}W$的基矢是$\{(e_1)^a,\cdots,(e_r)^a\}$,
则空间$W$的基矢为:$\mathcal{B}=\{(e_1)^a,\cdots,(e_r)^a,(e_{r+1})^a,\cdots,(e_m)^a\}$.
考虑到任意的${\rm Rad}W$矢量都正交于$W$,故可以得出$g$在基矢$\mathcal{B}$下的矩阵是:
\begin{equation}
    g=\begin{pmatrix}
        0_{r,r} & 0_{r,m-r} \\ 0_{m-r,r} &  \eta_{ij}
    \end{pmatrix},\quad
    \eta_{ij}=\pm \delta_{ij},\quad  r< i,j \leqslant m .
\end{equation}



\begin{proposition}\label{chnull:thm_RadWWP}
    设$(V,g)$是$m$维广义欧几里得空间,$W$是$V$的子空间;则有    
    \begin{equation}
        {\rm Rad}W={\rm Rad}W^\perp=W\cap W^\perp ,\quad
        W^\perp \text{是垂直子空间\ref{chmla:def_oc}}.
    \end{equation}
\end{proposition}
\begin{proof}
    令$v^a\in W \cap W^\perp\subset W^\perp$,则有$g_{ab}(v^a,w^b)=0,\, \forall w^b\in W$,
    这就是说$v^a\in {\rm Rad}W$.反之,$\forall  v^a\in {\rm Rad}W\subset W$,
    可以得到$g_{ab}(v^a,w^b)=0,\, \forall w^b\in W$,这意味着$v^a\in W\cap W^\perp$.
    因此${\rm Rad}W=W\cap W^\perp $.应用命题\ref{chmla:thm_WcW}中的(2)可得
    ${\rm Rad}W={\rm Rad}W^\perp=W\cap W^\perp $.
\end{proof}

写到这里,读者应能看出,只有在退型情形下,${\rm Rad}W$才能包含非零矢量.
若$V$的度规是非退化、正定的,则$V$及其子空间不会有退化情形,根空间是平庸的(只含零矢量).
若空间$V$的度规$g$非退化但不定,则$V$本身的根空间也是平庸的;但$V$的子空间$W$可能
是退化的,一般情形下$W$包含了$V$中的部分零模矢量,不是全部,可参见例\ref{chnull:exm_R41}.


\begin{proposition}\label{chnull:thm_maxnull}
    设$g$是$m$维实矢量空间$V$上的一个非退化、不定度规,它的负特征值个数是$q$.
    则存在$V$的子空间$W$,其维数是$\min(q, m - q)$,使得$g|_W=0$;
    并且没有维数更大的子空间$W'$使得$g|_{W'}=0$.
\end{proposition}
\begin{proof}
    设$V$有正交归一基矢$\{(e_i)^a\}$,注意$g={\rm diag}(-\cdots -+\cdots +)$在$V$上是非退化的,
    它的负特征值个数是$q$,正特征值个数是$m-q$.
    
    先设$q<m-q$.我们构建如下空间:
    \begin{equation}
        \overline{W}\equiv {\rm Span}\left\{(u_1)^a= (e_1)^a+(e_{q+1})^a,
        \cdots,(u_q)^a= (e_q)^a+(e_{2q})^a\right\} .
    \end{equation}
    容易验证$g|_{\overline{W}}=0$;
    很明显$\overline{W}$的维数是$q$.
    
    对$V$中满足$g_{ab}x^a (u_j)^b=0,\, j=1,\cdots,q$的零模矢量$x^a=x^i(e_i)^a$而言,
    它的分量满足$x^1=x^{q+1},\cdots,x^q=x^{2q}$.因为$\{(e_1)^a,\cdots,(e_q)^a\}$和
    $\{(e_{q+1})^a,\cdots,(e_m)^a\}$分别是类时、类空基矢量,再从$x^a$是零模矢量可以
    得到$x^{2q+1}=\cdots=x^m=0$.
    最终$x^a$可以表示成$\{(u_1)^a,\cdots,(u_q)^a\}$的常系数线性组合,故$x^a\in \overline{W}$.
    由此可见不存在维数更大的子空间$W'\supset \overline{W}$使得$g|_{W'}=0$.
    
    仿照上述可证:当$q\geqslant m-q$时,命题也成立.
\end{proof}
    
\begin{example}\label{chnull:exm_Em-bases}
给固有欧氏空间$\mathbb{R}^m_q$构建包含零模矢量的基矢组.
\end{example}

设$\mathbb{R}^m_q$有正交归一基矢$\{(e_i)^a\}$,度规$g={\rm diag}(-\cdots -+\cdots +)$;
我们借用这些矢量构建包含零模矢量的基矢组.

\fbox{甲}$q< m-q$.构建如下矢量(其中$i=1,\cdots,q$)
\begin{equation}\label{chnull:eqn_I-nulls}
    (l_i)^a \overset{def}{=} \frac{1}{\sqrt{2}} \bigl[ (e_{i})^a + (e_{q+i})^a \bigr] ,\quad  
    (n_i)^a \overset{def}{=} \frac{1}{\sqrt{2}} \bigl[ (e_{i})^a - (e_{q+i})^a \bigr] .
\end{equation}
容易验证$g_{ab}(l_i)^a (l_j)^b=0=g_{ab}(n_i)^a (n_j)^b$以
及$g_{ab}(l_i)^a (n_j)^b=-\delta_{ij}$;其中$i,j=1,\cdots,q$.
因此可以看出$\{(l_1)^a,\cdots,(l_q)^a,(n_1)^a,\cdots,(n_q)^a,(e_{2q+1})^a,\cdots,(e_m)^a\}$是
空间$\mathbb{R}^m_q$的一组基矢量,它包含$2q$个零模矢量,以及$m-2q$个类空矢量.
在这组基矢下,度规矩阵是:
\begin{equation}
    g=\begin{pmatrix}
        0 & -I_q &0 \\
        -I_q & 0 &0 \\
        0& 0 & I_{m-2q}
    \end{pmatrix}.
\end{equation}

\fbox{乙}$q> m-q$.构建如下矢量
\begin{equation*}%\label{chnull:eqn_II-nulls}
    (l_j)^a \overset{def}{=} \frac{1}{\sqrt{2}} \bigl[ (e_{j})^a + (e_{q+j})^a \bigr] ,\quad  
    (n_j)^a \overset{def}{=} \frac{1}{\sqrt{2}} \bigl[ (e_{j})^a - (e_{q+j})^a \bigr] ;\quad j=1,\cdots,m-q.
\end{equation*}
可验证$g_{ab}(l_i)^a (l_j)^b=0=g_{ab}(n_i)^a (n_j)^b$以
及$g_{ab}(l_i)^a (n_j)^b=-\delta_{ij}$;其中$i,j=1,\cdots,m-q$.
因此可以看出类光矢量$(l_1)^a,\cdots,(l_{m-q})^a$、$ (n_1)^a,\cdots,(n_{m-q})^a$和
类空矢量$(e_{2(m-q)+1})^a,\cdots,(e_m)^a$的组合是
空间$\mathbb{R}^m_q$的一组基矢量,它包含$2(m-q)$个零模矢量,以及$2q-m$个类时矢量.

\fbox{丙}$2q= m$.可为空间$V$构建
$\{(l_1)^a,\cdots,(l_q)^a,(n_1)^a,\cdots,(n_q)^a\}$个零模矢量组成的基矢组,
可参见式\eqref{chnull:eqn_I-nulls}.

虽然上述三类基矢均包含零模矢量,但度规在这些基矢下是非退化的.
\qed

\begin{example}\label{chnull:exm_R41}
    给定四维闵氏时空$(\mathbb{R}^4_1,\eta)$,$\eta=(-+++)$.
    利用上面三种分类,分情况讨论它的包含类光矢量的子空间$W$.
\end{example}

根据\fbox{甲}可设闵氏时空基矢为$\{l^a,n^a, u^a, v^a\}$,其中$l^a,n^a$是类光基矢,
$u^a, v^a$是类空基矢;它们都是实基矢.
根据命题\ref{chnull:thm_maxnull}可知四维闵氏时空最大退化子空间维数是$1$.

{\noindent\heiti \bfseries (1)} 设$W$是1维的.可将$W$的基矢选为$l^a$,
此时$W$是退化的;$W^\perp ={\rm Span}\{l^a,u^a,v^a\}$;可将$l^a$换成$n^a$.
此时${\rm Rad}W=W$,根补空间$SW={\rm Span}\{0\}$.

{\noindent\heiti \bfseries (2)} 设$W$是2维的.那么$W$的基矢可能是$l^a$和$n^a$;
也可能是$l^a$、$n^a$两者之一与$u^a$、$v^a$两者之一;分情况讨论.

(i) $W={\rm Span}\{l^a,n^a\}$.$W^\perp ={\rm Span}\{u^a,v^a\}$.
${\rm Rad}W={\rm Span}\{0\}$,$SW=W$.

(ii) $W={\rm Span}\{l^a,u^a\}$.$W^\perp ={\rm Span}\{l^a,v^a\}$.
${\rm Rad}W={\rm Span}\{l^a\}$,$SW={\rm Span}\{u^a\}$.
可将此种情下的$l^a$换成$n^a$;以及$u^a$换成$v^a$.

{\noindent\heiti \bfseries (3)} 设$W$是3维的.
那么$W$的基矢可能是$l^a$、$n^a$两者之一,再加上$u^a$、$v^a$;
也可能是$l^a$和$n^a$,再加上$u^a$、$ v^a$两者之一;分情况讨论.

(i) $W={\rm Span}\{l^a,u^a, v^a\}$.$W^\perp ={\rm Span}\{l^a\}$.
${\rm Rad}W={\rm Span}\{l^a\}$,$SW={\rm Span}\{u^a,v^a\}$.
可将$l^a$换成$n^a$.

(ii) $W={\rm Span}\{l^a,n^a,u^a\}$.$W^\perp ={\rm Span}\{v^a\}$.
${\rm Rad}W={\rm Span}\{0\}$,$SW=W$.可将$u^a$换成$v^a$.

{\noindent\heiti \bfseries (4)} 设$W$是4维的.$W^\perp ={\rm Span}\{0\}$.
${\rm Rad}W={\rm Span}\{0\}$,$SW=W$.
\qed

%我们会在\S\ref{chnull:sec_NP1}专门介绍一种特殊的类光基矢场:Newman--Penrose型式.





\index[physwords]{超曲面!类光超曲面}

\section{类光超曲面} \label{chnull:sec_hypersurface-null}

设有四维广义黎曼流形$(M,g)$,其度规是Lorentz型度规($(-+++)$);它有类光超曲面$\Sigma$.
本节大部分结论可以毫无困难地推广到广义Lorentz度规.

首先,类光超曲面的$\Sigma$法矢量${n}^{a}$自然是法丛$T^\bot\Sigma$中的矢量;
其次它还是零模的,即${n}^{a}{n}_{a}=0$,这说明$n^a$还属于切丛$T\Sigma$.
这种双重身份,使得矢量场$n^a$即切于又法于超曲面$\Sigma$;这在非零模情形是不可能的!

\subsection{法矢量}
在\S\ref{chsm:sec_Phi-n}中我们用标量函数$\Phi=0$来描述超曲面,这种方式同样适用于类光超曲面.
但此时$g^{ab}\nabla_a \Phi\nabla_b\Phi=0$,故我们没有类似于
式\eqref{chsm:eqn_unit-normal-Phi}的描述法矢量的方式.
类光超曲面的法矢量自然不能归一化到$\pm 1$,也就是说它没有单位法矢量的概念.
我们令
\begin{equation}\label{chnull:eqn_unit-normal-Phi-null}
    n^a \equiv \nabla^a \Phi.
\end{equation}
把它当成类光超曲面的法矢量.


\subsection{诱导退型}
我们已经知道在$\mathbb{R}^m_1$中的零模子空间上度规是退化的;
在一般流形上,零模子空间上的度规也是退化的,下面验证一下.
设四维闵氏流形$(M,g)$上有类光超曲面$\Sigma$,其零模法矢量$n^a$具有双重身份,
它既是$\Sigma$的法矢量又是$\Sigma$的切矢量.首先,我们把$n^a$看成法矢量,
那么$\forall u^a \in \mathfrak{X}(\Sigma)$,有
\begin{equation}\label{chnull:eqn_tmp-im01}
    g_{ab} n^a u^b = 0.
\end{equation}
根据法矢量与切矢量正交,可得此式为零.然而,$n^a$的另一身份是:$n^a \in \mathfrak{X}(\Sigma)$.
当用这个身份来看时,诱导度规是(注意$\imath_* u^a = u^a, \  \forall u^a \in \mathfrak{X}(\Sigma)$)
\begin{equation}
    h_{ab} n^a u^b = (\imath^*g_{ab}) n^a u^b = g_{ab} (\imath_* n^a) (\imath_*u^b)
    =g_{ab} n^a u^b \xlongequal{\ref{chnull:eqn_tmp-im01}} 0.
\end{equation}
可见诱导退型$h_{ab}$是退化的,它不能称为真正的度规.

%依据文献\parencite{Duggal-Bejancu-1996}\S 4.1-\S4.2内容可作如下分解:
%\begin{equation}\label{chnull:eqn_decomp}
%    TM = S\oplus (T^\perp\Sigma + E)=T\Sigma + E,\quad
%    T\Sigma=S\oplus T^\perp\Sigma .
%\end{equation}
%
%\begin{equation}
%    T\Sigma= \bigcup_{p\in \Sigma} T_p \Sigma,\quad
%    T^\perp \Sigma= \bigcup_{p\in \Sigma} T_p^\perp \Sigma
%    = \bigcup_{p\in \Sigma}\{n^a \}_p .
%\end{equation}
%上式说明零模空间$T_p^\perp \Sigma$是$T_p\Sigma$的退化子空间,
%也就是说${\rm Rad}(T_p\Sigma) = T_p^\perp \Sigma$.
%由命题\ref{chnull:thm_screen-radical}可令$S$是$T^\perp \Sigma$在$T\Sigma$中
%的类空补空间,称为{\heiti 根补分布}(screen distribution).
%由于$S$是非退化的,故有
%\begin{equation}\label{chnull:eqn_TMSS}
%    TM = S\oplus S^\perp, \quad S\cap S^\perp = \{0\},\quad
%    S^\perp \, \text{是$S$的正交补空间}.
%\end{equation}
%不难看出,$S$、$S^\perp$都是$2$维空间.
%采用上面的记号,有如下定理:
%\begin{theorem}\label{chnull:thm_lnX}
%    在四维闵氏流形$(M,g)$上有类光超曲面$(\Sigma,h,S)$.
%    则在$\Sigma$的任意局部开邻域$U\subset \Sigma$上,
%    存在唯一类光分布$E\equiv \cup_{p\in \Sigma}E_p$,
%    且$E$是由类光矢量场$l^a$张成;同时有
%    \begin{equation}\label{chnull:eqn_lnX}
%        g_{ab}n^a l^b =-1, \quad  g_{ab}l^a l^b  =0=g_{ab}X^a l^b ,
%        \quad \forall X^a \in \mathfrak{X}(S|_U).
%    \end{equation}
%\end{theorem}
%\begin{proof}
%    由于$T^\perp \Sigma\subset S^\perp$,令$E$是$T^\perp \Sigma$在$S^\perp$中的补空间,它是一维的.
%    因$S^\perp$的维数是2,故在$E$中存在一个非零矢量场$W^a$使得$g_{ab}W^a n^b\neq 0$;
%    否则,从式\eqref{chnull:eqn_TMSS}可知度规$g$在$TU$上是退化的.将$l^a$定义为:
%    \begin{equation}\label{chnull:eqn_l}
%        l^a \equiv \frac{1}{ W_b n^b} \left( n^a \frac{W_b W^b}{2 W_c n^c}-W^a\right) .
%    \end{equation}
%    经过直接计算可以验证$l^a$是零模的,即
%    \begin{align*}
%        l^a l_a =& \frac{1}{ W_b n^b} \left( n^a \frac{W_b W^b}{2 W_c n^c}-W^a\right) 
%        \frac{1}{ W_d n^d} \left( n_a \frac{W_d W^d}{2 W_e n^e}-W_a\right) \\
%        =&\frac{1}{(W_b n^b)^2 }\left(W^aW_a - n^a W_a\frac{W_b W^b}{2 W_c n^c}
%        -W^a n_a \frac{W_d W^d}{2 W_e n^e} + n^a n_a \left(\frac{W_d W^d}{2 W_e n^e}\right)^2\right) \\
%        \xlongequal{n^a n_a=0}&\frac{1}{(W_b n^b)^2 }\left(W^aW_a - W_b W^b\right) =0 .
%    \end{align*}
%    继续验证式\eqref{chnull:eqn_lnX}中剩下的两式.
%    \begin{align*}
%        n_a l^a = \frac{1}{ W_b n^b} \left( n_a n^a \frac{W_b W^b}{2 W_c n^c}-n_aW^a\right) 
%        = -\frac{1}{ W_b n^b} \left(n_a W^a\right) = -1 . 
%    \end{align*}
%    $W^a$、$n^a$都是$\mathfrak{X}(S^\perp)$中的切矢量,它们必然
%    与$\mathfrak{X}(S|_U)$中的切矢量正交.
%    
%    下面证明唯一性.
%    我们考虑另一个局部坐标邻域$U^*\subset M$使得$U\cap U^* \neq \varnothing$;
%    不难得到在$U\cap U^*$中有$W^{*a}=\alpha W^a$.
%    令$l^{*a}$表示在$U^*$上由$W^{*a}$诱导
%    出来的矢量场,它的具体表达式仍由式\eqref{chnull:eqn_l}给出.
%    容易得到$\alpha l^a= l^{*a}$;此式表明
%    两个不同邻域诱导出的切矢量场是$C^\infty(U\cap U^*)$-相关的;
%    作为分布而言,$l^a$是唯一的,与邻域的选取无关.
%\end{proof}
%
%由定理\ref{chnull:thm_lnX}和式\eqref{chnull:eqn_TMSS}、\eqref{chnull:eqn_l}可以有如下分解

下面叙述一下关于退型的一些内容;
我们着手在四维闵氏流形$(M,g)$的类光超曲面$(\Sigma,h)$上建立活动标架场.
根据文献\parencite[p.90-92]{Duggal-Bejancu-1996}论述可知,
在的$M$的局部开邻域$U$上一定存在基矢:
\begin{equation}\label{chnull:eqn_bases}
    \mathcal{B} \equiv \left\{ l^a, \  n^a,
    \   (E_1)^a,\  (E_2)^a \right\} , \qquad
    l^a, n^a \, \text{类光,} (E_1)^a, (E_2)^a\, \text{类空}
\end{equation}
使得类光超曲面$\Sigma$切丛由
\begin{equation}\label{chnull:eqn_SigmaAll}
    T\Sigma={\rm Span}_{\infty}\left\{ n^a, \   (E_1)^a,\   (E_2)^a \right\}
\end{equation}
张成;其中$n^a$是$\Sigma$的类光法矢量.
四维闵氏流形$M$的切丛由
\begin{equation}
    TM={\rm Span}_{\infty}\left\{ l^a,  \   n^a, \   (E_1)^a,\   (E_2)^a \right\}
\end{equation}
张成.除此以外,这四个基矢还可以张成如下子空间    %\setlength{\mathindent}{0em}
\begin{equation*}
    T^\perp\Sigma={\rm Span}_{\infty} \left\{ n^a \right\},\
    S={\rm Span}_{\infty}\left\{  (E_1)^a,  (E_2)^a \right\},\
    S^\perp={\rm Span}_{\infty}\left\{ n^a,  l^a \right\} . 
\end{equation*} %\setlength{\mathindent}{2em}
$\forall p\in \Sigma \subset M$,每个$T_p\Sigma$都是$T_pM$的类光子空间.不难得到
\begin{equation}
    T_p\Sigma \cap T_p^\perp \Sigma = T_p^\perp\Sigma,\quad 
    {\rm dim}\bigl(T^\perp \Sigma\bigr)_p =1,
    \quad {\rm dim}\bigl(T \Sigma\bigr)_p =3.
\end{equation}



在标架场\eqref{chnull:eqn_bases}下,流形$M$的度规$g$矩阵为
\begin{equation}\label{chnull:eqn_gab}
    g_{\mu\nu}=\begin{pmatrix}
        0& -1&0 &0 \\ -1& 0&0 &0 \\ 
        0&0& 1&0 \\0&0& 0 & 1
    \end{pmatrix}.
\end{equation}

式\eqref{chnull:eqn_bases}与Newman--Penrose型式\eqref{chnull:eqn_np-metric}非常相近;
这里的基矢全部是实的,NP型式中有两个是复基矢.


%存在局部坐标系$\{x^0,x^1,x^2\}$使得$(\frac{\partial}{\partial x^0})^a \in T^\perp\Sigma|_U$.
%我们可以取$\{(\frac{\partial}{\partial x^0})^a,(\frac{\partial}{\partial x^1})^a,
%(\frac{\partial}{\partial x^2})^a\}$为$\Sigma$的自然标架场,

在标架场下\eqref{chnull:eqn_SigmaAll}退型$h_{ab}$的表示矩阵为
\begin{equation}\label{chnull:eqn_hij}
    h=\begin{pmatrix}
        0&0&0 \\ 0& 1 &0 \\0& 0 & 1
    \end{pmatrix}.
\end{equation}




%\begin{example}
%    考虑$\mathbb{R}^4_1$中的单位伪球面$S^3_1$.
%\end{example}
%伪球面定义见\S\ref{chsm:sec_pseudo-sphere},它满足的方程是
%\begin{equation}
%    -(x^0)^2+(x^1)^2+(x^2)^2+(x^3)^2 = 1.
%\end{equation}
%我们用超平面$x^0-x^1=0$将这个伪球面分开,从而得到$S^3_1$的超曲面$\Sigma$:
%\begin{equation}
%    T^\perp\Sigma = {\rm Span}\{ n^a\},\quad
%    n^a=\left(\frac{\partial}{\partial x^0}\right)^a+\left(\frac{\partial}{\partial x^1}\right)^a .
%\end{equation}
%根补分布是
%\begin{equation}
%    S=\left\{x^3\left(\frac{\partial}{\partial x^2}\right)^a
%    -x^2\left(\frac{\partial}{\partial x^3}\right)^a \right\} .
%\end{equation}
%%由定理\ref{chnull:thm_lnX}可得
%$l^a$最一般的表达式为
%\setlength{\mathindent}{0em}
%\begin{equation*}
%    l^a=\frac{1}{2}\left[\bigl(1+(x^0)^2\bigr)\left(\frac{\partial}{\partial x^0}\right)^a
%    +\bigl((x^0)^2-1\bigr)\left(\frac{\partial}{\partial x^1}\right)^a 
%    +2x^0 x^2 \left(\frac{\partial}{\partial x^2}\right)^a 
%    +2x^0 x^3 \left(\frac{\partial}{\partial x^3}\right)^a \right] .
%\end{equation*}\setlength{\mathindent}{2em}

\subsection{基本方程}
$(\Sigma,h)$是$(M,g)$的类光超曲面,$\nabla_a$是$g_{ab}$的Levi-Civita联络,
$\Sigma$上的联络用${\rm D}_a$表示.%根据分解\eqref{chnull:eqn_decomp},
非零模曲面的Gauss方程\eqref{chsm:eqn_Gauss}和Weingarten方程\eqref{chsm:eqn_Weingarten}在
零模曲面上变为:
\begin{align}
    {\nabla}_{X} Y^a =& {\rm D}_{X} Y^a + l^a (K_{bc}  X^b Y^c), 
    \qquad \text{Gauss方程}  \label{chnull:eqn_Gauss} \\
    {\nabla}_{X} l^a =& - S^a_{l b} X^b + (\tau_b X^b)  l^a .
    \qquad \text{Weingarten方程}  \label{chnull:eqn_Weingarten} 
\end{align}
其中$X^a,Y^a, {\rm D}_{X} Y^a, S^a_{l b} X^b \in \mathfrak{X}(\Sigma)$;
$l^a \in E$;$\tau_a$是$\Sigma$上的余切场;
$K_{bc}$是超曲面$\Sigma$的{\heiti 第二基本型式},
$S^a_{l b}$是超曲面$\Sigma$的{\heiti 形状因子}.
得到Gauss方程\eqref{chnull:eqn_Gauss}的方式与得到式\eqref{chsm:eqn_IIbc}过程大体相仿;
此时不属于$T\Sigma$的切矢量必然是$l^a$,不是$n^a$.
Weingarten方程\eqref{chnull:eqn_Weingarten}自然也就对$l^a$求导了;
所得结果中形状因子部分属于$\mathfrak{X}(\Sigma)$;
还有一部分属于$l^a$,其系数是与超曲面$\Sigma$相关的标量函数场,
此处转化成了余切矢量场$\tau_b$.

由Gauss方程\eqref{chnull:eqn_Gauss}可得(参见式\eqref{chnull:eqn_SigmaAll})
\begin{equation}
    g_{ab} ({\nabla}_{X} Y^a) n^b = - K_{bc}  X^b Y^c .
\end{equation}
由上式容易验证第二基本型式是退化的:
\begin{equation}\label{chnull:eqn_K-deg}
    K_{bc}  X^b n^c = -g_{ab} ({\nabla}_{X} n^a) n^b
    =-n_a \nabla_X n^a =- \frac{1}{2}\nabla_X(n_a n^a)=0.
\end{equation}

非退化超曲面上诱导联络是Levi-Civita联络,见定理\ref{chsm:thm_Dg}.
然而在退化(即类光)超曲面上,联络变得略显复杂;
$\forall X^a,Y^a,Z^a \in \mathfrak{X}(\Sigma)$有
\setlength{\mathindent}{0em}
\begin{equation}\label{chnull:eqn_DK}
\begin{aligned}
    0=& ({\nabla}_{X}g_{ab}) Y^a Z^b \xlongequal{\ref{chgd:eqn_connection-compatibility}}
    \nabla_{X}(g_{ab}Y^a Z^b)-g_{ab}( \nabla_{X}Y^a) Z^b- g_{ab}Y^a\nabla_{X} Z^b \\
%    =& X(h_{ab}Y^a Z^b)-g_{ab}\bigl( {\rm D}_{X} Y^a + l^a K_{ec}  X^e Y^c \bigr) Z^b
%    - g_{ab}Y^a \bigl( {\rm D}_{X} Z^b + l^b K_{ec}  X^e Z^c \bigr) \\
%    =&X(h_{ab}Y^a Z^b)-h_{ab}( {\rm D}_{X} Y^a ) Z^b - h_{ab}Y^a  {\rm D}_{X} Z^b 
%    -g_{ab} l^a K_{ec}  X^e Y^c Z^b  - g_{ab}Y^a l^b K_{ec}  X^e Z^c \\
    =&({\rm D}_{X}h_{ab}) Y^a Z^b
    -(g_{ab} l^a Z^b) K_{ec}  X^e Y^c   - (g_{ab}Y^a l^b) K_{ec}  X^e Z^c .
\end{aligned}
\end{equation}\setlength{\mathindent}{2em}
从上式可见类光超曲面上的诱导联络${\rm D}_a$一般不与退型$h_{ab}$相互容许.易得

\begin{proposition}
    设$(\Sigma,h)$是四维闵氏流形$(M,g)$的类光超曲面.
    则$\Sigma$上诱导联络${\rm D}_a$与退型$h_{ab}$相互容许
    充要条件是:$\Sigma$是$M$的全测地类光超曲面(定义见\ref{chsm:def_total-geodesic}).
\end{proposition}


\begin{proposition}
    设$(\Sigma,h)$是四维闵氏流形$(M,g)$的类光超曲面.则下述陈述相互等价:    
    {\bfseries (1)} $\Sigma$是全测地超曲面.    
    {\bfseries (2)} $\Sigma$的第二基本型式恒为零.    
    {\bfseries (3)} $\Sigma$上存在唯一的由度规$h_{ab}$诱导的无挠联络.    
    {\bfseries (4)} $T^\perp(\Sigma)$是$\Sigma$上的Killing分布.    
    {\bfseries (5)} $T^\perp(\Sigma)$是${\rm D}_a$上平行移动不变的分布.
\end{proposition}

\begin{proof}
    证明可参考\parencite{Duggal-Bejancu-1996}第88页定理2.2.

仅对于第(4)条略作说明.$\forall X^a,Y^a,Z^a \in \mathfrak{X}(\Sigma)$,
由式\eqref{chnull:eqn_DK}可得
\begin{align}
    &0=({\rm D}_{X}h_{ab}) n^a Z^b
    -(g_{ab} l^a Z^b) K_{ec}  X^e n^c   - (g_{ab}n^a l^b) K_{ec}  X^e Z^c 
     \notag \\
    &\xLongrightarrow{\ref{chnull:eqn_K-deg}} 
    0= - h_{ab} ({\rm D}_X n^a) Z^b + K_{ec}  X^e Z^c .
\end{align}
由上式及\eqref{chrg:eqn_killing-2}可得
\begin{align}
    %(\Lie_{n} h_{ab} ) (X^a,Y^b)  =
    ({\rm D}_a n_b + {\rm D}_b n_a ) X^a Y^b
    =Y^b {\rm D}_X n_b + X^a {\rm D}_Y n_a 
    =2 K_{ab} X^a Y^b.
\end{align}
由上式即可证明第(4)条.
\end{proof}


%本书中,在涉及类光超曲面时,将只在全测地类光超曲面上研究问题,以便让退型和联络相互容许;否则问题会变得很复杂.

%依据文献\parencite{Duggal-Bejancu-1996}\S 4.3内容可知:
%与式\eqref{chsm:eqn_curvature-Gauss-Frame-HS}和\eqref{chsm:eqn_Codazzi-Frame-HS}相同.

类光超曲面的Gauss方程和Codazzi方程请参见\parencite[\S 4.3]{Duggal-Bejancu-1996}.
需要指出的是:在一般情形下,类光超曲面上的Ricci张量不再是对称的;满足一些条件后,
Ricci张量再次对称,详见\parencite[\S 4.3]{Duggal-Bejancu-1996}定理3.2.


\subsection{几条属性}
\begin{proposition}\label{chnull:thm_tl-sl}
    在广义Lorentz度规下,与类时矢量正交的矢量必定是类空矢量.
\end{proposition}
\begin{proof}
    设在流形$M$上有类时矢量$t^a$,将其选为第零基矢量,即$(e_0)^a=t^a$;然后将这个基矢扩充为
    整个流形的基矢组$(e_\mu)^a(0\leqslant \mu \leqslant m)$;假设这个基矢组是正交归一的,
    这一点总是可以做到的.在此正交归一基矢组$(e_\mu)^a$下,度规是对角矩阵$g={\rm diag}(-1,1,\cdots,1)$.
    
    假设非零矢量$v^a$与$t^a$正交,即$g_{ab}t^a v^b=0$.在基矢组$(e_\mu)^a$下,
    矢量$v^a$可以展开为$v^a=v^\mu (e_\mu)^a$,则有
    \begin{equation}
        0=g_{ab}t^a v^b=g_{ab}(e_0)^a  v^\mu (e_\mu)^b= g_{0\mu}v^\mu =-1\cdot v^0
        {\quad \Rightarrow \quad  } v^0 = 0.
    \end{equation}
    再计算
    \begin{equation}
        g_{ab}v^a v^b = -(v^0)^2 + \sum_{i=1}^{m} (v^i)^2 =\sum_{i=1}^{m} (v^i)^2 >0.
    \end{equation}
    上式说明矢量$v^a$是类空矢量.
    这个命题只在广义Lorentz度规下成立;比如在度规$g={\rm diag}(-1,-1,+1,+1)$下便可举出反例.
\end{proof}



因广义闵氏空间的矢量只分为类时、类空和零模,故由命题\ref{chnull:thm_tl-sl}必有推论:
\begin{corollary}\label{chnull:thm_n-s}
    在广义Lorentz度规下,类时矢量与零模矢量不正交!
\end{corollary}

\begin{corollary}\label{chnull:thm_n-t}
    切于类光超曲面$\Sigma$的矢量不能是类时的,只能是类空的或零模的.
\end{corollary}

令广义Minkowski流形$M$的基底是正交归一的;那么必有一基底是类时的(否则$M$的度规就变成正定的了).
设流形$M$中有类时超曲面$\Sigma$,它的法矢量是类空的;
那么这个类时基底必然切于类时超曲面$\Sigma$.
因此可以得到结论:类时超曲面$\Sigma$每一点都有类时矢量切于它.


\begin{proposition}\label{chnull:thm_null-tn}
    在广义Lorentz度规下,超曲面$\Sigma$每点都有零模切矢量并且没有类时切矢量,
    那么$\Sigma$必定是类光超曲面.
\end{proposition}
\begin{proof}
    用反证法.假设$\Sigma$是类时的,与上面结论矛盾!
    
    再假设$\Sigma$是类空的,那么它的法矢量$t^a$是类时的;而切于$\Sigma$的矢量必定正交于$t^a$.
    命题\ref{chnull:thm_tl-sl}告诉我们切于$\Sigma$的矢量都是类空的,
    不可能是零模的!
    这与命题中的条件(每点都有零模切矢量)矛盾!
\end{proof}


\begin{proposition}\label{chnull:thm_null-self}
    在广义Lorentz度规下,设有零模矢量$\xi^a, \eta^a \in \mathfrak{X}(M)$;
    如果$g_{ab}\xi^a \eta^b=0$,那么必有$\xi^a\propto \eta^a$.
\end{proposition}
\begin{proof}
    选正交归一基矢组$(e_\mu)^a$,度规是对角矩阵$g={\rm diag}(-1,1,\cdots,1)$.
    不失一般性,假设在上述基底下,两个零模矢量的分量表达是为
    \begin{align}
        \xi=&(\kappa,\kappa,0 \cdots, 0),\qquad \text{通过旋转基矢,总可以做到此点}; \\
        \eta=&(\eta^0,\eta^1, \cdots, \eta^m),\qquad\text{其中}\  -(\eta^0)^2 + \sum_{i=1}^{m}(\eta^i)^2 =0.
    \end{align}
    将$g_{ab}\xi^a \eta^b=0$用上述分量展开,有 %(即使$\xi,\eta$第零分量不是$1$,也不影响下式证明)
    \begin{equation}
        g_{ab}\xi^a \eta^b= -\kappa \eta^0 + \kappa \eta^1 \overset{?}{=} 0.
    \end{equation}
    上式只有在$\xi^a\propto \eta^a$情形下才可能等于零.
\end{proof}
类时矢量与零模矢量不正交;与零模矢量$\xi^a$正交的零模矢量只能是$f\cdot \xi^a(f \in C^\infty(M))$;
类空矢量可以与零模矢量正交,且没有太多限制.由此可得如下命题:

\begin{proposition}\label{chnull:thm_null-sxi}
    在广义Lorentz度规下,有一类光超曲面$\Sigma$,它的法矢量是$n^a$;
    那么切于$\Sigma$的矢量场只能是类空矢量场,或者是$f\cdot n^a(f \in C^\infty(M))$.
\end{proposition}


%\begin{proposition}\label{chnull:thm_null-etana}
%    在广义Lorentz度规下,设有一零模矢量$\xi^a \in \mathfrak{X}(M)$,
%    它切于类光超曲面$\Sigma$;那么它必定正比于法矢量,即$\xi^a \propto n^a$;
%    从而$\xi^a$也法于超曲面$\Sigma$.
%\end{proposition}
%\begin{proof}
%    由于类光超曲面$\Sigma$的法矢量$n^a$是切于$\Sigma$的,所以可以假设
%    \begin{equation}
    %        \xi^a = f\cdot n^a + s^a, \qquad f \in C^\infty(\Sigma), \ s^a \in \mathfrak{X}(\Sigma).
    %    \end{equation}
%    因$\xi^a$切于$\Sigma$,而零模法矢量$n^a$法于$\Sigma$,故有
%    \begin{equation}\label{chnull:eqn_tmp-sn01}
    %        0=g_{ab} \xi ^a n^b = (f\cdot n^a + s^a)n_a = s^a n_a.
    %    \end{equation}
%    由推论\ref{chsm:thm_n-s}可知,$s^a$不是类时矢量;如果是零模的,那必然正比于$n^a$,这已包含在$f$中,
%    应排除;所以$s^a$必是类空矢量.而$\xi^a$自身还是零模的,即
%    \begin{equation}
    %        0=\xi^a \xi_a = (f\cdot n^a + s^a)(f\cdot n_a + s_a)
    %        =f^2 \cdot n^a n_a + s^a n_a + f\cdot n^a s_a + s^a s_a = s^a s_a.
    %    \end{equation}
%    最后一个等号用了式\eqref{chsm:eqn_tmp-sn01}.由此可知类空矢量$s^a=0$,故有$\xi^a = f\cdot n^a$,证毕.
%\end{proof}

%\begin{example}\label{chnull:exam_time-null}
%    在广义闵氏流形$(\mathbb{R}^{m+1},\eta)$中至少包含两个线性无关的零模切矢量.
%\end{example}
%设此流形有正交归一基矢组$\{(e_\alpha)^a\}$,其中有(且仅有一个)类时基矢$(e_0)^a$,
%其余基矢$(e_i)^a\,(1\leqslant i \leqslant m)$是类空的.
%不难验证$(e_0)^a\pm (e_1)^a$是两个线性无关的零模矢量.
%
%其实$(e_0)^a\pm (e_i)^a$都是零模矢量,且当$i$不同时线性无关.
%我们仅验证$(e_0)^a+ (e_1)^a$和$(e_0)^a+ (e_2)^a$线性无关性;
%假设两者线性相关,即存在非零实常数$\lambda$使得
%\begin{align*}
%    &(e_0)^a+ (e_1)^a= \lambda \bigl((e_0)^a+ (e_2)^a\bigr) {\quad \color{red}\Rightarrow} \\
%    &\eta_{ab}\bigl((e_0)^a+ (e_1)^a\bigr)\bigl((e_0)^b+ (e_1)^b\bigr)
%    = \eta_{ab}\lambda \bigl((e_0)^a+ (e_2)^a\bigr) \bigl((e_0)^b+ (e_1)^b\bigr)
%    {\color{red}\Rightarrow}
%    0=\lambda (-1).
%\end{align*}
%与$\lambda\neq 0$的假设矛盾,故两者线性无关.
%
%读者需注意零模矢量是可以作为基矢的.比如对于流形$(\mathbb{R}^{2},\eta)$而言,
%我们可以选取$\{(e_0)^a,\, (e_1)^a\}$当基矢,
%也可以选取线性无关的零模矢量组$\{(e_0)^a\pm (e_1)^a\}$当基矢.
%\qed







%\subsection{零模测地线}
%我们先计算下式
%\begin{equation}
%    n^b \nabla_b n_a = (\nabla^b \Phi) \nabla_b \nabla_a \Phi
%    = (\nabla^b \Phi) \nabla_a \nabla_b \Phi
%    =  \frac{1}{2}\nabla_a (\nabla^b \Phi \nabla_b \Phi).
%\end{equation}
%因超曲面是零模的,故$\nabla^b \Phi \nabla_b \Phi=0$;它的梯度也是零模的,有(将指标升上来)
%\begin{equation}
%    n^b \nabla_b n^a =\kappa n^a.
%\end{equation}
%这正是测地线方程的一般形式(见测地线一章开头论述),称为非仿射参数化测地线.
%因$n^a$也是切于超曲面$\Sigma$的,故可以把超曲面$\Sigma$看成由零测地线生成的,
%也就是$\Sigma$是由$n^a$的积分曲线构成.


\index[physwords]{Newman--Penrose型式}
\index[physwords]{NP型式}

\section{Newman--Penrose基本结构}\label{chnull:sec_NP1}
在\S\ref{chnull:sec_indef-algebra}末尾引入了包含零模基矢在内的基矢组,
本节给出一种全部是零模的基矢组:
Newman--Penrose Formalism\cite{newman-Penrose-1962},简称NP型式.
NP型式是一种特定的四维“型式理论”,\S\ref{chccr:sec_form1}和\S\ref{chrg:sec_form2}中的
结果适用于NP型式.NP型式基底是由四个零模矢量来构造的,并非一般的实正交归一基底场.
Penrose引入零模基矢最重要的目的是讨论{\kaishu 旋量}在广义相对论中的
表示\cite{penrose-Rindler1984,penrose-Rindler1986}.
对于经典(非量子)广义相对论而言,NP理论更像一种数学技巧
来计算黎曼曲率张量,进而求解爱因斯坦方程;
这种技巧对于讨论(黑洞)类光测地线十分方便,因为其基矢是类光的;另一个
可见的方便之处是列出的方程个数减少近一半,见命题\ref{chnull:thm_NP-conj-34}.


NP理论只针对四维流形(及其子流形)而言,我们将使用类光矢量这一名词,而不用零模矢量这一术语,
以显得更物理一点儿.需要强调的是,需先
了解\S\ref{chccr:sec_form1}和\S\ref{chrg:sec_form2}中的内容,才能更好理解NP理论.


\subsection{基矢构造}
设已有Lorentz型的正交归一实标架场$\{(E_\mu)^a\}$及其对偶实标架场$\{(E^\mu)_a\}$,
度规在这个标架场可以表示为
\setlength{\mathindent}{0em}
\begin{subequations}
    \begin{align}
        g_{ab} = & -(E^0)_a (E^0)_b + \sum\limits_{i=1}^3 {(E^i)_a (E^i)_b }
        = -l_a n_b - n_a l_b + m_a \overline{m}_b + \overline{m}_a {m}_b , \\
        g^{ab} = & -(E_0)^a (E_0)^b + \sum\limits_{i=1}^3 {(E_i)^a (E_i)^b }
        = -l^a n^b - n^a l^b + m^a \overline{m}^b + \overline{m}^a {m}^b .
    \end{align}
\end{subequations}\setlength{\mathindent}{2em}
其中$\{l^a, n^a, m^a, \overline{m}^a\}$定义如下\parencite[Ch. 3]{penrose-Rindler1984}:
\begin{subequations}\label{chnull:eqn_Newman-Penrose-bases}
    \begin{align}
        {l^a} & \equiv (e_1)^a \overset{def}{=} \frac{1}{\sqrt{2}}
        \bigl[ (E_0)^a + (E_3)^a \bigr] ,   \label{chnull:eqn_Newman-Penrose-base-l} \\
        {n^a} & \equiv (e_2)^a \overset{def}{=} \frac{1}{\sqrt{2}}
        \bigl[ (E_0)^a - (E_3)^a \bigr] ,  \label{chnull:eqn_Newman-Penrose-base-n}\\
        {m^a} & \equiv (e_3)^a \overset{def}{=} \frac{1}{\sqrt{2}}
        \bigl[ (E_1)^a - \mathbbm{i} (E_2)^a \bigr]  , \label{chnull:eqn_Newman-Penrose-base-m}\\
        {\overline{m}^a} & \equiv (e_4)^a \overset{def}{=} \frac{1}{\sqrt{2}}
        \bigl[ (E_1)^a + \mathbbm{i} (E_2)^a \bigr] . \label{chnull:eqn_Newman-Penrose-base-mb}
    \end{align}
\end{subequations}
有了逆变矢量场,自然可以定义其对偶矢量场
\begin{subequations}\label{chnull:eqn_Newman-Penrose-cov-bases}
    \begin{align}
        {l_a}  &\equiv (e_1)_a  = g_{ab} (e_1)^b
        =  -\frac{1}{\sqrt{2}}\left( (E^0)_a - (E^3)_a  \right) , \\
        {n_a}  &\equiv (e_2)_a  = g_{ab} (e_2)^b
        =  -\frac{1}{\sqrt{2}}\left( (E^0)_a + (E^3)_a  \right) , \\
        {m}_a  &\equiv (e_3)_a  = g_{ab} (e_3)^b
        =  \frac{1}{\sqrt{2}}\left( (E^1)_a - \mathbbm{i} (E^2)_a  \right) , \\
        \overline{m}_a  &\equiv (e_4)_a  = g_{ab} (e_4)^b
        =  \frac{1}{\sqrt{2}}\left( (E^1)_a + \mathbbm{i} (E^2)_a  \right) .
    \end{align}
\end{subequations}
容易验证,这些基矢有如下关系
\begin{subequations}\label{chnull:eqn_NP-lnmmb}
    \begin{align}
        l_a l^a =& n_a n^a = m_a m^a = \overline{m}_a \overline{m}^a = 0 , \\
        l_a m^a =& l_a \overline{m}^a = n_a m^a = n_a \overline{m}^a = 0 , \\
        l_a n^a =& -1 = - m_a  \overline{m}^a  .\label{chnull:eqn_lm1}
    \end{align}
\end{subequations}
显然,四个基矢是{\kaishu 类光}的.
可看出在类光标架场$\{l^a, n^a, m^a, \overline{m}^a\}$(即$(e_\mu)^a$)下
度规分量是
\begin{equation}\label{chnull:eqn_np-metric}
    g_{\mu \nu } = \left( {\begin{array}{*{20}{c}}
            0&-1&0&0\\
            -1&0&0&0\\
            0&0&0&1\\
            0&0&1&0
    \end{array}} \right)
    \quad = g^{\mu \nu } .
\end{equation}
这个矩阵的逆是其自身,已标注在上面公式中;显然此标架是{\heiti 刚性标架};
这样便可用\S\ref{chrg:sec_form2}中的理论来计算黎曼曲率张量.
有了度规分量表达式,便可以将前面基矢的内指标进行升降,
\begin{align*}
        (e^1)_a =& g^{1\rho}(e_\rho)_a=-(e_2)_a=-n_a, &
        (e^2)_a =& g^{2\rho}(e_\rho)_a=-(e_1)_a=-l_a;  \\
        (e^3)_a =& g^{3\rho}(e_\rho)_a=+(e_4)_a=+\overline{m}_a, &
        (e^4)_a =& g^{4\rho}(e_\rho)_a=+(e_3)_a=+{m}_a .
\end{align*}
下述对偶关系自然成立
\begin{equation}
    (e^\mu)_a (e_\nu)^a = \delta_\nu^\mu, \qquad
    (e^\mu)_a (e_\mu)^b = \delta_a^b .
\end{equation}

需要注意,标架场$\{(E_\mu)^a\}$是{\kaishu 实}正交归一的;
标架场$\{l^a, n^a\}$是{\kaishu 实}类光的;
标架场$\{m^a, \overline{m}^a\}$是{\kaishu 复}类光的,且互为复共轭.
\begin{remark}
    式\eqref{chnull:eqn_Newman-Penrose-bases}给出了类光基矢量的一种取法,
    但不一定都按照此种方法选取.NP基矢只要类光,
    即满足式\eqref{chnull:eqn_NP-lnmmb}皆可.
\end{remark}

依照式\eqref{chrg:eqn_ricci0},可以得到NP型式的Ricci旋转系数
\begin{equation}\label{chnull:eqn_NP-spin-coef}
    \omega_{\mu\nu\rho}= (\omega_{\mu\nu})_a(e_\rho)^a
    = (e_\mu)^a (e_\rho)^b \nabla_b (e_\nu)_a .
\end{equation}

\begin{proposition}\label{chnull:thm_NP-conj-34}
    如果把Ricci旋转系数$\omega_{\mu\nu\rho}$下标中的指标3和4互换(保持1和2不变),
    那么得到其共轭复数,例如$\omega_{123}=\bar{\omega}_{124}$,
    $\omega_{343}=\bar{\omega}_{434}$,$\omega_{122}=\bar{\omega}_{122}$,等等.
    其它量,比如黎曼张量,在NP标架下的分量也有相同变化规律,例如
    $R_{1212}=\bar{R}_{1212}$,$R_{1224}=\bar{R}_{1223}$,$R_{3412}=\bar{R}_{4312}$,等等.
\end{proposition}
\begin{proof}
    只需把类光标架的四个基矢量\eqref{chnull:eqn_Newman-Penrose-bases}带入
    就可证明,因为其第三、四基矢是互为共轭复数.
    例如$R_{1234}=R_{abcd}l^a n^b m^c \overline{m}^d $,
    那么$\bar{R}_{1234}=\overline{R_{abcd}l^a n^b m^c \overline{m}^d}
    =R_{abcd}\overline{l^a n^b m^c \overline{m}^d}
    =R_{abcd}l^a n^b \overline{m}^c {m}^d=R_{1243}$.
    需注意,抽象指标下的曲率$R_{abcd}$是实流形上的张量,
    其本身是实的,只有其NP分量才是复的.
\end{proof}
\begin{remark}
    Newman--Penrose型式的底流形仍是{\kaishu 实流形},不是{\kaishu 复流形}.
    在目前的框架下,复数共轭属于额外定义的运算,不是{\kaishu 代数学}上的运算,
    所以$z$和其复共轭$\bar{z}$在代数学上是线性独立的;
    见例题\ref{chmla:exm_rc1}和\ref{chmla:exm_rc2}.
    因复共轭运算的引入,由命题\ref{chnull:thm_NP-conj-34}可知
    约一半的方程可省略不写出;但需注意,这些方程仍是线性独立的.
    复流形的初步知识可参考第\ref{chcx}章.    %\textcite{cc2001-zh}教材.
\end{remark}


\subsection{参量定义}
本节定义NP型式理论中用到的各种参数.
黎曼曲率等量会带着较为繁杂的角标,在实际求解方程时略显麻烦;
用不带上下标(或较少上下标)的记号来标记这些量,可使方程显得简洁一些.
因符号数量有限,而NP理论占用了大量符号,
在后面章节中同一符号可能代表不同量的事情,需要谨慎区分.

\index[physwords]{NP型式!自旋系数}

\paragraph{自旋系数}
注解\ref{chrg:remark_ricci-coef-num}指出
Ricci旋转系数$\omega_{\mu\nu\rho}$有$\frac{1}{2}m^2(m-1)$个;
对于四维空间,便是24个;由命题\ref{chnull:thm_NP-conj-34}知
只需12个记号便可标记所有Ricci旋转系数.
Newman和Penrose称之为自旋系数(Spin Coefficients),定义如下
\begin{subequations}\label{chnull:eqn_spincoef-all}
    \begin{align}
        -\kappa &\equiv \omega_{311}=m^a l^b \nabla_b l_a,  \\
        -\tau   &\equiv \omega_{312}=m^a n^b \nabla_b l_a, \\
        -\sigma &\equiv \omega_{313}=m^a m^b \nabla_b l_a, \\
        -\rho   &\equiv \omega_{314}=m^a \overline{m}^b \nabla_b l_a, \\
        \pi    &\equiv \omega_{421}=\overline{m}^a l^b \nabla_b n_a,  \\
        \nu    &\equiv \omega_{422}=\overline{m}^a n^b \nabla_b n_a,   \\
        \mu    &\equiv \omega_{423}=\overline{m}^a m^b \nabla_b n_a,   \\
        \lambda&\equiv \omega_{424}=\overline{m}^a \overline{m}^b \nabla_b n_a,   \\
        \epsilon&\equiv \frac{1}{2}(\omega_{431} - \omega_{211})
        =\frac{1}{2}(\overline{m}^a l^b\nabla_b m_a - n^a l^b\nabla_b l_a ), \\
        \gamma &\equiv \frac{1}{2}(\omega_{432} - \omega_{212})
        =\frac{1}{2}(\overline{m}^a n^b\nabla_b m_a - n^a n^b\nabla_b l_a ), \\
        \beta &\equiv \frac{1}{2}(\omega_{433} - \omega_{213})
        =\frac{1}{2}(\overline{m}^a m^b\nabla_b m_a - n^a m^b\nabla_b l_a ),  \\
        \alpha &\equiv \frac{1}{2}(\omega_{434} - \omega_{214})
        =\frac{1}{2}(\overline{m}^a \overline{m}^b\nabla_b m_a- n^a \overline{m}^b\nabla_b l_a ) .
    \end{align}
\end{subequations}
还可用自旋系数表示Ricci旋转系数
\begin{equation}\label{chnull:eqn_omegas-by-spin-coefs}
    \begin{aligned}
        \omega_{121} &= \epsilon + \bar{\epsilon}, & \omega_{431} &= \epsilon - \bar{\epsilon}, &
        \omega_{122} &= \gamma + \bar{\gamma},  \\
        \omega_{124} &= \alpha + \bar{\beta},  & \omega_{434} &= \alpha - \bar{\beta}, &
        \omega_{432} &= \gamma - \bar{\gamma} .
    \end{aligned}
\end{equation}
其它Ricci旋转系数直接一望而知,或通过命题\ref{chnull:thm_NP-conj-34}一望而知.

\paragraph{Ricci曲率}
Ricci曲率独立分量个数是10个;
定义9个Ricci张量分量,结合标量曲率$R$后,便可描述10个Ricci曲率
\begin{equation}\label{chnull:eqn_ricci-all}
    \begin{aligned}
        \Phi_{00}&\equiv \frac{1}{2} R_{11}, &
        \Phi_{01}&\equiv \frac{1}{2} R_{13}, &
        \Phi_{02}&\equiv \frac{1}{2} R_{33}, \\
        \Phi_{10}&\equiv \frac{1}{2} R_{14}, &
        \Phi_{11}&\equiv \frac{1}{4} (R_{12} + R_{34}), &
        \Phi_{12}&\equiv \frac{1}{2} R_{23}, \\
        \Phi_{20}&\equiv \frac{1}{2} R_{44} , &
        \Phi_{21}&\equiv \frac{1}{2} R_{24}, &
        \Phi_{22}&\equiv \frac{1}{2} R_{22}.
        %        \Phi_{00}&\equiv \frac{1}{2} R_{ab}l^a l^b, &
        %        \Phi_{01}&\equiv \frac{1}{2} R_{ab}l^a m^b, &
        %        \Phi_{02}&\equiv \frac{1}{2} R_{ab}m^a m^b, \\
        %        \Phi_{10}&\equiv \frac{1}{2} R_{ab}l^a \overline{m}^b, &
        %        \Phi_{11}&\equiv \frac{1}{4} (R_{ab}l^a n^b + R_{ab}m^a \overline{m}^b), &
        %        \Phi_{12}&\equiv \frac{1}{2} R_{ab}n^a m^b, \\
        %        \Phi_{20}&\equiv \frac{1}{2} R_{ab}\overline{m}^a \overline{m}^b , &
        %        \Phi_{21}&\equiv \frac{1}{2} R_{ab}n^a \overline{m}^b, &
        %        \Phi_{22}&\equiv \frac{1}{2} R_{ab}n^a n^b.
    \end{aligned}
\end{equation}
其中标量曲率可表示为
\begin{equation}
    R= g^{ab}R_{ab}=-R_{12} -R_{21} +R_{34}+R_{43} = 2(R_{34}-R_{12}) .
\end{equation}

\paragraph{Weyl张量}
由式\eqref{chrg:eqn_WeylConform-d4},可得黎曼曲率与Weyl张量、Ricci曲率关系
\begin{equation}\label{chnull:eqn_weyl4dim}
    \begin{aligned}
    R_{abcd}{=}& C_{abcd} + \frac{1}{2} ( g_{ac}R_{bd} + g_{bd}R_{ac}
     - g_{ad}R_{bc} - g_{bc}R_{ad} ) \\
     &- \frac{1}{6} (g_{ac}g_{bd}-g_{ad}g_{bc}) R .
    \end{aligned}
\end{equation}
Weyl张量是无迹的,且遵循循环恒等式
\begin{align}
    0 &= g^{ad}C_{abcd} =- C_{1bc2}- C_{2bc1}+C_{3bc4}+C_{4bc3}, \label{chnull:eqn_weyl0tr} \\
    0 &= C_{1234}+C_{1342}+C_{1423}. \label{chnull:eqn_weyl-cycle}
\end{align}
当$b=c$时,式\eqref{chnull:eqn_weyl0tr}给出
\begin{equation}
    C_{1314} = C_{2324} = C_{1332} = C_{1442} = 0.
\end{equation}
当$b\neq c$时,式\eqref{chnull:eqn_weyl0tr}和式\eqref{chnull:eqn_weyl-cycle}一起给出
\begin{align}
    C_{1343}&=C_{1213}, \quad  C_{1434}=C_{1214}, \quad  C_{2343}=C_{1232}, \quad
    C_{2434}=C_{1242}, \\
    C_{3434}&= C_{1212} = C_{1342} + C_{1432}, \quad
    C_{1234}=C_{1432}- C_{1342}.
\end{align}
参考表\ref{chrg:tab-riemann},再综合上两式,可取独立10个分量为
\begin{equation}\label{chnull:eqn_ind-C}
    \begin{aligned}
        &C_{1313}, \quad C_{1213}, \quad C_{1224}, \quad C_{1324}, \quad C_{2424};  \\
        &{\color{red} C_{1414},\quad C_{1214},\quad C_{1223},\quad C_{1423},
            \quad  C_{2323}. }
    \end{aligned}
\end{equation}
第二行是第一行的复共轭.
通常,用如下5个复标量来表示Weyl张量分量
\begin{subequations}\label{chnull:eqn_weylscalars-all}
    \begin{align}
        \Psi_0 &\equiv C_{abcd}l^a m^b l^c m^d = C_{1313} , \label{chnull:eqn_weylscalars0} \\
        \Psi_1 &\equiv C_{abcd}l^a n^b l^c m^d = C_{1213}, \label{chnull:eqn_weylscalars1} \\
        \Psi_2 &\equiv C_{abcd}l^a m^b \overline{m}^c n^d = C_{1342}, \label{chnull:eqn_weylscalars2} \\
        \Psi_3 &\equiv C_{abcd}l^a n^b \overline{m}^c n^d = C_{1242}, \label{chnull:eqn_weylscalars3} \\
        \Psi_4 &\equiv C_{abcd}n^a \overline{m}^b n^c \overline{m}^d = C_{2424}. \label{chnull:eqn_weylscalars4}
    \end{align}
\end{subequations}
利用上面结果及式\eqref{chnull:eqn_weyl4dim},
可将黎曼曲率分量表示成Ricci曲率和Weyl张量,
计算过程略显繁琐,但并不困难,主要是符号代换.以$R_{1212}$为例来说明.
\setlength{\mathindent}{0em}
    \begin{align}
        &R_{1212} = C_{1212} + \frac{1}{2}(g_{11}R_{22} - g_{12}R_{21} + g_{22}R_{11} - g_{21}R_{12})
        - \frac{1}{6}(g_{11}g_{22}-g_{12}g_{21}) R    \notag\\
        &=C_{1342} + \bar{C}_{1342}  + \frac{1}{2}(R_{21}  +R_{12})      + \frac{1}{6} R
        = \Psi_{2} + \bar{\Psi}_2 + 2\Phi_{11} -\frac{1}{12}R .
        \label{chnull:eqn_Riemann-by-Weyl-Ricc-R1212}
    \end{align}
其它黎曼曲率分量是
\begin{equation}\label{chnull:eqn_Riemann-by-Weyl-Ricc-all}
    \begin{aligned}
        R_{1213} %=& C_{1213} + \frac{1}{2}( g_{11}R_{23} - g_{13}R_{21}  + g_{23}R_{11} - g_{21}R_{13})
        %- \frac{1}{6}(g_{11}g_{23}-g_{13}g_{21}) R    \\
        &= \Psi_{1} + \Phi_{01},      \quad
        R_{1224} %=& C_{1224} + \frac{1}{2}( g_{12}R_{24} - g_{14}R_{22} + g_{24}R_{12} - g_{22}R_{14})
        %- \frac{1}{6}(g_{12}g_{24}-g_{14}g_{22}) R    \\
        = -\Psi_{3} -\Phi_{21} ,  \quad
        R_{1313} %=& C_{1313} + \frac{1}{2}( g_{11}R_{33} - g_{13}R_{31} + g_{33}R_{11} - g_{31}R_{13})
        % - \frac{1}{6}(g_{11}g_{33}-g_{13}g_{32}) R    \\
        = \Psi_{0} ,  \quad
        R_{1314} %=& C_{1314} + \frac{1}{2}( g_{11}R_{34} - g_{14}R_{31} + g_{34}R_{11} - g_{31}R_{14})
        % - \frac{1}{6}(g_{11}g_{34}-g_{14}g_{31}) R    \\
        =  \Phi_{00}, \\
        R_{1323} %=& C_{1323} + \frac{1}{2}( g_{12}R_{33} - g_{13}R_{32} + g_{33}R_{12} - g_{32}R_{13})
        %  - \frac{1}{6}(g_{12}g_{33}-g_{13}g_{32}) R    \\
        &= -\Phi_{02}  , \
        R_{1324} %=& C_{1324} + \frac{1}{2}( g_{12}R_{34} - g_{14}R_{32} + g_{34}R_{12} - g_{32}R_{14})
        %- \frac{1}{6}(g_{12}g_{34}-g_{14}g_{32}) R    \\
        = -\Psi_{2} - \frac{1}{12} R  ,   \
        R_{1334} %=& C_{1334} + \frac{1}{2}( g_{13}R_{34} - g_{14}R_{33} + g_{34}R_{13} - g_{33}R_{14})
        %- \frac{1}{6}(g_{13}g_{34}-g_{14}g_{33}) R    \\
        = \Phi_{01}-\Psi_{1}  , \
        R_{2324} %=& C_{2324} + \frac{1}{2}( g_{22}R_{34} - g_{24}R_{32} + g_{34}R_{22} - g_{32}R_{24})
        %- \frac{1}{6}(g_{22}g_{34}-g_{24}g_{32}) R    \\
        =  \Phi_{22},   \\
        R_{2334} %=& C_{2334} + \frac{1}{2}( g_{23}R_{34} - g_{24}R_{33} + g_{34}R_{23} - g_{33}R_{24})
        %- \frac{1}{6}(g_{23}g_{34}-g_{24}g_{33}) R    \\
        &= -\bar{\Psi}_{2} + \Phi_{12},  \quad
        R_{2424} %=& C_{2424} + \frac{1}{2}( g_{22}R_{44} - g_{24}R_{42} + g_{44}R_{22} - g_{42}R_{24})
        %- \frac{1}{6}(g_{22}g_{44}-g_{24}g_{42}) R \\
        = \Psi_{4} ,  \quad
        R_{3434} %=& C_{3434} + \frac{1}{2}( g_{33}R_{44} - g_{34}R_{43} + g_{44}R_{33} - g_{43}R_{34})
        %- \frac{1}{6}(g_{33}g_{44}-g_{34}g_{43}) R    \\
        = \Psi_{2} + \bar{\Psi}_2  -2\Phi_{11}- \frac{1}{12} R   .
    \end{aligned}
\end{equation} \setlength{\mathindent}{2em}



需要注意,这里的度规与原始论文的度规\cite{newman-Penrose-1962}整体差一负号,
为了保证NP方程组\eqref{chnull:eqn_NP-all}、Bianchi恒等式\eqref{chnull:eqn_Bianchi-4Q}、
麦克斯韦方程组\eqref{chnull:eqn_maxwell-4Q}、爱因斯坦引力场方程组\eqref{chnull:eqn_einstein-4Q}等
形式不变;我们将式\eqref{chnull:eqn_lm1}、式\eqref{chnull:eqn_np-metric}、
自旋系数\eqref{chnull:eqn_spincoef-all}、Weyl标量\eqref{chnull:eqn_weylscalars-all}和
Ricci标量\eqref{chnull:eqn_ricci-all}的定义取得与原论文\cite{newman-Penrose-1962}差一负号.

\subsection{四基矢协变导数}
有了自旋系数定义,直接计算便可得到基矢协变导数.
两个实类光矢量的协变导数是
\begin{equation}\label{chnull:eqn_covD-l}
    \begin{aligned}
        \nabla_{b} l_a =& \nabla_{b} (e_1)_a =
        \omega_{\nu 1 \tau}(e^\tau)_b (e^\nu)_a  \\
        %        =&-\omega_{211}(e^1)_b l_a - \omega_{212}(e^2)_b l_a
        %        -\omega_{213}(e^3)_b l_a - \omega_{214}(e^4)_b l_a  \\
        %        & +\omega_{311}(e^1)_b \overline{m}_a  +\omega_{312}(e^2)_b \overline{m}_a
        %        +\omega_{313}(e^3)_b \overline{m}_a  +\omega_{314}(e^4)_b \overline{m}_a  \\
        %        & +\omega_{411}(e^1)_b m_a +\omega_{412}(e^2)_b m_a
        %        +\omega_{413}(e^3)_b m_a+\omega_{414}(e^4)_b m_a  \\
        =& -(\epsilon+\bar{\epsilon}) n_b l_a -(\gamma+\bar{\gamma}) l_b l_a
        +(\bar{\alpha}+\beta) \overline{m}_b l_a + (\alpha+\bar{\beta}) m_b l_a  \\
        & +\kappa n_b \overline{m}_a  +\tau l_b \overline{m}_a
        -\sigma \overline{m}_b \overline{m}_a  -\rho m_b \overline{m}_a  \\
        & +\bar{\kappa} n_b m_a + \bar{\tau} l_b m_a
        -\bar{\rho} \overline{m}_b m_a -\bar{\sigma} m_b m_a  .
    \end{aligned}
\end{equation}
和
\begin{equation}\label{chnull:eqn_covD-n}
    \begin{aligned}
        \nabla_{b} n_a =& \nabla_{b} (e_2)_a =
        \omega_{\nu 2 \tau}(e^\tau)_b (e^\nu)_a  \\
        %        =&-\omega_{121}(e^1)_b n_a - \omega_{122}(e^2)_b n_a
        %        -\omega_{123}(e^3)_b n_a - \omega_{124}(e^4)_b n_a  \\
        %        & +\omega_{321}(e^1)_b \overline{m}_a  +\omega_{322}(e^2)_b \overline{m}_a
        %        +\omega_{323}(e^3)_b \overline{m}_a  +\omega_{324}(e^4)_b \overline{m}_a  \\
        %        & +\omega_{421}(e^1)_b m_a +\omega_{422}(e^2)_b m_a
        %        +\omega_{423}(e^3)_b m_a+\omega_{424}(e^4)_b m_a  \\
        =& (\epsilon+\bar{\epsilon}) n_b n_a + (\gamma+\bar{\gamma}) l_b n_a
        -(\bar{\alpha}+\beta)\overline{m}_b n_a - (\alpha+\bar{\beta})m_b n_a  \\
        & -\bar{\pi} n_b \overline{m}_a  -\bar{\nu} l_b \overline{m}_a
        +\bar{\lambda}\overline{m}_b \overline{m}_a  +\bar{\mu}m_b \overline{m}_a  \\
        & -\pi n_b m_a -\nu l_b m_a
        +\mu \overline{m}_b m_a+\lambda m_b m_a .
    \end{aligned}
\end{equation}
只需给出一个复类光矢量的协变导数,另一个可以通过取复共轭得到.
\begin{equation}\label{chnull:eqn_covD-m}
    \begin{aligned}
        \nabla_{b} m_a =& \nabla_{b} (e_3)_a =
        \omega_{\nu 3 \tau}(e^\tau)_b (e^\nu)_a  \\
        %        =&-\omega_{131}(e^1)_b n_a - \omega_{132}(e^2)_b n_a
        %        -\omega_{133}(e^3)_b n_a - \omega_{134}(e^4)_b n_a  \\
        %        & -\omega_{231}(e^1)_b l_a  -\omega_{232}(e^2)_b l_a
        %        -\omega_{233}(e^3)_b l_a  -\omega_{234}(e^4)_b l_a  \\
        %        & +\omega_{431}(e^1)_b m_a +\omega_{432}(e^2)_b m_a
        %        +\omega_{433}(e^3)_b m_a+\omega_{434}(e^4)_b m_a  \\
        =&\kappa n_b n_a + \tau l_b n_a
        -\sigma \overline{m}_b n_a - \rho m_b n_a  \\
        & -\bar{\pi} n_b l_a  -\bar{\nu}l_b l_a
        +\bar{\lambda}\overline{m}_b l_a  + \bar{\mu}m_b l_a
        -(\epsilon-\bar{\epsilon})n_b m_a \\
        &  -(\gamma-\bar{\gamma})l_b m_a
        + (\beta-\bar{\alpha})\overline{m}_b m_a+(\alpha-\bar{\beta})m_b m_a .
    \end{aligned}
\end{equation}

%请读者补齐式\eqref{chnull:eqn_covD-l}、\eqref{chnull:eqn_covD-n}和\eqref{chnull:eqn_covD-m}的计算.



\section{Newman--Penrose基本方程}\label{chnull:sec_NP2}
基本方程包括如下三组关系式:对易关系、NP方程和Bianchi恒等式.
\subsection{对易关系}
定义如下方向导数:
\begin{equation}\label{chnull:eqn_direct-D}
    {\rm D}\equiv l^a\nabla_a , \quad
        {\Delta}\equiv n^a\nabla_a , \quad
        {\updelta}\equiv m^a\nabla_a , \quad
        {\bar{\updelta}}\equiv \overline{m}^a\nabla_a .
\end{equation}
由式\eqref{chrg:eqn_EmuEnucommutator}可得基矢间对易关系,以$n^a$和$l^a$为例,
\begin{align*}
    [n, l]^a = &\bigl[(e_2), (e_1)\bigr]^a
    = \bigl(\omega_{\sigma 12} - \omega_{\sigma 21}\bigr) (e^\sigma)^a
    =\bigl(\omega_{112} - \omega_{121}\bigr) (e^1)^a  \\
    &+\bigl(\omega_{212} - \omega_{221}\bigr) (e^2)^a
    +\bigl(\omega_{312} - \omega_{321}\bigr) (e^3)^a
    +\bigl(\omega_{412} - \omega_{421}\bigr) (e^4)^a  \\
    =&  (\epsilon + \bar{\epsilon}) n^a
    +(\gamma + \bar{\gamma}) l^a
    +(-\tau - \bar{\pi}) \overline{m}^a
    +(-\bar{\tau} - \pi) m^a
\end{align*}
将上式与联络$\nabla_a$缩并,便可得到下列第一式,
\begin{subequations}\label{chnull:eqn_DDcomunicator}
    \begin{align}
        \Delta {\rm D}-{\rm D}\Delta =&(\gamma +\bar{\gamma}){\rm D}+(\epsilon +\bar{\epsilon})\Delta
        -(\bar{\tau}+\pi )\updelta -(\tau +\bar{\pi})\bar{\updelta},
        \label{chnull:eqn_DDcomunicator-a}\\
        \updelta {\rm D}-{\rm D}\updelta =&(\bar{\alpha}+\beta -\bar{\pi}){\rm D}+\kappa\Delta
        -(\bar{\rho}+\epsilon -\bar{\epsilon})\updelta -\sigma\bar{\updelta},
        \label{chnull:eqn_DDcomunicator-b}\\
        \updelta \Delta-\Delta \updelta =&-\bar{\nu}{\rm D}+(\tau -\bar{\alpha} -\beta )\Delta
        +(\mu -\gamma +\bar{\gamma})\updelta +\bar{\lambda}\bar{\updelta },
        \label{chnull:eqn_DDcomunicator-c}\\
        \bar{\updelta}\updelta-\updelta \bar{\updelta} =&(\bar{\mu}-\mu ){\rm D}+(\bar{\rho} -\rho )\Delta
        +(\alpha -\bar{\beta})\updelta -(\bar{\alpha}-\beta )\bar{\updelta} .
        \label{chnull:eqn_DDcomunicator-d}
    \end{align}
\end{subequations}
通过类似方法,可以得到剩余三式.

\index[physwords]{NP方程}

\subsection{NP方程}
把第二嘉当结构方程在类光标架上展开得到36个方程,称之为NP方程,是NP型式理论的核心.
因命题\ref{chnull:thm_NP-conj-34},只需写出如下18个(与NP原论文相同),具体推导见后.
\begin{small}
\setlength{\mathindent}{0em}
\begin{subequations}\label{chnull:eqn_NP-all}
    \begin{align}
        &{\rm D}\tau -\Delta \kappa =  \rho(\tau +\bar{\pi }) +\sigma(\bar{\tau }+\pi )
        +\tau(\epsilon -\bar{\epsilon }) - \kappa(3\gamma +\bar{\gamma } )
        +\Psi _{1} +\Phi_{01},   & [R_{1312}] \label{chnull:eqn_NP-all-R1312} \\
        &{\rm D}\sigma -\updelta \kappa = \sigma(\rho +\bar{\rho } + 3\epsilon - \bar{\epsilon })
        -\kappa(\tau -\bar{\pi }+\bar{\alpha } +3\beta ) +\Psi_{0},
        &[R_{1313}] \label{chnull:eqn_NP-all-R1313} \\
        &{\rm D}\rho -\bar{\updelta }\kappa =  \rho ^{2}+\sigma \bar{\sigma }
        + \rho(\epsilon+\bar{\epsilon }) -\bar{\kappa }\tau
        -\kappa (3\alpha +\bar{\beta }-\pi ) +\Phi_{00},
        &[R_{1314}] \label{chnull:eqn_NP-all-R1314}  \\
        &\Delta \rho -\bar{\updelta }\tau = \tau (\bar{\beta }
        -\alpha -\bar{\tau })+\rho(\gamma +\bar{\gamma })-(\rho \bar{\mu }+\sigma \lambda )
        +\kappa \nu  -\Psi_{2} - \tfrac{1}{12}R,
        & [R_{1324}] \label{chnull:eqn_NP-all-R1324} \\
        &\updelta \tau -\Delta \sigma =\mu \sigma +\rho \bar{\lambda }+\tau (\tau +\beta -\bar{\alpha })
        -\sigma (3\gamma -\bar{\gamma } )-\kappa \bar{\nu } + \Phi_{02},
        & [R_{1332}] \label{chnull:eqn_NP-all-R1332} \\
        &{\rm D}\nu -\Delta \pi = \mu(\bar{\tau }+\pi ) + \lambda(\tau +\bar{\pi } )
        + \pi(\gamma -\bar{\gamma }) - \nu(3\epsilon +\bar{\epsilon } )
         + \Psi _{3} +\Phi_{21},   & [R_{2421}] \label{chnull:eqn_NP-all-R2421} \\
        &\updelta \nu -\Delta \mu =\mu ^{2}+\lambda \bar{\lambda }+\mu (\gamma + \bar{\gamma })
        -\bar{\nu }\pi +\nu (\tau -3\beta -\bar{\alpha }) + \Phi_{22},
        & [R_{2423}] \label{chnull:eqn_NP-all-R2423} \\
        &{\rm D}\mu -\updelta \pi =\bar{\rho }\mu +\sigma \lambda +\pi \bar{\pi }
        - \mu(\epsilon +\bar{\epsilon }) -\pi (\bar{\alpha }-\beta )
        - \nu \kappa +\Psi_{2}+ \tfrac{1}{12}R,
        & [R_{2431}] \label{chnull:eqn_NP-all-R2431} \\
        &{\rm D}\lambda -\bar{\updelta }\pi =\rho \lambda +\bar{\sigma }\mu +\pi ^{2}
        +\pi (\alpha -\bar{\beta }) -\nu \bar{\kappa }
        -\lambda (3\epsilon -\bar{\epsilon }) +\Phi_{20}  ,
        & [R_{2441}] \label{chnull:eqn_NP-all-R2441} \\
        &\Delta \lambda -\bar{\updelta }\nu =-\lambda (\mu +\bar{\mu } + 3\gamma -\bar{\gamma })
        + \nu (3\alpha +\bar{\beta }+\pi -\bar{\tau }) -\Psi _{4} ,
        & [R_{2442}] \label{chnull:eqn_NP-all-R2442} \\
        &\updelta \lambda -\bar{\updelta }\mu = \nu(\rho -\bar{\rho }) + \pi(\mu - \bar{\mu })
        +\mu (\alpha +\bar{\beta })+\lambda (\bar{\alpha }-3\beta )
        -\Psi _{3} + \Phi_{21}, & [R_{2443}] \label{chnull:eqn_NP-all-R2443} \\
        &\updelta \rho -\bar{\updelta }\sigma =\rho (\bar{\alpha }+\beta)-\sigma (3\alpha -\bar{\beta })
        +\tau(\rho -\bar{\rho }) +\kappa (\mu -\bar{\mu })
         -\Psi _{1} + \Phi_{01}, & [R_{3143}] \label{chnull:eqn_NP-all-R3143}
    \end{align}
    \begin{align}
        {\rm D}\alpha -\bar{\updelta }\epsilon = &   \alpha(\rho +\bar{\epsilon }-2\epsilon )
        +\beta \bar{\sigma }-\bar{\beta }\epsilon -\kappa \lambda \notag \\
        &-\bar{\kappa }\gamma +\pi (\epsilon +\rho ) +\Phi_{10},
        &[\tfrac{1}{2}(R_{3414}-R_{1214})] \label{chnull:eqn_NP-all-R3414-1214} \\
        {\rm D}\beta -\updelta \epsilon =& \sigma(\alpha +\pi ) + \beta(\bar{\rho }
        -\bar{\epsilon }) -\kappa(\mu +\gamma ) \notag \\
        &- \epsilon(\bar{\alpha }-\bar{\pi } ) +\Psi _{1},
        &[\tfrac{1}{2}(R_{1213}-R_{3413})] \label{chnull:eqn_NP-all-R1213-3413} \\
        {\rm D}\gamma -\Delta \epsilon =& \alpha(\tau +\bar{\pi }) + \beta(\bar{\tau }+\pi )
        - \gamma(\epsilon +\bar{\epsilon }) - \epsilon(\gamma +\bar{\gamma }) & \notag \\
        &+ \tau \pi -\nu \kappa +\Psi _{2}+\Phi_{11}-\frac{1}{24}R,
        & [\tfrac{1}{2}(R_{1212}-R_{3412})] \label{chnull:eqn_NP-all-R1212-3412} \\
        \updelta \alpha -\bar{\updelta }\beta =&\mu \rho -\lambda \sigma +\alpha\bar{\alpha }
        +\beta \bar{\beta }-2\alpha \beta +\gamma (\rho -\bar{\rho }) & \notag \\
        & +\epsilon (\mu -\bar{\mu })-\Psi _{2} + \Phi_{11} +\frac{1}{24}R,
        &[\tfrac{1}{2}(R_{1234}-R_{3434})] \label{chnull:eqn_NP-all-R1234-3434} \\
        \updelta \gamma -\Delta \beta =& \gamma (\tau -\bar{\alpha }-\beta )+\mu \tau -\sigma \nu
        -\epsilon \bar{\nu } \notag \\
        &-\beta (\gamma -\bar{\gamma } -\mu )  +\alpha \bar{\lambda } + \Phi_{12},
        & [\tfrac{1}{2}(R_{1232}-R_{3432})]  \label{chnull:eqn_NP-all-R1232-3432} \\
        \Delta \alpha -\bar{\updelta }\gamma =&\nu (\rho +\epsilon )-\lambda (\tau +\beta )
        +\alpha (\bar{\gamma }-\bar{\mu }) \notag \\
        &+\gamma (\bar{ \beta }-\bar{\tau })-\Psi _{3}.
        & [\tfrac{1}{2}(R_{1242}-R_{3442})] \label{chnull:eqn_NP-all-R1242-3442}
    \end{align}
\end{subequations} \setlength{\mathindent}{2em} 
\end{small}
每个公式后面的方括号内已指明此NP方程是由哪个黎曼分量所产生的.

\subsubsection{黎曼曲率分量}
NP方程的推导过程是:首先,由\S\ref{chrg:sec_Riemann_rr}节中的
第二嘉当结构方程\eqref{chrg:eqn_Riemann_rr}得到
黎曼曲率$R_{\rho\sigma\mu\nu}$的Ricci旋转系数表示方式,通过代换
变成NP自旋系数形式;
其次,通过式\eqref{chnull:eqn_Riemann-by-Weyl-Ricc-all}得到第二种表示方式,
由Weyl标量、Ricci标量等表出.
两者必然相等,合并同类项后得到18个NP方程.

下面以两个分量为例,给出计算过程.
\paragraph{黎曼曲率分量$R_{1212}$:}
由式\eqref{chrg:eqn_Riemann_rr}计算
\begin{align*}
    R_{1212} =& \nabla_{1}\omega_{122}
    -\nabla_{2} \omega_{121}
    + g^{\xi\zeta} \bigl[ \omega_{12\zeta} (\omega_{2\xi 1}
    -\omega_{1\xi 2}  ) + \omega _{\zeta 22} \omega _{1\xi 1}
    - \omega _{\zeta 21} \omega _{1\xi 2} \bigr] \\
    =& {\rm D}(\gamma + \bar{\gamma}) -{\Delta} (\epsilon+\bar{\epsilon})
    + g^{12} \bigl[ \omega_{122} (\omega_{211}
    -\omega_{112}  ) + \omega _{222} \omega _{111}
    - \omega _{221} \omega _{112} \bigr] \\
    &+ g^{21} \bigl[ \omega_{121} (\omega_{221}
    -\omega_{122}  ) + \omega _{122} \omega _{121}
    - \omega _{121} \omega _{122} \bigr] \\
    &+ g^{34} \bigl[ \omega_{124} (\omega_{231}
    -\omega_{132}  ) + \omega _{422} \omega _{131}
    - \omega _{421} \omega _{132} \bigr] \\
    &+ g^{43} \bigl[ \omega_{123} (\omega_{241}
    -\omega_{142}  ) + \omega _{322} \omega _{141}
    - \omega _{321} \omega _{142} \bigr] \\
    =& {\rm D}(\gamma + \bar{\gamma}) -{\Delta} (\epsilon+\bar{\epsilon})
    + 2\omega_{121}  \omega_{122} \\
    &+ \omega_{124} (\omega_{231} -\omega_{132}  )
    + \omega _{422} \omega _{131}
    - \omega _{421} \omega _{132}  \\
    &+ \omega_{123} (\omega_{241} -\omega_{142}  )
    + \omega _{322} \omega _{141}
    - \omega _{321} \omega _{142}  \\
    =& {\rm D}(\gamma + \bar{\gamma}) -{\Delta} (\epsilon+\bar{\epsilon})
    + 2(\epsilon+\bar{\epsilon}) (\gamma + \bar{\gamma}) \\
    &+ (\alpha+\bar{\beta}) (-\bar{\pi} -\tau  )   + \nu \kappa  - \pi \tau
    + (\bar{\alpha}+\beta) (-\pi -\bar{\tau}  )
    + \bar{\nu} \bar{\kappa} - \bar{\pi} \bar{\tau}
\end{align*}
其次,还可以表示为式\eqref{chnull:eqn_Riemann-by-Weyl-Ricc-R1212}的形式,
两者必然相等,所以有
\begin{equation}\label{chnull:eqn_npR1212}
    \begin{aligned}
        \Psi_{2} &+ \bar{\Psi}_2 + 2\Phi_{11} -\frac{1}{12}R =
        {\rm D}(\gamma + \bar{\gamma}) -{\Delta} (\epsilon+\bar{\epsilon})
        + 2(\epsilon+\bar{\epsilon}) (\gamma + \bar{\gamma}) \\
        &- (\alpha+\bar{\beta}) (\bar{\pi} +\tau  )   + \nu \kappa  - \pi \tau
        - (\bar{\alpha}+\beta) (\pi +\bar{\tau}  )
        + \bar{\nu} \bar{\kappa} - \bar{\pi} \bar{\tau}
    \end{aligned}
\end{equation}

%\paragraph{黎曼曲率分量$R_{1213}$:}
%由式\eqref{chrg:eqn_Riemann_rr}计算
%\begin{align*}
%    R_{1213} =& \nabla_{1}\omega_{123} -\nabla_{3} \omega_{121}
%             + g^{12} \bigl[ \omega_{122} (\omega_{311} -\omega_{113}  )
%              + \omega _{223} \omega _{111} - \omega _{221} \omega _{113} \bigr] \\
%             &+ g^{21} \bigl[ \omega_{121} (\omega_{321}-\omega_{123}  )
%              + \omega _{123} \omega _{121}- \omega _{121} \omega _{123} \bigr] \\
%             &+ g^{34} \bigl[ \omega_{124} (\omega_{331}-\omega_{133}  )
%              + \omega _{423} \omega _{131}- \omega _{421} \omega _{133} \bigr] \\
%             &+ g^{43} \bigl[ \omega_{123} (\omega_{ 341}-\omega_{143}  )
%              + \omega _{323} \omega _{141} - \omega _{321} \omega _{143} \bigr] \\
%         =& {\rm D}(\bar{\alpha}+\beta) - \updelta (\epsilon+\bar{\epsilon})
%              +\kappa (\gamma+\bar{\gamma})
%             - (\epsilon+\bar{\epsilon}) (\bar{\pi}-\bar{\alpha}-\beta ) \\
%             &- ({\alpha}+\bar{\beta}) \sigma
%             + \mu \kappa - \pi \sigma
%             + (\bar{\alpha}+\beta) (\bar{\epsilon}-\epsilon-\bar{\rho}  )
%             + \bar{\lambda} \bar{\kappa} - \bar{\pi} \bar{\rho}
%\end{align*}
%由式\eqref{chnull:eqn_Riemann-by-Weyl-Ricc-all}得到另外
%一种表示$R_{1213} = \Psi_{1} + \Phi_{01}$,综合两式得
%\begin{equation}\label{chnull:eqn_npR1213}
%\begin{aligned}
%    \Psi_{1} + \Phi_{01}    =& {\rm D}(\bar{\alpha}+\beta) - \updelta (\epsilon+\bar{\epsilon})
%    +\kappa (\gamma+\bar{\gamma})
%    - (\epsilon+\bar{\epsilon}) (\bar{\pi}-\bar{\alpha}-\beta ) \\
%    &- ({\alpha}+\bar{\beta}) \sigma
%    + \mu \kappa - \pi \sigma
%    + (\bar{\alpha}+\beta) (\bar{\epsilon}-\epsilon-\bar{\rho}  )
%    + \bar{\lambda} \bar{\kappa} - \bar{\pi} \bar{\rho}
%\end{aligned}
%\end{equation}
%
%\paragraph{黎曼曲率分量$R_{1223}$:}
%由式\eqref{chrg:eqn_Riemann_rr}计算
%\begin{align*}
%   R_{1223} = & \nabla_{2}\omega_{123} -\nabla_{3} \omega_{122}
%      + g^{12} \bigl[ \omega_{122} (\omega_{312}-\omega_{213}  )
%        + \omega_{223} \omega_{112}- \omega_{222} \omega_{113} \bigr] \\
%      &+ g^{21} \bigl[ \omega_{121} (\omega_{322}-\omega_{223}  )
%        + \omega_{123} \omega_{122}- \omega_{122} \omega_{123} \bigr] \\
%      &+ g^{34} \bigl[ \omega_{124} (\omega_{332}-\omega_{233}  )
%        + \omega_{423} \omega_{132}- \omega_{422} \omega_{133} \bigr] \\
%      &+ g^{43} \bigl[ \omega_{123} (\omega_{342}-\omega_{243}  )
%        + \omega_{323} \omega_{142}- \omega_{322} \omega_{143} \bigr] \\
%    = & \Delta(\bar{\alpha}+{\beta}) -\updelta (\gamma+\bar{\gamma})
%        +(\gamma+\bar{\gamma})  (\tau - \bar{\alpha}- {\beta}  )
%       -(\epsilon+\bar{\epsilon}) \bar{\nu}  \\
%      &+ (\bar{\beta}+\alpha) \bar{\lambda}
%      + \mu \tau- \nu \sigma
%       + (\bar{\alpha}+{\beta}) (\bar{\gamma}-\gamma+\mu  )
%      + \bar{\lambda} \tau - \bar{\nu} \bar{\rho}
%\end{align*}
%由式\eqref{chnull:eqn_Riemann-by-Weyl-Ricc-all}得到另外
%一种表示$R_{1223} = -\bar{\Psi}_{3} -\Phi_{12} $,综合两式得
%\begin{equation}\label{chnull:eqn_npR1223}
%    \begin{aligned}
    %        -\bar{\Psi}_{3} -\Phi_{12} =& \Delta(\bar{\alpha}+{\beta}) -\updelta (\gamma+\bar{\gamma})
    %        +(\gamma+\bar{\gamma})  (\tau - \bar{\alpha}- {\beta}  )
    %        -(\epsilon+\bar{\epsilon}) \bar{\nu}  \\
    %        &+ (\bar{\beta}+\alpha) \bar{\lambda}
    %        + \mu \tau- \nu \sigma
    %        + (\bar{\alpha}+{\beta}) (\bar{\gamma}-\gamma+\mu  )
    %        + \bar{\lambda} \tau - \bar{\nu} \bar{\rho}
    %    \end{aligned}
%\end{equation}
%
%
%\paragraph{黎曼曲率分量$R_{1234}$:}
%由式\eqref{chrg:eqn_Riemann_rr}计算
%\begin{align*}
%%    R_{1234} =& \nabla_{3}\omega_{124} -\nabla_{4} \omega_{123}
%%     + g^{12} \bigl[ \omega_{122} (\omega_{413}-\omega_{314}  )
%%    + \omega_{224} \omega_{113}- \omega_{223} \omega_{114} \bigr] \\
%%    &+ g^{21} \bigl[ \omega_{121} (\omega_{423}-\omega_{324}  )
%%    + \omega_{124} \omega_{123}- \omega_{123} \omega_{124} \bigr] \\
%%    &+ g^{34} \bigl[ \omega_{124} (\omega_{433}-\omega_{334}  )
%%    + \omega_{424} \omega_{133}- \omega_{423} \omega_{134} \bigr] \\
%%    &+ g^{43} \bigl[ \omega_{123} (\omega_{443}-\omega_{344}  )
%%    + \omega_{324} \omega_{143}- \omega_{323} \omega_{144} \bigr] \\
%    R_{1234}=& \updelta(\bar{\beta}+\alpha) -\bar{\updelta} (\bar{\alpha}+{\beta})
%      - (\gamma+\bar{\gamma}) (\rho -\bar{\rho} )
%      - (\epsilon+\bar{\epsilon}) (\mu-\bar{\mu}  )  \\
%    & + (\bar{\beta}+\alpha) ({\beta}-\bar{\alpha})
%    + \lambda \sigma - \mu\rho
%    - (\bar{\alpha}+{\beta}) (\bar{\beta}-\alpha)
%    + \bar{\mu} \bar{\rho}- \bar{\lambda} \bar{\sigma}
%\end{align*}
%由式\eqref{chnull:eqn_Riemann-by-Weyl-Ricc-all}得到另外
%一种表示$R_{1234} = \bar{\Psi}_2-\Psi_2 $,综合两式得
%\begin{equation}\label{chnull:eqn_npR1234}
%    \begin{aligned}
    %       \bar{\Psi}_2-\Psi_2 =& \updelta(\bar{\beta}+\alpha) -\bar{\updelta} (\bar{\alpha}+{\beta})
    %       - (\gamma+\bar{\gamma}) (\rho -\bar{\rho} )
    %       - (\epsilon+\bar{\epsilon}) (\mu-\bar{\mu}  )  \\
    %       & + (\bar{\beta}+\alpha) ({\beta}-\bar{\alpha})
    %       + \lambda \sigma - \mu\rho
    %       - (\bar{\alpha}+{\beta}) (\bar{\beta}-\alpha)
    %       + \bar{\mu} \bar{\rho}- \bar{\lambda} \bar{\sigma}
    %    \end{aligned}
%\end{equation}

%
%\paragraph{黎曼曲率分量$R_{1312}$:}
%由式\eqref{chrg:eqn_Riemann_rr}计算
%\begin{align*}
%    R_{1312} %= & \nabla_{1}\omega_{132} -\nabla_{2} \omega_{131}
%%    + g^{12} \bigl[ \omega_{132} (\omega_{211}-\omega_{112}  )
%%    + \omega_{232} \omega_{111}- \omega_{231} \omega_{112} \bigr] \\
%%    &+ g^{21} \bigl[ \omega_{131} (\omega_{221}-\omega_{122}  )
%%    + \omega_{132} \omega_{121}- \omega_{131} \omega_{122} \bigr] \\
%%    &+ g^{34} \bigl[ \omega_{134} (\omega_{231}-\omega_{132}  )
%%    + \omega_{432} \omega_{131}- \omega_{431} \omega_{132} \bigr] \\
%%    &+ g^{43} \bigl[ \omega_{133} (\omega_{241}-\omega_{142}  )
%%    + \omega_{332} \omega_{141}- \omega_{331} \omega_{142} \bigr] \\
%    = {\rm D}\tau -\Delta \kappa
%      -\tau (\epsilon-\bar{\epsilon})  +  \rho (-\bar{\pi}-\tau )
%    + \kappa(\gamma-\bar{\gamma} + 2\gamma+2\bar{\gamma})
%    + \sigma (-\pi-\bar{\tau}  )
%\end{align*}
%由式\eqref{chnull:eqn_Riemann-by-Weyl-Ricc-all}得到另外
%一种表示$R_{1312} = \Psi_{1} + \Phi_{01}  $,
%综合两式立即可得式\eqref{chnull:eqn_NP-all-R1312}.
%
%
%
%\paragraph{黎曼曲率分量$R_{1313}$:}
%由式\eqref{chrg:eqn_Riemann_rr}计算
%\begin{align*}
%    R_{1313} %=& \nabla_{1}\omega_{133} -\nabla_{3} \omega_{131} \\
%%    &+ g^{12} \bigl[ \omega_{132} (\omega_{311}-\omega_{113}  )
%%    + \omega_{233} \omega_{111}- \omega_{231} \omega_{113} \bigr] \\
%%    &+ g^{21} \bigl[ \omega_{131} (\omega_{321}-\omega_{123}  )
%%    + \omega_{133} \omega_{121}- \omega_{131} \omega_{123} \bigr] \\
%%    &+ g^{34} \bigl[ \omega_{134} (\omega_{331}-\omega_{133}  )
%%    + \omega_{433} \omega_{131}- \omega_{431} \omega_{133} \bigr] \\
%%    &+ g^{43} \bigl[ \omega_{133} (\omega_{341}-\omega_{143}  )
%%    + \omega_{333} \omega_{141}- \omega_{331} \omega_{143} \bigr] \\
%    = {\rm D}\sigma -\updelta \kappa -\kappa(\bar{\pi}-\tau-\bar{\alpha}-3\beta)
%      -\sigma (3\epsilon-\bar{\epsilon}+\bar{\rho}+\rho)
%\end{align*}
%由式\eqref{chnull:eqn_Riemann-by-Weyl-Ricc-all}得到另外
%一种表示$R_{1313} = \Psi_{0}$,
%综合两式立即可得式\eqref{chnull:eqn_NP-all-R1313}.
%
%
%\paragraph{黎曼曲率分量$R_{1314}$:}
%由式\eqref{chrg:eqn_Riemann_rr}计算
%\begin{align*}
%    R_{1314}
%%    =&   \nabla_{1} \omega_{134} - \nabla_{4} \omega_{131} \\
%%     & + g^{12} \bigl[ \omega_{132} (\omega_{411}-\omega_{114}  )
%%       + \omega _{234} \omega _{111}- \omega _{231} \omega _{114} \bigr] \\
%%     & + g^{21} \bigl[ \omega_{131} (\omega_{421}-\omega_{124}  )
%%       + \omega _{134} \omega _{121}- \omega _{131} \omega _{124} \bigr] \\
%%     & + g^{34} \bigl[ \omega_{134} (\omega_{431}-\omega_{134}  )
%%       + \omega _{434} \omega _{131}- \omega _{431} \omega _{134} \bigr] \\
%%     & + g^{43} \bigl[ \omega_{133} (\omega_{441}-\omega_{144}  )
%%       + \omega _{334} \omega _{141}- \omega _{331} \omega _{144} \bigr] \\
%    =  {\rm D} \rho - \bar{\updelta} \kappa
%     - \kappa (\pi-\bar{\beta}-3\alpha)
%     - \rho (\epsilon+\bar{\epsilon} + \rho)
%     + \tau \bar{\kappa} - \sigma \bar{\sigma}
%\end{align*}
%由式\eqref{chnull:eqn_Riemann-by-Weyl-Ricc-all}得到另外
%一种表示$R_{1314} = \Phi_{00}$,
%综合两式立即可得式\eqref{chnull:eqn_NP-all-R1314}.
%
%
%\paragraph{黎曼曲率分量$R_{1324}$:}
%由式\eqref{chrg:eqn_Riemann_rr}计算
%\begin{align*}
%    R_{1324} %=&
%%    \nabla_{2} \omega_{134} - \nabla_{4} \omega_{132} \\
%%    & + g^{12} \bigl[ \omega_{132} (\omega_{412}-\omega_{214}  )
%%    + \omega _{234} \omega _{112}- \omega _{232} \omega _{114} \bigr] \\
%%    & + g^{21} \bigl[ \omega_{131} (\omega_{422}-\omega_{224}  )
%%    + \omega _{134} \omega _{122}- \omega _{132} \omega _{124} \bigr] \\
%%    & + g^{34} \bigl[ \omega_{134} (\omega_{432}-\omega_{234}  )
%%    + \omega _{434} \omega _{132}- \omega _{432} \omega _{134} \bigr] \\
%%    & + g^{43} \bigl[ \omega_{133} (\omega_{442}-\omega_{244}  )
%%    + \omega _{334} \omega _{142}- \omega _{332} \omega _{144} \bigr] \\
%    = \Delta \rho - \bar{\updelta} \tau
%     + \tau(\alpha-\bar{\beta} + \bar{\tau})
%     - \kappa \nu
%     - \rho(-\bar{\mu} + \gamma + \bar{\gamma})
%     + \sigma \lambda
%\end{align*}
%由式\eqref{chnull:eqn_Riemann-by-Weyl-Ricc-all}得到另外
%一种表示$R_{1324} = -\Psi_{2} - \frac{1}{12} R$,
%综合两式立即可得式\eqref{chnull:eqn_NP-all-R1324}.
%
%
%\paragraph{黎曼曲率分量$R_{1332}$:}
%由式\eqref{chrg:eqn_Riemann_rr}计算
%\begin{align*}
%    -R_{1332}
%%    =& \nabla_{2}\omega_{133} -\nabla_{3} \omega_{132} \\
%%    &+ g^{12} \bigl[ \omega_{132} (\omega_{312}-\omega_{213}  )
%%    + \omega_{233} \omega_{112}- \omega_{232} \omega_{113} \bigr] \\
%%    &+ g^{21} \bigl[ \omega_{131} (\omega_{322}-\omega_{223}  )
%%    + \omega_{133} \omega_{122}- \omega_{132} \omega_{123} \bigr] \\
%%    &+ g^{34} \bigl[ \omega_{134} (\omega_{332}-\omega_{233}  )
%%    + \omega_{433} \omega_{132}- \omega_{432} \omega_{133} \bigr] \\
%%    &+ g^{43} \bigl[ \omega_{133} (\omega_{342}-\omega_{243}  )
%%    + \omega_{333} \omega_{142}- \omega_{332} \omega_{143} \bigr] \\
%    =& \Delta \sigma -\updelta \tau
%     + \tau ( \beta-\bar{\alpha} + \tau)
%     - \kappa \bar{\nu} + \rho \bar{\lambda}
%     + \sigma (\bar{\gamma}-3\gamma +\mu)
%\end{align*}
%由式\eqref{chnull:eqn_Riemann-by-Weyl-Ricc-all}得到另外
%一种表示$R_{1332} = \Phi_{02}  $,综合两式得
%综合两式立即可得式\eqref{chnull:eqn_NP-all-R1332}.
%
%
%
%
%\paragraph{黎曼曲率分量$R_{1334}$:}
%由式\eqref{chrg:eqn_Riemann_rr}计算
%\begin{align*}
%    R_{1334}
%%    =& \nabla_{3}\omega_{134} -\nabla_{4} \omega_{133} \\
%%    &+ g^{12} \bigl[ \omega_{132} (\omega_{413}-\omega_{314}  )
%%    + \omega_{234} \omega_{113}- \omega_{233} \omega_{114} \bigr] \\
%%    &+ g^{21} \bigl[ \omega_{131} (\omega_{423}-\omega_{324}  )
%%    + \omega_{134} \omega_{123}- \omega_{133} \omega_{124} \bigr] \\
%%    &+ g^{34} \bigl[ \omega_{134} (\omega_{433}-\omega_{334}  )
%%    + \omega_{434} \omega_{133}- \omega_{433} \omega_{134} \bigr] \\
%%    &+ g^{43} \bigl[ \omega_{133} (\omega_{443}-\omega_{344}  )
%%    + \omega_{334} \omega_{143}- \omega_{333} \omega_{144} \bigr] \\
%     = \updelta \rho -\bar{\updelta} \sigma   + \tau \bar{\rho}
%     - \kappa (\mu-\bar{\mu} ) - \rho (\tau + \bar{\alpha}+\beta )
%     + \sigma (3\alpha -\bar{\beta} )
%\end{align*}
%由式\eqref{chnull:eqn_Riemann-by-Weyl-Ricc-all}得到另外
%一种表示$R_{1334} = -\Psi_{1} + \Phi_{01} $,
%综合两式立即可得式\eqref{chnull:eqn_NP-all-R3143}.
%
%
%
%
%\paragraph{黎曼曲率分量$R_{2421}$:}
%由式\eqref{chrg:eqn_Riemann_rr}计算
%\begin{align*}
%    -R_{2421}
%%    =& \nabla_{1}\omega_{242} -\nabla_{2} \omega_{241} \\
%%    &+ g^{12} \bigl[ \omega_{242} (\omega_{211}-\omega_{112}  )
%%    + \omega_{242} \omega_{211}- \omega_{241} \omega_{212} \bigr] \\
%%    &+ g^{21} \bigl[ \omega_{241} (\omega_{221}-\omega_{122}  )
%%    + \omega_{142} \omega_{221}- \omega_{141} \omega_{222} \bigr] \\
%%    &+ g^{34} \bigl[ \omega_{244} (\omega_{231}-\omega_{132}  )
%%    + \omega_{442} \omega_{231}- \omega_{441} \omega_{232} \bigr] \\
%%    &+ g^{43} \bigl[ \omega_{243} (\omega_{241}-\omega_{142}  )
%%    + \omega_{342} \omega_{241}- \omega_{341} \omega_{242} \bigr] \\
%    = -{\rm D}\nu + \Delta \pi
%    +\pi (\gamma -\bar{\gamma}) -\nu(3\epsilon +\bar{\epsilon})
%    +\mu(\pi+\bar{\tau}) +\lambda (\bar{\pi}+\tau  )
%\end{align*}
%由式\eqref{chnull:eqn_Riemann-by-Weyl-Ricc-all}得到另外
%一种表示$R_{2421} = \Psi_{3} +\Phi_{21} $,
%综合两式立即可得式\eqref{chnull:eqn_NP-all-R2421}.
%
%
%
%\paragraph{黎曼曲率分量$R_{2431}$:}
%由式\eqref{chrg:eqn_Riemann_rr}计算
%\begin{align*}
%    -R_{2431}
%%    =& \nabla_{1}\omega_{243} -\nabla_{3} \omega_{241} \\
%%    &+ g^{12} \bigl[ \omega_{242} (\omega_{311}-\omega_{113}  )
%%    + \omega_{243} \omega_{211}- \omega_{241} \omega_{213} \bigr] \\
%%    &+ g^{21} \bigl[ \omega_{241} (\omega_{321}-\omega_{123}  )
%%    + \omega_{143} \omega_{221}- \omega_{141} \omega_{223} \bigr] \\
%%    &+ g^{34} \bigl[ \omega_{244} (\omega_{331}-\omega_{133}  )
%%    + \omega_{443} \omega_{231}- \omega_{441} \omega_{233} \bigr] \\
%%    &+ g^{43} \bigl[ \omega_{243} (\omega_{341}-\omega_{143}  )
%%    + \omega_{343} \omega_{241}- \omega_{341} \omega_{243} \bigr] \\
%     =-{\rm D} \mu +\updelta \pi -\nu\kappa
%     +\pi(\beta-\bar{\alpha} + \bar{\pi}) +\lambda \sigma
%     +\mu (\bar{\rho} - \epsilon-\bar{\epsilon} )
%\end{align*}
%由式\eqref{chnull:eqn_Riemann-by-Weyl-Ricc-all}得到另外
%一种表示$R_{2431} = \Psi_{2} + \frac{1}{12} R  $,
%综合两式立即可得式\eqref{chnull:eqn_NP-all-R2431}.
%
%
%
%
%\paragraph{黎曼曲率分量$R_{2423}$:}
%由式\eqref{chrg:eqn_Riemann_rr}计算
%\begin{align*}
%    R_{2423}
%%    =& \nabla_{2}\omega_{243} -\nabla_{3} \omega_{242} \\
%%    &+ g^{12} \bigl[ \omega_{242} (\omega_{312}-\omega_{213}  )
%%    + \omega_{243} \omega_{212}- \omega_{242} \omega_{213} \bigr] \\
%%    &+ g^{21} \bigl[ \omega_{241} (\omega_{322}-\omega_{223}  )
%%    + \omega_{143} \omega_{222}- \omega_{142} \omega_{223} \bigr] \\
%%    &+ g^{34} \bigl[ \omega_{244} (\omega_{332}-\omega_{233}  )
%%    + \omega_{443} \omega_{232}- \omega_{442} \omega_{233} \bigr] \\
%%    &+ g^{43} \bigl[ \omega_{243} (\omega_{342}-\omega_{243}  )
%%    + \omega_{343} \omega_{242}- \omega_{342} \omega_{243} \bigr] \\
%    = -\Delta \mu + \updelta \nu - \nu ( \tau - \bar{\alpha}-3\beta  )
%    + \pi \bar{\nu} - \lambda \bar{\lambda}- \mu (\mu + \gamma + \bar{\gamma} )
%\end{align*}
%由式\eqref{chnull:eqn_Riemann-by-Weyl-Ricc-all}得到另外
%一种表示$R_{2423} = \Phi_{22} $,
%综合两式立即可得式\eqref{chnull:eqn_NP-all-R2423}.
%
%
%
%\paragraph{黎曼曲率分量$R_{2441}$:}
%由式\eqref{chrg:eqn_Riemann_rr}计算
%\begin{align*}
%%    -R_{2441} =& \nabla_{1}\omega_{244} -\nabla_{4} \omega_{241} \\
%%    &+ g^{12} \bigl[ \omega_{242} (\omega_{411}-\omega_{114}  )
%%    + \omega_{244} \omega_{211}- \omega_{241} \omega_{214} \bigr] \\
%%    &+ g^{21} \bigl[ \omega_{241} (\omega_{421}-\omega_{124}  )
%%    + \omega_{144} \omega_{221}- \omega_{141} \omega_{224} \bigr] \\
%%    &+ g^{34} \bigl[ \omega_{244} (\omega_{431}-\omega_{134}  )
%%    + \omega_{444} \omega_{231}- \omega_{441} \omega_{234} \bigr] \\
%%    &+ g^{43} \bigl[ \omega_{243} (\omega_{441}-\omega_{144}  )
%%    + \omega_{344} \omega_{241}- \omega_{341} \omega_{244} \bigr] \\
%    -R_{2441} = \nabla_{1}\omega_{244} -\nabla_{4} \omega_{241} \\
%&+ g^{12} \bigl[ \omega_{242} (\omega_{411}-\omega_{114}  )
%+ \omega_{244} \omega_{211}- \omega_{241} \omega_{214} \bigr] \\
%&+ g^{21} \bigl[ \omega_{241} (\omega_{421}-\omega_{124}  )
%+ \omega_{144} \omega_{221}- \omega_{141} \omega_{224} \bigr] \\
%&+ g^{34} \bigl[ \omega_{244} (\omega_{431}-\omega_{134}  )
%+ \omega_{444} \omega_{231}- \omega_{441} \omega_{234} \bigr] \\
%&+ g^{43} \bigl[ \omega_{243} (\omega_{441}-\omega_{144}  )
%+ \omega_{344} \omega_{241}- \omega_{341} \omega_{244} \bigr] \\
%\end{align*}
%由式\eqref{chnull:eqn_Riemann-by-Weyl-Ricc-all}得到另外
%一种表示$R_{2441} = \Phi_{20}$,
%综合两式立即可得式\eqref{chnull:eqn_NP-all-R2441}.
%
%
%
%
%\paragraph{黎曼曲率分量$R_{2442}$:}
%由式\eqref{chrg:eqn_Riemann_rr}计算
%\begin{align*}
%    -R_{2442} =& \nabla_{2}\omega_{244} -\nabla_{4} \omega_{242} \\
%    &+ g^{12} \bigl[ \omega_{242} (\omega_{412}-\omega_{214}  )
%    + \omega_{244} \omega_{212}- \omega_{242} \omega_{214} \bigr] \\
%    &+ g^{21} \bigl[ \omega_{241} (\omega_{422}-\omega_{224}  )
%    + \omega_{144} \omega_{222}- \omega_{142} \omega_{224} \bigr] \\
%    &+ g^{34} \bigl[ \omega_{244} (\omega_{432}-\omega_{234}  )
%    + \omega_{444} \omega_{232}- \omega_{442} \omega_{234} \bigr] \\
%    &+ g^{43} \bigl[ \omega_{243} (\omega_{442}-\omega_{244}  )
%    + \omega_{344} \omega_{242}- \omega_{342} \omega_{244} \bigr] \\
%\end{align*}
%由式\eqref{chnull:eqn_Riemann-by-Weyl-Ricc-all}得到另外
%一种表示$R_{2442} = - \Psi_{4}$,
%综合两式立即可得式\eqref{chnull:eqn_NP-all-R2442}.
%
%
%
%\paragraph{黎曼曲率分量$R_{2443}$:}
%由式\eqref{chrg:eqn_Riemann_rr}计算
%\begin{align*}
%    -R_{2443} =& \nabla_{3}\omega_{244} -\nabla_{4} \omega_{243} \\
%    &+ g^{12} \bigl[ \omega_{242} (\omega_{413}-\omega_{314}  )
%    + \omega_{244} \omega_{213}- \omega_{243} \omega_{214} \bigr] \\
%    &+ g^{21} \bigl[ \omega_{241} (\omega_{423}-\omega_{324}  )
%    + \omega_{144} \omega_{223}- \omega_{143} \omega_{224} \bigr] \\
%    &+ g^{34} \bigl[ \omega_{244} (\omega_{433}-\omega_{334}  )
%    + \omega_{444} \omega_{233}- \omega_{443} \omega_{234} \bigr] \\
%    &+ g^{43} \bigl[ \omega_{243} (\omega_{443}-\omega_{344}  )
%    + \omega_{344} \omega_{243}- \omega_{343} \omega_{244} \bigr] \\
%\end{align*}
%由式\eqref{chnull:eqn_Riemann-by-Weyl-Ricc-all}得到另外
%一种表示$R_{2443} = -{\Psi}_{2} + {\Phi}_{21} $,
%综合两式立即可得式\eqref{chnull:eqn_NP-all-R2443}.



\paragraph{黎曼曲率分量$R_{3412}$:}
由式\eqref{chrg:eqn_Riemann_rr}计算
\begin{align*}
    R_{3412}
    %    =& \nabla_{1}\omega_{342}
    %    -\nabla_{2} \omega_{341}
    %    + g^{\xi\zeta} \bigl[ \omega_{34\zeta} (\omega_{2\xi 1}
    %    -\omega_{1\xi 2}  ) + \omega _{\zeta 42} \omega _{3\xi 1}
    %    - \omega _{\zeta 41} \omega _{3\xi 2} \bigr] \\
    %    =& {\rm D} (\bar{\gamma}-\gamma) +\Delta (\epsilon-\bar{\epsilon})
    %    + g^{12} \bigl[ \omega_{342} (\omega_{211} -\omega_{112}  )
    %    + \omega _{242} \omega _{311}
    %    - \omega _{241} \omega _{312} \bigr] \\
    %    &+ g^{21} \bigl[ \omega_{341} (\omega_{221}-\omega_{122}  )
    %    + \omega _{142} \omega _{321}
    %    - \omega _{141} \omega _{322} \bigr] \\
    %    &+ g^{34} \bigl[ \omega_{344} (\omega_{231} -\omega_{132}  )
    %    + \omega _{442} \omega _{331}
    %    - \omega _{441} \omega _{332} \bigr] \\
    %    &+ g^{43} \bigl[ \omega_{343} (\omega_{241} -\omega_{142}  )
    %    + \omega _{342} \omega _{341}
    %    - \omega _{341} \omega _{342} \bigr] \\
    =& {\rm D} (\bar{\gamma}-\gamma) -\Delta (\bar{\epsilon}-\epsilon)
    + (\bar{\gamma}-\gamma) (\bar{\epsilon}+\epsilon)
    - \nu \kappa  + \tau\pi
    +  (\bar{\epsilon}-\epsilon) (\bar{\gamma}+\gamma)\\
    &- \bar{\tau} \bar{\pi} + \bar{\kappa} \bar{\nu}
    +  (\bar{\beta}-\alpha) (-\bar{\pi} -\tau  )
    +  (\bar{\alpha}-\beta) (-\pi -\bar{\tau}  )
\end{align*}
由\eqref{chnull:eqn_Riemann-by-Weyl-Ricc-all}得到
另一种表示$R_{3412}=R_{1432}-R_{1342}=\bar{\Psi}_{2} - \Psi_2 $,
两式必相等.
\begin{align}
        \bar{\Psi}_{2} &- \Psi_2  =
        {\rm D} (\bar{\gamma}-\gamma) -\Delta (\bar{\epsilon}-\epsilon)
        + (\bar{\gamma}-\gamma) (\bar{\epsilon}+\epsilon)
        - \nu \kappa  + \tau\pi
        +  (\bar{\epsilon}-\epsilon) (\bar{\gamma}+\gamma) \notag \\
        &- \bar{\tau} \bar{\pi} + \bar{\kappa} \bar{\nu}
        +  (\bar{\beta}-\alpha) (-\bar{\pi} -\tau  )
        +  (\bar{\alpha}-\beta) (-\pi -\bar{\tau}  ) \label{chnull:eqn_npR3412}
    \end{align}
现在可以计算$(R_{1212}-R_{3412})/2$,
即式\eqref{chnull:eqn_npR1212}减去\eqref{chnull:eqn_npR3412},
经过计算得\eqref{chnull:eqn_NP-all-R1212-3412}.

%
%\paragraph{黎曼曲率分量$R_{3413}$:}
%由式\eqref{chrg:eqn_Riemann_rr}计算,
%\begin{align*}
%    R_{3413} =& \nabla_{1}\omega_{343} -\nabla_{3} \omega_{341} \\
%    &+ g^{12} \bigl[ \omega_{342} (\omega_{311}-\omega_{113}  )
%    + \omega_{243} \omega_{311}- \omega_{241} \omega_{313} \bigr] \\
%    &+ g^{21} \bigl[ \omega_{341} (\omega_{321}-\omega_{123}  )
%    + \omega_{143} \omega_{321}- \omega_{141} \omega_{323} \bigr] \\
%    &+ g^{34} \bigl[ \omega_{344} (\omega_{331}-\omega_{133}  )
%    + \omega_{443} \omega_{331}- \omega_{441} \omega_{333} \bigr] \\
%    &+ g^{43} \bigl[ \omega_{343} (\omega_{341}-\omega_{143}  )
%    + \omega_{343} \omega_{341}- \omega_{341} \omega_{343} \bigr] \\
%\end{align*}
%由式\eqref{chnull:eqn_Riemann-by-Weyl-Ricc-all}得到另外
%一种表示$R_{3413} = -\Psi_{1} + \Phi_{01} $,综合两式得
%\begin{equation}\label{chnull:eqn_npR3413}
%    \begin{aligned}
    %        -\Psi_{1} + \Phi_{01} =& {\rm D}
    %    \end{aligned}
%\end{equation}
%
%
%
%
%\paragraph{黎曼曲率分量$R_{3432}$:}
%由式\eqref{chrg:eqn_Riemann_rr}计算,
%\begin{align*}
%    -R_{3432} =& \nabla_{2}\omega_{343} -\nabla_{3} \omega_{342} \\
%    &+ g^{12} \bigl[ \omega_{342} (\omega_{312}-\omega_{213}  )
%    + \omega_{243} \omega_{312}- \omega_{242} \omega_{313} \bigr] \\
%    &+ g^{21} \bigl[ \omega_{341} (\omega_{322}-\omega_{223}  )
%    + \omega_{143} \omega_{322}- \omega_{142} \omega_{323} \bigr] \\
%    &+ g^{34} \bigl[ \omega_{344} (\omega_{332}-\omega_{233}  )
%    + \omega_{443} \omega_{332}- \omega_{442} \omega_{333} \bigr] \\
%    &+ g^{43} \bigl[ \omega_{343} (\omega_{342}-\omega_{243}  )
%    + \omega_{343} \omega_{342}- \omega_{342} \omega_{343} \bigr] \\
%\end{align*}
%由式\eqref{chnull:eqn_Riemann-by-Weyl-Ricc-all}得到另外
%一种表示$R_{3432} = \bar{\Psi}_{2} - \Phi_{12} $,综合两式得
%\begin{equation}\label{chnull:eqn_npR3423}
%    \begin{aligned}
    %        \bar{\Psi}_{2} - \Phi_{12} =& {\rm D}
    %    \end{aligned}
%\end{equation}




%\paragraph{黎曼曲率分量$R_{3434}$:}
%由式\eqref{chrg:eqn_Riemann_rr}计算,
%\begin{align*}
%    R_{3434} =& \nabla_{3}\omega_{344} -\nabla_{4} \omega_{343} \\
%    &+ g^{12} \bigl[ \omega_{342} (\omega_{413}-\omega_{314}  )
%    + \omega_{244} \omega_{313}- \omega_{243} \omega_{314} \bigr] \\
%    &+ g^{21} \bigl[ \omega_{341} (\omega_{423}-\omega_{324}  )
%    + \omega_{144} \omega_{323}- \omega_{143} \omega_{324} \bigr] \\
%    &+ g^{34} \bigl[ \omega_{344} (\omega_{433}-\omega_{334}  )
%    + \omega_{444} \omega_{333}- \omega_{443} \omega_{334} \bigr] \\
%    &+ g^{43} \bigl[ \omega_{343} (\omega_{443}-\omega_{344}  )
%    + \omega_{344} \omega_{343}- \omega_{343} \omega_{344} \bigr] \\
%\end{align*}
%由式\eqref{chnull:eqn_Riemann-by-Weyl-Ricc-all}得到另外
%一种表示$R_{3434} =\Psi_{2} + \bar{\Psi}_2  -2\Phi_{11}- \frac{1}{12} R $,综合两式得
%\begin{equation}\label{chnull:eqn_npR3434}
%    \begin{aligned}
    %        \Psi_{2} + \bar{\Psi}_2  -2\Phi_{11}- \frac{1}{12} R =& {\rm D}
    %    \end{aligned}
%\end{equation}




\subsection{Bianchi恒等式}
第\ref{chrg:sec_NumOfBianchi}节指出Bianchi恒等式只有20个代数上独立的恒等式,它们是
\begin{subequations}
    \begin{align}
        R_{12[12;3]}=&0, & R_{12[12;4]}=&0, & R_{12[13;4]}=&0, & R_{12[23;4]}=&0;  \\
        R_{13[12;3]}=&0, & R_{13[12;4]}=&0, & R_{13[13;4]}=&0, & R_{13[23;4]}=&0;  \\
        R_{34[12;3]}=&0, & R_{34[12;4]}=&0, & R_{34[13;4]}=&0, & R_{34[23;4]}=&0;  \\
        R_{24[12;3]}=&0, & R_{24[12;4]}=&0, & R_{24[13;4]}=&0, & R_{24[23;4]}=&0.  \\
        R_{14[12;3]}=&0, & R_{14[12;4]}=&0, & R_{14[13;4]}=&0, & R_{14[23;4]}=&0; \label{chnull:eqn_tp1} \\
        R_{23[12;3]}=&0, & R_{23[12;4]}=&0, & R_{23[13;4]}=&0, & R_{23[23;4]}=&0. \label{chnull:eqn_tp2}
    \end{align}
\end{subequations}
根据命题\ref{chnull:thm_NP-conj-34}可不写出\eqref{chnull:eqn_tp1}和\eqref{chnull:eqn_tp2},
而用其它式子的复共轭表示;
$R_{12[12;4]}= 0$也可表示为$\bar{R}_{12[12;3]}=0$.
考虑到\S\ref{chrg:sec_NumOfBianchi}节中的那4个代数恒等式,在类光标架下上述方程最终
独立的方程是
\begin{subequations}
    \begin{align}
        R_{13[12;3]}=&0, & R_{13[12;4]}=&0, & R_{13[13;4]}=&0, & R_{13[23;4]}=&0;  \\
        R_{24[12;3]}=&0, & R_{24[12;4]}=&0, & R_{24[13;4]}=&0, & R_{24[23;4]}=&0; \\
        R_{34[13;4]}=&0, & R_{34[23;4]}=&0, & R_{12[12;3]}=&0. & {} &
    \end{align}
\end{subequations}
将它们写成分量方程\cite[\S 8e]{chandrasekhar-1983}\cite[\S 7.3]{stephani-exe-2003},为
\setlength{\mathindent}{0em} 
\begin{subequations}\label{chnull:eqn_Bianchi-4Q}
    \begin{align}
        \bar{\updelta }\Psi _{0}&-{\rm D}\Psi _{1} +{\rm D} \Phi_{01} -\updelta \Phi_{00}
        =(4\alpha -\pi )\Psi _{0}-2(2\rho +\epsilon )\Psi _{1} +3\kappa \Psi _{2} \notag \\
        &+(\bar{\pi}-2\bar{\alpha}-2\beta)\Phi_{00} + 2(\epsilon+\bar{\rho})\Phi_{01}
        +2\sigma\Phi_{10} - 2\kappa \Phi_{11} -\bar{\kappa}\Phi_{02}, \\
        \Delta \Psi_{0}&-\updelta \Psi_{1} + {\rm D} \Phi_{02} -\updelta \Phi_{01}
        =(4\gamma -\mu )\Psi_{0}-2(2\tau +\beta )\Psi _{1} +3\sigma \Psi _{2} \notag\\
        &+(2\epsilon -2\bar{\epsilon}+\bar{\rho}) \Phi_{02} + 2(\bar{\pi}-\beta)\Phi_{01}
        +2\sigma \Phi_{11} -2 \kappa \Phi_{12} -\bar{\lambda}\Phi_{00}, \\
        \Delta \Psi_{1}&-\updelta \Psi_{2} - {\Delta} \Phi_{01} + \bar{\updelta} \Phi_{02}
        -\frac{1}{12}{\updelta}R = \nu \Psi_{0} + 2(\gamma -\mu )\Psi_{1}
        -3\tau \Psi_{2} + 2\sigma \Psi _{3} \notag\\
        & + (\bar{\tau}-2\bar{\beta}+2\alpha)\Phi_{02} + 2(\bar{\mu}-\gamma)\Phi_{01}
        +2\tau\Phi_{11} - 2\rho\Phi_{12} -\bar{\nu}\Phi_{00}, \\
        {\rm D}\Psi _{2} &-\bar{\updelta }\Psi _{1}  +{\Delta} \Phi_{00} -\bar{\updelta} \Phi_{01}
        +\frac{1}{12}{\rm D}R = -\lambda \Psi _{0} - 2(\alpha -\pi )\Psi _{1}
        +3\rho \Psi _{2}-2\kappa \Psi _{3} \notag\\
        &+(2\gamma+2\bar{\gamma}-\bar{\mu})\Phi_{00} - 2(\bar{\tau}+\alpha)\Phi_{01}
        -2\tau \Phi_{10} + 2\rho\Phi_{11} +\bar{\sigma}\Phi_{02}, \\
        \bar{\updelta }\Psi_{3}&-{\rm D}\Psi_{4}  +\bar{\updelta} \Phi_{21} -\Delta \Phi_{20}
        =3\lambda \Psi _{2}-2(\alpha +2\pi)\Psi _{3}+(4\epsilon -\rho )\Psi _{4} \notag\\
        &+(2\gamma-2\bar{\gamma}+\bar{\mu})\Phi_{20} +2(\bar{\tau}-\alpha)\Phi_{21}
        +2\lambda\Phi_{11}-2\nu\Phi_{10}-\bar{\sigma}\Phi_{22}, \\
        \Delta \Psi_{3}&-\updelta \Psi _{4}  +\bar{\updelta} \Phi_{22} - {\Delta} \Phi_{21}
        =3\nu \Psi _{2}-2(\gamma +2\mu )\Psi_{3}+(4\beta -\tau )\Psi _{4} \notag\\
        &+(\bar{\tau}-2\bar{\beta}-2\alpha)\Phi_{22}+2(\gamma+\bar{\mu})\Phi_{21}
        +2\lambda \Phi_{12}-2\nu\Phi_{11}-\bar{\nu}\Phi_{20}, \\
        \Delta \Psi_{2}&-\updelta \Psi_{3}  +{\rm D} \Phi_{22} -\updelta \Phi_{21}
        +\frac{1}{12}{\Delta}R =2\nu \Psi_{1}-3\mu \Psi_{2}+2(\beta -\tau )
        \Psi_{3}+\sigma \Psi_{4} \notag\\
        &+(\bar{\rho}-2\epsilon-2\bar{\epsilon})\Phi_{22}+2(\bar{\pi}+\beta)\Phi_{21}
        +2\pi\Phi_{12} -2\mu \Phi_{11} -\bar{\lambda}\Phi_{20},  \\
        {\rm D}\Psi_{3}&-\bar{\updelta }\Psi_{2} - {\rm D} \Phi_{21} + \updelta \Phi_{20}
        -\frac{1}{12}\bar{\updelta}R =-2\lambda \Psi_{1} + 3\pi \Psi_{2}
        -2(\epsilon -\rho )\Psi_{3} -\kappa \Psi_{4} \notag\\
        &+(2\bar{\alpha}-2\beta-\bar{\pi})\Phi_{20} - 2(\bar{\rho}-\epsilon)\Phi_{21}
        -2\pi\Phi_{11}+2\mu\Phi_{10} + \bar{\kappa}\Phi_{22}, \\
        {\rm D} \Phi_{11}&-\updelta \Phi_{10} - \bar{\updelta} \Phi_{01} + {\Delta} \Phi_{00}
        +\frac{1}{8}{\rm D}R = \bar{\sigma}\Phi_{02} +\sigma\Phi_{20}
        -\bar{\kappa}\Phi_{12} - \kappa \Phi_{21} \notag \\
        & +(2\gamma+2\bar{\gamma} -\mu-\bar{\mu})\Phi_{00} +(\pi-2\alpha -2\bar{\tau})\Phi_{01}
        + (\bar{\pi}-2\bar{\alpha}-2\tau)\Phi_{10},  \\
        {\rm D} \Phi_{12}&-\updelta \Phi_{11} - \bar{\updelta} \Phi_{02} + {\Delta} \Phi_{01}
        +\frac{1}{8}{\updelta}R = \bar{\nu}\Phi_{00} - \bar{\lambda}\Phi_{10}
        +{\sigma}\Phi_{21} - \kappa \Phi_{22} +2(\bar{\pi}-\tau)\Phi_{11} \notag \\
        & +(-2\alpha+2\bar{\beta} +\pi-\bar{\tau})\Phi_{02}
        +(\bar{\rho}+2\rho -2\bar{\epsilon})\Phi_{12}
        +(2\gamma-2\bar{\mu}-\mu)\Phi_{01},  \\
        {\rm D} \Phi_{22}&-\updelta \Phi_{21} - \bar{\updelta} \Phi_{12} + {\Delta} \Phi_{11}
        +\frac{1}{8}{\Delta}R = {\nu}\Phi_{01} +\bar{\nu}\Phi_{10} -\bar{\lambda}\Phi_{20}
        -{\lambda}\Phi_{02} -2(\mu +\bar{\mu})\Phi_{11} \notag \\
        & +(\rho+\bar{\rho} - 2\epsilon-2\bar{\epsilon})\Phi_{22}
        +(2\bar{\beta}+2\pi -\bar{\tau})\Phi_{12}
        +(2{\beta}+2\bar{\pi} -{\tau})\Phi_{21} .
    \end{align}
\end{subequations} \setlength{\mathindent}{2em} 
当只考虑真空爱因斯坦引力场方程时,上式最后三个恒为零,
前八个式子及其复共轭可描述全部Bianchi恒等式.










\index[physwords]{NP型式!爱氏、麦氏方程}

\subsection{麦克斯韦方程组和爱因斯坦方程组}
我们还没有系统介绍麦氏、爱氏理论;在这里,先给出两个方程的NP形式以备后用;
详细推导过程可参考\parencite[\S 1.8(f)]{chandrasekhar-1983}或\parencite{newman-Penrose-1962}.
实反对称电磁场张量$F_{ab}$有六个独立分量,现在用复数来表示,
则只需三个复变量即可,它们是
\begin{subequations}
    \begin{align}
        \Phi_{0}=&F_{13}=F_{ab}l^{a}m^{b},\\
        \Phi_{1}=&\frac{1}{2}(F_{12}+F_{43})=\frac{1}{2}F_{ab}(l^{a}n^{b}+\overline{m}^{a}m^{b}),\\
        \Phi_{2}=&F_{42}=F_{ab}\overline{m}^{a}n^{b}.
    \end{align}
\end{subequations}
无源麦克斯韦方程组
($\nabla_b F^{ab}= 0 , \ \nabla_{[a} F_{ bc ] } =0 $)
在NP形式中具体表示为
\begin{subequations}\label{chnull:eqn_maxwell-4Q}
    \begin{align}
        {\rm D}\Phi_1 -\bar{\updelta}\Phi_0 &=(\pi-2\alpha)\Phi_0+2\rho\Phi_1-\kappa\Phi_2 , \\
        {\rm D}\Phi_2 -\bar{\updelta}\Phi_1 &=-\lambda\Phi_0+2\pi\Phi_1+(\rho-2\epsilon)\Phi_2 ,   \\
        \Delta\Phi_0-\updelta\Phi_1 &=(2\gamma-\mu)\Phi_0-2\tau\Phi_1+\sigma\Phi_2,  \\
        \Delta\Phi_1-\updelta\Phi_2 &=\nu\Phi_0-2\mu\Phi_1+(2\beta-\tau)\Phi_2 .\label{chnull:eqn_maxwell-4Q-d}
    \end{align}
\end{subequations}
%我们仅以\eqref{chnull:eqn_maxwell-4Q-d}为例来说明




我们只考虑电磁能动张量 %\eqref{chlh:eqn_EM-tensor}
($T_{ab}= F_{ac} F_{b}^{\cdot c} - \frac{1}{4} g_{ab} F_{cd} F^{cd} $),
此时,爱因斯坦引力场方程组 %\eqref{chfd:eqn_Einstein}
($R_{ab} = 8\pi (T_{ab} - \frac{1}{2}g_{ab}T )$)
在NP形式中表示为
\begin{equation}\label{chnull:eqn_einstein-4Q}
    \Phi_{\mu\nu} = \Phi_{\mu}\bar{\Phi}_{\nu}, \qquad
    \mu, \nu =0, 1, 2.
\end{equation}
其中$\Phi_{\mu\nu}$定义见\eqref{chnull:eqn_ricci-all}.






\index[physwords]{Petrov分类}

\section{Petrov分类}
对Weyl张量进行代数分类会令黑洞研究变得简单,
Petrov最早研究了这个问题;
此处采用文献\parencite[\S 1.9]{chandrasekhar-1983}中的叙述.

\subsection{四类光基矢变换}
给定四维闵氏时空$(M,g)$,其切空间是
平直的,即Minkowski时空;若对其实基矢$(E_\mu)^a$实施
Lorentz变换,可对NP四个类光基矢\eqref{chnull:eqn_Newman-Penrose-bases}产生
如下三种类型变换:
\fbox{甲}、\fbox{乙}和\fbox{丙}.
我们分别叙述它们.

\noindent\fbox{\heiti 甲}:保持实类光基矢$l^a$不变,四个基矢变换是:
\begin{equation*}
    l^a\to l^a,\quad m^a \to m^a + z l^a, \quad \overline{m}^a \to \overline{m}^a + \bar{z} l^a,
    \ \text{和}\ n^a \to n^a + \bar{z} m^a + z \overline{m}^a + z\bar{z} l^a .
\end{equation*}
其中$z$是流形$M$上的复数函数.

%(此式度规号差是“$-2$”)相似,我们将号改为“$+2$”的相应公式,即下式

\fbox{甲}类旋转对应的Lorentz变换为式\eqref{chlar:eqn_Sab},即下式
\begin{equation} \label{chnull:eqn_Sab1}
    S_1( {\alpha ,\beta } ) =
    \begin{bmatrix}
        1 + \zeta & \alpha&\beta&{ - \zeta }  \\
        \alpha &   1&0&{ - \alpha }\\
        \beta  &   0&1&{ - \beta  } \\
        \zeta  &   \alpha&\beta&{1 - \zeta }
    \end{bmatrix},
    \qquad \text{其中}\ \zeta = \frac{1}{2}(\alpha^2+\beta^2).
\end{equation}
可以验证$S_1$是Lorentz变换($S_1^T \eta S_1 = \eta$).将它作用在四个实基矢上
\begin{equation*}
    \begin{bmatrix}
        (E'_0)^a \\ (E'_1)^a \\ (E'_2)^a \\ (E'_3)^a
    \end{bmatrix}^T =
    \begin{bmatrix}
        (E_0)^a \\ (E_1)^a \\ (E_2)^a \\ (E_3)^a
    \end{bmatrix} ^T
    \begin{bmatrix}
        1 + \zeta & \alpha&\beta&{ - \zeta }  \\
        \alpha &   1&0&{ - \alpha }\\
        \beta  &   0&1&{ - \beta  } \\
        \zeta  &   \alpha&\beta&{1 - \zeta }
    \end{bmatrix},
\end{equation*}
用上式得到的新基矢$(E'_\mu)^a$构造NP类光基矢符合\fbox{甲}中的变换,
对应的$z=\alpha -\mathbbm{i} \beta$.

\fbox{甲}类变换会给NP中各量带来改变,比如考虑Weyl标量$\Psi_{0}$,有
\begin{align*}
    \Psi_0 =C_{1313}\to C_{abcd}l^a (m^b + z l^b) l^c (m^d + z l^d)
    %    =C_{abcd}\bigl(l^a  l^c (m^b m^d + z l^b m^d+  z m^b l^d + z^2 l^b l^d )   \bigr)
    %    =C_{1313} + z C_{1113} + zC_{1311} + z^2C_{1111}
    =C_{1313} .
\end{align*}
经过简单计算,$\Psi_{0}$是不变的.再计算一个略微复杂一点的
\begin{align*}
    \Psi_1 =& C_{1213} \to C_{abcd}l^a (n^b + \bar{z} m^b + z \overline{m}^b + z\bar{z} l^b)
    l^c (m^d + z l^d)     \\
%    =& C_{abcd} l^a l^c (n^bm^d + \bar{z} m^bm^d + z \overline{m}^b m^d+ z\bar{z} l^bm^d
%    +z l^d n^b + \bar{z} z l^d m^b + z^2 l^d \overline{m}^b + z^2 l^d\bar{z} l^b) \\
    =& C_{1213} + \bar{z} C_{1313} + z C_{1413} + z\bar{z} C_{1113}
    + zC_{1112} + z\bar{z} C_{1113} + z^2 C_{1411} +z^2 C_{1111} \\
    =& \Psi_1 + \bar{z} \Psi_0 .
\end{align*}
其它各量变化如下
\begin{equation}\label{chnull:eqn_Weyl-I}
    \begin{cases}
        \Psi_{0} \to \Psi_{0}, \quad \Psi_{1} \to \Psi_{1} + \bar{z} \Psi_{0}, \quad
        \Psi_{2} \to \Psi_{2} + 2 \bar{z} \Psi_{1} + \bar{z}^2 \Psi_{0}, \\
        \Psi_{3} \to \Psi_{3} + 3 \bar{z} \Psi_{2} + 3 \bar{z}^2 \Psi_{1} + \bar{z}^3 \Psi_{0}, \\
        \Psi_{4} \to \Psi_{4} + 4 \bar{z} \Psi_{3} + 6 \bar{z}^2 \Psi_{2}
        + 4 \bar{z}^3 \Psi_{1} + \bar{z}^4 \Psi_{0} .
    \end{cases}
\end{equation}
用类似方法可以得出自旋系数变换如下
\begin{equation}\label{chnull:eqn_Spin-I}
    \begin{cases}
        \kappa\to \kappa, \quad \sigma\to \sigma + z \kappa, \quad
        \rho \to \rho + \bar{z} \kappa, \quad \epsilon\to \epsilon +\bar{z}\kappa,\\
        \tau\to \tau + z \rho + \bar{z}\sigma + z\bar{z} \kappa,\quad
        \pi \to \pi + 2 \bar{z}\epsilon +\bar{z}^2 \kappa + {\rm D} \bar{z}, \\
        \alpha\to \alpha + \bar{z}(\rho +\epsilon) + \bar{z}^2\kappa,\quad
        \beta \to \beta + z \epsilon +\bar{z}\sigma + z\bar{z} \kappa, \\
        \gamma\to \gamma + z\alpha + \bar{z}(\beta + \tau ) + z\bar{z}(\rho+\epsilon)
        + \bar{z}^2\sigma + z\bar{z}^2\kappa,\\
        \lambda\to \lambda + \bar{z}(2\alpha + \pi)+ \bar{z}^2 (\rho+2\epsilon)
        + \bar{z}^3 \kappa + \bar{\updelta}\bar{z} \bar{z}{\rm D} \bar{z}, \\
        \mu\to \mu + z \pi +2 \bar{z}\beta +2z\bar{z}\epsilon +\bar{z}^2\sigma
        +z\bar{z}^2\kappa + \updelta \bar{z} + z{\rm D}\bar{z}, \\
        \nu \to \nu + z\lambda + \bar{z}(\mu+2\lambda) + \bar{z}^2 (\tau+2\beta)
        +\bar{z}^3\sigma + z\bar{z}(\pi+2\alpha) \\
        \quad +z\bar{z}^2 (\rho+2\epsilon) + z\bar{z}^3 \kappa
        +(\Delta + \bar{z}\updelta + z \bar{\updelta}+ z\bar{z}{\rm D})\bar{z} .
    \end{cases}
\end{equation}
表示麦克斯韦场的标量变成
\begin{equation}\label{chnull:eqn_Maxwell-I}
    \Phi_0\to \Phi_0,\quad \Phi_1\to \Phi_1 + \bar{z}\Phi_0, \quad
    \Phi_2 \to \Phi_2 + 2 \bar{z}\Phi_1 + \bar{z}^2 \Phi_0 .
\end{equation}
用于爱因斯坦场方程的Ricci曲率\eqref{chnull:eqn_ricci-all}变化是
\begin{align*}
    \Phi_{00}=& \frac{1}{2} R_{11} \to \frac{1}{2}R_{11}= \Phi_{00} \\
    \Phi_{01}=& \frac{1}{2} R_{13} \to \frac{1}{2}R_{ab}l^a (m^b + z l^b)
    =\frac{1}{2}R_{13} + z\frac{1}{2}R_{11} = \Phi_{01} + z \Phi_{00} \\
    \Phi_{02}=& \frac{1}{2} R_{33} \to %\frac{1}{2}R_{ab}(m^a + z l^a)(m^b + z l^b)=
    \frac{1}{2}R_{ab}(m^am^b + z l^am^b+ zm^al^b + z^2 l^al^b) 
    =\Phi_{02}+ 2z\Phi_{01} + z^2 \Phi_{00} \\
    \Phi_{10}=& \frac{1}{2} R_{14} \to \frac{1}{2}R_{ab}l^a(\overline{m}^b + \bar{z} l^b)
    = \Phi_{01} + \bar{z} \Phi_{00} \\
    \Phi_{12}=& \frac{1}{2} R_{23} \to %\frac{1}{2}R_{ab}(n^a + \bar{z} m^a +
    %z \overline{m}^a + z\bar{z} l^a)(m^b + z l^b)\\
    %=&\frac{1}{2}R_{ab}(n^am^b + \bar{z} m^am^b + z \overline{m}^am^b + z\bar{z} l^am^b
    %+z n^a l^b+ z\bar{z} m^a l^b+ z^2 \overline{m}^a l^b+ z^2\bar{z} l^al^b) \\
    %=& \frac{1}{2} (R_{23}+\bar{z}R_{33}+zR_{43}+z\bar{z}R_{13}
    %+zR_{21}+z\bar{z} R_{31}+z^2 R_{41}+ z^2\bar{z} R_{11})\\
    %=& 
    \Phi_{12}+\bar{z}\Phi_{02}+2z\Phi_{11} +2z\bar{z}\Phi_{01}
    +z^2 \Phi_{10} + z^2\bar{z} \Phi_{00} \\
    \Phi_{20}=& \frac{1}{2} R_{44} \to \frac{1}{2}R_{ab}
    (\overline{m}^a\overline{m}^b + \bar{z} l^a\overline{m}^b
    +\bar{z}\overline{m}^al^b + \bar{z}^2 l^al^b)
    =\Phi_{20}+ 2\bar{z}\Phi_{10} + \bar{z}^2 \Phi_{00} \\
    \Phi_{21}=& \frac{1}{2} R_{24} \to  %\frac{1}{2}R_{ab}(n^a + \bar{z} m^a +
    %    z \overline{m}^a + z\bar{z} l^a)(\overline{m}^b + \bar{z} l^b)\\
    %    =&\frac{1}{2}R_{ab}(n^a\overline{m}^b + \bar{z} m^a\overline{m}^b
    %      + z \overline{m}^a\overline{m}^b + z\bar{z} l^a\overline{m}^b
    %    +\bar{z} n^a l^b+ \bar{z}^2 m^a l^b+ z\bar{z} \overline{m}^a l^b+ z\bar{z}^2 l^al^b) \\
    %    =& \frac{1}{2} (R_{24}+\bar{z}R_{34}+zR_{44}+z\bar{z}R_{14}
    %    +\bar{z} R_{21}+\bar{z}^2 R_{31}+z\bar{z} R_{41}+ z\bar{z}^2 R_{11})\\
    %    =&
    \Phi_{21}+ {z}\Phi_{20}+2\bar{z}\Phi_{11}
    +2z\bar{z}\Phi_{10} +\bar{z}^2 \Phi_{01} + z\bar{z}^2 \Phi_{00}\\
    \Phi_{22}=& \frac{1}{2} R_{22} \to %\frac{1}{2}R_{ab}
%    (+n^a + \bar{z} m^a + z \overline{m}^a + z\bar{z} l^a)
%    (n^b + \bar{z} m^b + z \overline{m}^b + z\bar{z} l^b)\\
%    =&\frac{1}{2}R_{ab} \bigl(
%    (n^an^b + \bar{z} m^an^b + z \overline{m}^an^b + z\bar{z} l^an^b)
%    +(\bar{z}n^am^b + \bar{z}^2 m^am^b + z \bar{z}\overline{m}^am^b + z\bar{z}^2 l^am^b)\\
%    &+(zn^a\overline{m}^b + z\bar{z} m^a\overline{m}^b
%    + z^2 \overline{m}^a\overline{m}^b + z^2\bar{z} l^a\overline{m}^b )
%    +(z\bar{z} n^al^b + z\bar{z}^2 m^a l^b+ z^2\bar{z}\overline{m}^a l^b
%    + z^2\bar{z}^2 l^al^b) \bigr) \\
%    =&\frac{1}{2}\bigl(
%    R_{22} + \bar{z} R_{32} + z R_{42} + z\bar{z} R_{12}
%    +\bar{z}R_{23} + \bar{z}^2 R_{33} + z \bar{z}R_{43} + z\bar{z}^2 R_{13} \\
%    &+ z R_{24} + z\bar{z} R_{34} + z^2 R_{44} + z^2\bar{z} R_{14}
%    + z\bar{z} R_{21} + z\bar{z}^2 R_{31}+ z^2\bar{z}R_{41}+ z^2\bar{z}^2 R_{11}  \bigr) \\
%    =& 
    \Phi_{22} + 2z\bar{z}^2\Phi_{01}  + \bar{z}^2 \Phi_{02}
    + 2\bar{z} \Phi_{12} + 4z\bar{z} \Phi_{11} + 2z \Phi_{21} \\
    &+ z^2 \Phi_{20} + 2z^2\bar{z} \Phi_{10}  + z^2\bar{z}^2 \Phi_{00}   \\
    \Phi_{11}=& \frac{1}{4} (R_{12} + R_{34})  \to %\frac{1}{4}
%    R_{ab}l^a(n^b + \bar{z} m^b + z \overline{m}^b + z\bar{z} l^b)
%    +\frac{1}{4} R_{ab} (m^a + z l^a)(\overline{m}^b + \bar{z} l^b ) \\
%    =& \frac{1}{4} ( R_{12}+\bar{z} R_{13} +z R_{14} +z\bar{z} R_{11}
%    + R_{34} + \bar{z}R_{31}+ zR_{14}+ z\bar{z}R_{11}) \\
%    =& 
    \Phi_{11} + \bar{z} \Phi_{01} +z \Phi_{10} +z\bar{z} \Phi_{00}  \\
    R= & 2(R_{34}-R_{12})\to R
    %    2(R_{34} + \bar{z}R_{31}+ zR_{14}+ z\bar{z}R_{11} -  R_{12}-\bar{z} R_{13} -z R_{14} -z\bar{z} R_{11} )
\end{align*}


\noindent\fbox{\heiti 乙}:保持实类光基矢$n^a$不变,四个基矢变换是
\begin{equation*}
    n^a\to n^a,\quad m^a \to m^a + w n^a, \quad \overline{m}^a \to \overline{m}^a + \bar{w} n^a,
    \ \text{和}\ l^a \to l^a + \bar{w} m^a + w \overline{m}^a + w\bar{w} n^a .
\end{equation*}
其中$w$是流形$M$上的复数函数.
\fbox{乙}类旋转对应的Lorentz变换为式\eqref{chnull:eqn_Sab1}的{\kaishu 转置},
即$S_2( {\alpha ,\beta } ) = S_1^T( {\alpha ,\beta } )$.
将它作用在四个实基矢上,
得到新实基矢$(E'_\mu)^a$,用此构造NP类光基矢符合\fbox{乙}中的变换,
对应的$w=\alpha -\mathbbm{i} \beta$.
这类变换只需将\fbox{甲}中的$l^a$和$n^a$互换即可,其效果是
\begin{equation}\label{chnull:eqn_Weyl-Spin-II}
    \begin{cases}
        \Psi_0 \leftrightarrows \overline{\Psi}_4 ,\quad
        \Psi_1 \leftrightarrows \overline{\Psi}_3 ,\quad
        \Psi_2 \leftrightarrows \overline{\Psi}_2 ;\quad
        \Phi_0 \leftrightarrows -\overline{\Phi}_2 ,\quad
        \Phi_1 \leftrightarrows -\overline{\Phi}_1 ;\\
        \kappa \leftrightarrows -\bar{\nu}, \
        \rho   \leftrightarrows -\bar{\mu}, \
        \sigma \leftrightarrows -\bar{\lambda}, \
        \alpha \leftrightarrows -\bar{\beta},\
        \epsilon\leftrightarrows-\bar{\gamma},\
        \pi    \leftrightarrows -\bar{\tau}
    \end{cases}
\end{equation}
特别的,\fbox{乙}类旋转对Weyl标量的影响是
\begin{equation}\label{chnull:eqn_Weyl-II}
    \begin{cases}
        \Psi_{0} \to \Psi_{0} + 4 w \Psi_{1} + 6 w^2 \Psi_{2}
        + 4 w^3 \Psi_{3} + w^4 \Psi_{4}, \\
        \Psi_{1} \to \Psi_{1} + 3 w \Psi_{2} + 3 w^2 \Psi_{3} + w^3 \Psi_{4}, \\
        \Psi_{2} \to \Psi_{2} + 2 w \Psi_{3} + w^2 \Psi_{4}, \quad
        \Psi_{3} \to \Psi_{3} + w \Psi_{4}, \quad \Psi_{4} \to \Psi_{4} .
    \end{cases}
\end{equation}
用于爱因斯坦场方程的Ricci曲率\eqref{chnull:eqn_ricci-all}变化是
\begin{align*}
    \Phi_{00}=& \frac{1}{2} R_{11} \to 
%    \frac{1}{2}R_{ab}
%    (l^a + \bar{w} m^a + w \overline{m}^a + w\bar{w} n^a)
%    (l^b + \bar{w} m^b + w \overline{m}^b + w\bar{w} n^b)\\
%    =&\frac{1}{2}\bigl(
%    R_{22} + 2 w\bar{w}^2 R_{13} + \bar{w}^2 R_{33}
%    + 2\bar{w} R_{32} + 2w\bar{w} R_{12} + 2w \bar{w}R_{43}  + 2w R_{42}  \\
%    & + w^2 R_{44} + 2w^2\bar{w} R_{14}
%    +  w^2\bar{w}^2 R_{11}  \bigr) \\    =& 
    \Phi_{22} + 2w\bar{w}^2\Phi_{01}  + \bar{w}^2 \Phi_{02}
    + 2\bar{w} \Phi_{12} + 4w\bar{w} \Phi_{11} + 2w \Phi_{21} \\
    & + w^2 \Phi_{20} + 2w^2\bar{w} \Phi_{10}  + w^2\bar{w}^2 \Phi_{00}   \\
    \Phi_{01}=& \frac{1}{2} R_{13} \to 
%    \frac{1}{2}R_{ab}
%    (l^a + \bar{w} m^a + w \overline{m}^a + w\bar{w} n^a) (m^b + w n^b) \\
%    =& \frac{1}{2} (R_{13}+\bar{w}R_{33}+wR_{43}+w\bar{w}R_{23}
%    +wR_{12}+w\bar{w} R_{32}+w^2 R_{42}+ w^2\bar{w} R_{22})\\   =& 
    \Phi_{01}+\bar{w}\Phi_{02}+2w\Phi_{11} +2w\bar{w}\Phi_{12}
    +w^2 \Phi_{21} + w^2\bar{w} \Phi_{22} \\
    \Phi_{02}=& \frac{1}{2} R_{33} \to
    \frac{1}{2}R_{ab}(m^am^b + w n^am^b+ wm^an^b + w^2 n^an^b)
    =\Phi_{02}+ 2w\Phi_{12} + w^2 \Phi_{22} \\
    \Phi_{10}=& \frac{1}{2} R_{14} \to 
%    \frac{1}{2}R_{ab}(l^a + \bar{w} m^a +
%    w \overline{m}^a + w\bar{w} n^a)(\overline{m}^b + \bar{w} n^b)\\
%    =& \frac{1}{2} (R_{14}+\bar{w}R_{34}+wR_{44}+w\bar{w}R_{24}
%    +\bar{w}R_{12}+\bar{w}^2 R_{32}+w\bar{w} R_{42}+ w\bar{w}^2 R_{22})\\    =& 
    \Phi_{10}+{w}\Phi_{20}+2\bar{w}\Phi_{11} +2w\bar{w}\Phi_{21}
    +\bar{w}^2 \Phi_{12} + w\bar{w}^2 \Phi_{22} \\
    \Phi_{12}=& \frac{1}{2} R_{23} \to  \frac{1}{2}R_{ab} n^a (m^b+w n^b)
    = \Phi_{12} + w \Phi_{22}\\
    \Phi_{20}=& \frac{1}{2} R_{44} \to \frac{1}{2}R_{ab}
    (\overline{m}^a\overline{m}^b + \bar{w} n^a\overline{m}^b
    +\bar{w}\overline{m}^an^b + \bar{w}^2 n^an^b)
    =\Phi_{20}+ 2\bar{w}\Phi_{21} + \bar{w}^2 \Phi_{22} \\
    \Phi_{21}=& \frac{1}{2} R_{24} \to  \frac{1}{2}R_{ab} n^a (\overline{m}^b+\bar{w} n^b)
    = \Phi_{21} + \bar{w} \Phi_{22}\\
    \Phi_{22}=& \frac{1}{2} R_{22} \to \frac{1}{2}R_{ab} n^a n^b = \Phi_{22}   \\
    \Phi_{11}=& \frac{1}{4} (R_{12} + R_{34})  \to 
%    \frac{1}{4}
%    R_{ab}(l^a + \bar{w} m^a + w \overline{m}^a + w\bar{w} n^a) n^b
%    +\frac{1}{4} R_{ab} (m^a + w n^a)(\overline{m}^b + \bar{w} n^b ) \\
%    =& \frac{1}{4} ( R_{12}+\bar{w} R_{23} +w R_{24} +w\bar{w} R_{22}
%    + R_{34} + \bar{w}R_{32}+ wR_{24}+ w\bar{w}R_{22}) \\    =& 
    \Phi_{11} + \bar{w} \Phi_{12} +w \Phi_{21} +w\bar{w} \Phi_{22}  \\
    R= & 2(R_{34}-R_{12})\to R
\end{align*}


\noindent\fbox{\heiti 丙}:保持实类光基矢$l^a$和$n^a$方向不变,
并在$(m^a,\overline{m}^a)$平面上将基矢$m^a$和$\overline{m}^a$旋转$\theta$角;
四个基矢变换是
\begin{equation*}
    l^a\to A^{-1}l^a,\quad n^a\to A n^a,\quad m^a \to e^{\mathbbm{i}\theta} m^a,
    \ \text{和}\ \overline{m}^a \to e^{-\mathbbm{i}\theta} \overline{m}^a .
\end{equation*}
其中$A$和$\theta$是两个实数函数.
\fbox{丙}类旋转对应的Lorentz变换是
\begin{equation} \label{chnull:eqn_Sab3}
    S_3( {A ,\theta } ) =
    \begin{bmatrix}
        \frac{1}{2} (A^{-1}+A) & 0 & 0 & \frac{1}{2} (A^{-1}-A) \\
        0 &  \cos \theta & \sin \theta  & 0 \\
        0 & -\sin \theta & \cos \theta  & 0 \\
        \frac{1}{2} (A^{-1}-A) & 0 & 0 & \frac{1}{2} (A^{-1}+A) \\
    \end{bmatrix}
\end{equation}
可验证$S_3$是Lorentz变换,将它作用在四个实基矢上,
得到新实基矢$(E'_\mu)^a$,用此构造NP类光基矢符合\fbox{丙}中变换.
此变换对Weyl标量和麦氏标量的影响是
\begin{equation}\label{chnull:eqn_Weyl-III}
    \begin{cases}
        \Psi_0 \to A^{-2} e^{2\mathbbm{i}\theta} \Psi_0, \
        \Psi_1 \to A^{-1} e^{\mathbbm{i}\theta} \Psi_1, \
        \Psi_2 \to \Psi_2, \
        \Psi_3 \to A e^{-\mathbbm{i}\theta} \Psi_3, \\
        \Psi_4 \to A^{2} e^{-2\mathbbm{i}\theta} \Psi_4 ;\
        \Phi_0 \to A^{-1} e^{\mathbbm{i}\theta} \Phi_0, \
        \Phi_1 \to \Phi_1, \
        \Phi_2 \to A e^{-\mathbbm{i}\theta} \Phi_2;\\
    \end{cases}
\end{equation}
对自旋系数的影响是
\setlength{\mathindent}{0em} 
\begin{equation}\label{chnull:eqn_Spin-III}
    \begin{cases}
        \kappa\to A^{-2} e^{\mathbbm{i}\theta}\kappa,\
        \sigma\to A^{-1} e^{2\mathbbm{i}\theta}\sigma,\
        \rho \to A^{-1}\rho \
        \tau \to e^{\mathbbm{i}\theta} \tau, \
        \pi \to e^{-\mathbbm{i}\theta} \pi,\\
        \lambda \to A e^{\mathbbm{i}\theta} \lambda,\quad
        \nu \to A^2 e^{-\mathbbm{i}\theta} \nu, \quad
        \epsilon\to A^{-1} \epsilon -\frac{1}{2}A^{-2}{\rm D}A
        + \frac{\mathbbm{i}}{2} A^{-1} {\rm D}\theta,       \\
        \gamma\to A \gamma -\frac{1}{2}\Delta A+ \frac{\mathbbm{i}}{2} A \Delta \theta,\quad
        \alpha\to e^{-\mathbbm{i}\theta}\alpha + \frac{\mathbbm{i}}{2} e^{-\mathbbm{i}\theta}
        \bar{\updelta}\theta -\frac{1}{2} A^{-1} e^{-\mathbbm{i}\theta}\bar{\updelta}A, \\
        \mu\to A\mu, \quad
        \beta\to e^{\mathbbm{i}\theta}\beta + \frac{\mathbbm{i}}{2} e^{\mathbbm{i}\theta}
        \bar{\updelta}\theta -\frac{1}{2} A^{-1} e^{\mathbbm{i}\theta}\bar{\updelta}A.
    \end{cases}
\end{equation}\setlength{\mathindent}{2em} 
用于爱因斯坦场方程的Ricci曲率\eqref{chnull:eqn_ricci-all}变化是
\begin{equation}\label{chnull:eqn_Einstein-III}
    \begin{cases}
        \Phi_{00}\to A^{-2} \Phi_{00} , \quad
        \Phi_{01}\to A^{-1} e^{\mathbbm{i}\theta} \Phi_{01}  , \quad
        \Phi_{02}\to e^{2\mathbbm{i}\theta} \Phi_{02}  , \\
        \Phi_{10}\to A^{-1} e^{-\mathbbm{i}\theta} \Phi_{10} , \quad
        \Phi_{11}\to \Phi_{11}, \quad
        \Phi_{12}\to A e^{\mathbbm{i}\theta} \Phi_{12}  , \\
        \Phi_{20}\to e^{-2\mathbbm{i}\theta} \Phi_{20}  , \
        \Phi_{21}\to A e^{-\mathbbm{i}\theta} \Phi_{12}  , \
        \Phi_{22}\to A^2 \Phi_{22}    , \
        R\to R.
    \end{cases}
\end{equation}



其中$z$和$w$是流形$M$上的两个复数函数,$A$和$\theta$是两个实数函数;
Lorentz群包含六个实参量,上面三种变化同样包含六个实参量,两者相互对应.


\subsection{主类光矢量方法}
有两种等价的方法来将Weyl张量分类,分别是本征值方法和主类光矢量方法;
我们只叙述主类光矢量方法.
在NP类光四标架中Weyl张量化为五个标量:$\Psi_0,\ \Psi_1,\ \Psi_2,\ \Psi_3,\ \Psi_4$;
这些标量取值依赖于标架方向的选取,也依赖于所经受的Lorentz变换的六个参数.
如果标架选取适当,则Weyl标量中会有多个为零;对这个问题的
回答导致了Petrov分类.

%当四维闵氏时空$M$是共性平直时,根据定理\ref{chrg:thm_conformal-flat}可知
%Weyl张量恒为零,此时没有必要对其分类;故我们假设时空并非共性平直.
假设$\Psi_4\neq 0$,若$\Psi_4 = 0$,对标架实施\fbox{甲}类变换,则它就不再为零了.

现在考虑有参数$w$的\fbox{乙}类变换,Weyl标量变换
方程是\eqref{chnull:eqn_Weyl-II},即下式
\begin{equation}\label{chnull:eqn_Weyl-Petrov}
    \begin{cases}
        \Psi_{0}^{(n)}=  \Psi_{0} + 4 w \Psi_{1} + 6 w^2 \Psi_{2}
        + 4 w^3 \Psi_{3} + w^4 \Psi_{4}, \\
        \Psi_{1}^{(n)}=  \Psi_{1} + 3 w \Psi_{2} + 3 w^2 \Psi_{3} + w^3 \Psi_{4}, \\
        \Psi_{2}^{(n)}= \Psi_{2} + 2 w \Psi_{3} + w^2 \Psi_{4}, \
        \Psi_{3}^{(n)}= \Psi_{3} + w \Psi_{4}, \
        \Psi_{4}^{(n)}= \Psi_{4} .
    \end{cases}
\end{equation}
其中标记上标“$(n)$”的$\Psi$是新值,它由旧值变换而来.
设有代数方程
\begin{equation}\label{chnull:eqn_Weyl-II-w4}
    \Psi_{4}w^4 +  4 \Psi_{3}  w^3 + 6  \Psi_{2} w^2
    + 4  \Psi_{1} w + \Psi_{0}= 0.
\end{equation}
这个方程有四个根:$w_1, w_2, w_3 ,w_4$;
当$w$取这四个中的任一个时,由式\eqref{chnull:eqn_Weyl-Petrov}第一个
方程可知$\Psi_0^{(n)}=0$.
此时,基矢$l^a$对应新的方向,即$l^a + \bar{w} m^a + w \overline{m}^a + w\bar{w} n^a $.
\begin{definition}
    $l^a$的新方向($l^a + \bar{w} m^a + w \overline{m}^a + w\bar{w} n^a $)
    被称为Weyl张量的{\heiti 主类光方向}.如果上述四个根互不相等,
    那么Weyl张量称为{\heiti 代数普通}的,否则称为{\heiti 代数特殊}的.
\end{definition}

方程\eqref{chnull:eqn_Weyl-II-w4}四个根($w_1, w_2, w_3 ,w_4$)不同重合方式导致了Petrov分类.

\paragraph{Petrov O型}
这种类型是指四个根全部为零,是平庸情形.
此时所有Weyl标量都是零,进而可得Weyl张量也是零;
根据定理\ref{chrg:thm_conformal-flat}可知时空是共形平坦的.


\paragraph{Petrov N型}
这种类型是指有非零四重根.
容易看出由参数$w$的\fbox{乙}类旋转能使$\Psi_0$、$\Psi_1$、$\Psi_2$和$\Psi_3$同时为零,
只留下$\Psi_4$一个非零量.



\paragraph{Petrov I 型}
这种类型是指四个根互不相同.
我们通过第一个根$w_1$(比如说)的\fbox{乙}类变换将$\Psi_0^{(1)}$变为恒零量.
再通过\fbox{甲}类旋转(不影响$\Psi_0$)能将$\Psi_4$为零(取
方程\eqref{chnull:eqn_Weyl-II-w4}根的倒数即可,见式\eqref{chnull:eqn_Weyl-I}).
这样使$\Psi_0$和$\Psi_4$为零后,保留$\Psi_1$、$\Psi_2$和$\Psi_3$非零.
我们注意到在\fbox{丙}类变换下$\Psi_1$、$\Psi_2$和$\Psi_3$保持不变,
但\fbox{丙}类旋转会改变$\Psi_0$和$\Psi_4$的值.


\paragraph{Petrov II 型}
这种类型是指有两个重根,比如$w_1=w_2(\neq w_3 \neq w_4)$且$w_3\neq w_4$.
在这种情形下,$w_1(=w_2)$为方程\eqref{chnull:eqn_Weyl-II-w4}的解,
并且是其导数(对$w$求导)的解,即是下式
\begin{equation}\label{chnull:eqn_Weyl-II-w4-d1}
    \Psi_{4} w^3 +  3 \Psi_{3}  w^2 + 3  \Psi_{2} w +  \Psi_{1} = 0
\end{equation}
的解.
此时,参数为$w=w_1=w_2$的\fbox{乙}类旋转能将$\Psi_0$和$\Psi_1$变成零;
再由\fbox{甲}类旋转(不改变$\Psi_0$和$\Psi_1$的值)能使$\Psi_4$为零;
只留下$\Psi_2$和$\Psi_3$不为零.
对于\fbox{丙}类旋转,$\Psi_2$保持不变,但$\Psi_3$会变.

\paragraph{Petrov D 型}
这种类型是指有两组相互可区分的重根,比如$w_1=w_3, w_2 = w_4$但$w_1\neq w_2$.
对于引力论来说这是最重要的一种类型;通过旋转,能令$\Psi_0$、$\Psi_1$、$\Psi_3$和$\Psi_4$同时
为零,只留下$\Psi_2$一个非零标量.


依照假定,在参数为$w$的\fbox{乙}类旋转后,$\Psi_0$可以化为如下形式
\begin{equation}
    \Psi_0^{(1)} = \Psi_4 (w-w_1)^2 (w-w_2)^2.
\end{equation}
我们对上式中的$w$求导数,求导后乘以适当系数使得$w$最高次幂前的系数是$\Psi_4$;
经过连续四次求导便可以得到$\Psi_1^{(1)},\Psi_2^{(1)},\Psi_3^{(1)},\Psi_4^{(1)}$.
\begin{subequations}\label{chnull:eqn_w4d}
    \begin{align}
        \Psi_1^{(1)} =& \frac{1}{2}\Psi_4 (w-w_1)(w-w_2)(2w-w_1-w_2)\\
        \Psi_2^{(1)} =& \frac{1}{3}\Psi_4 \bigl((w-w_1)(w-w_2)+\frac{1}{2}(2w-w_1-w_2)^2\bigr)\\
        \Psi_3^{(1)} =& \frac{1}{2}\Psi_4 (2w-w_1-w_2) \\
        \Psi_4^{(1)} =& \Psi_4
    \end{align}
\end{subequations}
选择$w=w_1$,由式\eqref{chnull:eqn_w4d}可知
\begin{equation*}
    \Psi_0^{(1)}=0=\Psi_1^{(1)},\
    \Psi_2^{(1)}=\frac{1}{6}\Psi_4 (w_1-w_2)^2, \
    \Psi_3^{(1)}=\frac{1}{2}\Psi_4 (w_1-w_2),\
    \Psi_4^{(1)} = \Psi_4 .
\end{equation*}

现在给标架一个参数为$\bar{z}$的\fbox{甲}类旋转;
新的Weyl标量值(上角标记为“$(2)$”)是
\begin{equation}
    \Psi_0^{(2)}= 0=\Psi_1^{(2)},\qquad
    \Psi_2^{(2)}=\Psi_2^{(1)}=\frac{1}{6}\Psi_4 (w_1-w_2)^2 .
\end{equation}
上述三个值较为简单,剩下两个值略显复杂,见下式
\begin{align*}
        \Psi_3^{(2)}=&\Psi_3^{(1)}  +  3 \bar{z} \Psi_2^{(1)}
        =\frac{1}{2}\Psi_4 (w_1 - w_2)\bigl(1+\bar{z}(w_1 - w_2)\bigr) \\
        \Psi_4^{(2)}=& \Psi_4 + 4\bar{z}\cdot
        \frac{1}{2}\Psi_4 (w_1 - w_2) + 6\bar{z}^2 \cdot
        \frac{1}{6}\Psi_4 (w_1 - w_2)^2
        =\Psi_4 \bigl(1+ \bar{z}(w_1 - w_2)\bigr)^2
\end{align*} %\setlength{\mathindent}{2em}
$\bar{z}$是自由参数,我们选取其为$\bar{z}= (w_2-w_1)^{-1}$ ;
这样的选择能使$\Psi_3^{(2)}$和$\Psi_4^{(2)}$为零.
因此,由一个参数为$w-1$的\fbox{乙}类旋转,
再接一个参数为$\bar{z}= (w_2-w_1)^{-1}$的\fbox{甲}类旋转,
$\Psi_0^{(2)}$、$\Psi_1^{(2)}$、$\Psi_3^{(2)}$和$\Psi_4^{(2)}$都
被简化为零,唯独$\Psi_2^{(2)}$不为零.
现在可以把上角标“$(2)$”丢掉,直接说:经过变换只有$\Psi_2$不为零;
并且$\Psi_2$对\fbox{丙}类旋转保持不变.



\paragraph{Petrov III型}
这种类型是指有三个重根,比如$w_1=w_2=w_3(\neq w_4)$.
不难看出由参数$w=w_1$的\fbox{乙}类旋转能令
$\Psi_0$、$\Psi_1$和$\Psi_2$同时为零;
再接一个\fbox{甲}类旋转能使$\Psi_4$为零
(不影响$\Psi_0$、$\Psi_1$和$\Psi_2$为零).
只留下$\Psi_3$一个非零量,它的值在\fbox{丙}类旋转下可变.








至此,主类光矢量式的Petrov分类讲述完毕.





\paragraph{本征值方法分类}
请参阅\parencite[\S 4.1-4.2]{stephani-exe-2003},
或中文书籍\parencite[附录\S 15.2]{chenbin2018}.



%\section{类光曲线}
%设有四维广义黎曼流形$(M,g)$,其度规是Lorentz型度规($(-+++)$).
%本节大部分结论可以毫无困难地推广到广义Lorentz度规,即$(-+\cdots +)$.
%
%现在,假设曲线$C(t)$是类光曲线,即
%\begin{equation}
%    0=g_{ab}T^a T^b = g_{ab}\left.\left(\frac{\partial }{\partial t}\right)^a\right|_{C(t)}
%    \left.\left(\frac{\partial }{\partial t}\right)^b\right|_{C(t)}
%    =g_{\alpha\beta}\frac{{\rm d} x^\alpha\bigl(C(t)\bigr)}{{\rm d} t}
%    \frac{{\rm d} x^\beta\bigl(C(t)\bigr)}{{\rm d} t} .
%\end{equation}
%在$TM$上选定{\heiti \bfseries Frenet 标架}:$F\equiv \{T^a, N^a, U^a, V^a\}$.
%$T^a$为式\eqref{chnull:eqn_T};$N^a$为式\eqref{chnull:eqn_N};
%$U^a,V^a$是两个相互正交归一的类空矢量场,它们是$S$的基,
%故$N^a$和$T^a$与$U^a,V^a$是正交的.Frenet导数关系是
%\begin{align}
%    \nabla_T T^a = & +h T^a + \sigma U^a, \label{chnull:eqn_Frenet-T} \\
%    \nabla_T N^a = & -h N^a +\tau U^a +\beta V^a, \label{chnull:eqn_Frenet-N} \\
%    \nabla_T U^a = & -\tau T^a -\sigma N^a + \gamma U^a, \label{chnull:eqn_Frenet-U} \\
%    \nabla_T V^a = & -\beta T^a -\gamma U^a . \label{chnull:eqn_Frenet-V}
%\end{align}
%其中$\nabla$是$(M,g)$上的Levi-Civita联络,它自然
%满足相容性条件\eqref{chgd:eqn_connection-compatibility}.
%
%由$g_{ab}T^a N^b =1$可得$0=N_a \nabla_T T^a + T_a \nabla_T N^a$
%
%不难验证Frenet标架是刚性标架(见\ref{chrg:def_rigid-frame}),
%由式\eqref{chrg:eqn_ricci3}
%\begin{equation}
%    \nabla_b (e_\mu)^a  = -\omega_{\mu\nu\rho} (e^\rho)_b (e^\nu)^a,\qquad
%    \omega_{\mu\nu\rho}=-\omega_{\nu\mu\rho} . \tag{\ref{chrg:eqn_ricci3}}
%\end{equation}
%由上式可以看到在刚性标架上对$(e_\mu)^a$求导后,等号右端抽象指标为“$a$”的基矢$(e^\nu)^a$
%不可能包含内指标为“$\mu$”的基矢了,因为反对称关系$\omega_{\mu\nu\rho}=-\omega_{\nu\mu\rho}$.



\section{类光曲线}\label{chnull:sec_llc}
%\S\ref{chfd:sec_tlc}讲述了类时线汇基本知识,本小节继续那里的讨论,给出类光线汇的基本内容.

令$C(t)$是$M$中光滑曲线,它在局部坐标系$(U;x^\alpha)$中的分量表达式为:
\begin{equation}
    x^\alpha\equiv x^\alpha(t),\qquad \alpha =0,1,2,3,
    \quad t\in (-\delta,\delta)\subset \mathbb{R} .
\end{equation}
曲线$C(t)$的切线切矢量记为
\begin{equation}\label{chnull:eqn_T}
    l^a \equiv \left.\left(\frac{\partial }{\partial t}\right)^a\right|_{C(t)}
    =\frac{{\rm d} x^\alpha\bigl(C(t)\bigr)}{{\rm d} t}
    \left(\frac{\partial }{\partial x^\alpha}\right)^a .
\end{equation}
现设$C(t)$是类光曲线,
在$\{l^a, n^a, m^a, \overline{m}^a\}$下,利用\eqref{chnull:eqn_covD-l}
%\begin{equation}
%    \begin{aligned}
%        \nabla_{b} l_a
%        =& -(\epsilon+\bar{\epsilon}) n_b l_a -(\gamma+\bar{\gamma}) l_b l_a
%        +(\bar{\alpha}+\beta) \overline{m}_b l_a + (\alpha+\bar{\beta}) m_b l_a  \\
%        & +\kappa n_b \overline{m}_a  +\tau l_b \overline{m}_a
%        -\sigma \overline{m}_b \overline{m}_a  -\rho m_b \overline{m}_a  \\
%        & +\bar{\kappa} n_b m_a + \bar{\tau} l_b m_a
%        -\bar{\rho} \overline{m}_b m_a -\bar{\sigma} m_b m_a  .
%    \end{aligned}        \tag{\ref{chnull:eqn_covD-l}}
%\end{equation}
和\eqref{chnull:eqn_NP-lnmmb}可得
\begin{equation}
    l^b\nabla_{b} l_a  = (\epsilon+\bar{\epsilon})  l_a
    -\kappa  \overline{m}_a  -\bar{\kappa}  m_a  .
\end{equation}
由上式可知:$C(t)$是类光测地线充要条件是$\kappa=0$;
若再要求$Re(\epsilon)=0$,则$t$是仿射参数.
%有些时候为简单起见直接要求$\epsilon=0$.

下面我们只关心仿射参数化的类光矢量$l^a$,即$\kappa=0=\epsilon+\bar{\epsilon}$.
此时,由式\eqref{chnull:eqn_covD-l}易得
\begin{align}
    \nabla_{[b} l_{a]}
    =&  (\bar{\alpha}+\beta-\tau) \overline{m}_{[b} l_{a]}
    + (\alpha+\bar{\beta}-\bar{\tau}) m_{[b} l_{a]}
    +(\rho-\bar{\rho}) \overline{m}_{[b} m_{a]} . \label{chnull:eqn_adl}\\
    l_{[c}\nabla_{b} l_{a]}  =& (\rho-\bar{\rho})
    l_{[c}\overline{m}_{b} m_{a]}. \label{chnull:eqn_aldl}
\end{align}
参考\S\ref{chsm:sec_hypersurface-orthogonal}知识,
由式\eqref{chnull:eqn_aldl}可知
类光测地线切矢量$l^a$是超曲面正交的充要条件是$\rho$为实数.
进一步,若在要求$\bar{\alpha}+\beta=\tau$,则
从式\eqref{chnull:eqn_adl}可知切矢量$l^a$就等于一个标量场梯度.

%当$\kappa=0=\epsilon+\bar{\epsilon}$,$\bar{\alpha}+\beta=\tau$且$\rho$为实数时,
由式\eqref{chnull:eqn_covD-l}易得仿射参数化类光测地线$C(t)$的切线切矢量$l^a$满足
\begin{equation}\label{chnull:eqn_ltheta}
    \frac{1}{2}\nabla^{a} l_a =-\frac{1}{2} (\rho +\bar{\rho})
    \overset{def}{=} \theta .
\end{equation}
最后一步,我们定义了$\theta$.
我们继续给出另一定义,并给出了计算结果
\begin{equation}\label{chnull:eqn_lomega}
    \omega^2 \overset{def}{=} \frac{1}{2} (\nabla_{[b} l_{a]}) \nabla^{b} l^{a}
    = -\frac{1}{4}  (\rho - \bar{\rho})^2 .
\end{equation}
经过一些计算可以得到
\begin{equation}\label{chnull:eqn_lsigma}
    \frac{1}{2} (\nabla_{(b} l_{a)}) \nabla^{b} l^{a}
    =|\sigma|^2 + \theta^2 .
\end{equation}
式\eqref{chnull:eqn_ltheta}、\eqref{chnull:eqn_lomega}和\eqref{chnull:eqn_lsigma}只
要求$l^a$是类光测地线(仿射参数化)切矢量即可,无需额外要求;
因对称、反对称性,上面三式计算过程中很多项直接相消.
$\theta$、$\omega$和$\sigma$被称为{\heiti 光学标量},其中$\sigma$就是
NP型式中的自旋系数.它们分别表示膨胀、旋转和剪切.



\index[physwords]{Goldberg--Sachs定理}

\section{Goldberg--Sachs定理}
%我们讨论没有物质场的真空情形,此时爱氏引力场方程\eqref{chfd:eqn_Einstein}简化为$R_{ab}=0$;
当Ricci曲率恒为零($R_{ab}=0$)时,那么黎曼曲率张量和Weyl张量重合.

\begin{theorem}\label{chnull:thm_Goldenberg-Sachs}
    如果Weyl张量是Petrov II型的,并且选择类光基矢是使得$l^a$为主类光方向,
    且$\Psi_0=0=\Psi_1$;那么$\kappa=0=\sigma$.
    反之,如果$\kappa=0=\sigma$,那么有$\Psi_0=0=\Psi_1$,并且Weyl张量是II型的.
\end{theorem}

这个定理将场的II型特征与无剪切、类光测地线的存在相互联系.

当$\Psi_0=0=\Psi_1$时,Bianchi恒等式\eqref{chnull:eqn_Bianchi-4Q}中
的$a,b,c,d,g,h$得
    \begin{align*}
        &  \sigma \Psi _{2} =0, \
        \updelta \Psi_{2}  =   3\tau \Psi_{2} - 2\sigma \Psi _{3}, \
        \Delta \Psi_{2}-\updelta \Psi_{3}
        =2(\beta -\tau )  \Psi_{3}+\sigma \Psi_{4} -3\mu \Psi_{2}.  \\
        &  \kappa \Psi _{2} =0,       \
        {\rm D}\Psi _{2} = 3\rho \Psi _{2}-2\kappa \Psi _{3},\
        {\rm D}\Psi_{3}-\bar{\updelta }\Psi_{2}
        = 3\pi \Psi_{2} -2(\epsilon -\rho )\Psi_{3} -\kappa \Psi_{4} .
    \end{align*}
若空间不是共形平坦的,即Weyl标量不都为零,则由上式
第一行可得$\sigma=0$,第二行可得$\kappa=0$.
也就是,类光矢量$l^a$是测地线($\kappa=0$)切矢量,以及它是无剪切的($\sigma=0$).

逆定理的证明则要费一番周折.我们已知$\kappa=0=\sigma$.
首先,通过\fbox{丙}类旋转(不改变$\kappa,\sigma$的零值),使得$\epsilon=0$.
从NP方程\eqref{chnull:eqn_NP-all}中的$a,b,c,l,n$可得(利用$\kappa=\sigma=\epsilon=0$)
\begin{equation}\label{chnull:eqn_abcln}
    \begin{aligned}
        &{\rm D}\tau  =  \rho(\tau +\bar{\pi }) + \Psi _{1} ,   \ (a); \quad
        0 = \Psi_{0},        \ (b); \quad
        {\rm D}\rho  =  \rho ^{2},  \ (c);        \\
        &\updelta \rho  =\rho (\bar{\alpha }+\beta)
        +\tau(\rho -\bar{\rho })       -\Psi _{1} , \ (l); \quad
        {\rm D}\beta  =   \beta\bar{\rho } +\Psi _{1} , \ (n) .
    \end{aligned}
\end{equation}
由上式已得到$\Psi_{0}=0$.
如果$\rho=0$,则由\eqref{chnull:eqn_abcln}的$(l)$式可得$\Psi_1=0$,
这正是我们想要的结果;但$\rho$可能不是零,故我们假设$\rho\neq 0$.

\fbox{甲}类旋转(见式\eqref{chnull:eqn_Weyl-I}和\eqref{chnull:eqn_Spin-I})并
不改变$\Psi_0$的值,也不影响$\kappa,\sigma,\epsilon$的零值;
但此类旋转能令$\tau=0$.

由Bianchi恒等式\eqref{chnull:eqn_Bianchi-4Q}中的$a,b$得
(利用$\kappa=\sigma=\epsilon=\tau=\Psi_0=0$)
\begin{equation}\label{chnull:eqn_tmpbi}
    {\rm D}\Psi _{1}  =4\rho \Psi _{1} ,\qquad
    \updelta \Psi_{1} =2\beta\Psi _{1} .
\end{equation}
由上式可得
\begin{equation}
    ({\rm D}\updelta - \updelta{\rm D}) \ln \Psi _{1}
    =2{\rm D}\beta -4 \updelta \rho .
\end{equation}
利用式\eqref{chnull:eqn_abcln}中的$(l),(n)$,以及$\tau=0$可得
\begin{equation}\label{chnull:eqn_tmpdps}
    ({\rm D}\updelta - \updelta{\rm D}) \ln \Psi _{1}
    =2 \bar{\rho} \beta -4 (\bar{\alpha}+\beta)\rho + 6\Psi_1 .
\end{equation}
此时,需要利用式\eqref{chnull:eqn_DDcomunicator-b}(去掉各种为零的量后)
\begin{equation}
    \updelta {\rm D}-{\rm D}\updelta =(\bar{\alpha}+\beta -\bar{\pi}){\rm D}
    -\bar{\rho}\updelta .
\end{equation}
将上式计算
\begin{equation}
    ({\rm D}\updelta-\updelta {\rm D}) \ln\Psi_1 =
    \bar{\rho}\updelta \ln\Psi_1 -
    (\bar{\alpha}+\beta -\bar{\pi}){\rm D}\ln\Psi_1
    =2\beta\bar{\rho} - 4 (\bar{\alpha}+\beta -\bar{\pi})\rho.
\end{equation}
计算过程中要用到式\eqref{chnull:eqn_tmpbi}.
将上式计算结果代入式\eqref{chnull:eqn_tmpdps}得
\begin{equation}\label{chnull:eqn_tmpallldps}
    \Psi_1=  \frac{2}{3} \bar{\pi} \rho .
\end{equation}
然而,式\eqref{chnull:eqn_abcln}中的$(a)$式(令$\tau=0$)给出
\begin{equation}
    \Psi _{1}  = - \rho \bar{\pi } .
\end{equation}
上两式是矛盾的,这说明前面假设$\rho\neq 0$是不成立的;
$\rho$必须是零,进而$\Psi_1=0$.
这便证明了Goldberg-Sachs定理.
\qed

这个定理有一个重要的推论:
\begin{corollary}\label{chnull:thm_BHD}
    如果场是代数特殊的,并且是Petrov D型,那么两个主类光方向$l^a$和$n^a$既是
    测地线切矢量又是无剪切的;即若有$\Psi_0=\Psi_1=\Psi_3=\Psi_4=0$,则
    必有$\kappa=\sigma=\nu=\lambda=0$.反之亦然.
\end{corollary}

目前,在广义相对论中得到黑洞解,比如史瓦西黑洞或克尔黑洞,都是Petrov D型的,
这种情形使得NP型式在分析黑洞类光测地线时有着较为方便之处.




\section*{小结}

更多零模问题请参考Duggal教授及其合作者写的几部专著,比如\parencite{Duggal-Bejancu-1996}.

Newman--Penrose型式是广义相对论中一种常用的方法,它将在黑洞物理的求解、分析中起到重要作用.

%\vspace{1cm}



\printbibliography[heading=subbibliography,title=第\ref{chnull}章参考文献]

\endinput