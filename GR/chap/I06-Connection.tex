% !TeX encoding = UTF-8
% 此文件从2021.8开始

\chapter{仿射联络与曲率}\label{chccr}
联络是微分流形上非常重要的几何概念,起源于二维曲面几何上的平行移动.
现在联络概念已推广到纤维丛上,而在此,只讨论切丛上的联络,其实质是
定义在切矢量场(切丛的截面)上的一种方向导数;并由此产生一种新的微分法则,
即所谓的“协变微分”(或共变微分).
有了仿射联络,便可定义黎曼曲率张量,它是微分几何中的核心概念.
我们还讨论了一般标架场上的曲率型式.


\index[physwords]{联络}
\index[physwords]{联络!仿射联络}

\section{仿射联络}\label{chccr:sec_affine-connection}

\begin{definition}\label{chccr:def_connection}
    设$M$是一个$m$维光滑流形,其上的$\nabla_a:\mathfrak{X}(M)\times \mathfrak{X}(M)\to \mathfrak{X}(M)$
    (或用${\rm D}_a$)是一个映射;
    下面式子中$ X^a, Y^a, Z^a\in \mathfrak{X}(M)$,
    $ f\in C^\infty(M)$,$ \lambda \in \mathbb{R}$.
    记$X^a\nabla_a \equiv \nabla_X $,$\nabla_b (X^b, Y^a) \equiv \nabla_X Y^a$.
    若$\nabla_a$满足如下条件:

    {\bfseries (1)}  $\mathbb{R}$-线性:$\nabla_X (Y^a +\lambda\cdot Z^a)
       \overset{def}{=}\nabla_X Y^a +\nabla_X (\lambda\cdot Z^a) \equiv  \nabla_X Y^a +\lambda\cdot\nabla_X Z^a$;

    {\bfseries (2)} 加法:$(f\cdot X^b+Y^b)\nabla_{b} Z^a \equiv \nabla_{f\cdot X+Y} Z^a
    \overset{def}{=} \nabla_{f\cdot X} Z^a + \nabla_{Y} Z^a $;

    {\bfseries (3)} $\nabla_{X} (f\cdot Y^a) \overset{def}{=} (\nabla_{X} f) Y^a + f\cdot \nabla_{X} Y^a
     \equiv X(f) Y^a + f\cdot \nabla_{X} Y^a$.

    则称$\nabla_a$(或${\rm D}_a$)是光滑流形$M$上的
    一个{\heiti 仿射联络},或{\heiti 协变微分};
    称$\nabla_X$为{\heiti 协变导数}.仿射联络专指切丛上的联络.
    定义了仿射联络的光滑流形$(M,\nabla_a)$被称为{\heiti 仿射联络空间}.
\end{definition}

定义\ref{chccr:def_connection}只给出了矢量场上的联络;下面把它拓展成张量场上的联络.

\begin{definition}\label{chccr:def_connection-tb}
    下面式子中$ X^a\in \mathfrak{X}(M)$,
    $ K \in \mathfrak{T}^r_s(M) $,$ L \in \mathfrak{T}^p_q(M) $.
    若$\nabla_a$满足:
    
{\bfseries (4)} $\nabla_a$将$M$上任意$\Tpq{r}{s}$型张量场映射为$\Tpq{r}{s+1}$型张量场.

{\bfseries (5)} Leibniz 法则:
$\nabla_X(K\otimes L)\overset{def}{=} (\nabla_X K)\otimes L+ K\otimes \nabla_X L$;

{\bfseries (6)} $\nabla_X$与缩并运算$C^\times_\times$可交换,即
$C^\times_\times(\nabla_X K_{\cdots}^{\cdots}) \overset{def}{=} \nabla_X(C^\times_\times K_{\cdots}^{\cdots})$.
    
  
    则称$\nabla_a$可以作用在光滑流形$M$上张量场的联络,仍称之为仿射联络.
\end{definition}


\begin{remark}\label{chccr:rek_ccl}
    仿射联络必然存在,请参考定理\ref{chfb:thm_existence}.
\end{remark}
\begin{remark}
    依定义,仿射联络$\nabla_X$将$\Tpq{r}{s}$型张量场映射为一个$\Tpq{r}{s}$型张量场.
\end{remark}

\begin{remark}\label{chccr:rek_abstnoa}
    为避免抽象指标不匹配,$\nabla_X$中的$X$不附加抽象指标,即不写成$\nabla_{X^a}$.
\end{remark}

\begin{remark}
定义\ref{chccr:def_connection}中的三个条件是最基本的,
定义\ref{chccr:def_connection-tb}中的几条是在前三条基础上的附加原则.
其中Leibniz律是为了把联络应用到各阶张量而引入的.
\end{remark}

我们进一步解释条目(3),联络作用在标量函数上,有
\begin{align}
    &X^a\nabla_a f = X(f) =
    X^i \frac{\partial f}{\partial x^i} =
    X^i \left(\frac{\partial }{\partial x^i}\right) ^a
    ({\rm d} x^j)_a \frac{\partial f}{\partial x^j} =
    X^a ({\rm d}f)_a  \notag \\
    {\color{red}\Rightarrow} \quad &
    \nabla_a f = ({\rm d}f)_a . \label{chccr:eqn_Df=df}
\end{align}
由此可知不同的联络$\nabla_a$、${\rm D}_a$作用在标量场上皆相等,
即$\nabla_a f = ({\rm d}f)_a={\rm D}_a f$.
也就是说:任何推广后的“导数”算符都必须能退化到普通微积分中的导数.





\subsection{联络系数}
为讨论联络系数,先要叙述一个命题,一般称之为{\heiti 局部性定理}.  \index[physwords]{局部性定理!仿射联络}
\begin{proposition}\label{chccr:thm_local}
    设有仿射联络空间$(M,\nabla_a)$,则仿射联络$\nabla_a$具有如下{\kaishu 局部性质}:
    设$X_1,X_2,Y_1,Y_2\in \mathfrak{X}(M)$,则如果存在$M$的开集$U$使
    得$X_1|_U =X_2|_U,\ Y_1|_U =Y_2|_U  $,那么必然
    有$(\nabla_{X_1} Y_1) |_U=(\nabla_{X_2} Y_2) |_U$.
\end{proposition}
\begin{proof}
    很明显只要证明$\nabla_{X_1} Y_1 |_U=\nabla_{X_1} Y_2 |_U$及
    $\nabla_{X_1} Y_1 |_U=\nabla_{X_2} Y_1 |_U$便能证明命题了;而且两者证明非常相似,只需证明其一即可.

    因流形$M$有局部紧致性,所以$\forall p\in U$都存在开子集$V$和紧致集$\overline{V}$使得
    $\forall p \in V \subset \overline{V} \subset U$.依命题\ref{chdm:thm_exp1UVM},
    存在$C^\infty$函数$h:M \to \mathbb{R}$使得$0\leqslant h \leqslant 1$,
    且$h|_V=1$,$h|_{M-U}=0$;因此有$h\cdot(Y_1-Y_2)\equiv 0$在整个流形$M$上成立;
    由定义\ref{chccr:def_connection}中Leibniz律可得联络$\nabla_a$作用在此式的结果为
    \begin{equation*}
        0=X_1^a \nabla_a \bigl( h\cdot(Y_1-Y_2) \bigr) = ( \nabla_{X_1}  h ) \cdot(Y_1-Y_2)
        +h\cdot \nabla_{X_1} (Y_1-Y_2) .
    \end{equation*}
    将此式限制在$p\in V$上,便可得到(注$h|_V=1$,$Y_1|_V =Y_2|_V$)
    \begin{equation*}
        0=\nabla_{X_1} (Y_1-Y_2)|_p 
        \quad {\color{red}\Rightarrow}\quad
        (\nabla_{X_1} Y_1) (p) = (\nabla_{X_1} Y_2) (p) .
    \end{equation*}
    因点$p$的任意性,可得$\nabla_{X_1} Y_1|_U = \nabla_{X_1} Y_2|_U$,进而可证明命题.
\end{proof}




关于局部性定理的理解,请先参考切矢量场的评注\ref{chdm:rmk_local}.
读者需要注意,定义\ref{chccr:def_connection}是针对整体流行$M$而言;
一般说来不能用一个坐标域覆盖住$M$,
目前我们不知道把定义在整个流形$M$上的仿射联络$\nabla_a$作用在局部坐标系上具体会是什么样子.
{\kaishu 局部性定理}说:只要$X_1|_U =X_2|_U,\ Y_1|_U =Y_2|_U  $,
那么必然有$(\nabla_{X_1} Y_1) |_U=(\nabla_{X_2} Y_2) |_U$.
换句话说仿射联络的计算只需要知道切矢量场的局部性状即可,
无需知道$p$点开邻域$V$外的性状;这和微积分中的“导数运算”完全一致.
这也就说明只需要一个坐标域就能解决仿射联络的局部表示,
无需多个坐标域结合使用,这与积分不同;
流形上的积分(见\S\ref{chdf:sec_integral-on-manifold})需要
用到单位分解定理来处理多个坐标域的情形.
现在假设$(U;x^i)$是$M$的一个局部坐标系,根据仿射联络定义可以得到
$(\frac{\partial }{\partial x^i}) ^b \nabla_b(\frac{\partial }{\partial x^j}) ^a \in \mathfrak{X}(U)$,
故可定义
\begin{equation}\label{chccr:eqn_Christoffel2}
    \nabla_{\frac{\partial }{\partial x^i}}
    \left(\frac{\partial }{\partial x^j}\right) ^a \overset{def}{=}
    \Gamma^{k}_{ji} \left(\frac{\partial }{\partial x^k}\right) ^a
      \ \Leftrightarrow \ 
    \nabla_c \left(\frac{\partial }{\partial x^j}\right) ^a \overset{def}{=}
    \Gamma^{k}_{ji} \left(\frac{\partial }{\partial x^k}\right) ^a ({\rm d} x^i)_c .
\end{equation} %\setlength{\mathindent}{2em}
其中$\Gamma^{k}_{ji} \in C^\infty(U)$被称为仿射联络$\nabla_a$在自然基底
场$\{(\frac{\partial }{\partial x^i}) ^a\}$下的
{\heiti 联络系数};需注意它的两个下标不能随意交换,即一般情
况下$\Gamma^{k}_{ji}\neq \Gamma^{k}_{ij}$.

由式\eqref{chccr:eqn_Christoffel2}的第一式两边乘以$({\rm d} x^i)_c$,有
\begin{equation}
    \left[({\rm d} x^i)_c \left(\frac{\partial }{\partial x^i}\right) ^b\right] \nabla_b
    \left(\frac{\partial }{\partial x^j}\right) ^a \overset{def}{=}
    \Gamma^{k}_{ji} \left(\frac{\partial }{\partial x^k}\right) ^a ({\rm d} x^i)_c .
\end{equation}
再对比\eqref{chccr:eqn_Christoffel2}的第二式,可得
\begin{equation}\label{chccr:eqn_deltaac}
    \sum_i \left[({\rm d} x^i)_c \left(\frac{\partial }{\partial x^i}\right) ^b\right] = \delta_c^b .
\end{equation}
目前,我们还没有引入度规;在引入度规后(\S \ref{chrg:sec_riemann}),会发现上式与度规相容.

\index[physwords]{Christoffel记号}
\index[physwords]{联络!联络系数|see{Christoffel记号}}

设$\nabla_a$是$M$上的一个仿射联络,再设$(U;x^i)$和$(V;y^\alpha)$是$M$的
两个局部坐标系,并且$U\cap V \neq \varnothing$;那么依
定义\eqref{chccr:eqn_Christoffel2},联络$\nabla_a$便有两个联络
系数$\Gamma^{k}_{ji}(x)$和$\Gamma^{\rho}_{\beta\alpha}(y)$.
在$U\cap V$上显然有
\begin{equation}
    \frac{\partial }{\partial x^i} = \frac{\partial y^\alpha}{\partial x^i}
    \frac{\partial }{\partial y^\alpha} {\quad \color{red}\Leftrightarrow \quad }
    \frac{\partial }{\partial y^\alpha} = \frac{\partial x^i} {\partial y^\alpha}
    \frac{\partial }{\partial x^i} .
\end{equation}
因此,
\begin{equation*}
    \nabla_{\frac{\partial }{\partial x^i}}\left(\frac{\partial }{\partial x^j}\right) ^a
    =\Gamma^{k}_{ji}(x) \left(\frac{\partial }{\partial x^k}\right) ^a
    =\Gamma^{k}_{ji}(x) \frac{\partial y^\rho}{\partial x^k}
    \left(\frac{\partial }{\partial y^\rho}\right)^a .
\end{equation*}
而上式左端还可以表示为
\begin{align*}
    \nabla_{\frac{\partial }{\partial x^i}}
    \left[\frac{\partial y^\beta}{\partial x^j}
    \left(\frac{\partial }{\partial y^\beta}\right)^a\right]
    &=\left(\frac{\partial }{\partial y^\beta}\right)^a
    \nabla_{\frac{\partial }{\partial x^i}}
    \left[\frac{\partial y^\beta}{\partial x^j} \right]
    +\frac{\partial y^\beta}{\partial x^j}
    \nabla_{\frac{\partial y^\alpha}{\partial x^i}
        \frac{\partial }{\partial y^\alpha} }
    \left[\left(\frac{\partial }{\partial y^\beta}\right)^a\right]  \\
    &=\left(\frac{\partial }{\partial y^\beta}\right)^a
    \frac{\partial^2 y^\beta}{\partial x^i \partial x^j}
    +\frac{\partial y^\beta}{\partial x^j} \frac{\partial y^\alpha}{\partial x^i}
    \Gamma^{\rho}_{\beta\alpha}(y)
    \left(\frac{\partial }{\partial y^\rho}\right)^a  .
\end{align*}
结合上两式,有
\begin{subequations}\label{chccr:eqn_Exchange-Christoffel}
    \begin{align}
        \Gamma^{k}_{ji}(x) \frac{\partial y^\rho}{\partial x^k} &=
        \frac{\partial y^\beta}{\partial x^j} \frac{\partial y^\alpha}{\partial x^i}
        \Gamma^{\rho}_{\beta\alpha}(y) + \frac{\partial^2 y^\rho}{\partial x^j \partial x^i}
          \label{chccr:eqn_Exchange-Christoffel-a}\\
        &= \frac{\partial y^\beta}{\partial x^j} \frac{\partial y^\alpha}{\partial x^i}
        \Gamma^{\rho}_{\beta\alpha}(y)
        -\frac{\partial y^\alpha }{\partial {x^i}}\frac{\partial {y^\beta }}{\partial {x^j}}
        \frac{\partial {y^\rho }}{\partial {x^k}}\frac{{{\partial ^2}{x^k}}}
        {\partial {y^\beta }\partial {y^\alpha }} . \label{chccr:eqn_Exchange-Christoffel-b}
    \end{align}
\end{subequations}
上式是联络系数的基本关系式.
式\eqref{chccr:eqn_Exchange-Christoffel-a}到\eqref{chccr:eqn_Exchange-Christoffel-b}可以
这样验证,设坐标$y$是$x$函数,而$x$是$z$函数,求如下导数
\begin{equation} \label{chccr:eqn_tmp21}
\begin{aligned}
    &\frac{{{\partial ^2}{y^\rho }\left( {x\left( z \right)} \right)}}{{\partial {z^A}\partial {z^B}}}
     = \frac{\partial }{{\partial {z^A}}}\left( {\frac{{\partial {y^\rho }
                \left( {x\left( z \right)} \right)}}{{\partial {z^B}}}} \right)
             = \frac{\partial }{{\partial {z^A}}}\left( {\frac{{\partial {y^\rho }}}
                 {{\partial {x^i}}}\frac{{\partial {x^i}\left( z \right)}}{{\partial {z^B}}}} \right)\\
    =& \frac{\partial }{{\partial {z^A}}}\left( {\frac{{\partial {y^\rho }}}
        {{\partial {x^i}}}} \right)\frac{{\partial {x^i}\left( z \right)}}{{\partial {z^B}}}
    + \frac{{\partial {y^\rho }}}{{\partial {x^i}}}\frac{{{\partial ^2}{x^i}
            \left( z \right)}}{{\partial {z^A}\partial {z^B}}} \\
    =& \frac{{{\partial ^2}{y^\rho }}}{{\partial {x^i}\partial {x^j}}}
    \frac{{\partial {x^i}\left( z \right)}}{{\partial {z^A}}}
    \frac{{\partial {x^j}\left( z \right)}}{{\partial {z^B}}}
     + \frac{{\partial {y^\rho }}}{{\partial {x^l}}}\frac{{{\partial ^2}{x^l}
             \left( z \right)}}{{\partial {z^A}\partial {z^B}}} .
\end{aligned}\end{equation}
令$z\equiv y$,则上式最左端恒为零,最右端的式子便可证明\eqref{chccr:eqn_Exchange-Christoffel}.

从式\eqref{chccr:eqn_Exchange-Christoffel}可以看到$\Gamma^{k}_{ij}$不满足张量分量
变换公式\eqref{chdm:eqn_tensor-component-trans},所以它不是张量场.
读者需注意,$\Gamma^{k}_{ij}$只在流形$M$上才能定义,在线性空间$V$上是无法定义的;
因为单个线性空间没有坐标系$\{x^i\}$的概念,也就无法诞生$\Gamma^{k}_{ij}$了.

我们人为约定它们在局部坐标$(U;x^i)$的抽象指标表达式为
\begin{equation}\label{chccr:eqn_Christoffel-naturalbases-ij}
    \Gamma_{ab}^c = \Gamma_{ij}^k ({\rm d}x^i)_a ({\rm d}x^j)_b
    \left(\frac{\partial}{\partial x^k} \right)^c; \quad
    \Gamma_{ij}^k  = \Gamma_{ab}^c  \left(\frac{\partial}{\partial x^i} \right)^a
    \left(\frac{\partial}{\partial x^j} \right)^b ({\rm d}x^k)_c .
\end{equation}
再次强调,虽然有抽象指标记号,但$\Gamma_{ab}^c$不是张量场.

上面从联络得到了联络系数.
反之,在$m$维光滑流形$M$的每个局部开集$(U;x^i)$中给定$m^3$个
光滑函数$\Gamma^{k}_{ji}\in C^\infty(U)$,且在坐标变换时这些系数满足
式\eqref{chccr:eqn_Exchange-Christoffel};
那么在$M$上存在唯一的仿射联络与之相对应.证明如下,
设$M$有开子集$(U;x^i)$和$(V;y^\alpha)$且$U\cap V \neq \varnothing$,
$\forall v^b\in \mathfrak{X}(U\cap V)$,设$v^b|_{U\cap V}
=v^i(\frac{\partial}{\partial x^i})^b
=v^\alpha(\frac{\partial}{\partial y^\alpha})^b$,令
\begin{equation}\label{chccr:eqn_tmpvc}
       \left. \nabla_a v^b \right|_{U}  \overset{def}{=}
    \left(\frac{\partial v^i}{\partial x^j} +v^k \Gamma_{kj}^{i}(x) \right)
           \Bigl(\frac{\partial}{\partial x^i}\Bigr)^b ({\rm d}x^j)_a ,
\end{equation}
下面证明式\eqref{chccr:eqn_tmpvc}与局部坐标系选取无关,
即在坐标变化下上式的形式协变:
\begin{align*}
    \left. \left(\left. \nabla_a v^b \right|_{U}\right) \right|_{U\cap V} = &
    \left(\frac{\partial v^i}{\partial x^j} +v^k \Gamma_{kj}^{i}(x) \right)
    \Bigl(\frac{\partial}{\partial x^i}\Bigr)^b ({\rm d}x^j)_a  \\
    =&  \left(\frac{\partial }{\partial x^j}\Bigl(v^\alpha \frac{\partial x^i}{\partial y^\alpha} \Bigr)
      + v^\mu \frac{\partial x^k}{\partial y^\mu} \Gamma_{kj}^{i}(x) \right)
      \frac{\partial y^\rho}{\partial x^i}  \Bigl(\frac{\partial}{\partial y^\rho}\Bigr)^b
      \frac{\partial x^j} {\partial y^\beta}    ({\rm d}y^\beta)_a  \\
%     \xlongequal{\ref{chccr:eqn_Exchange-Christoffel-a}}&
%      \left( \frac{\partial v^\alpha}{\partial y^\beta} \frac{\partial x^i}{\partial y^\alpha}
%      + v^\alpha \frac{\partial^2 x^i}{\partial y^\beta \partial y^\alpha} \right)
%      \frac{\partial y^\rho}{\partial x^i}
%      \Bigl(\frac{\partial}{\partial y^\rho}\Bigr)^b({\rm d}y^\beta)_a \\
%     & + v^\mu \frac{\partial x^k}{\partial y^\mu}  \frac{\partial x^j} {\partial y^\nu}
%     \left(\frac{\partial^2 y^\rho}{\partial x^k \partial x^j}
%     +\frac{\partial y^\beta}{\partial x^k} \frac{\partial y^\alpha}{\partial x^j}
%     \Gamma^{\rho}_{\beta\alpha}(y) \right)
%     \Bigl(\frac{\partial}{\partial y^\rho}\Bigr)^b   ({\rm d}y^\nu)_a \\
%     =& \left( \frac{\partial v^\alpha}{\partial y^\beta} \frac{\partial x^i}{\partial y^\alpha} \frac{\partial y^\rho}{\partial x^i}
%     + v^\alpha \frac{\partial^2 x^i}{\partial y^\beta \partial y^\alpha} \frac{\partial y^\rho}{\partial x^i}   \right)
%     \Bigl(\frac{\partial}{\partial y^\rho}\Bigr)^b({\rm d}y^\beta)_a \\
%     & +  \left(v^\mu \frac{\partial x^k}{\partial y^\mu}  \frac{\partial x^j} {\partial y^\nu}
%     \frac{\partial^2 y^\rho}{\partial x^k \partial x^j}
%     +v^\mu \frac{\partial x^k}{\partial y^\mu}  \frac{\partial x^j} {\partial y^\nu}
%     \frac{\partial y^\beta}{\partial x^k} \frac{\partial y^\alpha}{\partial x^j}
%     \Gamma^{\rho}_{\beta\alpha}(y) \right)
%     \Bigl(\frac{\partial}{\partial y^\rho}\Bigr)^b   ({\rm d}y^\nu)_a  \\
%     =& \left( \frac{\partial v^\rho}{\partial y^\beta}
%     + v^\alpha \frac{\partial^2 x^i}{\partial y^\beta \partial y^\alpha} \frac{\partial y^\rho}{\partial x^i}   \right)
%     \Bigl(\frac{\partial}{\partial y^\rho}\Bigr)^b({\rm d}y^\beta)_a \\
%     & +  \left(v^\mu \frac{\partial x^k}{\partial y^\mu}  \frac{\partial x^j} {\partial y^\nu}
%     \frac{\partial^2 y^\rho}{\partial x^k \partial x^j}
%     +v^\mu \Gamma^{\rho}_{\mu\nu}(y) \right)
%     \Bigl(\frac{\partial}{\partial y^\rho}\Bigr)^b   ({\rm d}y^\nu)_a  \\
%     =& \left( \frac{\partial v^\rho}{\partial y^\nu}
%     + v^\mu \frac{\partial^2 x^i}{\partial y^\nu \partial y^\mu} \frac{\partial y^\rho}{\partial x^i}
%     + v^\mu \frac{\partial x^k}{\partial y^\mu}  \frac{\partial x^j} {\partial y^\nu}
%     \frac{\partial^2 y^\rho}{\partial x^k \partial x^j}
%     +v^\mu \Gamma^{\rho}_{\mu\nu}(y) \right)
%     \Bigl(\frac{\partial}{\partial y^\rho}\Bigr)^b({\rm d}y^\nu)_a    \\
     \xlongequal[\ref{chccr:eqn_tmp21}]{\ref{chccr:eqn_Exchange-Christoffel-a}} &
      \left( \frac{\partial v^\rho}{\partial y^\nu}
     +v^\mu \Gamma^{\rho}_{\mu\nu}(y) \right)
     \Bigl(\frac{\partial}{\partial y^\rho}\Bigr)^b({\rm d}y^\nu)_a
     = \left. \left(\left. \nabla_a v^b \right|_{V}\right) \right|_{U\cap V}  .
\end{align*}
这说明定义式\eqref{chccr:eqn_tmpvc}是大范围(整个流形$M$)适用的.
不难验证式\eqref{chccr:eqn_tmpvc}的定义满足{\kaishu 切矢量场}的
联络定义\ref{chccr:def_connection},故它是一个仿射联络.

\begin{remark}\label{chccr:remark_con-Nabla}
上面论述说明:当式\eqref{chccr:eqn_Exchange-Christoffel}满足时,
仿射联络与联络系数相互唯一确定.
\end{remark}



%联络系数是一组函数场,并非流形自然携带的信息,而是人为指定的,那它必然有相当的任意性.
%比如在$\mathbb{R}^3$中,可以考虑两个坐标系,笛卡尔坐标系和球坐标系;
%笛卡尔坐标系中联络系数全为零;球坐标系中联络系数见例\ref{chrg:exm_S3};这相当于定义了两种不同的联络.

    微分结构(即覆盖整个流形的、相互容许的、极大坐标卡集合)能够自动产生切矢量场的概念,无需其它附加结构;
    有了切矢量场,便有了它的积分曲线,进而可以定义李导数了,见\S\ref{chdm:sec_Lie-Derivative}.
    仿射联络则不然,首先,它需要微分结构;其次,还需人为给定联络系数;
    联络系数是一种不同于微分结构的新结构.




\subsection{张量场协变导数}
式\eqref{chccr:eqn_Christoffel2}给出了自然坐标基矢的协变导数定义,
可从此式导出自然坐标对偶基矢的协变导数计算公式;从计算下式开始.
\begin{equation}
    \left(\frac{\partial }{\partial x^i}\right) ^b \nabla_b
    \left[\left(\frac{\partial }{\partial x^j}\right) ^a ({\rm d}x^k)_a \right]{=}
    \left(\frac{\partial }{\partial x^i}\right) ^b \nabla_b \left[\delta_j^k \right]
    = \frac{\partial \delta_j^k}{\partial x^i} =0.
\end{equation}
上式左端还可用Leibnitz律展开,有
\begin{equation*}
    \nabla_{\frac{\partial }{\partial x^i}}
    \left[\left(\frac{\partial }{\partial x^j}\right) ^a ({\rm d}x^k)_a \right]
   =\left(\frac{\partial }{\partial x^j}\right) ^a  \left[
    \nabla_{\frac{\partial }{\partial x^i}}  ({\rm d}x^k)_a \right]
    +\left[\nabla_{\frac{\partial }{\partial x^i}}
    \left(\frac{\partial }{\partial x^j}\right) ^a  \right]({\rm d}x^k)_a .
\end{equation*}
综合以上两式,有
\begin{equation}\label{chccr:eqn_Ddualbase}
    \nabla_{\frac{\partial }{\partial x^i}}  ({\rm d}x^k)_a = -  \Gamma_{ji}^k ({\rm d}x^j)_a
    {\quad \color{red}\Leftrightarrow \quad}
    \nabla_b  ({\rm d}x^k)_a = - \Gamma_{ji}^k ({\rm d}x^j)_a ({\rm d}x^i)_b.
\end{equation}
这是与式\eqref{chccr:eqn_Christoffel2}同等重要的一个计算公式.

\subsubsection{普通导数算符}
给定$m$维仿射联络空间$(M,\nabla_a)$的局部坐标系$(U;x^i)$,引入一个常用记号.
$T_{ij}^k$表示张量$T_{ab}^c$在自然坐标基底下的分量,
用逗号“,”表示其在坐标域$\{x\}$上的偏导数,即
\begin{equation}
    T^k_{ij,l} \equiv \frac{\partial T^k_{ij}}{\partial x^l} .
\end{equation}

\index[physwords]{联络!普通导数}

我们把偏导数$\partial_a\equiv {\rm d }_a$用抽象指标表示为
\begin{equation}\label{chccr:eqn_partial-d}
    \partial_a \equiv {\rm d }_a = ({\rm d }x^j)_a \frac{\partial}{\partial x^j} .
\end{equation}
微积分中,同一坐标域内两个偏导数次序可交换;
这条原则在此变成(下式中第一条):在坐标域$(U;x^i)$内偏导数$\partial_a$满足如下规则,
\begin{equation}\label{chccr:eqn_dpdd=0}
    \partial_a \left[ \left(\frac{\partial }{\partial x^i}\right) ^b \right] =0, \quad
    \partial_a \left[ ({\rm d }x^i)_b \right] = 0; \qquad 1 \leqslant i \leqslant m .
\end{equation}
上式中第二条可以这样证明:
\begin{align*}
  &0=\partial_a(\delta^j_i)=\partial_a\left( \left(\frac{\partial }{\partial x^i}\right) ^b
 ({\rm d }x^j)_b \right)=({\rm d }x^j)_b\partial_a \left[ \left(\frac{\partial }{\partial x^i}\right) ^b \right]
 + \left(\frac{\partial }{\partial x^i}\right) ^b\partial_a \left[ ({\rm d }x^j)_b \right] \\
 &\Rightarrow \quad 0=\left(\frac{\partial }{\partial x^i}\right) ^b \partial_a \left[ ({\rm d }x^j)_b \right]
 \quad \Rightarrow\quad  \partial_a \left[ ({\rm d }x^j)_b \right] = 0 .
\end{align*}
由此可得在局部坐标域$(U;x^i)$内张量场的偏导数为
\begin{equation}
    \partial_a T^c_b = \partial_a \left[T^k_j \left(\frac{\partial }{\partial x^k}\right) ^c
      ({\rm d }x^j)_b \right] = T^k_{j,i} \left(\frac{\partial }{\partial x^k}\right) ^c
      ({\rm d }x^j)_b  ({\rm d }x^i)_a.
\end{equation}
在坐标域$(U;x^i)$内,算符$\partial_a$满足仿射联络\ref{chccr:def_connection}的各条性质.
联络不能只在一个坐标域内成立,在进行坐标变换时,联络形式必须是协变的(不同
坐标系下,数学表达式“长得一样”才行).
再给$M$的另一个局部坐标系$(V;y^\alpha)$,并且$U\cap V \neq \varnothing$,
那么对$V$中自然坐标基矢的偏导数是:
\begin{equation}
\begin{aligned}
    \partial_a \left[ \left(\frac{\partial }{\partial y^\alpha}\right) ^b \right]
    =& \partial_a \left[\frac{\partial x^k}{\partial y^\alpha}
      \left(\frac{\partial }{\partial x^k}\right) ^b \right]
    = ({\rm d }x^j)_a \left(\frac{\partial }{\partial x^k}\right) ^b \times
      \frac{\partial}{\partial x^j}\frac{\partial x^k}{\partial y^\alpha} \\
    =& ({\rm d }x^j)_a \left(\frac{\partial }{\partial x^k}\right) ^b\times
     \frac{\partial \delta^k_j}{\partial y^\alpha}  =0 .
\end{aligned}
\end{equation}
最后一步用到了常数的偏导数为零这一性质.$\forall v^b \in \mathfrak{X}(U\cap V)$,计算其偏导数
    \begin{align*}
        \partial_a (v^b|_V ) =&    ({\rm d }x^j)_a\frac{\partial }{\partial x^j} \left[v^\alpha
        \left(\frac{\partial }{\partial y^\alpha}\right) ^b \right]
        = \left(\frac{\partial }{\partial y^\alpha}\right) ^b ({\rm d }x^j)_a
          \frac{\partial }{\partial x^j}\left(v^i \frac{\partial y^\alpha}{\partial x^i}\right) \\
%        =& \left(\frac{\partial }{\partial y^\alpha}\right) ^b ({\rm d }x^j)_a
%          \left(\frac{\partial v^i}{\partial x^j} \frac{\partial y^\alpha}{\partial x^i}
%          +v^i \frac{\partial^2 y^\alpha}{\partial x^j\partial x^i}\right) \\
        =& \left(\frac{\partial }{\partial x^i}\right) ^b ({\rm d }x^j)_a
        \frac{\partial v^i}{\partial x^j}
        +\left(\frac{\partial }{\partial y^\alpha}\right) ^b ({\rm d }x^j)_a
        \times v^i \frac{\partial^2 y^\alpha}{\partial x^j\partial x^i}   \\
        =& \partial_a (v^b|_U ) +\left(\frac{\partial }{\partial y^\alpha}\right) ^b ({\rm d }x^j)_a
        \times v^i \frac{\partial^2 y^\alpha}{\partial x^j\partial x^i}   .
    \end{align*}
上面给出了详尽的计算过程.一般说
来$\frac{\partial^2 y^\alpha}{\partial x^j\partial x^i}\neq 0$,
所以$\partial_a v^b$在坐标变换时不具有协变属性.
因此可以得到结论:\uwave{$\partial_a$不是联络}.

联络定义\ref{chccr:def_connection}隐藏了局部坐标卡,但隐含着:
在任意局部坐标卡中,联络必须满足定义中的三条公理.
从上面的推导可以看出,算符$\partial_a$只能在某个坐标卡满足三条公理,
在与其相邻的不同坐标卡中可能不满足;这也说明$\partial_a$不是联络.


虽然$\partial_a$不是联络,但是在局部坐标$(U;x^i)$上,当作用在标量函数场$f$上时,
它们是相同的,即$\partial_a f = \nabla_a f = {\rm d}_a f$.
其实,几乎所有“导数”的算符,作用在标量场上时所得结果必须与微积分中结果相同;否则不能称为“导数”.

\index[physwords]{协变导数|see{联络}}
\index[physwords]{联络!协变导数}
\subsubsection{协变导数算符}
有了联络系数,逆变切矢场和协变矢量的协变导数的分量表示(计算见后)
\begin{align}
    \nabla_a X^b & = \Bigl(\frac{\partial}{\partial x^i}\Bigr)^b ({\rm d}x^j)_a
    \left(\frac{\partial X^i}{\partial x^j} +X^k \Gamma_{kj}^i \right)
    \equiv \Bigl(\frac{\partial}{\partial x^i}\Bigr)^b ({\rm d}x^j)_a  X^i_{\hphantom{i};j}  .
    \label{chccr:eqn_X-covariantD} \\
    \nabla_a \omega_b & = ({\rm d}x^j)_a  ({\rm d}x^i)_b \left(\frac{\partial \omega_i}{\partial x^j}
    -\omega_k \Gamma_{ij}^k \right) \equiv({\rm d}x^j)_a  ({\rm d}x^i)_b  \omega_{i;j} .
    \label{chccr:eqn_w-covariantD}
\end{align}
上两式最后的恒等号“$\equiv$”定义了:\uwave{用分号“;”代表协变导数}.即有
\begin{align}
    X^i_{\hphantom{i};j} \equiv& \frac{\partial X^i}{\partial x^j} +X^k \Gamma_{kj}^i
       =X^i_{\hphantom{i}, j} +X^k \Gamma_{kj}^i, \\
    \omega_{i;j} \equiv& \frac{\partial \omega_i}{\partial x^j} -\omega_k \Gamma_{ij}^k
       =\omega_{i,j} -\omega_k \Gamma_{ij}^k .
\end{align}
下面给出式\eqref{chccr:eqn_X-covariantD}的计算过程:
\begin{equation}
\begin{aligned}
    \nabla_a X^b & = \nabla_a \left[X^i \left(\frac{\partial }{\partial x^i}\right) ^b \right]
    \xlongequal{\ref{chccr:def_connection} (3)}
    \left[\nabla_a X^i\right] \left(\frac{\partial }{\partial x^i}\right) ^b
      +X^i \nabla_a \left(\frac{\partial }{\partial x^i}\right) ^b \\
    & \xlongequal[\ref{chccr:eqn_Christoffel2}]{\ref{chccr:eqn_partial-d}}
    \frac{\partial X^i}{\partial x^j}({\rm d}x^j)_a \left(\frac{\partial }{\partial x^i}\right) ^b
      +X^k \Gamma^{i}_{kj} \left(\frac{\partial }{\partial x^i}\right) ^b ({\rm d} x^j)_a  .
\end{aligned}
\end{equation}
整理之后便是式\eqref{chccr:eqn_X-covariantD}.
式\eqref{chccr:eqn_w-covariantD}的计算类似,留给读者当练习.

有逆变与协变矢量的协变导数容易推广到高阶张量,
下面给出$\Tpq{1}{2}$型张量的协变导数计算公式(推导过程并不复杂,请读者自行完成).
\begin{equation}\label{chccr:eqn_T-covariantD}
    \nabla_a T^b_{\hphantom{a} cd} = \partial_a T^b_{\hphantom{a} cd} + \Gamma^b_{ea} T^e_{\hphantom{a}  cd}
     -\Gamma^e_{ca} T^b_{\hphantom{a} ed} -\Gamma^e_{da} T^b_{\hphantom{a} ce} .
\end{equation}
更高阶张量的协变导数公式,请读者自行写出,需注意系数的正负号.

在式\eqref{chccr:eqn_tmpvc}中,我们已经证明了$\nabla_a X^b$的分量表达式在
坐标系变换时具有协变性,再结合$\nabla_a X^b$表达式形式可以断定它是一个$\Tpq{1}{1}$型张量;
与此类似也可证明式\eqref{chccr:eqn_w-covariantD}是一个$\Tpq{0}{2}$型张量;
对更高阶张量场(比如\eqref{chccr:eqn_T-covariantD})也可作类似证明.
这些便验证了联络$\nabla_a$把$\Tpq{r}{s}$型张量场映射为$\Tpq{r}{s+1}$型张量场.

\begin{example}\label{chccr:exm_c-c=t}
    注\ref{chccr:rek_ccl}中说切丛上的联络不止一个,现设切丛$TM$上有
    两个不同的仿射联络$\nabla_a$和${\rm D}_a$,与它们相对应的联络
    系数分别是$\Gamma^c_{ab}$和$\widetilde{\Gamma}^c_{ab}$.
    由评注\ref{chccr:remark_con-Nabla}可知,在满足式\eqref{chccr:eqn_Exchange-Christoffel}的
    前提下,联络和联络系数相互唯一确定;所以不同联络是指其对应的联络系数不同.
    取流形$M$的局部坐标系$(U;x)$,直接计算,有
    \begin{equation}\label{chccr:eqn_c-c=t}
        (\nabla_a-{\rm D}_a) X^b  %= \Bigl(\frac{\partial}{\partial x^i}\Bigr)^b ({\rm d}x^j)_a
          %X^k \left( \Gamma_{kj}^i - \widetilde{\Gamma}_{kj}^i \right)
        = X^c \left( \Gamma_{ca}^b - \widetilde{\Gamma}_{ca}^b \right)
        \equiv X^c \Xi_{ca}^b .
    \end{equation}
    虽然联络系数不是张量,但由上式,
    利用定理\ref{chdm:thm_Tensor-Characterization-Lemma}容易证
    明$\Xi_{ca}^b$是$\Tpq{1}{2}$型张量场.
    
    比如,三维欧几里得空间上有笛卡尔坐标系$\{x^i\}$,我们取${\rm D}_a \equiv \partial_a$,
    再取$\Xi_{ca}^b=x_c x_a x^b$.由定理\ref{chrg:thm_Levi-Civita-Connetion}可
    知${\rm D}_a$是欧几里得空间的Levi-Civita联络;
    而由式\eqref{chccr:eqn_c-c=t}算出的$\nabla_a$是不同于${\rm D}_a$的一个仿射联络.\qed
\end{example}


\index[physwords]{挠率}

\subsection{挠率}\label{chccr:sec_torsion}
\begin{definition}
    设$(M,\nabla_a)$是$m$维仿射空间,定义挠率张量$T_{bc}^a$:
    \begin{equation}\label{chccr:eqn_Ttorsion}
        T_{bc}^a{X^b}{Y^c} \overset{def}{=} {\nabla _X}{Y^a} - {\nabla _Y}{X^a} - {\left[ {X,Y} \right]^a} ,
        \qquad \forall X^a, Y^a \in \mathfrak{X}(M) .
    \end{equation}
\end{definition}
用推论\ref{chdm:thm_Tensor-Characterization-Lemma-1}可证明$T_{bc}^a$是一张量场,请读者补齐.
它关于下标反对称,在局部坐标系的分量表达式为
\begin{equation}\label{chccr:eqn_Ttorsion-comp}
    T_{jk}^i=T_{bc}^a{\left(\frac{\partial}{\partial x^j}\right)^b}
    {\left(\frac{\partial}{\partial x^k}\right)^c}
    ({\rm d} x^i)_a = \Gamma^{i}_{kj} - \Gamma^{i}_{jk} .
\end{equation}
当$\Gamma^{i}_{kj}$下标对称时,挠率恒为零.
反之,若$T_{bc}^a\equiv 0$,则$\Gamma^{i}_{kj}$下标对称.
\begin{definition}
    设$(M,\nabla_a)$是仿射空间,若挠率$T_{bc}^a$恒为零,
    则称$\nabla_a$为{\heiti 无挠联络}.
\end{definition}

\begin{remark}\label{chccr:remark_connection-coef-num}
    联络有挠时,系数$\Gamma^{i}_{jk}$有$m^3$个;
    无挠时,系数有$\frac{1}{2}m^2(m+1)$个.
\end{remark}


给定流形$(M,\nabla_a)$上标量场$f$,有
\begin{equation}
    \nabla_a\nabla_b f= \partial_a\nabla_b f - \Gamma_{ba}^c \nabla_c f
    = \partial_a\partial_b f - \Gamma_{ba}^c \partial_c f .
\end{equation}
由此,得
\begin{equation}\label{chccr:eqn_scalar-torsion}
    \nabla_a\nabla_b f - \nabla_b\nabla_a f=  - (\Gamma_{ba}^c-\Gamma_{ab}^c) \nabla_c f
    = - T_{ab}^c \nabla_c f .
\end{equation}
上式是协变导数对易子作用在标量场计算公式.
此式也可当成挠率的定义式.

%挠率的几何意义并不是十分明确,用处较少.无挠联络有较好的性质,比如下面的定理,
%\begin{theorem}
%\end{theorem}





\subsection{无挠联络的优点}\label{chccr:sec_eqn-Nalba}
给定仿射空间$(M,\nabla_a)$,其联络无挠,则其联络系数的下指标是对称的,会给表达式带来诸多好处.
本节中,$f\in C^\infty(M)$,$X^a,Y^a\in \mathfrak{X}(M)$,$\omega \in A^r(M)$,
$T_{ab} \in \mathfrak{T}^0_2(M)$.所有证明留给读者当练习.

%\subsubsection{协变导数的简化}
反对称化操作可令(无挠)协变导数退化为偏导数
\begin{align}
    \nabla_{[b}T_{cd]} &= \partial_{[b}T_{cd]}, \label{chccr:eqn_anti-covD-parD1}\\
    \nabla_{[a}\partial_{b]}f &= \partial_{[a}\partial_{b]}f=0, \qquad
    \nabla_{[a}\nabla_{b}T_{cd]} = \partial_{[a}\partial_{b}T_{cd]}=0. \label{chccr:eqn_anti-covD-parD2}
\end{align}
需注意,上式必须是协变张量,不能有逆变部分.张量的协变指标可以是任意多个,
协变导数也可以是任意多个.


\index[physwords]{Poisson括号!无挠联络}

下式中第一个等号是对易子的定义;联络无挠时第二个等号才成立.
\begin{equation}\label{chccr:eqn_XYcommutator}
    \left[ {X,Y} \right]^a = {X^k}\frac{{\partial {Y^j}}}{{\partial {x^k}}}
    {\left( {\frac{\partial }{{\partial {x^j}}}} \right)^a} -
    {Y^k}\frac{{\partial {X^j}}}{{\partial {x^k}}}{\left( {\frac{\partial }{{\partial {x^j}}}} \right)^a}
    = \nabla_X Y^a - \nabla_Y X^a .
\end{equation}



\subsubsection{李导数}\label{chccr:sec_LieD-Nabla}
李导数定义并不需要联络,但当联络无挠时表示更简洁.
矢量场的李导数可由式\eqref{chccr:eqn_XYcommutator}来表示(见\eqref{chdm:eqn_LieD-tangent}).
与式\eqref{chdm:eqn_LieD-cotangent}对应的余切矢量场表达式为:
\begin{align}
    \Lie_{X} \omega_{b} &= X^c\nabla_c \omega_{b} +  \omega_{c} \nabla_b X^c
     =X^c\partial_c \omega_{b} + \omega_{c} \partial_b X^c . \label{chccr:eqn_LieD-cotangent-Nabla} \\
    \Lie_{X} T^{a}_{\hphantom{a} bc} &= X^e\nabla_e T^{a}_{\hphantom{a} bc}
    - T^{e}_{\hphantom{a} bc} \nabla_e X^a + T^{a}_{\hphantom{a} ec} \nabla_b X^e
    + T^{a}_{\hphantom{a} be} \nabla_c X^e ,  \label{chccr:eqn_LieD-tensor-Nabla} \\
    &= X^e\partial_e T^{a}_{\hphantom{a} bc}
    - T^{e}_{\hphantom{a} bc} \partial_e X^a + T^{a}_{\hphantom{a} ec} \partial_b X^e
    + T^{a}_{\hphantom{a} be} \partial_c X^e .  \label{chccr:eqn_LieD-tensor-Partial}
\end{align}
上面第二式为$\Tpq{1}{2}$型张量场的李导数;易将这个表达式推广到高阶张量.

%由于李导数很重要,下面再给出几个关于李导数的公式.
下面几式通常称为Henri Cartan(\'Elie Cartan 儿子)公式.
\begin{align}
    &\Lie_{X}  ( \omega_{ba_2 \cdots a_r} Y^b)
    -  ( \Lie_{X}  \omega_{ba_2 \cdots a_r}  ) Y^b
    =  \omega_{ba_2 \cdots a_r} [X,Y]^b . \label{chccr:eqn_Lie-iiLie=ixy} \\
    &\Lie_{X} \omega_{a_1 \cdots a_r} = {\rm d}_{a_1}( \omega_{ba_2 \cdots a_r} X^b )
    +( {\rm d}\omega)_{ba_1 \cdots a_r} X^b. \label{chccr:eqn_Lie-i(X)-d} \\
    &\Lie_{X} ({\rm d} \omega)_{b a_1 \cdots a_r} = {\rm d}_b
    \Lie_{X} \omega_{a_1 \cdots a_r} . \label{chccr:eqn_Lie-d=d-Lie} \\
    & \Lie_{X} \circ \Lie_{Y} -\Lie_{Y}\circ \Lie_{X} = \Lie_{[X,Y]} .
      {\quad \text{此式可作用在任意张量场上}} \label{chccr:eqn_Lie-XYYX=Liexy} \\
    & \Lie_{X}(\omega \wedge \psi ) = (\Lie_{X} \omega) \wedge \psi
       + \omega \wedge \Lie_{X}\psi, \quad \forall
       \omega\in A^r(M),\ \psi\in A^s(M). \label{chccr:eqn_Lie-OPPO}
\end{align}

\index[physwords]{李导数!无挠联络}


\subsubsection{外微分}\label{chccr:sec_exD-Nalba}
联络无挠时,式\eqref{chdf:eqn_exterior-differential}与下式等价,
定义\eqref{chdf:eqn_exterior-differential}是最基本的;下式更好用.
\begin{equation}\label{chccr:eqn_exterior-differential-covD}
    {\rm d}\omega_{b a_{1}\cdots a_{r}} = (r+1) \nabla _{[b}\omega_{a_{1}\cdots a_{r}]} .
\end{equation}

\index[physwords]{外微分“${\rm d}$”!无挠联络}


\subsubsection{无挠联络系数为零的局部坐标系}\label{chccr:sec_0-Christoffel}
当联络无挠时,$\forall p\in M$存在局部坐标系使得$p$点联络系数为零.
设流形有局部坐标$(U;x^i)$,联络系数$\Gamma'^k_{ij}$不为零,令
\begin{equation}\label{chccr:eqn_tmp30}
    y^k=x^k+\frac{1}{2} \Gamma'^k_{ij}(p) \bigl(x^i- x^i(p)\bigr) \bigl(x^j- x^j(p)\bigr) ,
\end{equation}
那么有
\begin{equation}
    \left. \frac{\partial y^k}{\partial x^l} \right|_{p} = \delta ^k_l, \qquad
    \left. \frac{\partial^2 y^k}{\partial x^l \partial x^n} \right|_{p}
    =\frac{1}{2}\Gamma'^k_{l n}(p) + \frac{1}{2}\Gamma'^k_{n l}(p)  = \Gamma'^k_{l n}(p) .
\end{equation}
如果联络有挠,上面第二式是不成立的.
显然,矩阵$\frac{\partial y^k}{\partial x^l}$在$p$点附近的小邻域内是非退化的,
式\eqref{chccr:eqn_tmp30}给出$p$点的一个局部坐标变换.
由\eqref{chccr:eqn_Exchange-Christoffel-a}得到在坐标系$\{y\}$下的
新联络系数$\Gamma^{k}_{ji}(p)=0$.
%需注意,此式只在$p$点成立;哪怕离开$p$一点儿,$\Gamma^{k}_{ji}$也可能不为零.



\index[physwords]{平行移动}
\subsection{平行移动}\label{chccr:sec_pxyd}
\uwave{仿射联络的几何意义便是平行移动};
平行移动在二维曲面论中有较为清晰的图像(见\S\ref{chcdg:sec_pt}),
高维空间只有公式,没有图像.
虽然图像不是十分清晰,我们仍然给出相应描述;先从定义开始.
\begin{definition}\label{chccr:def_px}
    给定$m$维仿射空间$(M,\nabla_a)$,$\gamma:[r_1,r_2]\to M$是流形$M$中
    的一条光滑曲线,$T^b=(\frac{\partial}{\partial t})^b|_{\gamma(t)}$是曲线$\gamma(t)$的切线切矢.
    $\forall Y^a \in \mathfrak{X}(\gamma(t))$,如果沿曲线$\gamma(t)$上有
    \begin{equation}\label{chccr:eqn_px}
        T^b \nabla _b Y^a \equiv \nabla_{\frac{\partial}{\partial t}} Y^a =0 ,
    \end{equation}
    那么,称切矢场$Y^a$沿曲线$\gamma(t)$是{\heiti 平行}的.
%    或称$Y^a$是沿曲线$\gamma(t)$的{\heiti 平行切矢量场}.
\end{definition}
可以认为$Y^a(t)$只在$\gamma(t)$上才有意义,在此条曲线之外(即$M-\gamma(t)$上)没有定义.
需注意,$\nabla _b Y^a(t)$本身没有定义,因为这个导数需要$Y^a$在$M-\gamma(t)$上值;
但$T^b \nabla _b Y^a(t)$是有良定义的,它只需要知道$\gamma(t)$上的$Y^a$值即可.
还有一点需要澄清,真正切于曲线$\gamma(t)$的矢量只有两个方向,即$\pm(\frac{\partial}{\partial t})^a$;
而$Y^a(t)$是切于流形$M$的,它可能切于$\gamma(t)$,也可能不切于$\gamma(t)$,
但上面定义中的$Y^a(t)$都是在曲线$\gamma(t)$上取值.


继续沿用定义\ref{chccr:def_px}中的符号,
设曲线$\gamma(t)$的局部坐标参数表达式为$\{x^i(t)\}$,
矢量场$Y^a$在此曲线上的分量表达式为$Y^a=Y^i(t)(\frac{\partial}{\partial x^i})^a$;
由式\eqref{chccr:eqn_X-covariantD}有
\begin{equation}\label{chccr:eqn_px-formula}
    T^b \nabla _b Y^a = \left(\frac{\partial}{\partial x^i}\right)^a
    \left(\frac{{\rm d} Y^i(t)}{{\rm d}t} +
    \frac{{\rm d}x^j(t)}{{\rm d} t} Y^k(t) \Gamma_{kj}^i(t) \right) .
\end{equation}

测地线定义为(我们会在第\ref{chgd}章专门讨论这个问题,在此就不展开了):
\begin{definition}\label{chccr:def_geodesic}
如果仿射空间$(M,\nabla_a)$中曲线$\gamma(t)$沿自身切线平行,即$T^b \nabla _b T^a=0$;
则称其为仿射空间$M$的{\heiti 测地线}.  \index[physwords]{测地线}
参数$t$为该曲线的{\heiti 仿射参数}.
\end{definition}



$Y^a(t)$是沿光滑曲线$\gamma(t)$平行的切矢量场,由式\eqref{chccr:eqn_px-formula}可知它满足
\begin{equation}\label{chccr:eqn_pxf1}
    \frac{{\rm d} Y^i(t)}{{\rm d}t} +
    \frac{{\rm d}x^j(t)}{{\rm d} t} Y^k(t) \Gamma_{kj}^i(t) =0;
    \qquad Y^i(0)=Y^i_0,\quad 1\leqslant i \leqslant m.
\end{equation}
很明显,一阶线性齐次常微分方程组\eqref{chccr:eqn_pxf1}的解构成一个向量空间,
并且它的每一个解是由$T_p M$中的元素$Y^a_0$唯一确定的(常微分方程组解存在唯一性定理).
因此,沿光滑曲线$\gamma$的平行矢量场的集合构成一个与$T_p M$线性同构的矢量空间.
特别地,对于任意取定的$t\in [0,b]$,
沿$\gamma$的平行矢量场给出了从 $T_p M$ 到 $T_{\gamma(t)} M$的
线性同构$P_0^t: T_p M \rightarrow T_{\gamma(t)} M$,
称为沿曲线$\gamma$从$t=0$到$t$的{\heiti 平行移动}.
这样,由切矢量$Y^a_0 \in T_p M$确定的沿曲线$\gamma$平行的矢量场$Y^a(t)$可以表示为:
\begin{equation}
    Y^a(t)=P_0^t \left(Y^a_0\right), \qquad 0 \leqslant t \leqslant b .
\end{equation}


总之,在光滑流形上只要指定了联络,就可以建立平行移动的概念.
下面的定理说明,切矢量场的协变导数(联络)也可以借 助于平行移动来表示.
\begin{theorem}\label{chccr:thm_parallel-transport}
    设$(M, \nabla_a)$是一个$m$维仿射联络空间,$\gamma:[0, b] \to M$是$M$中任意一条光滑曲线.
    则对于任意的$Y^a \in \mathfrak{X}(M)$,有(其中$T^b=(\frac{\partial}{\partial t})^b|_{\gamma(t)}$)
    \begin{equation}\label{chccr:eqn_parallel-transport}
        T^b\nabla_b Y^a= \lim _{\Delta t \to 0} \frac{P_{t+\Delta t}^t
            \bigl(Y^a \circ \gamma(t+\Delta t)\bigr)-Y^a \circ \gamma(t)}{\Delta t} .
    \end{equation}    
\end{theorem}
\begin{proof}
记 $p=\gamma(0)$.在$T_p M$中取定一个基底 $\{(e_i)^a\}$,并且设
$\bigl(e_i(t)\bigr)^a=P_0^t\bigl((e_i)^a\bigr)$,
则$\bigl(e_i(0)\bigr)^a=(e_i)^a$,而且$\bigl(e_i(t)\bigr)^a$是沿 $\gamma$ 的平行矢量场,
即有$T^b\nabla_b \bigl(e_i(t)\bigr)^a \equiv 0$.
由于$P_0^t: T_p M \rightarrow T_{\gamma(t)} M$是线性同构,
故$\{\bigl(e_i(t)\bigr)^a\}$是$T_{\gamma(t)} M$的基底.
于是,$Y^a$在 $\gamma$上的限制可以表示为$Y^a\bigl(\gamma(t)\bigr)=\sum_{i=1}^m Y^i(t) \bigl(e_i(t)\bigr)^a$,
其中$Y^i(t)$是$t$的光滑函数.故有
\begin{align*}
    \nabla_T Y^a\bigl(\gamma(t)\bigr)= \nabla_T\bigl(Y^i(t)\bigr) \cdot \bigl(e_i(t)\bigr)^a
    +Y^i(t) \cdot \nabla_T \bigl(e_i(t)\bigr)^a
    = \frac{\mathrm{d} Y^i(t)}{\mathrm{d} t}   \bigl(e_i(t)\bigr)^a .
\end{align*}
在另一方面,因为$P_{t+\Delta t}^t: T_{\gamma(t+\Delta t)} M \rightarrow T_{\gamma(t)} M$是线性同构,故
\begin{align*}
    P_{t+\Delta t}^t\Bigl(Y^a\bigl(\gamma(t+\Delta t)\bigr)\Bigr)  
    =P_{t+\Delta t}^t\left(Y^i(t+\Delta t)   \bigl(e_i(t+\Delta t)\bigr)^a \right) 
    =Y^i(t+\Delta t) \bigl(e_i(t) \bigr)^a.
\end{align*}
从而
\begin{align*}
    \frac{P_{t+\Delta t}^t\bigl(Y^a \circ \gamma(t+\Delta t)\bigr)- Y^a \circ \gamma(t)}{\Delta t}
    =\frac{Y^i(t+\Delta t)-Y^i(t)}{\Delta t} \bigl(e_i(t)\bigr)^a .
\end{align*}
令上式中的$\Delta t \rightarrow 0$, 则得
\begin{align*}
    \lim _{\Delta t \to 0} \frac{P_{t+\Delta t}^t\bigl(Y^a \circ \gamma(t+\Delta t)\bigr)-Y^a \circ \gamma(t)}{\Delta t} 
    =\frac{\mathrm{d} Y^i(t)}{\mathrm{d} t} \bigl(e_i(t)\bigr)^a= \nabla_T Y^a\bigl(\gamma(t)\bigr) .
\end{align*}
证毕.
\end{proof}

公式\eqref{chccr:eqn_parallel-transport}告诉我们:
$\nabla_T Y^a$是切矢量场$Y^a$在邻近两点$\gamma(t)$、$\gamma(t+\Delta t)$的值之差
与$\Delta t$的商的极限;只不过 $Y^a\bigl(\gamma(t+\Delta t)\bigr)$是$M$在$\gamma(t+\Delta t)$处的切矢量,
必须借助于线性同构$P_{t+\Delta t}^t$把它变为在$\gamma(t)$处的切矢量之后,才能与$Y^a\bigl(\gamma(t)\bigr)$相减. 
这也是“联络”的一个表观解释,两点$\gamma(t)$、$\gamma(t+\Delta t)$的值原来是没有关联,无法作差;
现在利用曲线$\gamma$上的平行移动将这两点“联系”起来,它们可以作差了;
需要强调的是,平行移动表示下的联络是曲线依赖的,体现在联络系数$\Gamma^i_{jk}\bigl(\gamma(t)\bigr)$上.


定理\ref{chccr:thm_parallel-transport}初步说明了:
由平行移动出发可以给仿射联络一个定义.
文献\parencite{wuhx2014cb}第一章的附录给出了一个更严谨的证明.
这说明:仿射联络和平行移动是同一事物的两种不同表述.


%\subsection{仿射变换}\label{chccr:sec_affine-transformation}
%本小节叙述切矢量场、仿射联络的仿射变换\cite[\S 1.4]{helgason-2001};
%请先参考\S\ref{chdm:sec_related}.
%设有仿射空间$(M,\nabla_a)$和$(N,{\rm D}_a)$,
%再设有微分同胚映射$\phi:M\to N$;$\forall X^a,Y^a\in \mathfrak{X}(M)$,如果
%\begin{equation}\label{chccr:D_affine-transformation}
%    \phi_{*}\bigl(\nabla_X Y^a\bigr) = {\rm D}_{\phi_{*}X} (\phi_{*} Y^a)
%\end{equation}
%成立,
%那么称$\nabla_a$与${\rm D}_a$在微分同胚$\phi$作用下是{\heiti 不变}的;
%此时的$\phi$称为流形$M\to N$的{\heiti 仿射变换}.
%流形$N$可以是$M$自身.

%由定理\ref{chdm:thm_fxxf}和\ref{chdm:thm_push-Poisson-related}可知上述定义是合理的.

\begin{exercise}
	证明式\eqref{chccr:eqn_w-covariantD}.
\end{exercise}

\begin{exercise}
	证明式\eqref{chccr:eqn_T-covariantD},并给出上下标分别为$p$、$q$个的一般性公式.
\end{exercise}

\begin{exercise}
	证明\S\ref{chccr:sec_eqn-Nalba}中所有公式;有数个公式证明很繁琐.
\end{exercise}

\index[physwords]{黎曼曲率}

\section{黎曼曲率张量}\label{chccr:sec_Curvatures}

现代几何学中的黎曼曲率张量似乎是一个神秘的东西.部分地是因为其名称显示它能够对难以捉摸
的“空间弯曲”性质给出一个精确度量;部分地是因为尽管看上去它很简单,但它似乎控制了黎曼
几何学的每一方面;又部分地因为虽然从过去到现在,学者们已花费极大努力,但距离完全
掌握它还相距甚远.文献\parencite[Ch.4]{spivak-dif-2}用了许多篇幅来描述这个定义的历史起源、
直观内容、不同情形下的种种姿态等等.
笔者学识所限,无法深入展示黎曼曲率的种种精妙,我们只给出它的定义并叙述它的一些性质.

\begin{definition}\label{chccr:def_tCurvatures}
    设$(M,\nabla_a)$是$m$维仿射空间,定义张量$R^{d}_{\hphantom{a} cab}$如下:
    \begin{equation}\label{chccr:eqn_RiemannianCurvature-13}
        R_{\hphantom{a} cab}^d{X^a}{Y^b}{Z^c} \overset{def}{=} {\nabla _X}{\nabla _Y}{Z^d} -
        {\nabla _Y}{\nabla _X}{Z^d} - {\nabla _{[X,Y]}}{Z^d}.
    \end{equation}
    其中$\forall X^a, Y^a, Z^a \in \mathfrak{X}(M)$.
    由上式定义的$\Tpq{1}{3}$型张量场$R^{d}_{\hphantom{a} cab}$是{\heiti 黎曼曲率张量}.
\end{definition}
\begin{remark}
    我们已经约定$\Tpq{1}{3}$型黎曼张量上指标降下来后放在第一个位置,
    以后为了节省角标空间,很多时候将$R_{\hphantom{a} cab}^d$写成$R_{cab}^d$.
    上指标降下位置没有特别意义,完全是人为规定;不同文献的习惯未必相同,读者需小心.
\end{remark}

我们来看一下$\Tpq{1}{3}$型黎曼张量定义可以怎样变化.
\begin{align*}
    &R_{\hphantom{a} cab}^d{X^a}{Y^b}{Z^c} = {X^a}{\nabla _a}\left( {{Y^b}{\nabla _b}{Z^d}} \right) -
    {Y^b}{\nabla _b}\left( {{X^a}{\nabla _a}{Z^d}} \right) - {[X,Y]^e}{\nabla _e}{Z^d} \\
    &\quad \xlongequal{\eqref{chccr:eqn_Ttorsion}} {X^a}\left( {{\nabla _a}{Y^b}} \right){\nabla _b}{Z^d}
    + {X^a}{Y^b}{\nabla _a}{\nabla _b}{Z^d}
    - {Y^b}\left( {{\nabla _b}{X^a}} \right){\nabla _a}{Z^d}  \\
    &\qquad - {X^a}{Y^b}{\nabla _b}{\nabla _a}{Z^d} + \left[ {T_{bf}^e{X^b}{Y^f}
         - {X^b}{\nabla _b}{Y^e} + {Y^b}{\nabla _b}{X^e}} \right]{\nabla _e}{Z^d} \\
    &\quad = {X^a}{Y^b}\left( {{\nabla _a}{\nabla _b}{Z^d} - {\nabla _b}{\nabla _a}{Z^d}} \right)
    + T_{ab}^e{X^a}{Y^b}{\nabla _e}{Z^d} .
\end{align*}
因矢量$X^a,Y^b$的任意性,由上式可得
\begin{equation}\label{chccr:eqn_Riemannian13-Vec-commutator}
    \nabla_a \nabla_b Z^d - \nabla_b \nabla_a Z^d = R_{\hphantom{a} cab}^d Z^c - T_{ab}^e \nabla_e Z^d .
\end{equation}
由式\eqref{chccr:eqn_Riemannian13-Vec-commutator}、
式\eqref{chccr:eqn_scalar-torsion}及Leibniz律,经计算(留给读者)可以得到
\begin{align}
    {\nabla_a}{\nabla_b}{\omega_c} - {\nabla_b}{\nabla_a}{\omega_c} &= -R_{\hphantom{a} cab}^e{\omega _e}
    - T_{ab}^e{\nabla _e}{\omega _c},  \label{chccr:eqn_Riemannian13-CoVec-commutator} \\
    {\nabla _a}{\nabla _b}S_{ \hphantom{a} c}^d - {\nabla _b}{\nabla _a}S_{ \hphantom{a} c}^d &= 
    +R_{\hphantom{a} eab}^d S_{ \hphantom{a} c}^e
    - R_{\hphantom{a} cab}^eS_{ \hphantom{a} e}^d - T_{ab}^e{\nabla _e}S_{ \hphantom{a} c}^d . 
    \label{chccr:eqn_Riemannian13-Tensor-commutator}
\end{align}
从式\eqref{chccr:eqn_Riemannian13-Tensor-commutator},不难将协变导数对易子推广到高维张量,
需注意系数的正负号;上述双联络对易子公式一般称为Ricci恒等式.
\begin{remark}
    由协变导数对易子公式\eqref{chccr:eqn_scalar-torsion}、\eqref{chccr:eqn_Riemannian13-Vec-commutator}、
    \eqref{chccr:eqn_Riemannian13-CoVec-commutator}和\eqref{chccr:eqn_Riemannian13-Tensor-commutator}
    易得:协变导数对易子公式完全由黎曼曲率和挠率确定.联络无挠时,挠率相关项消失.
\end{remark}
\begin{remark}\label{chccr:rek_DD}
    已知联络$\nabla_a$作用在张量积上遵循Leibniz法则,经计算可以确定
    双联络对易子$(\nabla_a \nabla_b - \nabla_b \nabla_a)$也遵循Leibniz法则,
    即对$M$上任意两个张量场$K,L$有
    $(\nabla_a \nabla_b - \nabla_b \nabla_a)(K\otimes L)
    =[(\nabla_a \nabla_b - \nabla_b \nabla_a) K]\otimes  L+
    K\otimes (\nabla_a \nabla_b - \nabla_b \nabla_a) L$.
\end{remark}

%\subsection{曲率局部坐标分量表示}

由式\eqref{chccr:eqn_RiemannianCurvature-13}可得黎曼曲率在局部坐标表达式:
\begin{align}
    R_{\hphantom{a} jln}^i &=R_{\hphantom{a} cab}^d {\left( {\frac{\partial }{{\partial {x^l}}}} \right)^a}
    {\left( {\frac{\partial }{{\partial {x^n}}}} \right)^b}
    {\left( {\frac{\partial }{{\partial {x^j}}}} \right)^c} {({\rm d}x^i)_d} \notag\\
    &= {({\rm d}x^i)_d} \left[\nabla _{\frac{\partial }{\partial x^l} }
    \nabla _{\frac{\partial }{\partial x^n}} {\left( {\frac{\partial }{{\partial {x^j}}}} \right)^d}
    - {\nabla _{\frac{\partial }{\partial x^n}}}
    {\nabla _{\frac{\partial }{\partial x^l}}}{\left( {\frac{\partial }{{\partial {x^j}}}} \right)^d}
    - {\nabla _{[\frac{\partial }{\partial x^l},\frac{\partial }{\partial x^n}]}}
    {\left( {\frac{\partial }{{\partial {x^j}}}} \right)^d}\right] \notag \\
    &= {({\rm d}x^i)_d}{\nabla _{\frac{\partial }{\partial x^l}}} \left(\Gamma_{jn}^{k}
    {\Bigl( {\frac{\partial }{{\partial {x^k}}}} \Bigr)^d} \right)
    -{({\rm d}x^i)_d} {\nabla _{\frac{\partial }{\partial x^n}}} \left(\Gamma_{jl}^{k}
    {\Bigl( {\frac{\partial }{{\partial {x^k}}}} \Bigr)^d} \right) \notag\\
    &= {\partial_l} \Gamma_{jn}^{i} -\partial_n \Gamma_{jl}^{i}+ \Gamma_{jn}^{k} \Gamma_{kl}^{i}
    - \Gamma_{jl}^{k}\Gamma_{kn}^{i} \label{chccr:eqn_Riemannian13-component}
\end{align}
为了对比黎曼曲率与Ricci曲率,将Ricci曲率写在这里(定义见\eqref{chrg:eqn_RicciCurvatured2}):
\begin{equation}\label{chccr:eqn_Ricci-component}
    R_{jn} = R_{\hphantom{a} jkn}^k = {\partial_k} \Gamma_{jn}^{k} -\partial_n \Gamma_{jk}^{k}
    + \Gamma_{jn}^{k} \Gamma_{kl}^{l}- \Gamma_{jl}^{k}\Gamma_{kn}^{l} .
\end{equation}
这两个表达式适用于有挠或无挠联络,需注意$\Gamma$下标的顺序.

%\begin{align*}
%     R_{cab}^d =& \partial_a \Gamma_{cb}^{d} -\partial_b \Gamma_{ca}^{d}
%     + \Gamma_{cb}^{e} \Gamma_{ea}^{d} - \Gamma_{ca}^{e} \Gamma_{eb}^{d} . \\
%     R_{\mu\alpha\beta}^\nu =& \partial_\alpha \Gamma_{\mu\beta}^{\nu} -\partial_\beta \Gamma_{\mu\alpha}^{\nu}
%     + \Gamma_{\mu\beta}^{\pi} \Gamma_{\pi\alpha}^{\nu} - \Gamma_{\mu\alpha}^{\pi} \Gamma_{\pi\beta}^{\nu} . \\
%     R_{cb} =& \partial_a \Gamma_{cb}^{a} -\partial_b \Gamma_{ca}^{a}
%     + \Gamma_{cb}^{e} \Gamma_{ea}^{a} - \Gamma_{ca}^{e} \Gamma_{eb}^{a} . \\
%     R_{\mu\beta} =& \partial_\nu \Gamma_{\mu\beta}^{\nu} -\partial_\beta \Gamma_{\mu\nu}^{\nu}
%     + \Gamma_{\mu\beta}^{\pi} \Gamma_{\pi\nu}^{\nu} - \Gamma_{\mu\nu}^{\pi} \Gamma_{\pi\beta}^{\nu} .
%\end{align*}

\index[physwords]{黎曼曲率!分量}

\subsection{黎曼曲率是张量之证明}\label{chccr:sec_rit}
从曲率的定义\ref{chccr:def_tCurvatures}与局部坐标无关,
我们用推论\ref{chdm:thm_Tensor-Characterization-Lemma-1}来证明它是$\Tpq{1}{3}$型张量;
只需验证它对三个切矢量是$C^\infty(M)$线性的即可.$\forall f\in C^\infty(M)$有
\begin{align*}
    &R_{ cab}^d{(f\cdot X^a)}{Y^b}{Z^c} = {\nabla _{f\cdot X}}{\nabla _Y}{Z^d} -
      {\nabla _Y}{\nabla _{f\cdot X}}{Z^d} - {\nabla _{[f\cdot X,Y]}}{Z^d} \\
    &\xlongequal[\ref{chccr:def_connection}]{\eqref{chdm:eqn_fg-PoissonBracket}}
    f\cdot{\nabla _{X}}{\nabla _Y}{Z^d} - {\nabla _Y}\bigl(f\cdot {\nabla _{ X}}{Z^d} \bigr)
    - {\nabla _{(f[X,Y]- Y(f)\cdot X)}}{Z^d}    \\
    &= f\cdot{\nabla _{X}}{\nabla _Y}{Z^d} - \bigl(Y(f) \bigr)\cdot {\nabla _{ X}}{Z^d}
    - f\cdot {\nabla _Y} {\nabla _{ X}}{Z^d}
     -f\cdot {\nabla _{[X,Y]}}{Z^d} +\bigl(Y(f)\bigr) \cdot{\nabla _{X}}{Z^d} \\
    &=  f\cdot R_{ cab}^d{ X^a}{Y^b}{Z^c} .
\end{align*}
接着验证,
\begin{align*}
    & R_{ cab}^d {X^a}{(f\cdot Y^b)}{Z^c} = {\nabla _{X}}{\nabla _{f \cdot Y}}{Z^d} -
    {\nabla _{f\cdot Y}}{\nabla _{X}}{Z^d} - {\nabla _{[X,f\cdot Y]}}{Z^d} \\
    = & -\bigl( R_{ cba}^d {(f\cdot Y^b)} {X^a}{Z^c} \bigr)
    = -f\cdot \bigl( R_{ cba}^d { Y^b} {X^a}{Z^c} \bigr)
     = f \cdot  R_{ cab}^d {X^a}{ Y^b}{Z^c} .
\end{align*}
验证最后一个位置,
\begin{align*}
    &R_{cab}^d {X^a}{Y^b}{(f\cdot Z^c)} = {\nabla _{X}}{\nabla _{Y}}{(f\cdot Z^d)} -
    {\nabla _{Y}}{\nabla _{X}}{(f\cdot Z^d)} - {\nabla _{[X,Y]}}{(f\cdot Z^d)} \\
    =&f\cdot {\nabla _{X}}{\nabla _{Y}}{Z^d} + {X(f)} \cdot {\nabla _{Y}}{Z^d}
     +X\left(  Y(f)\right) \cdot Z^d +Y(f)\cdot {\nabla _{X}}  Z^d \\
    &-f\cdot {\nabla _{Y}}{\nabla _{X}}{Z^d} - {Y(f)} \cdot {\nabla _{X}}{Z^d}
     -Y\left(  X(f)\right) \cdot Z^d -X(f)\cdot {\nabla _{Y}}  Z^d \\
    & - [X,Y](f) \cdot Z^d - f\cdot {\nabla _{[X,Y]}}{Z^d} \\
    =& f \cdot  R_{ cab}^d {X^a}{ Y^b}{Z^c} .
\end{align*}
它对三个位置均有$C^\infty(M)$线性,它是$\Tpq{1}{3}$型张量.

%\subsection{证明黎曼曲率是张量之二}\label{chccr:sec_rit2}

鉴于这个张量的重要性,我们从局部坐标的分量语言再次
证明它与局部坐标图选取无关,即在坐标变换时按四重线性变换规律变换.

设仿射空间$(M,\nabla_a)$中有两个相互容许的坐标图$(U;x^i)$和$(V;y^\alpha)$,
且$U\cap V \neq \varnothing$;那么在$U\cap V$上
有式\eqref{chccr:eqn_Exchange-Christoffel-a}成立,
我们将利用此式来证明命题.首先将此式两边求导,得
\begin{align*}
    &\frac{\partial \Gamma^{k}_{ji}(x)}{\partial x^n} \frac{\partial y^\rho}{\partial x^k} +
    \Gamma^{k}_{ji}(x) \frac{\partial^2 y^\rho}{\partial x^k\partial x^n} =
    \frac{\partial^3 y^\rho}{\partial x^j \partial x^i \partial x^n}
    +\frac{\partial^2 y^\beta}{\partial x^j\partial x^n}
      \frac{\partial y^\alpha}{\partial x^i} \Gamma^{\rho}_{\beta\alpha}(y) \\
    &\qquad +\frac{\partial y^\beta}{\partial x^j} \frac{\partial^2 y^\alpha}{\partial x^i\partial x^n}
      \Gamma^{\rho}_{\beta\alpha}(y)
    +\frac{\partial y^\beta}{\partial x^j} \frac{\partial y^\alpha}{\partial x^i}
    \frac{\partial y^\gamma}{\partial x^n} \frac{\Gamma^{\rho}_{\beta\alpha}(y)}{\partial y^\gamma} .
\end{align*}
将上式中的指标$i$和$n$互换得到新式子,并用上式减新式得
\setlength{\mathindent}{0em}
\begin{align}
    &\left(\frac{\partial \Gamma^{k}_{ji}(x)} {\partial x^n} -\frac{\partial \Gamma^{k}_{jn}(x)}{\partial x^i}\right)
     \frac{\partial y^\rho}{\partial x^k} +
    \Gamma^{k}_{ji}(x) \frac{\partial^2 y^\rho}{\partial x^k\partial x^n}
    -\Gamma^{k}_{jn}(x) \frac{\partial^2 y^\rho}{\partial x^k\partial x^i} =  \label{chccr:eqn_tmprgg} \\
    & \frac{\partial^2 y^\beta}{\partial x^j\partial x^n}
       \frac{\partial y^\alpha}{\partial x^i} \Gamma^{\rho}_{\beta\alpha}(y)
    -  \frac{\partial^2 y^\beta}{\partial x^j\partial x^i}
       \frac{\partial y^\alpha}{\partial x^n} \Gamma^{\rho}_{\beta\alpha}(y)
     +\frac{\partial y^\beta}{\partial x^j} \frac{\partial y^\alpha}
      {\partial x^i} \frac{\partial y^\gamma}{\partial x^n}
      \left( \frac{\Gamma^{\rho}_{\beta\alpha}(y)}{\partial y^\gamma}
    - \frac{\Gamma^{\rho}_{\beta\gamma}(y)}{\partial y^\alpha} \right) . \notag
\end{align}
将式\eqref{chccr:eqn_tmprgg}中等号左端的后两项用式\eqref{chccr:eqn_Exchange-Christoffel-a}代换,经运算得
\begin{align*}
    &\Gamma^{k}_{ji}(x) \frac{\partial^2 y^\rho}{\partial x^k\partial x^n}
    -\Gamma^{k}_{jn}(x) \frac{\partial^2 y^\rho}{\partial x^k\partial x^i} \\
    =& \Gamma^{k}_{ji}(x) \left(\Gamma^{l}_{kn}(x) \frac{\partial y^\rho}{\partial x^l} -
    \frac{\partial y^\beta}{\partial x^k} \frac{\partial y^\alpha}{\partial x^n}
    \Gamma^{\rho}_{\beta\alpha}(y)\right)
    -\Gamma^{k}_{jn}(x) \left(\Gamma^{l}_{ki}(x) \frac{\partial y^\rho}{\partial x^l} -
    \frac{\partial y^\beta}{\partial x^k} \frac{\partial y^\alpha}{\partial x^i}
    \Gamma^{\rho}_{\beta\alpha}(y)\right) \\
    =& \frac{\partial y^\rho}{\partial x^l} \left(\Gamma^{k}_{ji}(x)\Gamma^{l}_{kn}(x)
     - \Gamma^{k}_{jn}(x) \Gamma^{l}_{ki}(x) \right)
     - \frac{\partial y^\beta}{\partial x^k} \frac{\partial y^\alpha}{\partial x^n}
       \Gamma^{k}_{ji}(x) \Gamma^{\rho}_{\beta\alpha}(y)
     + \frac{\partial y^\beta}{\partial x^k} \frac{\partial y^\alpha}{\partial x^i}
       \Gamma^{k}_{jn}(x)\Gamma^{\rho}_{\beta\alpha}(y) .
\end{align*}\setlength{\mathindent}{2em}
将式\eqref{chccr:eqn_tmprgg}中等号右端的前两项用式\eqref{chccr:eqn_Exchange-Christoffel-a}代换,经运算得
\begin{align*}
    &\frac{\partial^2 y^\beta}{\partial x^j\partial x^n}
    \frac{\partial y^\alpha}{\partial x^i} \Gamma^{\rho}_{\beta\alpha}(y)
    -  \frac{\partial^2 y^\beta}{\partial x^j\partial x^i}
    \frac{\partial y^\alpha}{\partial x^n} \Gamma^{\rho}_{\beta\alpha}(y)
    =\frac{\partial y^\alpha}{\partial x^i}\Gamma^{\rho}_{\beta\alpha}(y) \times \\
    &\left(\Gamma^{l}_{jn}(x) \frac{\partial y^\beta}{\partial x^l} -
      \frac{\partial y^\xi}{\partial x^j} \frac{\partial y^\zeta}{\partial x^n}
      \Gamma^{\beta}_{\xi\zeta}(y) \right)
    -\frac{\partial y^\alpha}{\partial x^n}\Gamma^{\rho}_{\beta\alpha}(y)
     \left(\Gamma^{l}_{ji}(x) \frac{\partial y^\beta}{\partial x^l} -
     \frac{\partial y^\xi}{\partial x^j} \frac{\partial y^\zeta}{\partial x^i}
     \Gamma^{\beta}_{\xi\zeta}(y) \right) \\
     =& \frac{\partial y^\alpha}{\partial x^i} \frac{\partial y^\beta}{\partial x^l}
        \Gamma^{l}_{jn}(x)\Gamma^{\rho}_{\beta\alpha}(y)
      - \frac{\partial y^\beta}{\partial x^l}\frac{\partial y^\alpha}{\partial x^n}
        \Gamma^{l}_{ji}(x) \Gamma^{\rho}_{\beta\alpha}(y) \\
      & +  \frac{\partial y^\alpha}{\partial x^i}
      \frac{\partial y^\xi}{\partial x^j}  \frac{\partial y^\zeta}{\partial x^n}
      \left(\Gamma^{\rho}_{\beta\zeta}(y) \Gamma^{\beta}_{\xi\alpha}(y)
      -\Gamma^{\rho}_{\beta\alpha}(y) \Gamma^{\beta}_{\xi\zeta}(y)\right) .
\end{align*}
将上两式带入式\eqref{chccr:eqn_tmprgg}得
\setlength{\mathindent}{0em}
\begin{align*}
    &\left(\frac{\partial \Gamma^{k}_{ji}(x)} {\partial x^n} -\frac{\partial \Gamma^{k}_{jn}(x)}{\partial x^i}\right)
    \frac{\partial y^\rho}{\partial x^k} +
\frac{\partial y^\rho}{\partial x^l} \left(\Gamma^{k}_{ji}(x)\Gamma^{l}_{kn}(x)
- \Gamma^{k}_{jn}(x) \Gamma^{l}_{ki}(x) \right) \\
    =& \frac{\partial y^\alpha}{\partial x^i}
    \frac{\partial y^\xi}{\partial x^j}  \frac{\partial y^\zeta}{\partial x^n}
    \left(\Gamma^{\rho}_{\beta\zeta}(y) \Gamma^{\beta}_{\xi\alpha}(y)
    -\Gamma^{\rho}_{\beta\alpha}(y) \Gamma^{\beta}_{\xi\zeta}(y)\right)
    +\frac{\partial y^\beta}{\partial x^j} \frac{\partial y^\alpha}
    {\partial x^i} \frac{\partial y^\gamma}{\partial x^n}
    \left( \frac{\Gamma^{\rho}_{\beta\alpha}(y)}{\partial y^\gamma}
    - \frac{\Gamma^{\rho}_{\beta\gamma}(y)}{\partial y^\alpha} \right) .
\end{align*}\setlength{\mathindent}{2em}
将此式整理后即有(参考式\eqref{chccr:eqn_Riemannian13-component})
\begin{equation}
     \frac{\partial y^\rho}{\partial x^k} \cdot R^{k}_{jin}(x) =
     \frac{\partial y^\alpha}{\partial x^i} \frac{\partial y^\xi}{\partial x^j}
     \frac{\partial y^\zeta}{\partial x^n} \cdot R^{\rho}_{\alpha\xi\zeta}(y)     .
\end{equation}
这便证明了黎曼曲率分量在坐标卡变换下具有协变性.

\subsection{黎曼曲率对称性}\label{chccr:sec_riemann-sym}
本节在{\kaishu 无挠联络}情形下证明几个黎曼曲率的对称性.
由式\eqref{chccr:eqn_Riemannian13-Vec-commutator}易得
\begin{align}
    R_{cab}^d Z^c =- (\nabla_b \nabla_a Z^d -\nabla_a \nabla_b Z^d)=-R_{cba}^d Z^c 
    \ \Rightarrow \ R_{cab}^d= -R_{cba}^d . \label{chccr:eqn_Rab=-Rba}
\end{align}

\index[physwords]{黎曼曲率!对称性}

从式\eqref{chccr:eqn_RiemannianCurvature-13}出发,将$X^a,Y^b,Z^c$进行轮换,有
\begin{align*}
    &R_{cab}^d{X^a}{Y^b}{Z^c} + R_{abc}^d{Y^b}{Z^c}{X^a}+R_{bca}^d{Z^c}{X^a}{Y^b} \\
    =& {\nabla _X}{\nabla _Y}{Z^d} - {\nabla _Y}{\nabla _X}{Z^d} - {\nabla _{[X,Y]}}{Z^d}
      +{\nabla _Y}{\nabla _Z}{X^d} - {\nabla _Z}{\nabla _Y}{X^d} - {\nabla _{[Y,Z]}}{X^d} \\
     &+{\nabla _Z}{\nabla _X}{Y^d} - {\nabla _X}{\nabla _Z}{Y^d} - {\nabla _{[Z,X]}}{Y^d} \\
    =&\left({\nabla _X}({\nabla _Y}{Z^d}-{\nabla _Z}{Y^d}) - {\nabla _{[Y,Z]}}{X^d} \right)
     +\left({\nabla _Y}({\nabla _Z}{X^d}-{\nabla _X}{Z^d}) - {\nabla _{[Z,X]}}{Y^d} \right) \\
    &+\left({\nabla _Z}({\nabla _X}{Y^d}-{\nabla _Y}{X^d}) - {\nabla _{[X,Y]}}{Z^d} \right) \\
    =& \bigl[X, [Y,Z]\bigr]^d+\bigl[Y, [Z,X]\bigr]^d +\bigl[Z, [X,Y]\bigr]^d=0.
\end{align*}
上面计算参考了式\eqref{chccr:eqn_XYcommutator}以及定理\ref{chdm:thm_poisson-Lie-bracket}中
的Jacobi恒等式.因$X^a,Y^b,Z^c$的任意性,又因上式最后恒为零,可知黎曼曲率张量有如下
关系式({\heiti\bfseries 第一Bianchi恒等式}或{\heiti 循环恒等式})
\begin{equation}\label{chccr:eqn_Bianchi-I-global}
    R_{cab}^d + R_{abc} + R_{bca}^d = 0 \quad 
    \xLongleftrightarrow[\ref{chmla:eqn_tmp592}]{\ref{chccr:eqn_Rab=-Rba}} \quad
    R_{[abc]}^d = 0.
\end{equation}

\index[physwords]{Bianchi恒等式}

我们从\eqref{chccr:eqn_Riemannian13-Vec-commutator}出发,对其再次求导,有
\begin{align*}
    \nabla_c\nabla_a \nabla_b Z^d - \nabla_c\nabla_b \nabla_a Z^d &=\nabla_c (R_{eab}^d Z^e)
    =Z^e \nabla_c R_{eab}^d + R_{eab}^d \nabla_c Z^e . \\
    \nabla_a\nabla_b \nabla_c Z^d - \nabla_a\nabla_c \nabla_b Z^d &=Z^e \nabla_a R_{ebc}^d + R_{ebc}^d \nabla_a Z^e .\\
    \nabla_b\nabla_c \nabla_a Z^d - \nabla_b\nabla_a \nabla_c Z^d &=Z^e \nabla_b R_{eca}^d + R_{eca}^d \nabla_b Z^e .
\end{align*}
后两式给出了将$a,b,c$轮换后的公式;将上面三个式子相加并重新组合,有
\setlength{\mathindent}{0em}
\begin{align*}
   &(\nabla_c\nabla_a \nabla_b Z^d - \nabla_a\nabla_c \nabla_b Z^d)+
    (\nabla_a\nabla_b \nabla_c Z^d - \nabla_b\nabla_a \nabla_c Z^d) +
    (\nabla_b\nabla_c \nabla_a Z^d - \nabla_c\nabla_b \nabla_a Z^d)  \\
   &= Z^e \nabla_c R_{eab}^d + R_{eab}^d \nabla_c Z^e  +
      Z^e \nabla_a R_{ebc}^d + R_{ebc}^d \nabla_a Z^e +
      Z^e \nabla_b R_{eca}^d + R_{eca}^d \nabla_b Z^e \\
   &{\color{red} \xLongrightarrow{\ref{chccr:eqn_Riemannian13-Tensor-commutator}}}
      +R_{eca}^d \nabla_b Z^e  -R_{bca}^e \nabla_e Z^d
      +R_{eab}^d \nabla_c Z^e  -R_{cab}^e \nabla_e Z^d
      +R_{ebc}^d \nabla_a Z^e  -R_{abc}^e \nabla_e Z^d \\
   &= Z^e \nabla_c R_{eab}^d + R_{eab}^d \nabla_c Z^e  +
      Z^e \nabla_a R_{ebc}^d + R_{ebc}^d \nabla_a Z^e +
      Z^e \nabla_b R_{eca}^d + R_{eca}^d \nabla_b Z^e .
\end{align*}\setlength{\mathindent}{2em}
因$Z^a$的任意性以及式\eqref{chccr:eqn_Bianchi-I-global},从上式可得
\begin{equation}\label{chccr:eqn_Bianchi-II-global}
    \nabla_a R_{ebc}^d + \nabla_b R_{eca}^d + \nabla_c R_{eab}^d = 0 \quad
    \xLongleftrightarrow[\ref{chmla:eqn_tmp592}]{\ref{chccr:eqn_Rab=-Rba}}\quad
    \nabla_{[a} R_{|e|bc]}^d = 0.
\end{equation}
这是{\heiti\bfseries 第二Bianchi恒等式},也称为{\heiti\bfseries Bianchi微分恒等式}.
根据最新的考古\cite{Darrigol-2015}发现,黎曼本人早于其他任何学者发现了这个微分恒等式.

%需再次提醒,本小节的对称性是在无挠联络下证明的.
%在引入度规后,黎曼曲率会有更多对称性,见定理\ref{chrg:thm_Riemann-Sym-Properties}.

\begin{exercise}
	证明式\eqref{chccr:eqn_Riemannian13-Tensor-commutator};并给出更一般性的公式.
\end{exercise}

\begin{exercise}
	证明注解\ref{chccr:rek_DD}中的公式.
\end{exercise}


\begin{exercise}
	用不同的方法再次证明Bianchi微分恒等式.
\end{exercise}


\section{曲率型式}\label{chccr:sec_form1}
给定$m$维仿射联络空间$(M,\nabla_a)$,除了由局部坐标系$(U;x^i)$确定的
自然基底标架场$\{(\frac{\partial}{\partial x^i})^a\}$外,还可以使用一般的局部
标架场$\{(e_i)^a\}$;它们是定义在流形$M$的某开集$U$上的切矢量场,并且对于
任意一点$p\in U$,$(e_i)^a|_p$是切空间$T_p M$的基底.
既然是基矢量,自然可以用自然标架场来展开
\begin{align}\label{chccr:eqn_natural-base_change}
    ({e} _i)^a = \left(\frac{\partial}{\partial x^\alpha}\right)^a C_{\hphantom{a} i}^{\alpha}(x) , \quad
    ({e} ^i)_a = \left({\rm d} x^\alpha \right)_a \left(C^{-1}\right)^{\hphantom{a} i}_{\alpha}   ;
    \ C_{\hphantom{a} i}^{\alpha}(x) \in C^{\infty}(U).
\end{align}
要求变换矩阵$C$的行列式$\det(C)>0$,用以保证$(e_i)^a$与$(\frac{\partial}{\partial x^i})^a$定向相同.

\index[physwords]{黎曼曲率!型式}


现在考虑一般的标架变换,
若$(\tilde{e}_i)^a$是另一局部标架,则存在光滑函数使得
\begin{align}\label{chccr:base_change}
    (\tilde{e} _i)^a = (e_\mu)^{a} A_{\hphantom{a} i}^{\mu} , \quad
    (\tilde{e} ^j)_a = (e^\nu)_{a} B^{\hphantom{a} j}_{\nu};
    \qquad A,B \in C^{\infty}(U) .
\end{align}
矩阵$A$和$B$的指标在前的是行指标,在后的是列指标(不论在上还是在下);
依照定理\ref{chmla:thm_inv-cov-base},变换矩阵$A$和$B$是转置逆关系($B^{-1} = A^T$).
设有张量$T^a _{\hphantom{a} b}$,则有
\begin{align}
    T^a _{\hphantom{a} b} &= T^{\mu} _{\hphantom{a} \nu}(e_\mu)^{a} (e^\nu)_{b}
    = \tilde{T}^{i} _{\hphantom{a} j} (\tilde{e} _i)^a (\tilde{e} ^j)_b
    = \tilde{T}^{i} _{\hphantom{a} j} (e_\mu)^{a} A_{\hphantom{a} i}^{\mu} (e^\nu)_{b} B^{\hphantom{a} j}_{\nu}
    \notag \\ {\color{red}\Rightarrow \ }
    T^{\mu} _{\hphantom{a} \nu} &= \tilde{T}^{i} _{\hphantom{a} j} A_{\hphantom{a} i}^{\mu} B^{\hphantom{a} j}_{\nu}
    {\ \color{red}\Leftrightarrow\ }
    T^{\mu} _{\hphantom{a} \nu} A_{\hphantom{a} k}^{\nu}  B^{\hphantom{a} n}_{\mu} = \tilde{T}^{n} _{\hphantom{a} k} .
\end{align}
这是不同标架场下张量分量的变换关系.
基矢场$(e_\mu)^{a}$和$(\tilde{e} _i)^a$可以是自然坐标基底,也可以是一般基底.

\index[physwords]{联络!型式}
\subsection{联络型式}\label{chccr:sec_connectionForm}
给定联络$\nabla _a$(有挠无挠均可),在
任意基底场$(e_i)^a$及其对偶标架场$(e^i)_a$中,
与坐标基底场系数\eqref{chccr:eqn_Christoffel2}类似,
同样可以定义联络系数(仍记为$\Gamma$)
    \begin{align}
        &(e_j)^b \nabla _b (e_i)^a  \overset{def}{=} \Gamma^{k}_{ij} (e_k)^a
        {\ \color{red} \Leftrightarrow \ }
        \nabla _b (e_i)^a = \Gamma^{k}_{ij} (e^j)_b (e_k)^a
        {\ \color{red} \Leftrightarrow \ }
        \Gamma^{k}_{ij}  =  (e^k)_a (e_j)^b \nabla _b (e_i)^a \notag \\
        &{\color{red} \Rightarrow }
        (e_j)^b \nabla _b (e^i)_a  = -\Gamma^{i}_{kj} (e^k)_a
        {\ \color{red} \Leftrightarrow \ }
        \nabla _b (e^i)_a = -\Gamma^{i}_{kj}(e^j)_b  (e^k)_a  . \label{chccr:eqn_concoef}
    \end{align}
自然坐标$\{x^i\}$基底的无挠联络系数\eqref{chccr:eqn_Christoffel2}关于下标是对称的,
即$\oversetmy{x}{\Gamma}^{k}_{ij}=\oversetmy{x}{\Gamma}^{k}_{ji}$.
然而,一般基底场$(e_i)^a$的联络系数$\Gamma^{k}_{ij}$通常不具有这种对称性;
参见式\eqref{chccr:eqn_GAMMA-trans}.

%由式\eqref{chccr:eqn_natural-base_change}出发,
%通过类似式\eqref{chccr:eqn_Exchange-Christoffel}的推导,
%可得(下式中$\Gamma^{k}_{ij}$一般不等于$\Gamma^{k}_{ji}$)
%\begin{equation}\label{chccr:eqn_Gamma-Cw}
%   \Gamma^{k}_{ji} C^\sigma_{\hphantom{a} k} = 
%     C^\beta_{\hphantom{a} i} \frac{\partial C^\sigma_{\hphantom{a} j}}{\partial x^\beta }
%   + C^\beta_{\hphantom{a} i} C^\alpha_{\hphantom{a} j} \oversetmy{x}{\Gamma}^{\sigma}_{\alpha\beta}  .
%\end{equation}

定义{\heiti 联络型式场}:
\begin{equation}\label{chccr:def_1form}
    (\omega_{\hphantom{a} j}^{k})_a \overset{def}{=} \Gamma_{ji}^k (e^i)_a
    \equiv (e^i)_a (e^k)_c (e_i)^b \nabla _b (e_j)^c
    \equiv (e^k)_c \nabla _a (e_j)^c .
\end{equation}
上指标降下位置是人为约定的.由上式可导出联络型式的另一种常用表示
\begin{equation}\label{chccr:eqn_D-omega-form}
    \nabla _a (e^i)_b= - (\omega_{\hphantom{a} j}^{i})_a (e^j)_b ,\qquad
    \nabla _a (e_i)^b= + (\omega_{\hphantom{a} i}^{j})_a (e_j)^b .
\end{equation}


\subsection{标架变换}
现在考虑在标架变换\eqref{chccr:base_change}下,联络型式如何变化.
设联络$\nabla_{a}$在标架场$(\tilde{e}^i)_a$下的联络系数及联络型式分别为
\begin{equation}\label{chccr:eqn_connection-form-tilde}
    (\tilde{e}_i)^b \nabla _b (\tilde{e}_j)^a  =
    \widetilde{\Gamma}^{k}_{ji} (\tilde{e}_k)^a,   \qquad
    (\tilde{\omega} _{\hphantom{a} j}^{k})_a  \equiv
    \widetilde{\Gamma}^{k}_{j i} (\tilde{e}^i)_a .
\end{equation}
依照定义直接计算
\begin{align*}
    &(\tilde{e}_i)^b \nabla _b (\tilde{e}_j)^a =\widetilde{\Gamma}^{k}_{ji} (\tilde{e}_k)^a
    \quad \xRightarrow{\ref{chccr:base_change}} \quad (\tilde{e}_i)^b \nabla _b
    \bigl( A_{\hphantom{a} j}^{\nu} (e_\nu)^{a} \bigr) =
    \widetilde{\Gamma}^{k}_{ji} \bigl( A_{\hphantom{a} k}^{\rho} (e_\rho)^{a} \bigr) \\
    {\color{red}\Rightarrow}  &
    (e_\nu)^{a} (\tilde{e}_i)^b \nabla _b  A_{\hphantom{a} j}^{\nu}
    + A_{\hphantom{a} j}^{\nu} A_{\hphantom{a} i}^{\mu} (e_\mu)^{b}  \nabla _b (e_\nu)^{a}
    =  \widetilde{\Gamma}^{k}_{ji}  A_{\hphantom{a} k}^{\rho} (e_\rho)^{a}     \\
     {\color{red}\Rightarrow}  &
    (\tilde{e}_i)^b \nabla _b   A_{\hphantom{a} j}^{\sigma}
    + A_{\hphantom{a} j}^{\nu} A_{\hphantom{a} i}^{\mu} {\Gamma}^{\sigma}_{\nu\mu}
    =  \widetilde{\Gamma}^{k}_{ji}  A_{\hphantom{a} k}^{\sigma} ,
    \qquad \text{已缩并掉基矢}(e_\nu)^{a} \\   {\color{red}\Rightarrow}  &
    (\tilde{e}^i)_a (\tilde{e}_i)^b \nabla _b   A_{\hphantom{a} j}^{\sigma}
    + A_{\hphantom{a} j}^{\nu} A_{\hphantom{a} i}^{\mu} {\Gamma}^{\sigma}_{\nu\mu} (\tilde{e}^i)_a
    =  \widetilde{\Gamma}^{k}_{ji}  A_{\hphantom{a} k}^{\sigma}  (\tilde{e}^i)_a,
    \qquad \text{准备缩并基矢}(\tilde{e}_i)^b  \\   {\color{red}\Rightarrow}  &
    \nabla _a   A_{\hphantom{a} j}^{\sigma}  + A_{\hphantom{a} j}^{\nu} A_{\hphantom{a} i}^{\mu}
    {\Gamma}^{\sigma}_{\nu\mu} (e^\rho)_{a} B^{\hphantom{a} i}_{\rho}
    =  \widetilde{\Gamma}^{k}_{ji}  A_{\hphantom{a} k}^{\sigma}  (\tilde{e}^i)_a  ,
    \quad {\text{已把基矢$(\tilde{e}^i)_a$表示为$(e^\rho)_{a}B^{\hphantom{a} i}_{\rho}$}}
\end{align*}
把上面最后一式的联络系数变成型式场,并应用矩阵$A,B$互逆,可以得到
\begin{equation}\label{chccr:eqn_1form-transformation}
    \nabla _a A_{\hphantom{a} j}^{\sigma} + (\omega ^{\sigma}_{\hphantom{a}\nu})_a  A_{\hphantom{a} j}^{\nu}
    =  A_{\hphantom{a} k}^{\sigma} (\tilde{\omega} _{\hphantom{a} j} ^{k})_a
     {\quad \color{red}\Leftrightarrow \quad }
    \tilde{\omega} = A^{-1}{\rm d} A + A^{-1} \omega A.
\end{equation}
最后把公式表示成矩阵型式,显得较为简洁.


将上上式中的$(e^\alpha)_{a}$取为$({\rm d}x^\alpha)_a$,
可将式\eqref{chccr:eqn_1form-transformation}表示成联络系数形式:
\begin{equation}\label{chccr:eqn_GAMMA-trans}
    A_{\hphantom{a} i}^{\alpha}\frac{\partial A_{\hphantom{a} j}^{\sigma} }{\partial x^\alpha}  
    + A_{\hphantom{a} i}^{\alpha} A_{\hphantom{a} j}^{\nu}\oversetmy{x}{\Gamma}^{\sigma}_{\nu \alpha} 
    =  A_{\hphantom{a} k}^{\sigma} \widetilde{\Gamma}^{k}_{j i}  .
\end{equation}
上式中$\oversetmy{x}{\Gamma}$表示自然坐标$\{x^i\}$的联络系数,
$\widetilde{\Gamma}$是一般标架$\{(\tilde{e}_i)^a\}$的联络系数.
由上式可以导出式\eqref{chccr:eqn_Exchange-Christoffel},请读者完成这个计算.

\index[physwords]{联络!型式!协变导数}
\subsection{协变导数}\label{chccr:sec_CovD-Form}
对于任意标量场$f$,有
\begin{align}
    {e} _i (f)&= A_{\hphantom{a} i}^{j}(x) \frac{\partial f}{\partial x^j}, \\
    {\rm d}_a (f) &=  \frac{\partial f}{\partial x^j} ({\rm d} x^j)_a
    = \frac{\partial f}{\partial x^i}  A_{\hphantom{a} \nu}^{i}\cdot
    B^{\hphantom{a} \nu}_{j} ({\rm d} x^j)_a
    = [{e} _\nu (f)] (e^\nu)_a .
\end{align}
依照约定,切矢场作用标量场上时,去掉抽象指标.
易得一般标架的协变导数
\begin{align}
    \nabla_a u^c &= \left[e_j (u^i) + u^l \Gamma^{i}_{lj} \right] (e^j)_a (e_i)^c
    =  \left[{\rm d}_a (u^i) + u^l (\omega_{\hphantom{a} l}^{i})_a \right] (e_i)^c .   \\
    \nabla_a \mu_b &= \left[e_j (\mu_k)-\mu_l \Gamma^{l}_{kj} \right] (e^j)_a (e^k)_b
    =  \left[{\rm d}_a (\mu_k) -\mu_l (\omega_{\hphantom{a} k}^{l})_a \right] (e^k)_b . \\
    \nabla_a T_b^c & = \left[e_j (T_k^i) -T_l^i \Gamma^{l}_{kj}+ T^l_k \Gamma^{i}_{lj} \right]
          (e^j)_a (e^k)_b (e_i)^c.    \label{chccr:eqn_eTg}  \\
      &= \left[{\rm d}_a (T_k^i) -T_l^i (\omega_{\hphantom{a} k}^{l})_a
         +T^l_k (\omega_{\hphantom{a} l}^{i})_a \right] (e^k)_b (e_i)^c  .  \label{chccr:eqn_eTw}
\end{align}


\index[physwords]{Cartan结构方程}
\subsection{Cartan结构方程}\label{chccr:sec_Cartan-Structure}
由式\eqref{chccr:eqn_Ttorsion}可知
\begin{align}
    T_{bc}^a (e^i)_a (e_j)^b (e_k)^c &= (e^i)_a{\nabla_{e_j}}{(e_k)^a} -
    (e^i)_a{\nabla _{e_k}}{(e_j)^a} - (e^i)_a{\left[ {e_j,e_k} \right]^a} \notag \\
    & = \Gamma_{kj}^l (e_l)^a(e^i)_a - \Gamma_{jk}^l (e_l)^a(e^i)_a-
    (e^i)_a{\left[ {e_j,e_k} \right]^a}.  \label{chccr:eqn_tmptor}
\end{align}
依照式\eqref{chdf:eqn_d1form-value}可知上式最后一项可以表示为
\setlength{\mathindent}{0em}
\begin{align*}
    -(e^i)_a{\left[ {e_j,e_k} \right]^a} =\bigl( {\rm d}_a (e^i)_b \bigr) (e_j)^a (e_k)^b
    -e_j \bigl( (e^i)_b (e_k)^b \bigr) + e_k \bigl( (e^i) _b (e_j)^b \bigr) 
    =\bigl( {\rm d}_a (e^i)_b \bigr) (e_j)^a (e_k)^b ,
\end{align*}\setlength{\mathindent}{2em}
带回\eqref{chccr:eqn_tmptor},得
\begin{align*}
    T_{bc}^a (e^i)_a (e_j)^b (e_k)^c & = \Gamma_{kj}^i -\Gamma_{jk}^i
    + \bigl( {\rm d}_a (e^i)_b \bigr) (e_j)^a (e_k)^b  {\ \color{red} \Rightarrow }   \\
%    T_{ab}^c (e^i)_c  & = \Gamma_{kj}^i (e^j)_a(e^k)_b -\Gamma_{jk}^i(e^j)_a(e^k)_b
%    +  {\rm d}_a (e^i)_b    {\ \color{red} \Rightarrow }   \\
    T_{ab}^c (e^i)_c  & = (\omega _{\hphantom{a} k}^i)_a(e^k)_b -(e^j)_a(\omega _{\hphantom{a} j}^i)_b
    +  {\rm d}_a (e^i)_b .
\end{align*}
由此,可得
\begin{equation}\label{chccr:eqn_Cartan-Structure-I}
    {\rm d}_a (e^i)_b - (e^j)_a \wedge (\omega _{\hphantom{a} j}^i)_b =T_{ab}^c (e^i)_c
    =\frac{1}{2}T_{jk}^i (e^j)_a \wedge(e^k)_b .
\end{equation}
这是{\heiti \bfseries Cartan第一结构方程}.

由黎曼曲率的定义\eqref{chccr:eqn_RiemannianCurvature-13},可得
\begin{align*}
    &R_{cab}^d{\left( {{e_i}} \right)^a}{\left( {{e_j}} \right)^b}{\left( {{e_k}} \right)^c}
    = {\nabla _{{e_i}}}{\nabla _{{e_j}}}{\left( {{e_k}} \right)^d}
    - {\nabla _{{e_j}}}{\nabla _{{e_i}}}{\left( {{e_k}} \right)^d}
    - {\nabla _{[{e_i},{e_j}]}}{\left( {{e_k}} \right)^d} \\
    =&  {\nabla _{{e_i}}}\left( {\Gamma _{kj}^p{{\left( {{e_p}} \right)}^d}} \right)
    - {\nabla _{{e_j}}}\left( {\Gamma _{ki}^p{{\left( {{e_p}} \right)}^d}} \right)
    - {[{e_i},{e_j}]^c}\Gamma _{kl}^p{\left( {{e^l}} \right)_c}{\left( {{e_p}} \right)^d} \\
    =& {\left( {{e_p}} \right)^d}{e_i}\left( {\Gamma _{kj}^p} \right)
    + \Gamma _{kj}^p\Gamma _{pi}^n{\left( {{e_n}} \right)^d}
    - {\left( {{e_p}} \right)^d}{e_j}\left( {\Gamma _{ki}^p} \right)
    - \Gamma _{ki}^p\Gamma _{pj}^n{\left( {{e_n}} \right)^d}
    - {\left( {{e^l}} \right)_c}{[{e_i},{e_j}]^c}\Gamma _{kl}^p{\left( {{e_p}} \right)^d} .
\end{align*}
对上式两边缩并$(e^n)_d$,得
\begin{equation}\label{chccr:eqn_tmp1}
    R_{kij}^n = {e_i} ( {\Gamma _{kj}^n} ) - {e_j}( {\Gamma _{ki}^n} )
    + \Gamma _{kj}^l\Gamma _{li}^n - \Gamma _{ki}^l\Gamma _{lj}^n
    - {{( {{e^l}} )}_c}{{[{e_i},{e_j}]}^c}\Gamma _{kl}^n .
\end{equation}
注意${e_i} ( {\Gamma _{kj}^n} )$是切矢场${e_i}$作用在标量函数场${\Gamma _{kj}^n}$上.
依照式\eqref{chdf:eqn_d1form-value},上式最后一项可表示为
\begin{align}
    - \left[ {\Gamma _{kl}^n{{\left( {{e^l}} \right)}_c}} \right]{[{e_i},{e_j}]^c}
    = {{\text{d}}_a}{\left( {\omega _{\hphantom{a} k}^n} \right)_b}{\left( {{e_i}} \right)^a}
    {\left( {{e_j}} \right)^b} - {e_i}\left( {\Gamma _{kj}^n} \right)
    + {e_j}\left( {\Gamma _{ki}^n} \right) .
\end{align}
将上式带入\eqref{chccr:eqn_tmp1},得
\begin{align}
    R_{\hphantom{a} kij}^{n} &= + \Gamma _{kj}^l\Gamma _{li}^n - \Gamma _{ki}^l\Gamma _{lj}^n +
    \bigl[{\rm d}_a (\omega_{\hphantom{a} k}^n)_b \bigr] (e_i)^a (e_j)^b
    {\quad \color{red} \Leftrightarrow } \label{chccr:eqn_Cartan-Structure-II-comp} \\
    \frac{1}{2}R_{\hphantom{a} kij}^n (e^i)_a \wedge (e^j)_b
    &=  {\rm d}_a ( \omega _{\hphantom{a} k} ^n  )_b  + ( \omega _{\hphantom{a} l}^n )_a 
    \wedge ( \omega _{\hphantom{a} k}^l )_b .  \label{chccr:eqn_Cartan-Structure-II}
\end{align}
这是{\heiti \bfseries Cartan第二结构方程}.两个结构方程适用于有挠或无挠联络.
\begin{definition}\label{chccr:def_TCF}
    定义联络$\nabla_a$在局部标架场$(e_i)^a$下的{\heiti 挠率型式}与{\heiti 曲率型式}如下:
    \begin{align}
        (\Theta ^{i})_{ab} \overset{def}{=} & {\rm d}_a (e^i)_b -
        (e^j)_a \wedge (\omega _{\hphantom{a} j}^i)_b , \label{chccr:eqn_Tform}\\
        (\Omega_{\hphantom{a} j} ^{i})_{ab} \overset{def}{=} & {\rm d}_a
        ( \omega _{\hphantom{a} j} ^i  )_b  + ( \omega _{\hphantom{a} l}^i )_a
        \wedge ( \omega _{\hphantom{a} j}^l )_b . \label{chccr:eqn_Cform}
    \end{align}
\end{definition}
它们都是2型式场.由结构方程可得上两式在一般标架场上的具体表示:
\begin{align}
    (\Theta ^{i})_{ab} = &\frac{1}{2}T_{jk}^i (e^j)_a \wedge(e^k)_b
    = T_{jk}^i (e^j)_a (e^k)_b, \label{chccr:eqn_Tform-bak}\\
    (\Omega_{\hphantom{a} j} ^{i})_{ab} = &\frac{1}{2}R_{\hphantom{a} jkl}^i
    (e^k)_a \wedge (e^l)_b = R_{\hphantom{a} jkl}^i (e^k)_a (e^l)_b . \label{chccr:eqn_Cform-bak}
\end{align}
\uwave{则两个结构方程可以表示成如下常用形式}:
\begin{subequations}\label{chccr:eqn_CST}
\begin{align}
    {\rm d}_a (e^i)_b =&(e^j)_a \wedge (\omega _{\hphantom{a} j}^i)_b +(\Theta ^{i})_{ab}, \label{chccr:eqn_CST-I} \\
    {\rm d}_a ( \omega _{\hphantom{a} j} ^i  )_b  =&  ( \omega _{\hphantom{a} j}^l )_a
    \wedge ( \omega _{\hphantom{a} l}^i )_b + (\Omega_{\hphantom{a} j} ^{i})_{ab} . \label{chccr:eqn_CSC-II}
\end{align}
\end{subequations}


\subsection{Bianchi恒等式}
\index[physwords]{Bianchi恒等式}

下两式分别称为联络的\uwave{第一、第二Bianchi恒等式}.
\begin{theorem}
    挠率形式和曲率形式满足下式(有挠、无挠联络均适用):
    \begin{subequations}\label{chccr:eqn_Bianchi-C}
    \begin{align}
        {\rm d}_c (\Theta ^{i})_{ab}   &=(e^j)_c \wedge (\Omega_{\hphantom{a} j} ^{i})_{ab}
        -(\Theta ^{j})_{ca} \wedge ( \omega _{\hphantom{a} j}^i )_b , \label{chccr:eqn_Bianchi-CI} \\
        {\rm d}_c (\Omega_{\hphantom{a} j} ^{i})_{ab}  &=  ( \omega _{\hphantom{a} j}^k )_c
        \wedge (\Omega_{\hphantom{a} k} ^{i})_{ab} - (\Omega_{\hphantom{a} j} ^{k})_{ca}\wedge
        ( \omega _{\hphantom{a} k}^i )_b .  \label{chccr:eqn_Bianchi-CII}
    \end{align}
    \end{subequations}
\end{theorem}
\begin{proof}
    对式\eqref{chccr:eqn_CST-I}再取外微分,有(见式\eqref{chdf:eqn_dwm})
    \setlength{\mathindent}{0em}
    \begin{align*}
        0 & = {\rm d}_c {\rm d}_a (e^i)_b =
        \bigl({\rm d}_c (e^j)_a\bigr) \wedge (\omega _{\hphantom{a} j}^i)_b
        -  (e^j)_c  \wedge \bigl({\rm d}_a(\omega _{\hphantom{a} j}^i)_b\bigr)
        +{\rm d}_c (\Theta ^{i})_{ab} \quad  {\color{red} \Rightarrow} \\
        {\rm d}_c (\Theta ^{i})_{ab} &=  -
        \bigl [(e^l)_c \wedge (\omega _{\hphantom{a} l}^j)_a +(\Theta ^{j})_{ca} \bigr]
        \wedge (\omega _{\hphantom{a} j}^i)_b +  (e^j)_c  \wedge
        \bigl[ ( \omega _{\hphantom{a} j}^l )_a \wedge
        ( \omega _{\hphantom{a} l}^i )_b + (\Omega_{\hphantom{a} j} ^{i})_{ab} \bigr] \\
        &= (e^j)_c \wedge(\Omega_{\hphantom{a} j} ^{i})_{ab}
        -(\Theta ^{j})_{ca} \wedge (\omega _{\hphantom{a} j}^i)_b .
    \end{align*} %\setlength{\mathindent}{2em}
    这便是第一式.再对\eqref{chccr:eqn_CSC-II}取外微分,有
    \begin{align*}
        0 & = {\rm d}_c{\rm d}_a ( \omega _{\hphantom{a} j} ^i  )_b =
        \bigl[{\rm d}_c ( \omega _{\hphantom{a} j}^l )_a \bigr]
        \wedge ( \omega _{\hphantom{a} l}^i )_b
        -( \omega _{\hphantom{a} j}^l )_c   \wedge
        \bigl[ {\rm d}_a ( \omega _{\hphantom{a} l}^i )_b \bigr]
        + {\rm d}_c (\Omega_{\hphantom{a} j} ^{i})_{ab} \quad  {\color{red} \Rightarrow} \\
        {\rm d}_c (\Omega_{\hphantom{a} j} ^{i})_{ab} &=
        -\bigl[( \omega _{\hphantom{a} j}^p )_c
        \wedge ( \omega _{\hphantom{a} p}^l )_a + (\Omega_{\hphantom{a} j} ^{l})_{ca} \bigr]
        \wedge ( \omega _{\hphantom{a} l}^i )_b +
        ( \omega _{\hphantom{a} j}^l )_c   \wedge
        \bigl[ ( \omega _{\hphantom{a} l}^p )_a
        \wedge ( \omega _{\hphantom{a} p}^i )_b + (\Omega_{\hphantom{a} l} ^{i})_{ab} \bigr] \\
        & = ( \omega _{\hphantom{a} j}^l )_c   \wedge (\Omega_{\hphantom{a} l} ^{i})_{ab}
        - (\Omega_{\hphantom{a} j} ^{l})_{ca}\wedge ( \omega _{\hphantom{a} l}^i )_b .
    \end{align*}\setlength{\mathindent}{2em}
    这便是第二式.
\end{proof}

\begin{theorem}\label{chccr:thm_bianchi-12}
    联络无挠时,式\eqref{chccr:eqn_Bianchi-C}分别是通常的第一、第二Bianchi恒等式.
\end{theorem}
\begin{proof}
    联络无挠等价于$(\Theta ^{i})_{ab} \equiv 0$,故式\eqref{chccr:eqn_Bianchi-CI}变为
    \begin{equation}\label{chccr:eqn_Bianchi-I}
        0 = (e^j)_c \wedge (\Omega_{\hphantom{a} j} ^{i})_{ab}
        \xlongequal{\ref{chccr:eqn_Cform-bak}} \frac{1}{2}R_{jkl}^i
        (e^j)_c \wedge (e^k)_a \wedge (e^l)_b
        \xRightarrow{\ref{chmla:eqn_xiv2}}  %{\ \color{red} \Rightarrow \ }
        R_{[jkl]}^i =0 .
    \end{equation}
    这是第一Bianchi恒等式,与式\eqref{chccr:eqn_Bianchi-I-global}相同.


    应用无挠联络的式\eqref{chccr:eqn_CST-I},直接计算式\eqref{chccr:eqn_Bianchi-CII},有
    \begin{align*}
        0=&  {\rm d}_c (\Omega_{\hphantom{a} j} ^{i})_{ab} -  ( \omega _{\hphantom{a} j}^k )_c
        \wedge (\Omega_{\hphantom{a} k} ^{i})_{ab} + (\Omega_{\hphantom{a} j} ^{k})_{ca}\wedge
        ( \omega _{\hphantom{a} k}^i )_b  \\
        =& \frac{1}{2}{\rm d}_c (R_{jpn}^i) \wedge (e^p)_a \wedge (e^n)_b +
        \frac{1}{2}R_{jkn}^i \bigl[{\rm d}_c(e^k)_a\bigr] \wedge (e^n)_b \\
        & -\frac{1}{2}R_{jsk}^i (e^s)_c\wedge {\rm d}_a (e^k)_b
        +\frac{1}{2}(-R_{kpn}^i \Gamma _{js}^k +  R_{jsp}^k \Gamma _{kn}^i )
        (e^s)_c \wedge (e^p)_a \wedge (e^n)_b \\
        =& \frac{1}{2} (e^s)_c\wedge (e^p)_a \wedge (e^n)_b  \big[ e_s (R_{jpn}^i)
        +R_{jkn}^i \Gamma_{sp}^k  - R_{jsk}^i \Gamma_{pn}^k
        -R_{kpn}^i \Gamma _{js}^k +  R_{jsp}^k \Gamma _{kn}^i \big] \\
        \overset{*}{=}&
        \frac{1}{2} (e^s)_c\wedge (e^p)_a \wedge (e^n)_b  \big[ e_s (R_{jpn}^i)
        +R_{jkn}^i {\color{red}(\Gamma_{sp}^k +\Gamma_{ps}^k) } \\
        &-R_{jkn}^i\Gamma_{ps}^k - R_{jsk}^i \Gamma_{pn}^k
        -R_{kpn}^i \Gamma_{js}^k +  R_{jsp}^k \Gamma _{kn}^i \big]     \\
        =& \frac{1}{2} (e^s)_c\wedge (e^p)_a \wedge (e^n)_b  \big[ e_s (R_{jpn}^i)
        -R_{jkn}^i \Gamma_{ps}^k  - R_{jsk}^i \Gamma_{pn}^k
        -R_{kpn}^i \Gamma _{js}^k +  R_{jsp}^k \Gamma _{kn}^i \big] .
    \end{align*}
    在“$\overset{*}{=}$”步中作了一个技巧性处理:
    先加减$R_{jkn}^i\Gamma_{ps}^k$项,然后$R_{jkn}^i\Gamma_{ps}^k$和
    $R_{jkn}^i\Gamma_{sp}^k$组成红色圆括号内的项;
    红色括号内的两项关于下标$s$和$p$对称,而方括号外面是外微分式(关于$s$和$p$反对称),
    故此项消失了.    由式\eqref{chmla:eqn_xiv2}继续计算有
    \begin{align*}
        0 =&  e_{[s} (R_{|j|pn]}^i)
        -R_{ jk[n}^i \Gamma_{ps]}^k - R_{ j[s|k|}^i \Gamma_{pn]}^k
        -R_{ k[pn}^i \Gamma_{|j|s]}^k + R_{ j[sp}^k \Gamma_{|k|n]}^i
        \quad {\xRightarrow[\ref{chmla:eqn_tmp592}]{\ref{chccr:eqn_Rab=-Rba}}}\\
        0=&  e_{s} (R_{jpn}^i) +e_{p} (R_{jns}^i)  +e_{n} (R_{jsp}^i)  \\
        & -R_{jkn}^i \Gamma_{ps}^k -R_{jkp}^i \Gamma_{sn}^k -R_{jks}^i \Gamma_{np}^k
        -R_{jsk}^i \Gamma_{pn}^k -R_{jpk}^i \Gamma_{ns}^k -R_{jnk}^i \Gamma_{sp}^k  \\
        & -R_{kpn}^i \Gamma_{js}^k -R_{kns}^i \Gamma_{jp}^k -R_{ksp}^i \Gamma_{jn}^k
        +R_{jsp}^k \Gamma_{kn}^i +R_{jpn}^k \Gamma_{ks}^i +R_{jns}^k \Gamma_{kp}^i  \\
        =& +e_{s} (R_{jpn}^i)+R_{jpn}^k \Gamma_{ks}^i -R_{kpn}^i \Gamma_{js}^k
        -R_{jkn}^i \Gamma_{ps}^k    -R_{jpk}^i \Gamma_{ns}^k \\
        &+e_{p} (R_{jns}^i)+R_{jns}^k \Gamma_{kp}^i  -R_{kns}^i \Gamma_{jp}^k
        -R_{jks}^i \Gamma_{np}^k       -R_{jnk}^i \Gamma_{sp}^k   \\
        &+e_{n} (R_{jsp}^i) +R_{jsp}^k \Gamma_{kn}^i -R_{ksp}^i \Gamma_{jn}^k
        -R_{jkp}^i \Gamma_{sn}^k       -R_{jsk}^i \Gamma_{pn}^k   .
    \end{align*}
    由此式,参考式\eqref{chccr:eqn_eTg}可得第二Bianchi恒等式
    \begin{equation}\label{chccr:eqn_Bianchi-II}
       R^i_{jpn;s} + R^i_{jns;p} + R^i_{jsp;n} =0 
       \ \Leftrightarrow \ 
        \nabla_a R^d_{ebc} + \nabla_b R^d_{eca}+ \nabla_c R^d_{eab} =0 .
    \end{equation}
    与式\eqref{chccr:eqn_Bianchi-II-global}相同.
\end{proof}

本节的型式理论摆脱了局部自然坐标,
是切标架丛(见\pageref{chfb:sec_tangent-frame-bundles}页)上的理论.

\begin{remark}\label{chccr:rek_abstract-i-n}
    抽象指标记号的优势在于协变导数计算以及缩并计算的清晰性.
    它的劣势在曲率型式理论中体现明显:角标、符号过于繁杂;使公式显得十分笨重.
\end{remark}




\section*{小结}
本章主要参考了\parencite{chen-li-2023-2ed-v1}相应章节.

%\parencite{cc2001-zh}




%%%%%%%%%%%%%%%%%%%%%%%%%%%%%%%%%%%%%%%%%%%%%%%%%%%%%%%%%%%%%%%%%%%%%%%%%%%%%%%%
\printbibliography[heading=subbibliography,title=第\ref{chccr}章参考文献]
\endinput