% !Mode:: "TeX:UTF-8"
% 此文件从2021.10.10开始写作


%本书包括十五章以及三个附录.前两章是数学准备知识;第三章描述了多重线性代数;
%第四章到第七章分别讲述了微分流形、型式场、联络以及矢量丛;
%之后详细描述了黎曼几何、测地线、子流形、超曲面、零模、复几何知识;
%继而较为详细地介绍了李群、李代数和对称空间内容;最后初步阐述了旋量.
%
%本书适用于物理系高年级本科生、研究生,尤其是相对论从业者.


\section*{自序}


本书是写给理论物理工作者的,主要是为相对论学者提供一份微分几何的{\kaishu 初级}读物.
与类似的纯数学书籍相比,在深度、广度以及严谨性方面自然是逊色的,这是物理工作者撰写数学书籍的必然结果.


%目录已经非常完备地反映了书籍体系、内容;仿照惯例,简介如下.
%
%古典微分几何是研究嵌入到三维欧氏空间的二维正则曲面的学科,一般说来这不属于现代微分几何,
%我们只在第一章予以简单的综述.
%之后,扼要介绍了微分几何所需的代数概念和点集拓扑知识.
%接着,较为详尽地给出了多重线性代数内容,并由此给出了张量的确切定义.
%以上三章是微分几何必要的准备知识.
%
%第四章阐述了微分流形体系;首先给出微分流形定义,并由此引入切空间、诱导映射以及子流形概念,
%其中切空间是整个微分几何随处会用到的概念;
%再者,介绍了切丛,这为后面引入仿射联络奠定了基础;
%然后,细致地描述了切矢量场和张量场的知识,这是微分流形中极为重要的概念;
%本章最后给出了单参数微分同胚群和李导数.
%第五章,首先引入外微分知识;然后简单叙述了Frobenius定理;最后给出了流形上积分的知识.
%第六章,先是定义了切丛上的仿射联络,并详尽讨论相关知识;接着定义了黎曼曲率,此曲率是整个微分几何的核心;
%然后讨论切标架丛上的黎曼曲率,这是Cartan的杰作.
%第七章,简要介绍了矢量丛(向量丛)的相关知识,它是切丛的推广.
%
%
%第八章书写了黎曼几何的相关知识.首先给光滑流形配以黎曼度量,这样流形就成了黎曼流形;
%再者更为详尽地讨论了各种曲率(包括相对论中用到的爱氏张量);
%接着描述了等距映射和Killing切矢量场知识,这是广义相对论中所需的最基本概念;
%然后给出体积元的定义,以及Hodge算子等概念;
%进而细致讲述了度量场下的黎曼曲率型式理论;
%最后初步讨论了截面曲率和共形映射.
%第九章阐述了黎曼流形中的测地线以及Jacobi场等相关知识,比如指数映射、法坐标系等.
%第十章先是介绍了子流形的一般内容,然后细致地讲解了超曲面知识,这是学习广义相对论所必需了解的.
%第十一章阐述了与零模(Null,类光)相关的几何,这是正定度量几何中没有的内容;
%先讲解了类光超曲面,以补上第十章所缺内容;
%然后详尽介绍了Newman--Penrose框架;
%最后介绍了Weyl张量的Petrov分类和Goldberg--Sachs定理.
%前十一章内容是微分几何中基础知识,是学习广义相对论必需要掌握内容.
%
%第十二章概述了复几何知识,将复几何放在此处是为李群作准备.
%第十三章是内容庞杂的一章,介绍了物理学中所需要的李群、李代数初步知识.
%第十四章讲解了对称空间内容,是广义相对论中的高阶内容.
%第十五章具体讲述了物理学中最为常用的两个李群$SU(2)$、$SL(2,\mathbb{C})$;
%进而阐述了Lorentz旋量的初步知识.


{\kaishu
    整本书中几乎没有我的原创或首创内容,但几乎每个公式我都计算过.
    唯一的原创、首创\cite{liu2018}(见\S\ref{chfd:sec_liu2018})是:  
    用解存在性需求阐述了源守恒定律为何独立于场方程;
    这纠正了过去一个多世纪以来理论物理最底层存在的错误.
    具体来说,对于电磁场,电荷守恒定律独立于麦克斯韦方程组;
    麦氏散度方程($\nabla\cdot \boldsymbol{D}=\rho,\ \nabla \cdot \boldsymbol{B}=0$)独立于麦氏旋度方程,
    它们不能看成旋度方程的初始条件.
    对于引力场,源运动方程($\nabla^a T_{ab}=0$)独立于爱因斯坦引力场方程($G_{ab}=8\pi T_{ab}$).
}


严格说来,整本书是我的读书笔记.
笔者主要研读了\textcite{cc2001-zh}名著,此教材以“浓缩”著称,不易阅读.
陈维桓教授的几本教材\cite{chenwh2001,chen-li-2023-2ed-v1}相对容易多了,  %,chen-li-2004v2
本书较多参考了这几本书.
除了上述书籍,笔者还参考了\textcite{oneill1983}的专著;
此书是深入学习广义相对论的必备数学书籍.


%,但愿概念性错误较少(不奢求没有)至于数学或物理概念理解不够深入、不到位的情形更是无法避免.
%;但我需理解你的批评、指正后才能回复你

由于笔者学识所限,肯定存在这样或那样的错误;欢迎建设性的批评、指正.

%\vspace{1em}
%\begin{flushright}
%    刘长礼(LIUCL78@qq.com)  %, ORCID: 0000-0002-3592-2449 )
%    
%%    {\today}于北京
%\end{flushright}
%\vspace{1em}



\printbibliography[heading=subbibliography,title=参考文献]


\endinput


%
%\newpage
%
%\section*{指标惯例}
%本书主要采用由Penrose创立的抽象指标记号;同时也采用数学家熟悉的记号;但在同一章内只用一种记号.
%从{\kaishu 第二部分}(相对论部分)开始,角标是拉丁字母时,如$i,j,\cdots$,表示$i=1,2,3$;
%角标是希腊字母时,如$\mu,\nu,\cdots$,表示$\mu=0,1,2,3$.
%{\kaishu 第一部分}是纯粹的微分几何内容,度规可正定也可不定,也不会限制在四维空间,
%所以没有理由作上述要求;在第一部分(微分几何部分),角标一般是$1\leqslant i, \mu \leqslant m$.
%
%
%\section*{部分数学符号表}
%\begin{tabular}{ll}
%    $\mathbb{R},\, \mathbb{C}$ & 全体实数,实数域;全体复数,复数域 \\
%    $\bar{z},\,z^*$ & 数$z$的复共轭 \\
%    $\overset{def}{=}$ & 定义,有时候“$\equiv$”也有定义的含义 \\
%    $f:A\to B$、$A\xrightarrow{f}B$ & 定义从$A$到$B$的映射$f$ \\
%    $\cong$ & 同构 \\
%    $\otimes$ & 张量积 \\
%    $\circ$ & 复合映射.例如$f\circ g$代表先作用$g$再作用$f$ \\
%    $C^r_p(U)$ & $p\in U$点的全体$r$阶连续可微函数集合 \\
%    $V^*$ & 矢量空间$V$的对偶空间 \\
%    $\frac{\partial }{\partial x^\mu}$或$(\frac{\partial }{\partial x^\mu})^a$ &  第$\mu$个坐标基矢 \\
%    ${\rm d}x^\mu$或$({\rm d}x^\mu)^a$ & 第$\mu$个对偶坐标基矢 \\
%    $\mathfrak{T}^p_q(V)$ & 矢量空间$V$上全体$(p,q)$型张量场 \\
%    $\mathfrak{X}(M),\, \mathfrak{X}^*(M)$ & 流形$M$上的全体切矢量场,余切矢量场 \\
%    $\nabla,\, {\rm D}$ & 联络 \\
%    $\phi^*$ & 映射$\phi$诱导出的余切映射或拉回映射 \\
%    $\phi_*$ & 映射$\phi$诱导出的切映射或推前映射 \\
%    $\ln, \log$ & 以$e$为底的对数 \\
%    ${\rm Span},\, {\rm Span}_{\mathbb{F}}$ & 组合系数是$\mathbb{F}$中的常数 \\
%    ${\rm Span}_{\infty}$ & 组合系数是$C^{\infty}(M)$中的函数 \\
%\end{tabular}






%\newpage
%
%\section*{广义黎曼几何}
%\noindent{\heiti\bfseries  甲}: 度规符号:$(- + + +)$.
%
%\noindent{\heiti\bfseries  乙}:
%Riemann(黎曼)曲率张量和Ricci曲率张量定义
%\begin{align*}
%    R_{\hphantom{d} cab}^d{X^a}{Y^b}{Z^c} &\equiv {\nabla _X}{\nabla _Y}{Z^d} -
%       {\nabla _Y}{\nabla _X}{Z^d} - {\nabla _{[X,Y]}}{Z^d} \\
%    R_{dcab} &\equiv g_{de}R_{\hphantom{d} cab}^e   \\
%    R_{cb} &\equiv R_{\hphantom{d} cab}^a = g^{ad}R_{dcab}  \ = - g^{ad}R_{dcba} \\
%    R &\equiv  g^{cb}R_{cb} = g^{cb} g^{ad}R_{dcab}
%\end{align*}
%
%Riemann曲率和Ricci曲率在自然坐标系的分量表达式为
%\begin{align*}
%    R_{ijkn} =& \frac{1}{2}\left(
%        \frac{\partial^2 g_{in}} {\partial x^j\partial x^k}
%        - \frac{\partial^2 g_{jn}} {\partial x^i\partial x^k}
%        - \frac{\partial^2 g_{ik}} {\partial x^j\partial x^n}
%        + \frac{\partial^2 g_{jk}} {\partial x^i\partial x^n} \right) 
%        + \Gamma_{jk}^l\Gamma _{lin} - \Gamma _{jn}^l\Gamma _{lik}  \\  
%    R_{\hphantom{d} jln}^i =& {\partial_l} \Gamma_{jn}^{i} -\partial_n \Gamma_{jl}^{i}+
%      \Gamma_{jn}^{k} \Gamma_{kl}^{i} - \Gamma_{jl}^{k}\Gamma_{kn}^{i}   \\
%    R_{jn} =& R_{\hphantom{d} jkn}^k = {\partial_k} \Gamma_{jn}^{k} -\partial_n \Gamma_{jk}^{k}
%    + \Gamma_{jn}^{k} \Gamma_{kl}^{l}- \Gamma_{jl}^{k}\Gamma_{kn}^{l}
%\end{align*}
%
%第二类Christoffel(克氏)记号:
%$\Gamma_{ab}^c = \frac{1}{2}{g^{ce}}\left( {\frac{{\partial {g_{ae}}}}{{\partial {x^b}}}
%    + \frac{{\partial {g_{eb}}}}{{\partial {x^a}}}  - \frac{{\partial {g_{ab}}}}
%    {{\partial {x^e}}}} \right) $
%
%普通导数:$f_{,\mu}\equiv\frac{\partial f}{\partial x^\mu}$
%
%协变导数:$v^\mu_{\hphantom{\nu} ;\nu}= v^\mu _{\hphantom{\mu} ,\nu}+ \Gamma^\mu_{\rho\nu}v^\rho,\qquad
%\omega_{\mu ;\nu}=\omega_{\mu ,\nu}- \Gamma^\rho_{\mu\nu}\omega_\rho$
%
%
%\noindent{\heiti\bfseries  丙}:
%正定能量密度,即能动张量的“时时”分量为正:$T_{00}>0$或$T^{00}>0$.
%
%理想流体:$T_{ab}=\bigl(\rho + p/c^2 \bigr) U_a U_b +p g_{ab}$
%
%电磁场:$T_{ab}=\frac{1}{\mu_0} \left( F_{ac} F_{b}^{\cdot c} - \frac{1}{4} g_{ab} F_{cd} F^{cd} \right)$
%
%\noindent{\heiti\bfseries  丁}:
%Einstein(爱因斯坦)引力场方程(度规号差不影响方程右端的符号)
%\begin{equation*}
%    G_{ab} \equiv  R_{ab} - \frac{1}{2}g_{ab} R  = +\frac{8\pi G}{c^4} T_{ab}
%\end{equation*}

%\noindent{\heiti\bfseries  戊}:





%\newpage
%%\vspace{0.1cm}
%\begin{longtable}{|*5{c|}}
%    \caption{号差习惯表}\label{sign-convention} \\ \hline 
%    文献 & 度规号差   &   $R^{d}_{abc}$   &  Einstein   &    位置   \\ \hline
%    \endfirsthead
%    \multicolumn{2}{l}{(续表)} \\ \hline
%    文献 & 度规号差   &   Riemann   &  Einstein   &    位置   \\ \hline
%    \endhead \hline
%    \multicolumn{2}{c}{(接下一页表格……)} \\[2ex]
%    \endfoot
%    \hline
%    \endlastfoot
%    %    % data begins here
%    本书 & + & + & + &  1  \\ \hline
%    \textcite{carroll-stg2019} & + & + & + &  1  \\
%    \textcite{Choquet-Bruhat-2009} & + & + & + &  3  \\
%    \textcite{dirac-rela-1975} & $-$ & + & $-$ &  1  \\
%    \textcite{hawking-ellis1973} & + & + & + &  1  \\
%    \textcite{mtw1973} & + & + & + &  1  \\
%    \textcite{oneill1983} & + & $-$ & + &  1  \\
%    \textcite{ohanian-ruffini-2013} & $-$ & + & $-$ &  1  \\
%    \textcite{padmanabhan2010}  & + & + & + &  1  \\
%    \textcite{penrose-Rindler1984} & $-$ & + & $-$ &  4  \\
%    \textcite{poisson-will-2014} & + & + & + &  1  \\
%    \textcite{sachs-wu-1977} & + & + & + &  1  \\
%    \textcite{schutz-2022} & + & + & + &  1  \\
%    \textcite{stephani-exe-2003} & + & + & + &  1  \\
%    \textcite{straumann2013} & + & + & + &  1  \\
%    \textcite{wald1984} & + & $-$ & + &  4  \\
%    \textcite{weinberg_grav-1972}  & + & $-$ & $-$ &  1  \\
%    \textcite{will_tegp-2018}  & + & + & + &  1  \\
%    \textcite{chenbin2018} & + & + & + &  1  \\
%    \textcite{liang_zhou2006_1} & + & $-$ & + &  4  \\
%    \textcite{wangyj-2011}  & $-$ & + & + &  1  \\
%    \textcite{xu-wu-1999}  & + & + & + &  1  \\
%    %
%    %    % more data here
%    \hline
%\end{longtable}


%\vspace{1em}
%
%我们仿照\textcite{mtw1973}所著《引力论》的扉页上的度规号差等约定,
%上表列出了1972年之后主要引力论书籍采用的符号习惯.
%第一列为参考文献;第二列是度规号差;
%第三列是$\Tpq{1}{3}$型黎曼张量$R^{d}_{abc}$前符号;
%第四列是爱因斯坦方程右端能量-动量张量$T_{ab}$前的正负号;
%第五列是$\Tpq{1}{3}$型黎曼张量的逆变指标降下来后所处的位置.










%\newpage

\printbibliography[heading=subbibliography,title=参考文献]


\endinput
