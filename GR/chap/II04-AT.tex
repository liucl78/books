% !TeX encoding = UTF-8
% 此文件从2025.11开始

\chapter{引力替代理论}\label{chat}

在\S\ref{chlh:sec_Gravity}中,我们从Hilbert作用量导出了爱氏场方程;
这一框架有如下基本假设:
{\bfseries (1)} 仿射联络是无挠的;
{\bfseries (2)} 仿射联络与度规相容;
{\bfseries (3)} 关于引力的动力学变量只有度规$g_{ab}$;
{\bfseries (4)} 引力场方程是二阶偏微分方程组.
在本章中,我们试着放宽上述要求来寻找爱氏引力的{\kaishu 引力替代理论}.

在本章中我们假设$(M,g)$是四维闵氏时空,其上有Levi-Civata联络$\nabla_a$.

为简单起见,我们假设非引力场拉式密度仍然只依赖$(M,g)$上度规场$g_{ab}$,
不会依赖新引进的与引力有关的动力学变量.






\section{$f(R)$理论}

我们先叙述$f(R)$理论\cite{Sotiriou-2010a}.

$f(R)$级数展开式为
\begin{equation}\label{chat:eqn_fR-series} 
	f(R)=\cdots+\frac{\alpha_2}{R^2} 
	+\frac{\alpha_1}{R}-2\Lambda+R+\frac{R^2}{\beta_2} 
	+\frac{R^3}{\beta_3}+\cdots, 
\end{equation} 
作用量为
\begin{equation}\label{chat:eqn_fR-action} 
I= \frac{1}{16\pi}\int  f(R) \sqrt{-g} {\rm d}^4 x  +I_{NG} .
\end{equation} 
其中$I_{NG}$是非引力场作用量\eqref{chlh:eqn_I-NG}.
由此得到场方程是(其中$\Box \equiv g^{ab} \nabla_a \nabla_b$):
\begin{equation}\label{chat:eqn_fR-feq}
f'(R)R_{ab}-\frac{1}{2} f(R) g_{ab}- \left(\nabla_a\nabla_b -g_{ab}\Box \right) f'(R) = 8\pi \, T_{ab} .
\end{equation} 
当$f(R)=R-2\Lambda$时,由上式可以得到广义相对论中的爱氏引力场方程:
\begin{equation}
	R_{ab}-\frac{1}{2} (R-2\Lambda) g_{ab}= 8\pi \, T_{ab} .
\end{equation} 



%下面叙述式\eqref{chat:eqn_fR-feq}的变分过程.
%\begin{align*}
%	0 &= \updelta I = \updelta \left[\frac{1}{16\pi}\int {\rm d}^4 x \sqrt{-g} \, f(R) \right] + \updelta I_{NG} \\
%	&= \frac{1}{16\pi } \int \left[ \left(\updelta f(R) \right)\sqrt{-g}
%	-f(R)\frac{\sqrt{-g}}{2} g_{ab} \updelta g^{ab} \right]	{\rm d}^{4}x
%	-\int \frac{1}{2} T_{ab} \sqrt{-g} \updelta g^{ab} \,{\rm d}^{4}x \\
%	&= \frac{1}{16\pi } \int \left[ f'(R) R_{ab} (\updelta g^{ab}) \sqrt{-g}
%	-f(R)\frac{\sqrt{-g}}{2} g_{ab} \updelta g^{ab} - 8\pi T_{ab} \sqrt{-g} 
%	\updelta g^{ab}\right]	{\rm d}^{4}x\\
%	&\quad + \frac{1}{16\pi } \int \left[ f'(R)  \sqrt{-g} g^{ab}\updelta R_{ab} \right]{\rm d}^{4}x \\
%	&= \frac{1}{16\pi } \int \left[ f'(R) R_{ab} -\frac{1}{2} g_{ab} f(R)
%	- 8\pi T_{ab} \right] \sqrt{-g} \updelta g^{ab} {\rm d}^{4}x \\
%	&\quad + \frac{1}{16\pi } \int \left[ f'(R)   
%	\nabla^d \left(\nabla^b \updelta g_{db} - g^{cb} \nabla_d \updelta g_{cb}
%	\right) \right] \sqrt{-g} {\rm d}^{4}x \\
%	&\xlongequal[\text{丢掉边界项}]{\text{分部积分并}}
%	\frac{1}{16\pi } \int \left[ f'(R) R_{ab} -\frac{1}{2} g_{ab} f(R)
%	- 8\pi T_{ab} \right] \sqrt{-g} \updelta g^{ab} {\rm d}^{4}x \\
%	&\quad - \frac{1}{16\pi } \int\left\{ 
%	\left(\nabla^b \updelta g_{db} - g^{cb} \nabla_d \updelta g_{cb}\right) \nabla^d f'(R)
%	\right\} \sqrt{-g} {\rm d}^{4}x \\
%	&\xlongequal[\text{并丢掉边界项}]{\text{再次分部积分}}
%	\frac{1}{16\pi } \int \left[ f'(R) R_{ab} -\frac{1}{2} g_{ab} f(R)
%	- 8\pi T_{ab} \right] \sqrt{-g} \updelta g^{ab} {\rm d}^{4}x \\
%	&\quad - \frac{1}{16\pi } \int\left\{ 
%	- \left[\nabla^b\nabla^d f'(R)\right] \updelta g_{db} 
%	+ \nabla_d \left[g^{cb} \nabla^d f'(R) \right]  \updelta g_{cb}
%	\right\} \sqrt{-g} {\rm d}^{4}x \\
%	&=\frac{1}{16\pi } \int \left[ f'(R) R_{ab} -\frac{1}{2} g_{ab} f(R)
%	- 8\pi T_{ab} \right] \sqrt{-g} \updelta g^{ab} {\rm d}^{4}x \\
%	&\quad - \frac{1}{16\pi } \int\left\{ 
%	(\nabla^c\nabla^d f'(R)) g_{db} g_{ca}\updelta g^{ba} 
%	- \left[ \nabla_d \nabla^d f'(R) \right]  g_{ab}\updelta g^{ab}
%	\right\} \sqrt{-g} {\rm d}^{4}x \\
%	&=\frac{1}{16\pi } \int \left[ f'(R) R_{ab} -\frac{1}{2} g_{ab} f(R)
%	- (\nabla_a \nabla_b - g_{ab}\Box) f'(R)
%	- 8\pi T_{ab} \right] \sqrt{-g} \updelta g^{ab} {\rm d}^{4}x .
%\end{align*}
%因度规变分独立,故可得式\eqref{chat:eqn_fR-feq}.


\section{度规-仿射联络引力论}

在$(M,g)$上已有Levi-Civata联络$\nabla_a$;
我们再引入一个一般仿射联络$\oversetmy{\bullet\circ}{\nabla}_a$(采用\S\ref{chrg:sec_metric-affine}记号),
它可能有挠、也可能与度规不相容.
这种额外引入一个仿射联络的引力理论被称为{\heiti 度规-仿射联络引力论}
(Metric-Affine Gravity).
相当于引进了$64$个标量场$\oversetmy{\bullet\circ}{\Gamma}^i_{jk}$当作
新的与引力相关的动力学变量(可能包括、也可能不包括挠率和度规不相容项).
也可以进一步假设物质场拉式密度依赖于这些量($\oversetmy{\bullet\circ}{\Gamma}^i_{jk}$),
这类理论可以查阅\parencite{Sotiriou-2010a},就不在此讨论了.


作用量为(注$\oversetmy{\bullet\circ}{R}\equiv \oversetmy{\bullet\circ}{R}_{ab} g^{ab}$):
\begin{equation}\label{chat:eqn_MAG-action} 
	I= \frac{1}{16\pi}\int f\left(\oversetmy{\bullet\circ}{R} \right) \sqrt{-g} {\rm d}^4 x  +I_{NG} .
\end{equation} 
我们需要对度规$g^{ab}$和$\oversetmy{\bullet\circ}{\Gamma}^i_{jk}$作变分.
\begin{align*}
	0=& \updelta I = \updelta \left[\frac{1}{16\pi}\int f\left(\oversetmy{\bullet\circ}{R} \right)
	\sqrt{-g} {\rm d}^4 x \right]  + \updelta I_{NG} \\
	=& \frac{1}{16\pi} \int \left[f'\ \oversetmy{\bullet\circ}{R}_{ab} (\updelta g^{ab}) \sqrt{-g} 
	+f\left(\oversetmy{\bullet\circ}{R}\right) \updelta \sqrt{-g}\right] {\rm d}^4 x   + \updelta I_{NG}\\	  
	&+ \frac{1}{16\pi}\int f'\ (\updelta\oversetmy{\bullet\circ}{R}_{ab}) g^{ab} \sqrt{-g} {\rm d}^4 x\\
	=& \frac{1}{16\pi} \int \left[f'\  \oversetmy{\bullet\circ}{R}_{(ab)} 
	-\frac{1}{2} f\left(\oversetmy{\bullet\circ}{R}\right) g_{ab} - 8\pi T_{ab}  \right] 
	\sqrt{-g} \updelta g^{ab} {\rm d}^4 x \\
	&+\frac{1}{16\pi}\int f'\  \left(
	\oversetmy{\bullet\circ}{\nabla}_e \updelta\oversetmy{\bullet\circ}{\Gamma}_{ab}^{e} 
	-\oversetmy{\bullet\circ}{\nabla}_b \updelta\oversetmy{\bullet\circ}{\Gamma}_{ae}^{e} \right) 
	g^{ab} \sqrt{-g} {\rm d}^4 x\\
	=& \frac{1}{16\pi} \int \left[f'\  \oversetmy{\bullet\circ}{R}_{(ab)} 
	-\frac{1}{2} f\left(\oversetmy{\bullet\circ}{R}\right) g_{ab} - 8\pi T_{ab} \right] 
	\sqrt{-g} \updelta g^{ab} {\rm d}^4 x \\
	&+\frac{1}{16\pi}\int  \left[
	- \oversetmy{\bullet\circ}{\nabla}_e (f'g^{ab}) 
	+ \oversetmy{\bullet\circ}{\nabla}_c (f'g^{c(a})  \delta_e^{b)}
	\right] \updelta\oversetmy{\bullet\circ}{\Gamma}_{ab}^{e} \sqrt{-g} {\rm d}^4 x .
\end{align*}
由此可得场方程:
\begin{align}
	f'\left(\oversetmy{\bullet\circ}{R}\right)\  \oversetmy{\bullet\circ}{R}_{(ab)} 
	-\frac{1}{2} f\left(\oversetmy{\bullet\circ}{R}\right) g_{ab} =&  8\pi T_{ab}.\\ 
	\oversetmy{\bullet\circ}{\nabla}_e \left(f' g^{ab} \right)
	- \oversetmy{\bullet\circ}{\nabla}_c \left( f' g^{c(a} \right) \delta_e^{b)} = &0.
\end{align}
很明显,当一般仿射联络($\oversetmy{\bullet\circ}{\nabla}_a$)是Levi-Civita联络,
且$f(R)=R-2\Lambda$时,上式就是广义相对论中的爱氏引力场方程.


\section{标量-张量理论}

作用量为
\begin{equation}
	I=\frac{1}{16 \pi } \int\left[\phi R-\frac{\omega(\phi)}{\phi} g^{ab} (\nabla_a\phi) \nabla_b\phi
	-U(\phi)\right] \sqrt{-g} {\rm d}^4 x + I_{NG} .
\end{equation}
变分后可得场方程(其中$\Box = g^{ab}\nabla_a \nabla_b$,$T= g^{ab}T_{ab}$):
\begin{align}
	& R_{ab} - \frac{1}{2}g_{ab}R + g_{ab} \frac{U (\phi)}{2 \phi} =  
	\frac{8 \pi}{\phi} T_{ab}+\frac{1}{\phi}\left(\nabla_a\nabla_b\phi-g_{ab} \Box \phi\right) \notag \\
	&\qquad\qquad +\frac{\omega(\phi)}{\phi^2}\left((\nabla_a \phi) \nabla_b\phi
	-\frac{1}{2} g_{ab} (\nabla_c\phi) \nabla^c\phi\right) , \label{chat:eqn_ST-GFE} \\
	&\Box \phi=  \frac{1}{3+2 \omega(\phi)}\left(8 \pi T-\frac{{\rm d} \omega(\phi)}{{\rm d} \phi} 
	(\nabla_c\phi) \nabla^c\phi+\phi\frac{{\rm d}U(\phi)}{{\rm d} \phi} 
	-2U(\phi)\right). \label{chat:eqn_ST-PhiFE}
\end{align}

%变分过程
%\begin{align*}
%	0 =& \updelta I = \updelta \left[\frac{1}{16 \pi } \int\left[\phi R-\frac{\omega(\phi)}{\phi} g^{ab} 
%	(\nabla_a\phi) \nabla_b\phi	-U(\phi)\right] \sqrt{-g} {\rm d}^4 x \right] + \updelta I_{NG} \\
%	=& \frac{1}{16\pi } \int \Bigl[ \Bigl( \phi \updelta R + R \updelta\phi
%	-\frac{\omega}{\phi} g^{ab} (\nabla_a\phi) \nabla_b \updelta \phi	
%	-\frac{\omega}{\phi} g^{ab} (\nabla_a \updelta \phi) \nabla_b\phi	\\
%	&-\frac{\omega}{\phi} (\updelta g^{ab}) (\nabla_a\phi) \nabla_b\phi	
%	-\left(\frac{\omega'}{\phi} \updelta \phi - \frac{\omega}{\phi^2} \updelta \phi \right)
%	g^{ab} (\nabla_a\phi) \nabla_b\phi	-U' \updelta \phi\Bigr) \sqrt{-g} \\
%	&-\left(\phi R-\frac{\omega(\phi)}{\phi} g^{ab} 
%	(\nabla_a\phi) \nabla_b\phi	-U(\phi)\right) \frac{\sqrt{-g}}{2} g_{ab} \updelta g^{ab} \Bigr]	{\rm d}^{4}x
%	-\int \frac{1}{2} T_{ab} \sqrt{-g} \updelta g^{ab} \,{\rm d}^{4}x \\
%	=&\frac{1}{16\pi } \int  \phi \sqrt{-g} \left( 
%	\nabla^d\nabla^b \updelta g_{db} - g^{cb} \nabla^d\nabla_d \updelta g_{cb}
%	+ R_{ab} (\updelta g^{ab}) \right) {\rm d}^{4}x \\
%	&-\int \frac{1}{32\pi} \left[ g_{ab}  \left(\phi R-\tfrac{\omega}{\phi} g^{ab} 
%(\nabla_a\phi) \nabla_b\phi	-U(\phi)\right) 
%+\frac{2\omega}{\phi} (\nabla_a\phi) \nabla_b\phi 
%+ 16 \pi T_{ab}\right] \sqrt{-g} \updelta g^{ab} \,{\rm d}^{4}x \\
%	&+ \frac{1}{16\pi } \int \Bigl[ 
%	-\frac{2\omega}{\phi} g^{ab} (\nabla_a\phi) \nabla_b \updelta \phi	\\
%	&+\Bigl( R -\left(\frac{\omega'}{\phi}  - \frac{\omega}{\phi^2} \right)
%	g^{ab} (\nabla_a\phi) \nabla_b\phi -U' \Bigr)\updelta\phi
%	  \Bigr] \sqrt{-g}{\rm d}^{4}x \\
%	 =&\frac{1}{16\pi } \int  \left( 
%	 - \nabla_a \nabla_b\phi + g_{ab} (\nabla_d \nabla^d\phi)  \right) 
%	 \sqrt{-g} \updelta g^{ab}  {\rm d}^{4}x \\
%	 &-\int \frac{1}{32\pi} \left[ g_{ab}  \left(\phi R-\tfrac{\omega}{\phi} g^{ab} 
%	 (\nabla_a\phi) \nabla_b\phi	-U(\phi)\right) 
%	 +\frac{2\omega}{\phi} (\nabla_a\phi) \nabla_b\phi - 2\phi R_{ab}
%	 + 16 \pi T_{ab}\right] \sqrt{-g} \updelta g^{ab} \,{\rm d}^{4}x \\
%	 &+ \frac{1}{16\pi } \int \Bigl[ 
%	 -\nabla_b \left(\tfrac{2\omega}{\phi} g^{ab} (\nabla_a\phi)  \updelta \phi\right) 
%	 + \nabla_b \left(\tfrac{2\omega}{\phi} g^{ab} (\nabla_a\phi)  \right) \updelta \phi\\
%	 &+\Bigl( R -\left(\tfrac{\omega'}{\phi}  - \tfrac{\omega}{\phi^2} \right)
%	 g^{ab} (\nabla_a\phi) \nabla_b\phi -U' \Bigr)\updelta\phi
%	  \Bigr] \sqrt{-g} {\rm d}^{4}x .
%\end{align*}
%可得场方程
%\begin{align*}
%	0 = & \nabla_a \nabla_b\phi -  g_{ab} \Box \phi + \frac{1}{2}g_{ab}  \left(\phi R-\tfrac{\omega}{\phi} 
%	(\nabla^c\phi) \nabla_c\phi	-U(\phi)\right) \\
%	&+\frac{\omega}{\phi} (\nabla_a\phi) \nabla_b\phi - \phi R_{ab}+ 8 \pi T_{ab}, \\
%	0=& \nabla_b \left(\tfrac{2\omega}{\phi} g^{ab} (\nabla_a\phi)  \right)
%	+R -\left(\tfrac{\omega'}{\phi}  - \tfrac{\omega}{\phi^2} \right)
%	g^{ab} (\nabla_a\phi) \nabla_b\phi -U' .
%\end{align*}
%对第一式求迹
%\begin{align*}
%	0 = & - 3 \Box \phi + \phi R - 2 U 
%	-\frac{\omega}{\phi} (\nabla^c\phi) \nabla_c\phi + 8 \pi T .
%\end{align*}
%带入$\phi$的动力学方程,将$R$消掉,有
%\begin{align*}
%	0=& \frac{2\omega}{\phi} \nabla_b \nabla^b \phi 
%	+ 2(\nabla^b \phi) \left( \frac{ \nabla_b\omega}{\phi} - \frac{\omega}{\phi^2}\nabla_b\phi \right)
%	-\left(\frac{\omega'}{\phi}  - \frac{\omega}{\phi^2} \right)
%	(\nabla^b\phi) \nabla_b\phi -U'  +R  \\
%	0=& 2\omega \Box \phi  
%	+\left(\omega'  - \frac{\omega}{\phi} \right)(\nabla^b\phi) \nabla_b\phi -\phi U' 
%	+ 3 \Box \phi + 2 U +\frac{\omega}{\phi} (\nabla^c\phi) \nabla_c\phi - 8 \pi T \\
%	=& (2\omega +3) \Box \phi  
%	+ \omega'  (\nabla^b\phi) \nabla_b\phi -\phi U'  + 2 U  - 8 \pi T 
%\end{align*}












\section*{小结}
本章主要参考了\parencite{will_tegp-2018}第五章,以及那里列出的原始文献.


\printbibliography[heading=subbibliography,title=第\ref{chat}章参考文献]

\endinput
