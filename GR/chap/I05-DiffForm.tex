%% !TeX encoding = UTF-8
%% June 2022

\chapter{外微分型式} \label{chdf}

外微分的现代形式是由法国数学家\'{E}lie Cartan(1869-1951)提出并发展的一种微分几何中常用方法.
本章主要介绍微分型式场的概念,以及外微分运算;
继而,介绍Frobenius定理及其外微分表示;
然后介绍流形定向与单位分解,最后引入流形上的积分概念.

\index[physwords]{外微分型式场}
\index[physwords]{外型式场|see{外微分型式场}}

\section{外微分型式场}\label{chdf:sec_exterior-diff-fields}
在\S \ref{chmla:sec_exterior-algebra}中,我们已经引入了线性空间$V$中外代数
的概念;那里只是针对一个特定空间$V$(或其对偶空间$V^*$),而本节我们将这些概念推广到\CJKunderwave{场}中.
在\S \ref{chdm:sec_tensor-fields}中引入了张量场的概念,微分型式场是全反对称的协变张量场.

\begin{definition}
设有$m$维光滑流形$M$,其上一个$C^\infty$\CJKunderwave{全反对称}$r$阶\CJKunderwave{协变}张量\CJKunderwave{场}称为
流形$M$上的一个$r$次{\heiti 外微分型式\CJKunderwave{场}},简称为{\heiti 外型式场};也称为{\heiti 外微分式}.
\end{definition}
上面已将\S \ref{chmla:sec_exterior-form}中“型式”定义\ref{chmla:def_exterior-form}推广到了\CJKunderwave{场}.
在不引起误解时,可以省略前缀“外”字.
此\CJKunderwave{场}自然是构建在余切丛$T^*M$上的,对于点$x\in M$,有局部坐标系$(U;x^i)$;
仿照式\eqref{chmla:eqn_expand-on-base},$r$型式场有如下表达式
\begin{subequations}\label{chdf:eqn_omega-expand-on-bases}
\begin{align}
    \left. \omega_{a_{1}\cdots a_{r}} \right|_U
    & =\sum_{\mu_{1}<\cdots<\mu_{r}} \omega_{\mu_{1}\cdots \mu_{r}}
    ({\rm d}x^{\mu_1})_{a_{1}} \wedge\cdots\wedge ({\rm d}x^{\mu_r})_{a_{r}},
    \quad \{\mu\} \text{有序} \label{chdf:eqn_omega-expand-on-oldbases} \\
    &= \frac{1}{r!} \omega_{\mu_{1}\cdots \mu_{r}}
    ({\rm d}x^{\mu_1})_{a_{1}} \wedge\cdots\wedge ({\rm d}x^{\mu_r})_{a_{r}} .
    \quad \{\mu\} \text{无序}\label{chdf:eqn_omega-expand-on-newbases}
\end{align}
\end{subequations}
上面公式中,$\omega_{a_{1}\cdots a_{r}}$和$\omega_{\mu_{1}\cdots \mu_{r}}$的下标
是全反对称的,即交换其中任意两个指标都会产生一个负号.
很明显,当$r>m$时,外微分式恒为零(为什么?).

我们将$M$上$r$次全体光滑外微分型式场的集合记为$A^r(M)$.
为方便起见,规定$M$上全体光滑函数场是$0$次微分型式场,即$A^0(M)=C^\infty(M)$;
通常的$1$次微分式便是$1$次外微分型式场$A^1(M)$或$\mathfrak{X}^{*}(M)$.

微分型式场自然也可以看成多重线性函数,也有类似于
公式\eqref{chmla:eqn_xiv0}和\eqref{chmla:eqn_xiv}的求值公式,
请读者自行写出表达式.

可将\S \ref{chdm:sec_tensor-fields}中定义的张量间的加法和数量乘法,以及$C^\infty(M)$-乘法
直接运用到微分型式场(微分型式场只是张量场的特例,自然可直接运用这些运算).
除此之外,还可仿照\S \ref{chmla:sec_exterior-product}逐点定义外积,使
\begin{equation}
    A(M)= \bigoplus_{r=0}^{m} A^r(M)
\end{equation}
成为{\heiti 外代数};一般说来它是无穷维的.

很容易把定义\ref{chdm:def_pullback-1form-onfield}(不需要映射是微分同胚的)用到微分型式场上,
所拉回的映射是同阶的外型式场.
\begin{theorem}\label{chdf:thm_dfs-fsd}
    设有光滑流形$M$和$N$,维数分别是$m$和$n$;流形间存在光滑映射$\phi:M\to N$,
    拉回映射$\phi^*: A^r(N) \to A^r(M)$( $0\leqslant r \leqslant \min(m,n)$)是
    线性的,且与外积可交换,即
    \begin{equation*}
        \phi^{*} (\alpha \wedge \beta ) = (\phi^{*}\alpha) \wedge \phi^{*}\beta,
        \qquad \forall \alpha \in A^p(N), \ \beta \in A^q(N), \quad 0\leqslant p+q \leqslant \min(m,n) .
    \end{equation*}
\end{theorem}
\begin{proof}
    设点$x\in M$有局部坐标$(U;x^i)$;点$y=\phi(x)\in N$有局部坐标$(V;y^i)$.
    \begin{align*}
        & \phi^{*}(\alpha_{a_1\cdots a_p} \wedge \beta_{b_1\cdots b_q}) =
           \frac{(p+q)!}{p! q!} \phi^{*}(\alpha_{[a_1\cdots a_p} \beta_{b_1\cdots b_q]} ) \\
        =& \frac{(p+q)!}{p! q!} \phi^{*}\left(
           a_{j_1\cdots j_p} b_{k_1\cdots k_q}
           ({\rm d}y^{j_1})_{[a_1}  \cdots  ({\rm d}y^{j_p})_{a_p}
           ({\rm d}y^{k_1})_{b_1}  \cdots  ({\rm d}y^{k_q})_{b_q]} \right) \\
        =& \frac{1}{p! q!} a_{j_1\cdots j_p} b_{k_1\cdots k_q}
           \delta^{j_1 \cdots j_p k_1 \cdots k_q}_{i_1 \cdots i_p l_1 \cdots l_q}
           \phi^{*}\left(
             ({\rm d}y^{i_1})_{a_1} \otimes \cdots \otimes ({\rm d}y^{i_p})_{a_p} \otimes
             ({\rm d}y^{l_1})_{b_1} \otimes \cdots \otimes ({\rm d}y^{l_q})_{b_q} \right) \\
        =& \frac{1}{p! q!} a_{j_1\cdots j_p} b_{k_1\cdots k_q}
           \delta^{j_1 \cdots j_p k_1 \cdots k_q}_{i_1 \cdots i_p l_1 \cdots l_q}
           \frac{\partial y^{i_1} }{\partial x^{\pi_1} } \cdots
           \frac{\partial y^{i_p} }{\partial x^{\pi_p} } \times
           \frac{\partial y^{l_1} }{\partial x^{\sigma_1} } \cdots
           \frac{\partial y^{l_q} }{\partial x^{\sigma_q} } \\
           & \qquad \times
           ({\rm d}x^{\pi_1})_{a_1} \otimes \cdots \otimes ({\rm d}x^{\pi_p})_{a_p} \otimes
           ({\rm d}x^{\sigma_1})_{b_1} \otimes \cdots \otimes ({\rm d}x^{\sigma_q})_{b_q}  \\
        =& \left( \frac{1}{p!} a_{j_1\cdots j_p}
           \frac{\partial y^{j_1} }{\partial x^{\pi_1} } \cdots
           \frac{\partial y^{j_p} }{\partial x^{\pi_p} }
           ({\rm d}x^{\pi_1})_{a_1} \wedge \cdots \wedge ({\rm d}x^{\pi_p})_{a_p} \right) \\
          & \qquad \wedge \left(\frac{1}{q!} b_{k_1\cdots k_q}
            \frac{\partial y^{k_1} }{\partial x^{\sigma_1} } \cdots
           \frac{\partial y^{k_q} }{\partial x^{\sigma_q} }
           ({\rm d}x^{\sigma_1})_{b_1} \wedge \cdots \wedge ({\rm d}x^{\sigma_q})_{b_q} \right)  \\
         =&   (\phi^{*}\alpha_{a_1\cdots a_p} ) \wedge \phi^{*}\beta_{b_1\cdots b_q} .
    \end{align*}
    证明过程需参阅式\eqref{chmla:eqn_alpha-beta}、\eqref{chdm:eqn_pull-bases}和式\eqref{chdf:eqn_omega-expand-on-bases}.
\end{proof}
证明过程中,我们还得到了一个常用公式:
\begin{equation}\label{chdf:eqn_fsalpha}
    \left. \phi^{*}\alpha_{a_1\cdots a_p} \right|_U =
      \left(\frac{1}{p!} \alpha_{j_1\cdots j_p} \circ \phi \right) 
    \frac{\partial y^{j_1} }{\partial x^{\pi_1} } \cdots
    \frac{\partial y^{j_p} }{\partial x^{\pi_p} } 
    ({\rm d}x^{\pi_1})_{a_1} \wedge \cdots \wedge ({\rm d}x^{\pi_p})_{a_p} .
\end{equation}


\begin{example}\label{chdf:exam_eiapk}
	计算$(e^{i_1})_{a_1} \wedge \cdots \wedge (e^{i_p})_{a_p} (e_\kappa)^{a_p}$和
	$(e_\kappa)^{a_1} (e^{i_1})_{a_1} \wedge \cdots \wedge (e^{i_p})_{a_p} $.
\end{example}
参考式\eqref{chmla:eqn_ep-base},有
\begin{equation*} %\label{chmla:eqn_eiap}
	(e^{i_1})_{a_1} \wedge \cdots \wedge (e^{i_p})_{a_p}
	= \delta^{i_1 \cdots i_p}_{j_1 \cdots j_p}
	(e^{j_1})_{a_1}  \otimes	\cdots \otimes (e^{j_p})_{a_p} ;
	\quad 1\leqslant \{i\},  \{j\}  \leqslant m .
\end{equation*}
利用式\eqref{chmla:eqn_gkd-95},有
\begin{align*}
	&(e^{i_1})_{a_1} \wedge \cdots \wedge (e^{i_p})_{a_p} (e_k)^{a_p} = 
	\sum_{s=1}^{p} (-)^{p+s} \delta_{j_p}^{i_s} \cdot
	\delta_{j_1 \cdots  j_{p-1}}^{i_1 \cdots\hat{i}_s \cdots i_{p}}
	(e^{j_1})_{a_1}  \otimes	\cdots \otimes (e^{j_p})_{a_p} (e_\kappa)^{a_p} \\
	=&\sum_{s=1}^{p} (-)^{p+s} \delta_{\kappa}^{i_s} \cdot
	\delta_{j_1 \cdots  j_{p-1}}^{i_1 \cdots\hat{i}_s \cdots i_{p}} 
	(e^{j_1})_{a_1}  \otimes	\cdots \otimes (e^{j_{p-1}})_{a_{p-1}} \\
	=& \sum_{s=1}^{p} (-)^{p+s} \delta_{\kappa}^{i_s} \cdot
	(e^{i_1})_{a_1} \wedge\cdots \wedge \widehat{e^{i_s}}
	\wedge \cdots \wedge (e^{i_p})_{a_{p-1}} .
\end{align*}
上式对$s$求和过程中至多有一项非零.给出两个具体的示例.
\setlength{\mathindent}{0em}
\begin{align*}
	&({\rm d}x^{0})_{a_{0}} \wedge({\rm d}x^{1})_{a_{1}}\wedge({\rm d}x^{2})_{a_{2}}
	\wedge ({\rm d}x^{3})_{a_{3}} (\partial_\kappa)^{a_3}
	=(-)^{3+\kappa} ({\rm d}x^{0})_{a_{0}} \wedge\cdots \wedge \widehat{{\rm d}{x}^{\kappa}}
	\wedge \cdots \wedge ({\rm d}x^{3})_{a_{2}} . \\
	& (\partial_\kappa)^{a_0} ({\rm d}x^{0})_{a_{0}} \wedge({\rm d}x^{1})_{a_{1}}
	\wedge({\rm d}x^{2})_{a_{2}}	\wedge ({\rm d}x^{3})_{a_{3}} 
	=(-)^{\kappa} ({\rm d}x^{0})_{a_{1}} \wedge\cdots \wedge \widehat{{\rm d}{x}^{\kappa}}
	\wedge \cdots \wedge ({\rm d}x^{3})_{a_{3}} . 
\end{align*}\setlength{\mathindent}{2em}
其中$0\leqslant \kappa \leqslant 3$.在广义相对论中会经常用到上两式中的缩并.
\qed

%\begin{align*}
%	&(e_\kappa)^{a_1} (e^{i_1})_{a_1} \wedge \cdots \wedge (e^{i_p})_{a_p}  = 
%	\sum_{s=1}^{p} (-)^{1+s} \delta_{j_1}^{i_s} \cdot
%	\delta_{j_2 \cdots  j_{p}}^{i_1 \cdots\hat{i}_s \cdots i_{p}} 
%	(e_\kappa)^{a_1} (e^{j_1})_{a_1}  \otimes	\cdots \otimes (e^{j_p})_{a_p} \\
%	=& \sum_{s=1}^{p} (-)^{1+s} \delta_{\kappa}^{i_s} \cdot
%	\delta_{j_2 \cdots  j_{p}}^{i_1 \cdots\hat{i}_s \cdots i_{p}} 
%	(e^{j_2})_{a_2}  \otimes	\cdots \otimes (e^{j_{p}})_{a_{p}} \\
%	=& \sum_{s=1}^{p} (-)^{1+s} \delta_{\kappa}^{i_s} \cdot
%	(e^{i_1})_{a_2} \wedge\cdots \wedge \widehat{e^{i_s}}
%	\wedge \cdots \wedge (e^{i_p})_{a_{p}} .
%\end{align*}

\begin{exercise}
	补全例\ref{chdf:exam_eiapk}中未给出的证明.
\end{exercise}



%%%%%%%%%%%%%%%%%%%%%%%%%%%%%%%%%%%%%%%%%%%%%%%%%%%%%%%%%%%%%%%%%%%%%%%%%%%%%%
\index[physwords]{外微分“${\rm d}$”}
\section{外微分}\label{chdf:sec_exterior-diff}
本节介绍微分型式场上的重要运算——外微分.
\begin{theorem}\label{chdf:thm_exterior-differential}
    设$M$是$m$维光滑流形,则存在唯一的映射${\rm d}: A(M)\to A(M)$,
    使得${\rm d}\bigl(A^r(M)\bigr)\subset A^{r+1}(M)$,
    并且满足如下条件:

    {\bfseries (1)} $\mathbb{R}$-线性性:$\forall \alpha, \beta \in A^r(M), \ \forall \lambda \in \mathbb{R}$,有
      ${\rm d}(\alpha+ \lambda \cdot \beta) = {\rm d} \alpha+ \lambda \cdot {\rm d} \beta$;

    {\bfseries (2)} 准Leibitz律:$\forall \alpha \in A^{r}(M)$,有
      ${\rm d}(\alpha \wedge \beta) = {\rm d} \alpha \wedge \beta + (-)^{r} \alpha \wedge {\rm d} \beta$;

    {\bfseries (3)} $\forall f \in A^0(M) \equiv C^\infty(M)$,${\rm d} f$是$f$的普通微分;

    {\bfseries (4)} $\forall \omega \in A^r(M)$,${\rm d}({\rm d} \omega)=0$.

    满足上面条件的映射“${\rm d}$”称为{\heiti 外微分}.
\end{theorem}
我们稍后证明此定理,先叙述一些必要信息.为令公式表述简洁,叙述此定理时未使用抽象指标记号.
这个定理是流形$M$的整体描述,我们需要一个\CJKunderwave{局部性定理},才能得到局部坐标系中的外微分表达式.
\begin{lemma}\label{chdm:thm_d-exiterior-local}
    假设满足上述四个条件的外微分${\rm d}$存在;
    如果$\alpha, \beta \in A(M)$,并且在开子集$U\subset M$上
    有$\alpha|_U = \beta|_U$,则必然有${\rm d}\alpha|_U = {\rm d}\beta|_U$.
\end{lemma}
\begin{proof}
    因并未指定$\alpha,\beta$的阶,故仍旧不使用抽象指标记号.

\index[physwords]{局部性定理!外微分}

%    利用条件(1)(即线性性),令$\lambda =1,\ \alpha=0,\ \beta=0$,则有
%    ${\rm d}(0+0)={\rm d}(0)+{\rm d}(0) {\color{red}\Rightarrow}
%    {\rm d}(0)=2{\rm d}(0){\color{red}\Rightarrow} {\rm d}(0)=0$;
%    这说明映射${\rm d}$将零微分型式场映射为零标量场.

    由条件(1)可知,引理等价于证明${\rm d}(\alpha-\beta)|_U = 0$;
    因此只需证明:若$\omega \in A(M)$且$\omega|_U=0$,则${\rm d}\omega|_U =0$.

    任取一点$p\in U$,因流形是局部紧致的,所以存在$p$点开邻域$V$使得$\overline{V}$紧致,
    并且有$p \in V \subset \overline{V} \subset U$;
    令$h\in C^\infty(M)$是满足命题\ref{chdm:thm_exp1UVM}中条件的标量场;
    则有$h\cdot \omega \in A(M)$且$ h\cdot \omega =0$(这两式是针对整个流形$M$而言).
    将外微分作用在此式上,并利用准Leibnitz律(条件(2))和条件(3),
    有(需注意:截止到此处,外微分${\rm d}$只能作用在流形$M$上的型式场,
    不能作用在局部邻域$U$上的型式场,即${\rm d}_U (\omega|_U)$意义不明)
    \begin{equation*}
        0={\rm d}\bigl(h\cdot \omega\bigr)=( {\rm d}h ) \cdot \omega + h \cdot {\rm d}\omega.
    \end{equation*}
    因$h$限制在$V$上是恒为$1$的,而$\omega|_V=0$;所以将上式限制在$V$上取值时,
    可以得到$({\rm d}\omega)|_V=0$.
    又因$p$点取值的任意性,所以必然可得$({\rm d}\omega)|_U=0$;也就证明了引理.
\end{proof}

%有了局部性定理,原来只作用在流形$M$上的型式场才有意义的外微分算符${\rm d}$,现在作用在局部邻域$U$的型式场也有意义了.

关于局部性定理,请先参考切矢量场的评注\ref{chdm:rmk_local}.
$\forall p\in U$,选择点$p$的一个足够小邻域$V$,使得$V\subset \overline{V} \subset U$;
取$\omega \in A(U)$,根据命题\ref{chdm:thm_Flocal-equiv-Fglobal}可以将$\omega$延拓至
整个流形$M$上的$\hat{\omega}\in A(M)$,并且有$\omega|_V= \hat{\omega}|_V$.
根据局部性定理,可将${\rm d}\omega$在$p\in U$点的局部值定义为
\begin{equation}\label{chdf:eqn_daflg}
    ({\rm d}\omega )(p) \overset{def}{=} ({\rm d}\hat{\omega} )(p) .
\end{equation}
定义号“$\overset{def}{=}$”右端是$A(M)$中的型式场,本来就有意义.
需要注意的是,这个定义与开集$V$选取无关;
可以将$V$换成任意其它开集$V'$,只要满足$V' \subset \overline{V'} \subset U$即可;
证明过程与上面完全相同.
也和$\hat{\omega}$的选取无关,可以将$\hat{\omega}$换成任意$\tilde{\omega}$,
只需满足$\omega|_V=\hat{\omega}|_V=\tilde{\omega}|_V$即可.
因此,如果${\rm d}|_M$是流形$M$上的外微分算子,那么通过式\eqref{chdf:eqn_daflg}可以定义出
局部开集$U$中的外微分算子${\rm d}|_U$,而且${\rm d}|_U$同样具有
定理\ref{chdf:thm_exterior-differential}中所陈述的四条性质.

下面开始证明定理\ref{chdf:thm_exterior-differential};先解决外微分的存在性问题,
自然是寻找符合定理中四个条件的外微分定义.
为此,先回顾一下数学分析中的(一元)二次普通微分${\rm d}({\rm d}x)=0$;
因$x$是自变量,所以${\rm d}x$是不依赖于$x$的任意数,那么对其再次求普通微分时自然为零.
多元微分学中,不同自变量$x^\mu$是相互独立的(例如$x^1$独立于$x^2$);
那么,${\rm d}x^\mu$是不依赖于$x^\mu$以及其它$x^\nu$($\mu \neq \nu$)的任意数,
自然也有${\rm d}({\rm d}x^\mu)=0$.

虽然${\rm d}({\rm d}x)=0$,但标量函数$f(x)$的两次普通微分一般是不为零的,
有${\rm d}^2f=f''(x) {\rm d}x^2$;可见普通微分算符不满足
条件(4)(但满足条件(3))不能当成外微分算符定义.其实略加修改,便可满足
条件(4);考虑多元标量函数场$f$,把外积“$\wedge$”引入到${\rm d}x\ {\rm d}x$之间:
\begin{align}
    {\rm d}_{c}({\rm d}_{a}f)&\equiv {\rm d}_{c}\left(\frac{\partial f}{\partial x^\mu} ({\rm d}x^{\mu})_{a}\right)
    \overset{def}{=}\frac{\partial^2 f}{\partial x^\mu \partial x^\sigma}
    ({\rm d}x^{\sigma})_{c}\wedge ({\rm d}x^{\mu})_{a} +
    \frac{\partial f}{\partial x^\mu} {\rm d}_{c}({\rm d}x^{\mu})_{a} \notag \\
    &\xlongequal{{\rm d}\circ{\rm d}x=0}
    \frac{\partial^2 f}{\partial x^\mu \partial x^\sigma}
    ({\rm d}x^{\sigma})_{c}\wedge ({\rm d}x^{\mu})_{a} = 0. \label{chdf:eqn_ddf=0}
\end{align}
上式倒数第二步用的是自变量二次普通微分恒为零(${\rm d}({\rm d}x)=0$)
{\footnote{条件(4)是自变量$x$两次普通微分恒为零(${\rm d}({\rm d}x)=0$)的推广;
    推广成任何全反对称协变张量场的两次外微分皆为零.
    把作为自变量的坐标的两次外微分定义成两次普通微分,
即${\rm d}_{c}({\rm d}_{a}x^\mu) \overset{def}{=} {\rm d}({\rm d}x^\mu) =0$,
是为了消除可能存在的循环定义.}};
最后一步用到了标量函数对偏导数具有对称性,而外积有反对称性,求和为零(展开即可验证);
对于标量函数场,这样定义的算符满足条件(4).我们将上述定义推广到
一般情形;设流形$M$有局部坐标系$(U;x^i)$,
微分型式场$\omega$有局部表达式\eqref{chdf:eqn_omega-expand-on-bases},
则可定义它的外微分局部表达式为:
\begin{equation}\label{chdf:eqn_exterior-differential}
    ({\rm d}\omega_{b a_{1}\cdots a_{r}})|_U \overset{def}{=}
    \frac{1}{r!} {\rm d}_{b}(\omega_{\mu_{1}\cdots \mu_{r}}) \wedge
    ({\rm d}x^{\mu_1})_{a_{1}} \wedge\cdots\wedge ({\rm d}x^{\mu_r})_{a_{r}}
\end{equation}
易见,这个局部定义满足第(1){\footnote{当条件(1)中的$\alpha$和$\beta$不是
同阶微分型式场时,条件(1)是定义,是无法证明的(至少笔者不知道如何证明)!}}、(3)条.
%定义\eqref{chdf:eqn_exterior-differential}中不能(也没必要)有
%对坐标$x^\mu$的两次(或更高阶)微分(${\rm d}{\rm d}x=0$!)!

\index[physwords]{外微分“${\rm d}$”!定义式}

首先,验证式\eqref{chdf:eqn_exterior-differential}满足条件(4);
为验证此条件,我们计算一般外型式场的两次外微分
(使公式简洁,下面诸式皆省略下标“$|_U$”).
\begin{align*}
    & {\rm d}_{c}{\rm d}_{b}(\alpha_{a_1\cdots a_r})
    = \frac{1}{r!} {\rm d}_{c}\left(\frac{\partial \alpha_{\mu_{1}\cdots \mu_{r}}}{\partial x^\rho}
    ({\rm d}x^\rho)_{b} \wedge
    ({\rm d}x^{\mu_1})_{a_{1}} \wedge\cdots\wedge ({\rm d}x^{\mu_r})_{a_{r}}  \right) \\
    =& \frac{1}{r!(r+1)!}
    \frac{\partial^2 \alpha_{\mu_{1}\cdots \mu_{r}}}{\partial x^\rho \partial x^\sigma}
    ({\rm d}x^\sigma)_{c} \wedge ({\rm d}x^\rho)_{b} \wedge
    ({\rm d}x^{\mu_1})_{a_{1}} \wedge\cdots\wedge ({\rm d}x^{\mu_r})_{a_{r}}  =0.
\end{align*}
与\eqref{chdf:eqn_ddf=0}同样的理由,上式最后一步为零;
这样便验证了定义\eqref{chdf:eqn_exterior-differential}符合条件(4).
%因局部坐标$x^{\mu}\in C^\infty(U)$,是标量场,由式\eqref{chdf:eqn_ddf=0}可知其值为零,即${\rm d}_{c}({\rm d}_{a}x^\mu) = 0$;

其次,验证式\eqref{chdf:eqn_exterior-differential}满足第(2)条
\setlength{\mathindent}{0em}
\begin{align*}
    & {\rm d}_{c}(\alpha_{a_1\cdots a_r} \wedge \beta_{b_1\cdots b_s}) \\
   =& \frac{1}{r! s!} {\rm d}_{c}\left(\alpha_{\mu_{1}\cdots \mu_{r}} \beta_{\sigma_1\cdots \sigma_q}
      ({\rm d}x^{\mu_1})_{a_{1}} \wedge\cdots\wedge ({\rm d}x^{\mu_r})_{a_{r}}  \wedge
      ({\rm d}x^{\sigma_1})_{b_{1}} \wedge\cdots\wedge ({\rm d}x^{\sigma_s})_{b_{s}}  \right) \\
   =& \frac{1}{r! s!} \left( {\rm d}_{c}(\alpha_{\mu_{1}\cdots \mu_{r}}) \beta_{\sigma_1\cdots \sigma_q}
     +\alpha_{\mu_{1}\cdots \mu_{r}} {\rm d}_{c}(\beta_{\sigma_1\cdots \sigma_q})\right) \\
      &\wedge \bigl( ({\rm d}x^{\mu_1})_{a_{1}} \wedge\cdots\wedge ({\rm d}x^{\mu_r})_{a_{r}}  \wedge
      ({\rm d}x^{\sigma_1})_{b_{1}} \wedge\cdots\wedge ({\rm d}x^{\sigma_s})_{b_{s}}  \bigr) \\
   =& \frac{1}{r! s!}  {\rm d}_{c}(\alpha_{\mu_{1}\cdots \mu_{r}}) \beta_{\sigma_1\cdots \sigma_q}
      \wedge \bigl( ({\rm d}x^{\mu_1})_{a_{1}} \wedge\cdots\wedge ({\rm d}x^{\mu_r})_{a_{r}}  \wedge
      ({\rm d}x^{\sigma_1})_{b_{1}} \wedge\cdots\wedge ({\rm d}x^{\sigma_s})_{b_{s}}  \bigr) \\
     &+ \frac{1}{r! s!} \alpha_{\mu_{1}\cdots \mu_{r}} {\rm d}_{c}(\beta_{\sigma_1\cdots \sigma_q})
      \wedge \bigl( ({\rm d}x^{\mu_1})_{a_{1}} \wedge\cdots\wedge ({\rm d}x^{\mu_r})_{a_{r}}  \wedge
      ({\rm d}x^{\sigma_1})_{b_{1}} \wedge\cdots\wedge ({\rm d}x^{\sigma_s})_{b_{s}}  \bigr) \\
    =& \frac{1}{r!}  {\rm d}_{c}(\alpha_{\mu_{1}\cdots \mu_{r}})
      \wedge  ({\rm d}x^{\mu_1})_{a_{1}} \wedge\cdots\wedge ({\rm d}x^{\mu_r})_{a_{r}}  \wedge
      \frac{1}{s!} \beta_{\sigma_1\cdots \sigma_q}
      ({\rm d}x^{\sigma_1})_{b_{1}} \wedge\cdots\wedge ({\rm d}x^{\sigma_s})_{b_{s}}  + \\
      & (-)^r \frac{1}{r!} \alpha_{\mu_{1}\cdots \mu_{r}}
       ({\rm d}x^{\mu_1})_{c} \wedge\cdots\wedge ({\rm d}x^{\mu_r})_{a_{r-1}}  \wedge
      \frac{1}{s!} {\rm d}_{a_r}(\beta_{\sigma_1\cdots \sigma_q})\wedge
      ({\rm d}x^{\sigma_1})_{b_{1}} \wedge\cdots\wedge ({\rm d}x^{\sigma_s})_{b_{s}}.
\end{align*}
\setlength{\mathindent}{2em}
上式最后一个等号用到了外积的反对称性,产生了$r$个负号;最终,上面推导可表示为
\begin{equation}\label{chdf:eqn_dabddatbpadb}
    {\rm d}_{c}(\alpha_{a_1\cdots a_r} \wedge \beta_{b_1\cdots b_s}) =
    ( {\rm d}_{c}\alpha_{a_1\cdots a_r}) \wedge \beta_{b_1\cdots b_s} + (-)^r
    \alpha_{ca_1\cdots a_{r-1}} \wedge {\rm d}_{a_r}\beta_{b_1\cdots b_s} .
\end{equation}
式\eqref{chdf:eqn_exterior-differential}给出的定义满足四个条件,这便证明了外微分算符的存在性.
将$\alpha$和$\beta$取为一次微分型式场,由式\eqref{chdf:eqn_dabddatbpadb}可得到一个常用公式
\begin{equation}\label{chdf:eqn_dwm}
    {\rm d}_c (\alpha_a \wedge \beta_b) =  ({\rm d}\alpha)_{ca} \wedge \beta_b
    - \alpha_c \wedge ({\rm d}\beta)_{ab} .
\end{equation}


%将公式\eqref{chdf:eqn_daflg}前后论述反过来便可从局部定义整体;
%首先,将$\omega \in A(U)$延拓至$\hat{\omega} \in A(M)$,并
%保持$\omega|_U= \hat{\omega}|_U$;那么通过式\eqref{chdf:eqn_daflg}(从右到左定义)
%使得只在局部$U$上有意义的式\eqref{chdf:eqn_exterior-differential}在整个流形$M$上都有意义.
虽然解决了外微分算子的局部、整体存在性问题(并且局部、整体定义是相容的),但
仍需证明式\eqref{chdf:eqn_exterior-differential}中的定义是唯一的,
以及与局部坐标的选取无关.

先证明唯一性.
假设另有一个算符$\bar{\rm d}$同样满足定理\ref{chdf:thm_exterior-differential}中的四个条件;
需要注意,标量函数场$f$的普通微分是唯一确定的(这是数学分析中的结论),由条件(3)可
知$\bar{\rm d} f = {\rm d}f$,也就是两个算符作用在标量场上是相等的.
由于微分型式场$\omega_{a_{1}\cdots a_{r}}$的分量$\omega_{\mu_{1}\cdots \mu_{r}}$及
坐标$x^{\mu}$都是标量函数场,即它们属于$C^\infty(U)$;因此,
从式\eqref{chdf:eqn_exterior-differential}可以
看到$\bar{\rm d}\omega_{b a_{1}\cdots a_{r}} = {\rm d}\omega_{b a_{1}\cdots a_{r}}$,
也就说明了外微分算符的唯一性!

再证,定义\eqref{chdf:eqn_exterior-differential}与局部坐标选取无关.
设有开子集$(W;y^i)$,并且$U\cap W \neq \varnothing$,那么
微分型式场$\omega$可有两组相等的表达式:
\begin{align}
    \left. \omega_{a_{1}\cdots a_{r}} \right|_{U\cap W}
    &= \frac{1}{r!} \omega_{\mu_{1}\cdots \mu_{r}}(x)
    ({\rm d}x^{\mu_1})_{a_{1}} \wedge\cdots\wedge ({\rm d}x^{\mu_r})_{a_{r}} ,
      \quad  \label{chdf:eqn_omega-UU} \\
    &= \frac{1}{r!} \tilde{\omega}_{\mu_{1}\cdots \mu_{r}}(y)
    ({\rm d}y^{\mu_1})_{a_{1}} \wedge\cdots\wedge ({\rm d}y^{\mu_r})_{a_{r}} .
      \quad  \label{chdf:eqn_omega-WW}
\end{align}
下面推导过程中,注意利用关系式\eqref{chdm:eqn_xy-transform}和\eqref{chdm:eqn_xy-cov-transform},
以及式\eqref{chdm:eqn_tensor-component-trans};不再单独提示.
对式\eqref{chdf:eqn_omega-WW}取外微分,有
\begin{align*}
    {\rm d}_b ( \omega_{a_{1}\cdots a_{r}}) =&
       \frac{1}{r!} {\rm d}_b (\tilde{\omega}_{\mu_{1}\cdots \mu_{r}}) \wedge
      ({\rm d}y^{\mu_1})_{a_{1}} \wedge\cdots\wedge ({\rm d}y^{\mu_r})_{a_{r}}  \\
    =& \frac{1}{r!} {\rm d}_b \left( {\omega}_{j_1\cdots j_r}
    \frac{\partial x^{j_1}}{\partial y^{\mu_1}} \cdots
    \frac{\partial x^{j_r}}{\partial y^{\mu_r}} \right) \wedge
    ({\rm d}y^{\mu_1})_{a_{1}} \wedge\cdots\wedge ({\rm d}y^{\mu_r})_{a_{r}}  \\
    =& \frac{1}{r!} \biggl( {\rm d}_b ({\omega}_{j_1\cdots j_r} )
     \times \frac{\partial x^{j_1}}{\partial y^{\mu_1}} \cdots
    \frac{\partial x^{j_r}}{\partial y^{\mu_r}}
    +  {\omega}_{j_1\cdots j_r}
    \frac{\partial^2 x^{j_1}}{\partial y^{\mu_1} \partial y^{\sigma} } \times\\
    &({\rm d}y^{\sigma})_{b}
    \frac{\partial x^{j_2}}{\partial y^{\mu_2}} \cdots
    \frac{\partial x^{j_r}}{\partial y^{\mu_r}} + \cdots  \biggr) \wedge
    ({\rm d}y^{\mu_1})_{a_{1}} \wedge\cdots\wedge ({\rm d}y^{\mu_r})_{a_{r}}  \\
    =& \frac{1}{r!} \frac{\partial x^{j_1}}{\partial y^{\mu_1}} \cdots
    \frac{\partial x^{j_r}}{\partial y^{\mu_r}}  \times
     {\rm d}_b ({\omega}_{j_1\cdots j_r} ) \wedge
    ({\rm d}y^{\mu_1})_{a_{1}} \wedge\cdots\wedge ({\rm d}y^{\mu_r})_{a_{r}}  \\
    =& \frac{1}{r!}  {\rm d}_b (\omega_{j_{1}\cdots j_{r}})\wedge
    ({\rm d}x^{j_1})_{a_{1}} \wedge\cdots\wedge ({\rm d}x^{j_r})_{a_{r}} .
\end{align*}
上面推导中用到了坐标$x^j$(标量场)二阶偏导是对称的,且外积“$\wedge$”是反对称的,
所以对坐标的二次偏导数求和之后都是零.最后一个等号显示外微分运算与坐标选取无关,具有形式不变性!
至此,定理\ref{chdf:thm_exterior-differential}证毕. \qed
\begin{remark}
    此定理表明,在光滑流形上,通过外微分运算能够把一个光滑{\kaishu 全反对称协变}张量场变成
    {\kaishu 高一阶}的光滑反对称协变张量场.

    若对一般的切矢量场,或混合张量场,或者非全反对称(只有部分指标反对称或没有任何对称性)的协变张量场
    进行{\kaishu 微分}运算,进而得到协变阶数增加一阶的光滑张量场;
    仅有微分结构(即相容的坐标图册)是不够的,还需附加上联络或者度规结构.
    但{\kaishu 全反对称协变}张量场不需要这些结构便能进行{\kaishu 微分}运算.
\end{remark}

下面给出外微分运算的一些性质.


\begin{theorem}
对于一次微分型式场$\omega_a$和任意切矢量场$X^a,Y^b$,有如下重要公式:
\begin{equation}\label{chdf:eqn_d1form-value}
    ({\rm d}_a \omega_b) X^a Y^b = X(\omega_b Y^b)-Y(\omega_b X^b) -\omega_b [X,Y]^b .
\end{equation}
\end{theorem}
\begin{proof}
    因式\eqref{chdf:eqn_d1form-value}两边对于$\omega_a$是线性的,故不妨假设它是单项式,
       即$\omega_a = g {\rm d}_a f$,其中$f,g \in C^\infty(M)$;
    因此,${\rm d}_a \omega_b = {\rm d}_a g \wedge {\rm d}_b f$.
    式\eqref{chdf:eqn_d1form-value}等号左端为
    \setlength{\mathindent}{0em}
    \begin{equation*}
        LHS = ({\rm d}_a g \wedge {\rm d}_b f)X^a Y^b
        = ({\rm d}_a g) {\rm d}_b f  X^a Y^b -({\rm d}_b g) {\rm d}_a f X^a Y^b
        = X(g)  Y(f) - X(f)  Y(g) .
    \end{equation*}\setlength{\mathindent}{2em}
    再计算式\eqref{chdf:eqn_d1form-value}等号右端
    \begin{align*}
        RHS &= X(g {\rm d}_b f  Y^b)-Y(g {\rm d}_b f X^b) - g {\rm d}_b f  [X,Y]^b \\
        &= X\bigl(g  Y(f)\bigr)-Y\bigl(g X(f)\bigr) - g  [X,Y](f)
        = X(g) \cdot Y(f) - X(f) \cdot Y(g) .
    \end{align*}
    综合以上两式,定理得证.
    更高次微分型式场的相关公式表达式十分复杂,需要更多的辅助切矢量场;
    在引入联络后有较为简洁的表达式,见\S \ref{chccr:sec_exD-Nalba}.
\end{proof}

之前,我们证明了公式\eqref{chdm:eqn_dfphi}
(即${\phi ^*} ({\rm d} f)_a={\rm d}_a ({\phi ^*}f)={\rm d}_a (f \circ \phi)$);
此公式说明外微分与拉回映射可交换,这个结论是以推广至一般情形.
\begin{theorem}\label{chdf:thm_Dphis-phisD}
    设光滑流形$M$和$N$的维数分别是$m$和$n$,两者间存在光滑映射$\phi:M\to N$,
    其诱导映射是$\phi^{*}: A^r(N)\to A^r(M)$,其中$0\leqslant r < \min(m,n)$.
    则$\forall \omega_{a_{1}\cdots a_{r}} \in A^r(N)$有
    ${\phi ^*} ({\rm d}_b \omega_{a_{1}\cdots a_{r}})={\rm d}_b ({\phi ^*}\omega_{a_{1}\cdots a_{r}})$.
\end{theorem}
\begin{proof}
    定理中对微分型式场阶数的要求保证其外微分不为零.
    外微分有局部性,其局部坐标表达式为式\eqref{chdf:eqn_exterior-differential},由于这个式子
    具有坐标变换不变性,可以看成整体上皆有意义的定义式.首先将其看成流形$N$上的外微分
    (有局部坐标系$(V;y)$),再将此式拉回到流形$M$上(有局部坐标系$(U;x)$,且$\phi(U)\subset V$).
    \setlength{\mathindent}{0em}
    \begin{align*}
        {\phi ^*}({\rm d}\omega_{b a_{1}\cdots a_{r}}){=}& {\phi ^*} \left(
        \frac{1}{r!} \frac{\partial \omega_{\mu_{1}\cdots \mu_{r}} }{\partial y^\rho}
        ({\rm d}y^\rho)_{b} \wedge ({\rm d}y^{\mu_1})_{a_{1}} \wedge\cdots\wedge ({\rm d}y^{\mu_r})_{a_{r}} \right) \\
        \xlongequal{\ref{chdf:eqn_fsalpha}} & \frac{1}{r!}
        \frac{\partial \omega_{\mu_{1}\cdots \mu_{r}}\circ \phi }{\partial y^\rho}
        \cdot \frac{\partial y^{\rho} }{\partial x^{j_0} }
        \frac{\partial y^{\mu_1} }{\partial x^{j_1} } \cdots
        \frac{\partial y^{\mu_r} }{\partial x^{j_r} } \cdot ({\rm d}x^{j_0})_{b}\wedge
        ({\rm d}x^{j_1})_{a_1} \wedge \cdots \wedge ({\rm d}x^{j_r})_{a_r} \\
        =& \frac{1}{r!} \frac{\partial \omega_{\mu_{1}\cdots \mu_{r}}\circ \phi }{\partial x^{j_0}}
        \cdot \frac{\partial y^{\mu_1} }{\partial x^{j_1} } \cdots
        \frac{\partial y^{\mu_r} }{\partial x^{j_r} } \cdot ({\rm d}x^{j_0})_{b}\wedge
        ({\rm d}x^{j_1})_{a_1} \wedge \cdots \wedge ({\rm d}x^{j_r})_{a_r} .
    \end{align*}\setlength{\mathindent}{2em}
    再对式\eqref{chdf:eqn_fsalpha}求外微分,有
    \begin{align*}
         {\rm d}_b (\phi^{*}\omega_{a_1\cdots a_r}) = & {\rm d}_b \left(
        \frac{1}{r!} (\omega_{j_1\cdots j_r} \circ \phi) \cdot
        \frac{\partial y^{j_1} }{\partial x^{\pi_1} } \cdots
        \frac{\partial y^{j_r} }{\partial x^{\pi_r} } \cdot
        ({\rm d}x^{\pi_1})_{a_1} \wedge \cdots \wedge ({\rm d}x^{\pi_r})_{a_r} \right) \\
        = & \frac{1}{r!} \frac{\partial \omega_{j_{1}\cdots j_{r}}\circ \phi }{\partial x^{\pi_0}}
        \cdot  \frac{\partial y^{j_1} }{\partial x^{\pi_1} } \cdots
        \frac{\partial y^{j_r} }{\partial x^{\pi_r} } \cdot ({\rm d}x^{\pi_0})_{b}
        \wedge ({\rm d}x^{\pi_1})_{a_1} \wedge \cdots \wedge ({\rm d}x^{\pi_r})_{a_r} .
    \end{align*}
    上式最后一步省略了$\frac{\partial^2 y^{j} }{\partial x^{\pi} \partial x^{\sigma}}
     ({\rm d}x^{\pi})_{a}\wedge ({\rm d}x^{\sigma})_{b}=0$(读者应该能看出此式为何为零吧?)这一步.
    综合以上两式可见,外微分与拉回映射可交换.
\end{proof}

\begin{definition}
    对于$\omega \in A^r(M)$,若${\rm d}\omega =0$,则称$\omega$是{\heiti 闭的}(closed).
    若存在$\sigma \in A^{r-1}(M)$使得$\omega = {\rm d}\sigma$,则称$\omega$是{\heiti 恰当的}(exact).
\end{definition}

很明显,如果微分型式场$\omega$是恰当的,那么它必然是闭的;(整体上来说)反之未必.
但我们有如下定理说明{\kaishu 局部上}是闭的,则它必然是{\kaishu 局部上}恰当的.

\index[physwords]{Poincar\'{e}引理}

\begin{theorem}\label{chdf:thm_poincare-lemma}
     (Poincar\'{e}引理)设$U=B_0(a)$是$\mathbb{R}^m$中以原点为中心、$a$为半径的
     球形开邻域(可弱化为星形开邻域);设$\omega \in A^r(U)$且${\rm d}\omega =0$,
     则存在$\sigma \in A^{r-1}(U)$使得$\omega = {\rm d}\sigma$.
\end{theorem}
\begin{proof}
    因$U$是$\mathbb{R}^m$中开集,可假定它由一个坐标域覆盖;设$\omega$的表达式为:
    \begin{equation}
         \omega_{a_{1}\cdots a_{r}} = \sum_{\mu_{1}<\cdots<\mu_{r}} \omega_{\mu_{1}\cdots \mu_{r}}
          ({\rm d}x^{\mu_1})_{a_{1}} \wedge\cdots\wedge ({\rm d}x^{\mu_r})_{a_{r}} .
    \end{equation}
    证明过程需利用一个具有技巧性的积分,定义映射$I:A^r(U)\to A^{r-1}(U)$如下
    \begin{equation}\label{chdf:eqn_I-omega}
    \begin{aligned}
        I_r(\omega )_{a_{1}\cdots a_{r-1}} \overset{def}{=}& \sum_{\mu_{1}<\cdots<\mu_{r}}
         \left( \int_{0}^{1}t^{r-1} \omega_{\mu_{1}\cdots \mu_{r}}(tx){\rm d}t \right) \times
         \sum_{i=1}^{r}  (-)^{i-1} x^{\mu_i} \times \\
         & \times ({\rm d}x^{\mu_1})_{a_{1}} \wedge\cdots\wedge
         (\widehat{{\rm d}x^{\mu_i}}) \wedge \cdots \wedge ({\rm d}x^{\mu_r})_{a_{r-1}} .
    \end{aligned}
    \end{equation}
    其中脱字符$\widehat{}$表示没有这个基矢$(\widehat{{\rm d}x^{\mu_i}})$;
    脱字符下的基矢不再占用抽象指标记号.
    需要证明此映射有如下性质:
    \begin{equation}\label{chdm:eqn_dIpId}
        {\rm d}_{b}\bigl( I_r (\omega)_{a_{1}\cdots a_{r-1}} \bigr)
        + I_{r+1}({\rm d}\omega )_{ba_{1}\cdots a_{r-1}} = \omega_{ba_{1}\cdots a_{r-1}} .
    \end{equation}
    直接计算便可证明;先计算
\setlength{\mathindent}{0em}
    \begin{small}
    \begin{align*}
        &{\rm d}_{b}\bigl( I_r (\omega)_{a_{1}\cdots a_{r-1}} \bigr) =  \sum_{\mu_{1}<\cdots<\mu_{r}}
        \left( \int_{0}^{1}t^{r-1} \omega_{\mu_{1}\cdots \mu_{r}}(tx){\rm d}t \right)\cdot r\cdot
        ({\rm d}x^{\mu_1})_{b}\wedge ({\rm d}x^{\mu_2})_{a_{1}} \wedge\cdots\wedge ({\rm d}x^{\mu_r})_{a_{r-1}}  + \\
        &\sum_{\mu_{0}<\cdots<\mu_{r}}
        \left( \int_{0}^{1}t^{r} \frac{\partial \omega_{\mu_{1}\cdots \mu_{r}}(tx)}
        {\partial x^{\mu_{0}}} {\rm d}t \right)  \sum_{i=1}^{r} (-)^{i-1}
        x^{\mu_i}({\rm d}x^{\mu_{0}})_{b}\wedge  ({\rm d}x^{\mu_1})_{a_{1}} \wedge\cdots\wedge
        (\widehat{{\rm d}x^{\mu_i}}) \wedge \cdots \wedge ({\rm d}x^{\mu_r})_{a_{r-1}}
    \end{align*}
    \end{small}\setlength{\mathindent}{2em}
    再借用式\eqref{chdf:eqn_exterior-differential}可以得到
    \begin{align*}
        &I_{r+1}({\rm d}\omega )_{ba_{1}\cdots a_{r-1}} = \sum_{\mu_{0}<\cdots<\mu_{r}}
         \left( \int_{0}^{1}t^{r}  \frac{\partial \omega_{\mu_{1}\cdots \mu_{r}}(tx) }{\partial x^{\mu_{0}}}
         {\rm d}t \right) \times   \sum_{i=0}^{r}  (-)^{i} x^{\mu_i} \times \\
        &\qquad ({\rm d}x^{\mu_{0}})_{b} \wedge
        ({\rm d}x^{\mu_1})_{a_{1}} \wedge\cdots \wedge(\widehat{{\rm d}x^{\mu_i}})\wedge \cdots
        \wedge ({\rm d}x^{\mu_r})_{a_{r-1}}
    \end{align*}
    上两式求和,有
    \begin{align*}
        &{\rm d}_{b}\bigl( I_r (\omega)_{a_{1}\cdots a_{r-1}} \bigr)+I_{r+1}({\rm d}\omega )_{ba_{1}\cdots a_{r-1}}\\
        =& \sum_{\mu_{1}<\cdots<\mu_{r}}
        \left( \int_{0}^{1}t^{r-1} \omega_{\mu_{1}\cdots \mu_{r}}(tx){\rm d}t \right)\cdot r\cdot
        ({\rm d}x^{\mu_1})_{b}\wedge ({\rm d}x^{\mu_2})_{a_{1}} \wedge\cdots\wedge ({\rm d}x^{\mu_r})_{a_{r-1}}   \\
        & + \sum_{\mu_{1}<\cdots<\mu_{r}}  \left( \int_{0}^{1}t^{r}
        \frac{\partial \omega_{\mu_{1}\cdots \mu_{r}}(tx) }{\partial x^{\mu_{0}}}   {\rm d}t \right)
        \times   x^{\mu_0} ({\rm d}x^{\mu_{1}})_{b} \wedge
        ({\rm d}x^{\mu_2})_{a_{1}} \wedge\cdots  \wedge ({\rm d}x^{\mu_r})_{a_{r-1}} \\
        =& \sum_{\mu_{1}<\cdots<\mu_{r}} \left( \int_{0}^{1}
        \frac{{\rm d} \bigl(t^{r} \omega_{\mu_{1}\cdots \mu_{r}}(tx)\bigr) } {{\rm d}t} {\rm d}t \right)
        ({\rm d}x^{\mu_1})_{b}\wedge ({\rm d}x^{\mu_2})_{a_{1}} \wedge\cdots\wedge ({\rm d}x^{\mu_r})_{a_{r-1}} \\
        =& \sum_{\mu_{1}<\cdots<\mu_{r}} \omega_{\mu_{1}\cdots \mu_{r}}(x)
         ({\rm d}x^{\mu_1})_{b}\wedge ({\rm d}x^{\mu_2})_{a_{1}} \wedge\cdots\wedge
         ({\rm d}x^{\mu_r})_{a_{r-1}} = \omega_{b a_{1} \cdots a_{r-1}}.
    \end{align*}
    这便证明了式\eqref{chdm:eqn_dIpId}.当$\omega$为闭型式时,也就是${\rm d}\omega=0$时,
    由式\eqref{chdm:eqn_dIpId}可得$\omega_{b a_{1} \cdots a_{r-1}}=
    {\rm d}_{b}\bigl( I_r (\omega)_{a_{1}\cdots a_{r-1}} \bigr) $,这说明$\omega$是恰当的;
    其中$\sigma = I_r (\omega)$.
\end{proof}

\begin{remark}\label{chdf:rek_poincare-lemma}
    从Poincar\'{e}引理的证明过程来看(见式\eqref{chdf:eqn_I-omega}),只有微分型式场
    $\omega$的分量$\omega_{\mu_{1}\cdots \mu_{r}}(x)$不为零时才能使得积分后的
    式\eqref{chdf:eqn_I-omega}不为零;如果$\omega$的分量为零,那么定理中的$\sigma$也为零.
\end{remark}










%%%%%%%%%%%%%%%%%%%%%%%%%%%%%%%%%%%%%%%%%%%%%%%%%%%%%%%%%%%%%%%%%%%%%%%%%%%%%%
\section{Frobenius定理}\label{chdf:sec_frobenius}

在\S \ref{chdm:sec_One-Parameter-Transformations-Groups}中,已经指出流形上切矢量场
与其积分曲线(单参数可微变换群)有一一对应关系,这其实是在说单个切矢量场是否可积;
本节将讲述多个切矢量场是否可积的问题,由常微分方程组过渡到偏微分方程组.
我们将省略定理证明过程,可参考文献\parencite[\S 1.4]{cc2001-zh}或\parencite[\S 3.4]{chenwh2001}或其它类似书籍. 
需要事先声明,本节所叙述的定理只在局部成立;关于Frobenius定理的大范围描述
可参阅\parencite[\S 6.6-6.7]{spivak-dif-1}.

设有一阶偏微分方程组
\begin{equation}\label{chdf:eqn_1d-eqs}
    \frac{\partial y^\alpha(x)}{\partial x^i} = f_i^\alpha (x^1,\cdots, x^m;y^1,\cdots,y^n),
    \quad 1 \leqslant \alpha \leqslant n, \ 1 \leqslant i \leqslant m .
\end{equation}
假定$f_i ^\alpha (x^j; y^\beta)$是定义在区域$D=U \times V $上的$C^r$阶连续可微函数,
其中$U$是$\mathbb{R}^m$开凸子集,$V$是$\mathbb{R}^n$开凸子集;并且这两个子集的
边界($\partial U$和$\partial  V$)是$C^{r-1}$阶连续可微的,除了有限个测度为零的不可微子区域.
\begin{theorem}\label{chdf:thm_1d-exist}
    对于任意给定初值$(x^1_0,\cdots, x^m _0;y^1_0,\cdots,y^n_0)\in D$,
    方程组\eqref{chdf:eqn_1d-eqs}在点$(x^1_0,\cdots, x^m _0)\in U$的一个
    邻域$\tilde{U}$内有连续可微解$y^\alpha=y^\alpha(x^1,\cdots, x^m)$,
    并且满足初始条件$y^\alpha(x^1_0,\cdots, x^m _0)=y^\alpha_0$(其中$ 1 \leqslant \alpha \leqslant n$)
    的充分条件是,在区域$D$上下述恒等式成立:
    \begin{equation}\label{chdf:eqn_1d-exist}
        \frac{\partial f_i^\alpha}{\partial x^j} - \frac{\partial f^\alpha_j}{\partial x^i}
        = \sum_{\sigma=1}^{n} \left(\frac{\partial f^\alpha_j}{\partial y^\sigma} f^\sigma_i
          - \frac{\partial f_i^\alpha}{\partial y^\sigma} f^\sigma_j \right),
        \quad  1 \leqslant \alpha \leqslant n, \ 1 \leqslant i,j \leqslant m.
    \end{equation}
    此时简称该方程组完全可积;并且该解是唯一的.
    反之,如果假设方程组\eqref{chdf:eqn_1d-eqs}解存在,那么
    条件\eqref{chdf:eqn_1d-exist}也是其必要条件(即推论)之一.
\end{theorem}
需要提醒读者注意,此定理在叙述中省略了区域$D$的拓扑性质(已在定理前给出),
这些性质也是解存在的充分条件.
只有单独一个条件\eqref{chdf:eqn_1d-exist}无法保证方程组\eqref{chdf:eqn_1d-eqs}解的存在,
仍需拓扑条件、初边值条件等.

\begin{example}\label{chdf:exm_Fij2d}
    令定理\ref{chdf:thm_1d-exist}中的$n=1,m=2$,即方程\eqref{chdf:eqn_1d-eqs}为
    \begin{equation}
        \partial_x u = f(x,y;u),\qquad \partial_y u=g(x,y;u).
    \end{equation}
    注意有两个方程,一个未知量$u$;但方程组是确定的,
    见\S \ref{chmla:sec_diff-linear-dependence}.
    解存在性条件\eqref{chdf:eqn_1d-exist}只有
    一个$\partial_x g -\partial_y f = g \partial_u f-f \partial_u g $,其它全部
    为$0=0$的恒等式;如果假设$f,g$与$u$无关,
    则上述解存在性条件变为$\partial_x g -\partial_y f =0$.\qed
\end{example}

\begin{example}\label{chdf:exm_Fij2d-2}
    将例题\ref{chdf:exm_Fij2d}稍微改变一下表述方式,令
    \begin{equation}
        F_{ij}=\begin{pmatrix}
            0 & -u \\ u &0
        \end{pmatrix} ,\qquad
        J_{i}= \begin{pmatrix}
            g\\ -f
        \end{pmatrix}.
    \end{equation}
    则例\ref{chdf:exm_Fij2d}中方程变为
    \begin{equation}
        \sum_{j}\frac{\partial}{\partial x^j} F_{ji} = J_{i} .
    \end{equation}
    假设$f,g$与$u$无关,则解存在性条件是
    $\sum_{i}\frac{\partial}{\partial x^i}J_i= \partial_x g -\partial_y f =0 $.     \qed
\end{example}

\begin{example}\label{chdf:exm_Fij2d-3}
    令
    \begin{equation}
        F_{ij}=\begin{pmatrix}
            0   & -u_3 &  u_2 \\
            u_3 &  0   & -u_1 \\
            -u_2&  u_1 & 0
        \end{pmatrix} ,\qquad
        J_{i}= \begin{pmatrix}
            J_1 \\  J_2 \\  J_3
        \end{pmatrix}.
    \end{equation}
    其中$J_i$与$u_i$无关.    则有如下方程
    \begin{equation}
        \sum_{j}\frac{\partial}{\partial x^j} F_{ji} = J_{i}
        {\quad \color{red}\Leftrightarrow \quad}
        \nabla\times \boldsymbol{u} = \boldsymbol{J} .
    \end{equation}
    解存在性条件变为
    $\sum_{i}\frac{\partial}{\partial x^i}J_i= \nabla \cdot \boldsymbol{J}=0 $.    \qed
\end{example}

定理\ref{chdf:thm_1d-exist}是Frobenius定理的经典形式;下面我们用切矢量场
以及微分型式场方式再次叙述此定理,为此需要做些准备.


\begin{definition}
    设光滑流形$M$是$m$维的.若对每一点$p\in M$都指定切空间$T_pM$的一个$h$维子空间$L^h(p)\subset T_pM$,
    则$L^h=\bigcup_{p\in M} L^h(p)$构成切丛$TM$的一个子集,称为$M$上的一个$h$维{\heiti 分布}.
    若上述$L^h$只在流形$M$的某开子集$U$上成立,则称$L^h$是$U$上的$h$维{\heiti 分布}.
\end{definition}
流形上每点切空间都与流形同维数,分布的含义便是先在$T_pU$ %(此式及下面诸式中的$U$皆可换成$M$)
找到$h$个线性无关的切矢量$X^a_1(p),\cdots,X^a_h(p)$;然后让点$p$遍历$U$,这样就有$h$个线性无关的
切矢量\CJKunderwave{场}$X^a_1,\cdots,X^a_h\ \in \mathfrak{X}(U)$;
分布便是由这些切矢量场张成(为简单起见,要求这些切矢量场是光滑的),记为
\begin{equation}
    L^h|_U \equiv \text{Span}_{\infty}\{X^a_1,\cdots,X^a_h \},\quad
    {\rm Span}_{\infty} \text{的组合系数是} C^\infty(U) \text{函数}.
\end{equation}
切矢量场能够进行非退化$C^\infty(U)$函数变换(包括$\mathbb{R}$-线性变换) %见\eqref{chdm:eqn_functimesv}式后论述
\begin{equation}
    Y^a_\alpha = c^\beta_\alpha X_\beta ^a, \qquad
    \forall c^\beta_\alpha \in C^\infty(U) {\ \text{且}\ }\det(c^\beta_\alpha)\neq 0.
\end{equation}
这$h$个切矢量$Y^a_\alpha$仍是$\mathbb{R}$-线性无关的;此时,分布$L^h|_U$也可记为
\begin{equation}
    L^h|_U \equiv \text{Span}_{\infty}\{Y^a_1,\cdots,Y^a_h \}.
\end{equation}

如果光滑流形$M$上存在$h$个处处线性无关的光滑切矢量场,那么在$U$上就存在一个$h$维光滑分布.
反过来,$U$上存在一个$h$维光滑分布,那么在$M$上未必存在$h$个处处线性无关的光滑切矢量场.
典型的例子便是二维球面$S^2$上的光滑切矢量场必有奇点,因此$S^2$整体上不可能存在两个
处处线性无关的切矢量场;然而$S^2$的局部上的切矢量场显然是个2维分布.
虽然在整个流形(或者称之为大范围情形)两者并不等价,但在局部开子集$U$上,两者是等价的,
即一个$h$维光滑分布等价于存在$h$个处处线性无关的光滑切矢量场.

\index[physwords]{积分流形}

\begin{definition}\label{chdf:def_solution}
    设$L^h$是定义在开子集$U\subset M$上的光滑分布,如果有包含映射$\imath:N\to U$对每一
    点$p \in N$有$\imath_{*}(T_p N)\subset L^h\bigl(\imath(p)\bigr)$,则
    称$(\imath,N)$是光滑分布$L^h$的一个{\heiti 积分流形}.
    若$\forall q\in U$,都有分布$L^h$的一个$h$维积分流形经过它,则称
    分布$L^h$是{\heiti 完全可积}的.
\end{definition}
由于包含映射一定是嵌入映射,那么必然是单射;同时流形$N$的维数不可能超过$h$.
积分流形只要求$\forall p \in N$可积,完全积分流形要求$\forall q \in U$可积.

当$h=1$时,$L^1$分布便是$U$上的切矢量场;很明显,分布是切矢量场向高维空间的推广;
矢量场的积分曲线正是它作为分布时的积分流形.
定理\ref{chdm:thm_1PDG-ppy}阐述了任意切矢量场(在坐标变换后)都可以表示
成$X^a |_U = (\frac{\partial }{\partial x^1} )^a$,即其积分曲线
是$x^1$-参数曲线,在此条曲线上只有自变量$x^1$是变化的,其它
自变量$x^i(2\leqslant i)$都是常数.一个自然的问题是定理\ref{chdm:thm_1PDG-ppy}能否
推广到高维,即是否存在局部坐标系使得$L^h={\rm Span}_{\infty}\{(\frac{\partial }{\partial x^1} )^a,
\cdots,(\frac{\partial }{\partial x^h} )^a\}$成立?这便是Frobenius定理.

\begin{definition}\label{chdf:def_frobenius-conditions}
    设$L^h$是定义在开子集$U\subset M$上的光滑分布.若$\forall p\in U$,存在$p$点一个
    开邻域$p\in V\subset U$,以及$h$个光滑切矢量场$X_1^a,\cdots,X^a_h\in \mathfrak{X}(V)$使
    得$L^h={\rm Span}_{\infty}\{ X_1^a,\cdots,X^a_h\}$;并且这些光滑切矢量场满足封闭
    条件$[X_\alpha,X_\beta]^a=c_{\alpha\beta}^{\gamma} X_\gamma^a$,
    其中$c_{\alpha\beta}^{\gamma} \in C^\infty(V)$、$1\leqslant \alpha,\beta\leqslant h$,
    则称$L^h$满足{\heiti \bfseries Frobenius条件}.
\end{definition}

\index[physwords]{Frobenius条件}
\index[physwords]{Frobenius定理}

\begin{theorem}\label{chdf:thm_frobenius}
    (Frobenius定理)设$L^h$是开子集$U\subset M$上的光滑分布,若$L^h$满足Frobenius条件,
    则$\forall p\in U$,存在局部坐标系$(V;x^i)$使得$p\in V\subset U$,且
    \begin{equation}\label{chdf:eqn_frobenius-h}
        L^h={\rm Span}_{\infty}\left\{\left(\frac{\partial }{\partial x^1} \right)^a,
        \cdots,\left(\frac{\partial }{\partial x^h} \right)^a \right\}.
    \end{equation}
    $L^h$有如此表述形式,自然是完全可积的.
\end{theorem}
定理\ref{chdf:thm_frobenius}告诉我们,开子集$V$中的坐标
面$\{(x^1,\cdots,x^m) \mid x^{h+1}=const, \cdots, x^{m}=const\}$是
分布$L^h$的$h$维积分流形.
此定理的逆也成立,即
\begin{theorem}\label{chdf:thm_frobenius-converse}
    如果$L^h$完全可积,则$L^h$必然满足Frobenius条件.
\end{theorem}

\section{Frobenius定理的外型式表述}\label{chdf:sec_frobenius-ed}
\S\ref{chdf:sec_frobenius}中是Frobenius定理的经典描述方式,本节用外型式场方式
再次表述此定理.同\S\ref{chdf:sec_frobenius}表述类似,
通过包含映射$\imath$将$n$维子流形$N$嵌入到$m$流形$M$中,且$n < m$;
我们需要用到拉回映射来表述$(\imath,N)$是积分流形这一概念,
因$N$的维数小于$M$的维数,所以从$M$拉回到$N$的一次
外型式场$\imath^*(\omega^i)_a,\  \forall(\omega^i)_a\in\mathfrak{X}^*(M),
\ 1\leqslant i \leqslant m$中必然有些为零.

\index[physwords]{Pfaff方程组}
%\subsection{Pfaff方程与定理表述}
定义在$m$维光滑流形$M$的开子集$U$上的$r$个一次外微分型式场
\begin{equation}\label{chdf:eqn_Pfaff}
    (\omega^\alpha)_a = 0;\qquad {\text{其中}}\ (\omega^\alpha)_a
    \in\mathfrak{X}^*(U), \quad 1\leqslant \alpha \leqslant r.
\end{equation}
称为$U$上的{\bfseries \heiti Pfaff方程组}.内指标标记微分式的个数.
这里的$(\omega^\alpha)_a = 0$不是指一次外微分式为恒零元,恒零元
是指$\forall v^a\in \mathfrak{X}(U)$有$(\omega^\alpha)_a v^a =\left<(\omega^\alpha)_a, v^a\right>=0$.
而Pfaff方程\eqref{chdf:eqn_Pfaff}是指寻找子流形$(\imath,N)$使得
\begin{equation}\label{chdf:eqn_solution-ed}
    \imath^* \bigl((\omega^\alpha)_a \bigr) = 0; \qquad \text{其中包含映射}\ \imath:N\to U .
\end{equation}
或者说$(\omega^\alpha)_a$在子流形$N$上的限制为零.
满足上述条件的映射$\imath$和流形$N$未必存在,其存在性条件就是Frobenius条件.


先看一个例子.若$r=1$,在二维流形上,Pfaff方程是
\begin{equation}
    (\omega)_a = P(x,y)({\rm d}x)_a +Q(x,y)({\rm d}y)_a =0, \qquad P,Q\in C^\infty(U).
\end{equation}
当$Q\neq 0$时,上述方程等价于(直接去掉抽象指标记号)
\begin{equation}
    \frac{{\rm d}y}{{\rm d}x}=-\frac{P(x,y)}{Q(x,y)}.
\end{equation}
这是通常的常微分方程式.它完全可积条件是(可查阅微积分书籍)
\begin{equation}
    \frac{\partial P}{\partial y}=\frac{\partial Q}{\partial x}.
\end{equation}
在$U$上处处成立.



%下面进一步来描述偏微分方程与Pfaff方程组\eqref{chdf:eqn_Pfaff}的关系.
在光滑流形$M$上开子集$U$的一次微分式\eqref{chdf:eqn_Pfaff}在各点的线性无关
个数未必相同,这给研究带来诸多不便.下面,我们只研究$U$上各点线性无关的
一次微分式($(\omega^\alpha)_a$)个数均为相同常数的情形,此常数称为Pfaff方程组的{\heiti 秩}.
我们不妨假设它的秩是$r$.到目前为止,积分流形的定义仍旧模糊,下述命题有助于清晰化此概念.

%下面指出分布$L^h$与Pfaff方程组有如下关系.
\begin{proposition}\label{chdf:thm_dwwd}
    给定$m$维光滑流形$M$,定义在开子集$U\subset M$上的秩为$r$的Pfaff方程组
    在局部上{\kaishu 等价于}$h(=m-r)$维光滑分布$L^h$.
\end{proposition}
\begin{proof}
    需要提醒读者这个命题只在局部成立.我们先从分布导出Pfaff方程组.

%与它们对偶的是$m$个切矢场$(v_i)^a (1\leqslant i\leqslant m)$.
%在两者作内积过程中(即$(v_i)^a(\omega^\alpha)_a$),
%至少有$r$个切矢场$(v_\beta)^a(1\leqslant \beta\leqslant r)$使得
%\begin{equation}
%    (v_\beta)^a(\omega^\alpha)_a= \delta_\beta^\alpha,\qquad 1\leqslant \alpha, \beta\leqslant r .
%\end{equation}
%而剩余的$h=m-r$个切矢场与$(\omega^\alpha)_a$作内积是恒为零的,将它们记为
%\begin{equation}
%    L^h(p)=\left\{(v_\lambda)^a\in T_pU| (\omega^\alpha)_a(v_\lambda)^a=0,\
%    1\leqslant \alpha\leqslant r,\ 1+r\leqslant \lambda\leqslant m \right\}.
%\end{equation}
%这便是Pfaff方程组的几何意义:式\eqref{chdf:eqn_Pfaff}在$U$上给出了
%一个$h=m-r$维光滑分布$L^h$.


假设点$p\in V\subset U$邻域有局部坐标系$(V;x^i)$(进一步缩小开集$U$的
主要目的是:缩小后的$p$点开邻域$V$可用一个坐标域覆盖,同时紧致集$\overline{V}\subset U$),
$h(=m-r)$维光滑分布$L^h$可用$V$上处处线性无关的光滑切矢量场来张成,即
\begin{equation}
    L^h|_V = {\rm Span}_{\infty} \left\{(v_{r+1})^a,\cdots, (v_m)^a \right\}.
\end{equation}
将此$h$个切矢量场延拓至$m$个在$V$上处处线性无关的切矢量场,
假设这些切矢量场的局部展开式为(其中$1\leqslant i\leqslant m$)
\begin{equation}\label{chdf:eqn_tmp-v1}
    (v_i)^a= \sum_{j=1}^{m} v_i^j \left(\frac{\partial}{\partial x^j}\right)^a
    =\sum_{\alpha=1}^{r}v_i^\alpha \left(\frac{\partial}{\partial x^\alpha}\right)^a
    +\sum_{\mu=r+1}^{m}v_i^\mu \left(\frac{\partial}{\partial x^\mu}\right)^a .
\end{equation}
因分布$L^h$要有$h$个处处线性无关的切矢量场,故不妨假设展开系数$\det(v_\lambda^\mu)\neq 0$,
$r< \mu,\lambda \leqslant m$.其逆记为$b_\lambda^{\mu'}$,
即$v^\mu_{\mu'} b_\lambda^{\mu'} =\delta^\mu_\lambda$;
将逆矩阵乘在式\eqref{chdf:eqn_tmp-v1}上,有
\begin{align*}
    (u_\lambda)^a \overset{def}{=}&  \sum_{\mu'=r+1}^{m} b_\lambda^{\mu'} (v_{\mu'})^a
    =\sum_{\alpha=1}^{r} b_\lambda^{\mu'} v_{\mu'}^\alpha \left(\frac{\partial}{\partial x^\alpha}\right)^a
    +\sum_{\mu,{\mu'}=r+1}^{m} b_\lambda^{\mu'} v_{\mu'}^\mu
        \left(\frac{\partial}{\partial x^\mu}\right)^a   \\
    =&-\sum_{\alpha=1}^{r} l_\lambda^{\alpha} \left(\frac{\partial}{\partial x^\alpha}\right)^a
    + \left(\frac{\partial}{\partial x^\lambda}\right)^a ,
    \qquad \text{其中}\ l_\lambda^{\alpha} 
   \equiv - \sum_{\mu'=r+1}^{m}b_\lambda^{\mu'} v_{\mu'}^\alpha .
\end{align*}
显然$l_\lambda^{\alpha} \in C^\infty(U)$,所以仍然有
\begin{equation}
    L^h|_V = {\rm Span}_{\infty} \left\{(u_{r+1})^a,\cdots, (u_m)^a \right\}.
\end{equation}
补上$r$个切基矢$(u_\alpha)^a= \left(\frac{\partial}{\partial x^\alpha}\right)^a$,
其中$1\leqslant \alpha \leqslant r$;这样切标架场$\{ (u_{1})^a,\cdots, (u_m)^a \}$便
构成了$V$上的切基矢场.容易验证与$\{u^a\}$对偶的余切标架场是
\begin{equation}
    (\theta^\alpha)_a =  ({\rm d}x^\alpha)_a + \sum_{\mu=r+1}^{m} l^\alpha_\mu ({\rm d}x^\mu)_a  ,
    \   (\theta^\lambda)_a = ({\rm d}x^\lambda)_a.  
\end{equation}
其中$1 \leqslant \alpha \leqslant r,\ r<\lambda\leqslant m$.
由上式可见,$h$维光滑分布$L^h$可以导出$r(=m-h)$个Pfaff方程
组$(\theta^\alpha)_a =  ({\rm d}x^\alpha)_a + \sum_{\mu=r+1}^{m} l^\alpha_\mu ({\rm d}x^\mu)_a $,
其系数$l^\alpha_\mu$是由分布$L^h$基矢场的展开系数决定的.

反之.假设先给定秩为$r$的Pfaff方程组$(\omega^\alpha)_a$,其中$1 \leqslant \alpha \leqslant r$.
在每一点$p\in V$最多的线性无关微分型式场自然是$m$个,
我们将$r$个一次外微分式延拓到$m$个处处相互独立的外微分式,并在局部坐标系展开,有
\begin{equation}\label{chdf:eqn_tmp-w1}
    (\omega^i)_a= \sum_{k=1}^{m} \omega^i_k ({\rm d}x^k)_a
    =\sum_{\beta=1}^{r} \omega^i_\beta ({\rm d}x^\beta)_a
    +\sum_{\nu=r+1}^{m} \omega^i_\nu ({\rm d}x^\nu)_a ,
    \ 1\leqslant i\leqslant m .
\end{equation}
同理,不妨假设$\det(\omega^\alpha_\beta)\neq 0$,其中$1 \leqslant \alpha,\beta \leqslant r$;
这个方阵自然有逆$b^\gamma_\alpha$,即$\omega^\alpha_\beta b^\gamma_\alpha = \delta^\gamma_\beta$,
其中$1 \leqslant \gamma \leqslant r$.将逆矩阵乘在式\eqref{chdf:eqn_tmp-w1}上,有
\begin{align}
    (\varphi^\gamma)_a \overset{def}{=}& \sum_{\alpha=1}^{r} b^\gamma_\alpha(\omega^\alpha)_a
    =\sum_{\alpha,\beta=1}^{r} b^\gamma_\alpha\omega^\alpha_\beta ({\rm d}x^\beta)_a
      +\sum_{\nu=r+1}^{m} b^\gamma_\alpha\omega^\alpha_\nu ({\rm d}x^\nu)_a \notag \\
    =&  ({\rm d}x^\gamma)_a + \sum_{\nu=r+1}^{m} c^\gamma_\nu ({\rm d}x^\nu)_a ,
    \qquad \text{其中} \  c^\gamma_\nu \equiv b^\gamma_\alpha\omega^\alpha_\nu .
\end{align}
由于乘的是可逆矩阵,所以$(\varphi^\gamma)_a$和$(\omega^\gamma)_a$是等价的.
$(\varphi^\gamma)_a (1 \leqslant \gamma \leqslant r)$在$V$上是处处
线性无关的,我们可以补上$h=(m-r)$个余切矢量
场$(\varphi^\lambda)_a=({\rm d}x^\lambda)_a (r< \lambda \leqslant m)$,
这样$m$个$\{(\varphi^1)_a,\cdots,(\varphi^m)_a \}$便构成了$\mathfrak{X}^*(V)$的基矢场.
容易求出与其对偶的切标架场$\{(X_1)^a,\cdots,(X_m)^a\}$是
(其中$1 \leqslant \alpha \leqslant r$,$r< \lambda \leqslant m$)
\begin{equation}
    (X_\alpha)^a = \left(\frac{\partial }{\partial x^\alpha}\right)^a, \qquad
    (X_\lambda)^a = \left(\frac{\partial }{\partial x^\lambda}\right)^a
    -\sum_{\beta=1}^{r}c^\beta_\lambda  \left(\frac{\partial }{\partial x^\beta}\right)^a .
\end{equation}
上式说明,从秩为$r$的Pfaff方程组$(\omega^\alpha)_a$出发,可以找到一个$h$维光滑分布
\begin{equation}
    L^h|_V = {\rm Span}_{\infty} \left\{(X_{r+1})^a,\cdots, (X_m)^a \right\}.
\end{equation}
其系数$c^\beta_\lambda$是由Pfaff方程组$(\omega^\alpha)_a$的展开系数决定的.

最终可以看到,秩为$r$的Pfaff方程组与$h(=m-r)$维的分布$L^h$是相互确定、相互伴随的.
这说明,当给定一个Pfaff方程组时,便同时给定了一个光滑分布;反之亦然.
\end{proof}

取$L^h|_V = {\rm Span}_{\infty} \left\{(v_{r+1})^a,\cdots, (v_m)^a \right\}$和
$(\omega^\alpha)_a \ (1 \leqslant \alpha \leqslant r)$是命题\ref{chdf:thm_dwwd}中
相互伴随的分布$L^h$和一次外型式场.
因两者是相互对偶的,$\forall X^a\in L^h|_V$必然有$(\omega^\alpha)_a X^a =0$.
按照\S \ref{chdf:sec_frobenius}中论述,假设光滑分布$L^h|_V$的积分流形
是$(\imath,N)$(见定义\ref{chdf:def_solution});其中包含映射$\imath$的定义
是$\imath:N\to V\subset U \subset M$.此时虽然$TN$中的切矢量场与$L^h|_V$是同构的,
但需要用包含映射推前到$TV$中才能与$V$中的余切矢量场(即一次外型式场)
$(\omega^\alpha)_a \in \mathfrak{X}^*(V)$作用.$\forall u^a\in TN \cong L^h|_V$,必然有
\begin{equation}\label{chdf:eqn_solution-ed2}
    (\omega^\alpha)_a \bigl(\imath_{*}u^a\bigr) =0 \quad \Leftrightarrow \quad
    \bigl(\imath^*(\omega^\alpha)_a\bigr) (u^a) =0.
\end{equation}
本节开头说过,Pfaff方程组是在寻找一个子流形$(\imath,N)$使得$\imath^*(\omega^\alpha)_a=0$成立,
从上式可以看到:命题\ref{chdf:thm_dwwd}中描述的、与Pfaff方程组\eqref{chdf:eqn_Pfaff}伴随的
光滑分布$L^h$的积分流形$(\imath,N)$就满足此要求.
$\imath^*(\omega^\alpha)_a$在子流形$N$的切空间上式恒零元,在$N$外的空间不是恒零元.

我们再大致叙述一下这一过程:
\textcircled{1} 给定$m$维光滑流形$M$的一个开子集$U$(根据需要可适当将其缩小至$V$);
\textcircled{2} 给定开子集$U$上秩为$r$的Pfaff方程组\eqref{chdf:eqn_Pfaff}(即$r$个余切矢量场);
\textcircled{3} 通过命题\ref{chdf:thm_dwwd}中论述的方法找到与Pfaff方程组\eqref{chdf:eqn_Pfaff}相
互伴随的光滑分布$L^h$,分布维数是$h=m-r$;
\textcircled{4} 光滑分布$L^h$会对应一个积分流形$(\imath,N)$,见定义\ref{chdf:def_solution},
这个积分流形中的切矢量场满足式\eqref{chdf:eqn_solution-ed2},同时我们也把这个积分流形称为
Pfaff方程组\eqref{chdf:eqn_Pfaff}的积分流形;
\textcircled{5} 积分流形$(\imath,N)$的存在性条件就是下面叙述的Frobenius定理.

在叙述Frobenius定理之前,先看一下Pfaff方程组与一般偏微分微分方程组的关系.
给定一般的一阶偏微分方程组
\begin{equation}\label{chdf:eqn_pde1d}
    \frac{\partial y^\alpha(x)}{\partial x^i} = f_i ^\alpha (x^1,\cdots, x^m;y^1,\cdots,y^n),
    \quad 1 \leqslant \alpha \leqslant n, \quad 1 \leqslant i \leqslant m .
\end{equation}
其中$f_i ^\alpha$是自变量的光滑函数.上式可改写为
开子集$U\times V\subset \mathbb{R}^m\times \mathbb{R}^n$的Pfaff方程组
\begin{equation}\label{chdf:eqn_pde1d-pfaff}
    (\omega^\alpha)_a = ({\rm d}y^\alpha)_a - \sum_{i=1}^{m}f_i ^\alpha(x,y)({\rm d}x^i)_a,
     \quad 1 \leqslant \alpha \leqslant n.
\end{equation}
如果方程组\eqref{chdf:eqn_pde1d}有解$y^\alpha=g^\alpha(x^1,\cdots, x^m)$,
那么构建映射$\imath:U\to U\times V$为
\begin{equation}
    \imath(x^1,\cdots, x^m)=\left(x^1,\cdots, x^m;\ g^1(x),\cdots,g^n(x)\right).
\end{equation}
可以看出$(\imath,U)$是$\mathbb{R}^m\times \mathbb{R}^n$的嵌入子流形.
式\eqref{chdf:eqn_pde1d-pfaff}中描述的Pfaff方程组是流形$\mathbb{R}^m\times \mathbb{R}^n$中
开子集$U\times V$上的余切矢量场,可用映射$\imath$将其拉回至$U$中
\begin{equation*}
    \imath^*\bigl((\omega^\alpha)_a\bigr)=({\rm d}g^\alpha(x))_a - f_i ^\alpha\bigl(x,g(x)\bigr)({\rm d}x^i)_a
    =\left( \frac{\partial g^\alpha(x)}{\partial x^i} -f_i ^\alpha\bigl(x,g(x)\bigr) \right)({\rm d}x^i)_a .
\end{equation*}
因为$g^\alpha(x^1,\cdots, x^m)$是方程组\eqref{chdf:eqn_pde1d}的解,
所以显然有$\imath^*\bigl((\omega^\alpha)_a\bigr)=0$.


\index[physwords]{Frobenius定理!外型式}

\begin{theorem}\label{chdf:thm_frobenius-Diff-Form}
    设式\eqref{chdf:eqn_Pfaff}是定义在开子集$U\subset M$上的秩为$r$的Pfaff方程组,
    则它完全可积的充分必要条件是:满足下述条件之一即可(其中$1 \leqslant \alpha \leqslant r$)
    \begin{subequations}\label{chdf:eqn_frobenius-Diff-Form}
    \begin{align}
        ({\rm d} \omega^\alpha)_{ab} =& \sum_{\beta=1}^{r}(\phi^\alpha_\beta)_a \wedge (\omega^\beta)_b ,
           \label{chdf:eqn_frobenius-Diff-Form-a} \\
        0 =& ({\rm d} \omega^\alpha)_{ab} \wedge (\omega^1)_{c_1} \wedge (\omega^2)_{c_2}
           \cdots \wedge (\omega^r)_{c_r}, \label{chdf:eqn_frobenius-Diff-Form-b} \\
        (\omega^\alpha)_{a} =& \sum_{\beta=1}^{r} g^\alpha_\beta(x^1,\cdots,x^r) \cdot ({\rm d} x^\beta)_a,
        \quad \text{存在}\  g^\alpha_\beta \in C^\infty(V).
        \label{chdf:eqn_frobenius-Diff-Form-c}
    \end{align}
    \end{subequations}
    式\eqref{chdf:eqn_frobenius-Diff-Form}称为Frobenius条件.
    条件\eqref{chdf:eqn_frobenius-Diff-Form-a}要求$\forall p\in U$存在
    点$p$的开邻域$V\subset U$使得存在$r^2$个$(\phi^\alpha_\beta)_a\in \mathfrak{X}^*(V)$满足
    式\eqref{chdf:eqn_frobenius-Diff-Form-a}.
    条件\eqref{chdf:eqn_frobenius-Diff-Form-c}是指存在$g^\alpha_\beta(x^1,\cdots,x^r)\in C^\infty(V)$和
    局部坐标系$\{x\}$使得式\eqref{chdf:eqn_frobenius-Diff-Form-c}成立即可.
\end{theorem}
\begin{proof}
    定理证明可参考文献\parencite[\S 7.14-7.17]{spivak-dif-1}、
    \parencite[\S 3.2]{cc2001-zh}或其它类似书籍. %\parencite[\S 4.3]{chenwh2001}

    我们简单论述一下几个Frobenius条件的等价性.
    通过定理\ref{chmla:thm_0mod}极易证明条件\eqref{chdf:eqn_frobenius-Diff-Form-a}与
    条件\eqref{chdf:eqn_frobenius-Diff-Form-b}的等价性.


    对条件\eqref{chdf:eqn_frobenius-Diff-Form-c}取外微分可得到
    条件\eqref{chdf:eqn_frobenius-Diff-Form-a};这是说如果
    条件\eqref{chdf:eqn_frobenius-Diff-Form-c}成立,那么Pfaff方程组完全可积.
    反之,如果那么Pfaff方程组完全可积,从定理\ref{chdf:thm_frobenius}可知
    存在开子集$V$中的坐标面$\{(x^1,\cdots,x^m) \mid x^{r+1}=const,\cdots,x^{m}=const\}$是
    分布$L^h$的$h$维积分流形.在此坐标系下,Pfaff方程中的一次外微分型式场只能表示为
    \begin{equation}
        (\omega^\alpha)_{a} = \sum_{\beta=1}^{r} g^\alpha_\beta(x^1,\cdots,x^r) \cdot ({\rm d} x^\beta)_a,
        \qquad g^\alpha_\beta(x^1,\cdots,x^r) \in C^\infty(V).
    \end{equation}
    这就是条件\eqref{chdf:eqn_frobenius-Diff-Form-c}.
\end{proof}


%\subsection{例题}
在本节开头,我们描述了二维平面上的一个例子,下面把它拓展到$\mathbb{R}^3$中.
\begin{example}
    取$r=1$,在$\mathbb{R}^3$的凸开子集$U$上,给定Pfaff方程
    \begin{equation*}
        (\omega)_a = P(x,y,z)({\rm d}x)_a +Q(x,y,z)({\rm d}y)_a +R(x,y,z)({\rm d}z)_a  , \quad P,Q,R\in C^\infty(U).
    \end{equation*}
    通过式\eqref{chdf:eqn_frobenius-Diff-Form-b},很容易得到上述方程的完全可积条件,先计算其外微分
    \begin{equation}
        {\rm d}_b (\omega)_a = ({\rm d}P)_b\wedge({\rm d}x)_a +({\rm d}Q)_b\wedge({\rm d}y)_a +({\rm d}R)_b\wedge ({\rm d}z)_a .
    \end{equation}
    然后与$(\omega)_c$作外积
    \begin{equation*}
        {\rm d}_b (\omega)_a \wedge(\omega)_c= ({\rm d}P)_b\wedge({\rm d}x)_a \wedge(\omega)_c
          +({\rm d}Q)_b\wedge({\rm d}y)_a\wedge(\omega)_c +({\rm d}R)_b\wedge ({\rm d}z)_a\wedge(\omega)_c .
    \end{equation*}
    将此式展开可得Frobenius条件,计算过程留给读者完成.
    \begin{equation}
        P\left(\frac{\partial R}{\partial y} - \frac{\partial Q}{\partial z} \right)
       +Q\left(\frac{\partial P}{\partial z} - \frac{\partial R}{\partial x} \right)
       +R\left(\frac{\partial Q}{\partial x} - \frac{\partial P}{\partial y} \right) =0.
    \end{equation}
\end{example}


%需要更深入了解Frobenius定理的读者可参考文献\parencite[Ch.8]{westenholz-1978};
%那里给出了更多(物理、力学)的例题(约30道例题),以及更加详尽、全面地描述.

%%%%%%%%%%%%%%%%%%%%%%%%%%%%%%%%%%%%%%%%%%%%%%%%%%%%%%%%%%%%%%%%%%%%%%%%%%%%%%
\section{流形定向与带边流形}\label{chdf:sec_oriented-manifold}
\subsection{流形定向}
在$\mathbb{R}^3$中的直角坐标系可以分为左手和右手两种,这称为坐标系的{\kaishu 定向};
定向概念可以向高维空间推广,但三维空间中直观的左手或右手定则很难推广.
设$\mathbb{R}^3$中有两个原点重合的、正交归一坐标系$\boldsymbol{e}_i$和$\boldsymbol{f}_i$,
联系这个坐标系的是一个实正交矩阵,即$\boldsymbol{f}_i = a_i^j \boldsymbol{e}_j$;
这个矩阵的行列式$\det (a_i^j)=\pm 1$,当其值为$+1$时,我们称两个坐标系是
定向相同的,否则是定向相反的.这个方式可以推广到高维.


设$V$是$m$维矢量空间,有两组基矢$(e_i)^a$和$(\epsilon_i)^a$(一般不要求它们
正交归一),联系它们的是一个非奇异$m\times m$矩阵$A$,
即$(\epsilon_i)^a = A_i^j (e_j)^a$;当$\det(A)>0$时,我们称两组
基矢量定向相同,当$\det(A)<0$时,我们称两组基矢量定向相反;
也就是行列式的正负将所有基矢组分为两个等价类,此时可称其一为右手的,另一个
为左手的.需要注意,矢量空间的变换矩阵$A$是常数矩阵,它的行列式只能大于或者
小于零;这一方式可以推广至微分流形的切空间(一类矢量空间),但流形切空间是逐点不同的.
用不那么严格的语言来说:

\index[physwords]{定向}
\index[physwords]{定向!微分流形}

\begin{definition}\label{chdf:def_orientation}
    设$M$是$m$维光滑流形,其相互容许的坐标卡集
    合$\{(U,\varphi;x^i), (V,\psi;y^i), \cdots \}$构成
    流形$M$的开覆盖;如果当$U \cap V\neq \varnothing$时,
    坐标变换$\psi \circ \varphi^{-1}: \varphi(U \cap V )\to \psi(U \cap V)$的Jacobi行列
    式$\det(\frac{\partial y^i }{\partial x^j}) >0$,
    则称$M$是{\heiti 可定向}的光滑流形;满足Jacobi行列式大于零的两个坐标
    卡$(U,\varphi)$和$(V,\psi)$称为{\heiti 定向相符的}或{\heiti 同向的};
    由定向相符的坐标卡组成的极大坐标卡集合(也就是微分结构)称为流形$M$的一个{\heiti 定向}.
\end{definition}

上面的定义大意是,先在流形$M$取一个开覆盖集合(同时取定坐标系),
当点$p\in U \cap V\neq \varnothing$时,$p$点有两个坐标系
{\footnote{单独一个开子集$(U;x^i)$内,定向是确定的,或者说不存在定向问题.
因为$U$是能够被一个坐标系覆盖住,定向一定存在且相符合;
如果非要讨论定向,可取恒等坐标变换,那么Jacobi行列式为$+1$.}};
在$p$点切空间便有两组基矢量$\{(\frac{\partial }{\partial x^i})^a\}$和
$\{(\frac{\partial }{\partial y^i})^a\}$,这两组基矢量
坐标变换的Jacobi行列式(取式\eqref{chdm:eqn_tensor-component-trans}中逆变部分即可)
肯定不是零;当$p$在流形$M$上移动时,此行列式的正负
可能逐点不同,也就是有的点是正的,有的点是负的.如果能够通过
调节坐标取向(比如令$y^1=-y^1$) { \footnote{比如通过调节坐标取向已将
开子集$U$和$V$调整成定向相符.而$V\cap W\neq \varnothing$,
此时$V$中坐标必须和$U$中一致,不能再次调整$V$的坐标与$W$定向相符合,
只能调整$W$的坐标与$V$定向相符合;以此类推.}}
可以令所有点Jacobi行列式都大于零,
则流形$M$是可定向的.否则就是不可定向的,典型例子是M\"{o}bius带.

上面定义是由切空间来描述的,这个定义能很好地退化到$\mathbb{R}^3$中的定义;
更为常用的定义是由余切空间来描述,此时会用到外微分型式场.
先给出定义\ref{chdf:def_orientation}的对偶定义:
\begin{definition}\label{chdf:def_orientation-dual}
    如果$m$阶光滑流形$M$上\CJKunderwave{存在}一个连续的、处处非零的$m$阶外微分型式场$\omega$,
    则称$M$是{\heiti 可定向}的光滑流形;$\omega$称为流形$M$的{\heiti 定向}.
    反之,流形$M$不可定向.
\end{definition}
下面来分析两个定义的等价性.
假设$M$有局部坐标系$(U;x^i)$和$(V;y^\alpha)$且$U\cap V\neq \varnothing$,
在$U\cap V$上满足定义\ref{chdf:def_orientation-dual}中的外微分型式$\omega$可表示为
\begin{align}
    \omega_{a_1 \cdots a_m}=&h(x)\cdot ({\rm d}x^1)_{a_1} \wedge \cdots \wedge ({\rm d}x^m)_{a_m}
      =f(y) \cdot ({\rm d}y^1)_{a_1} \wedge \cdots \wedge ({\rm d}y^m)_{a_m} \notag \\
      =& f(y)\frac{\partial (y^1,\cdots, y^m)} {\partial (x^1, \cdots, x^m)}
      ({\rm d}x^1)_{a_1} \wedge \cdots \wedge({\rm d}x^m)_{a_m}. \label{chdf:eqn_hfJ}
\end{align}
定义\ref{chdf:def_orientation-dual}中还要求$h(x)$和$f(y)$是连续的、处处非零的,
那么必然有$h(x)>0$或$h(x)<0$,不存在任何一个开子集$U$,其上的$h(x)$可变正负号;$f(y)$也如此.
不妨假设$h>0$且$f>0$,如果任何一个小于零(比如$f<0$),只需令坐标取负号($y^1=-y^1$,其它坐标不变),
然后新的系数大于零了($-f(y)>0$).因$h>0$且$f>0$,故由式\eqref{chdf:eqn_hfJ}可知
坐标变换产生的Jacobi行列式$J'=\frac{\partial (y^1,\cdots, y^m)} {\partial (x^1, \cdots, x^m)}$处处大于零;
$J'$是定义\ref{chdf:def_orientation}中Jacobi行列式$J$的逆,即$J'=J^{-1}$;可见$J>0$等价于$J'>0$.
这便说明了两个定义是等价的.

如果$h(x)$可变正负号,那么$f(y)$也是可变正负号的,那么不能得出Jacobi行列式$J'>0$了,
定义\ref{chdf:def_orientation-dual}也就不等价于定义\ref{chdf:def_orientation}了.
即便只有一个点$p\in U$为零,其它点都大于零,那么$p$点的$J'$的正负号也是不能确定的.

我们要求$\omega$是$m$阶的,而流形$M$也是$m$阶的;由此可知
流形$M$上任意两个连续且处处非零的$m$阶外微分型式场$\omega$和$\omega'$只相差一个非零倍数,
即$\omega'=c\cdot \omega$,其中$c\in C^0(M)$.因为$c$连续,所以
要么$c>0$要么$c<0$,在任何点都有$c\neq 0$.
这更加清晰地说明了:定向只有两种,可称为左手定向或右手定向.

外微分型式场$\omega$与局部坐标系$(U;x)$定向相符合或者说同向是指
式\eqref{chdf:eqn_hfJ}中$h(x)>0$;如果$h(x)<0$则称定向相反.

%有了定向概念之后,我们只考虑可定向的流形.

%
%假定已按照定义\ref{chdf:def_orientation}取好定向,
%下面来描述如何诱导出用余切矢量给出的对偶定义.流形$M$任一点$p$都有
%余切空间作为切空间的对偶空间,在余切空间$T_p^*M$(当然是$m$维的)上可以定义一个$m$维外微分型式,
%比如$\omega_{a_1 \cdots a_m}=h(x)\cdot({\rm d}x^1)_{a_1} \wedge \cdots \wedge ({\rm d}x^m)_{a_m}$,
%需要注意这个$m$维外微分型式同构于实标量场.
%当切标架场在进行坐标变换时,产生的Jacobi行列式记为$J$.
%相应的余切标架场也在进行坐标变换,此时外微分型式场$\omega_{a_1 \cdots a_m}$会
%自动产生一个Jacobi行列式$J'$,此Jacobi行列式是
%定义\ref{chdf:def_orientation-dual}要求$\omega_{a_1 \cdots a_m}$连续
%且处处不为零(比如大于零),尤其是两个坐标卡重叠区域(如$U\cap V$),
%重叠区的坐标变换前后的$\omega_{a_1 \cdots a_m}$都大于零,
%$m$阶协变矢量间的变换公式可由式\eqref{chdm:eqn_tensor-component-trans}得到,刚好
%是定义\ref{chdf:def_orientation}中切矢量变换矩阵之逆;
%再把$m$阶协变矢量取全反对称得到外微分型式场,则这些变换矩阵恰好
%组成Jacobi行列式,记为$J'$,见命题\ref{chmla:thm_b2b}及注解\ref{chmla:rmk_b2b};
%因定义\ref{chdf:def_orientation-dual}中的外微分型式场处处非零且连续,
%所以$J$与$J'$的正负号是相同的,不会因坐标变换而变得不同;从而可知两个定义本质相同.


\index[physwords]{微分流形!带边}
\subsection{带边流形}\label{chdf:sec_manifold-boundary}
定义\ref{chdm:def_Dmanifold}中的开集不包含边界,比如包含边界的实心立方体在此定义下
便不是流形;为了更好地描述现实中的物体,需要给出带边流形的定义.引入记号:
\begin{align}
    \mathbb{R}^m_{+} \equiv & \{(x^1,\cdots,x^m)\in \mathbb{R}^m | x^1 \geqslant 0 \}, \label{chdf:eqn_Rn+} \\
    \partial \mathbb{R}^m_{+} \equiv &\{(x^1,\cdots,x^m)\in \mathbb{R}^m | x^1 = 0 \}
    = \mathbb{R}^{m-1} \times \{0\} . \label{chdf:eqn_pRn+}
\end{align}
如果局部坐标是从$x^0$开始的,则把上两式中的$x^1 \geqslant 0$换成$x^0 \geqslant 0$,及$x^1 = 0$换成$x^0 = 0$.
很明显$\mathbb{R}^m_{+}$是$\mathbb{R}^m$的子集,如果用$\mathbb{R}^m$中标准拓扑的定义来看,$\mathbb{R}^m_{+}$不是
开子集.我们采用如下策略:使用诱导拓扑(见定义\ref{chtop:def_induced-top})定义$\mathbb{R}^m_{+}$中的开集.
此时$\mathbb{R}^m_{+}$本身便是诱导拓扑下的开集;$\partial \mathbb{R}^m_{+}$称为$\mathbb{R}^m_{+}$的{\heiti 边界},
${\rm i}(\mathbb{R}^m_{+})=\mathbb{R}^m_{+}- \partial \mathbb{R}^m_{+}$称为$\mathbb{R}^m_{+}$的{\heiti 内部}.

\begin{definition}
    只需把定义\ref{chdm:def_Dmanifold}中的$\mathbb{R}^m$换成$\mathbb{R}^m_{+}$便
    可得到{\heiti 带边流形}定义,就不再重复叙述了.
    有了带边流形定义,定义\ref{chdm:def_Dmanifold}中的流形可称为无边流形,以示区分.
\end{definition}


在带边流形定义中,若有局部坐标$(U,\phi)$将点$p\in U$映射到$\partial \mathbb{R}^m_{+}$,
即$\phi(p)\in \partial \mathbb{R}^m_{+}$,则称点$p$为流形$M$的{\heiti 边界点},
所有这样的点构成的集合记为$\partial M$.

\begin{theorem}\label{chdf:thm_induced-orientation}
    设$M$是$m$为光滑带边流形,且$\partial M\neq \varnothing$.那么可由$M$的微分结构$\mathscr{A}$诱导
    出$\partial M$上的微分结构$\widetilde{\mathscr{A}}$,使得$\partial M$成为$m-1$维光滑无边流形.
    此时包含映射$(\imath,\partial M)$是$M$的正则嵌入闭子流形.
    如果$M$是可定向的,那么$\partial M$也是可(诱导)定向的.
\end{theorem}

上述定理证明可参阅\parencite[\S3.4]{cc2001-zh}定理4.1.
超曲面的诱导定向见\S \ref{chsm:sec_induced-VE}.

%我们将$M$的定向与$\partial M$的诱导定向作如下规定: %将在\S\ref{chrg:sec_volume}叙述.
%设带边流形$M$有局部坐标系$(U;x^i)$,
%%\begin{equation}
%%  \left\{((-)^m\cdot x^1,x^2,\cdots,x^{m-1},(-)^m\cdot x^m)\right\}.
%%\end{equation}
%对应的自然标架场表示的定向取为
%\begin{subequations}\label{chdf:eqn_induced-orientation-nt}
%\begin{align}
%    &\left\{  \left(\frac{\partial }{\partial x^1}\right)^a,
%    \left(\frac{\partial }{\partial x^2}\right)^a,\cdots,
%    \left(\frac{\partial }{\partial x^{m-1}}\right)^a,
%     \left(\frac{\partial }{\partial x^m}\right)^a \right\}
%    {\quad \color{red}\Leftrightarrow} \\
%    &({\rm d}x^1)_{a_1} \wedge ({\rm d}x^2)_{a_2} \wedge \cdots\wedge({\rm d}x^m)_{a_m}.
%\end{align}
%\end{subequations}
%%在以外微分型式场表示定向时,两个$(-)^m$相乘后为$+1$.
%当$\widetilde{U}=U\cap \partial M=\{(x^1,\cdots,x^m)\in U
%\ | \ x^1=0 \}\neq \varnothing $时,
%%\begin{equation}
%%    \widetilde{U}=U\cap \partial M=\left\{(x^1,\cdots,x^m)\in U|\ x^m=0\right\}\neq \varnothing
%%\end{equation}
%在$\widetilde{U}$上{\heiti 诱导定向}表示为
%\begin{subequations}\label{chdf:eqn_induced-orientation-nc}
%\begin{align}
%    &\left\{
%    \left(\frac{\partial }{\partial x^2}\right)^a,\cdots,
%    \left(\frac{\partial }{\partial x^{m}}\right)^a\right\}
%    {\quad \color{red}\Leftrightarrow} \\
%    & ({\rm d}x^2)_{a_2} \wedge \cdots\wedge({\rm d}x^{m})_{a_{m}}.
%\end{align}
%\end{subequations}
%将$M$和$\partial M$定向规定的如此复杂,是为了Stokes--Cartan定理\ref{chdf:thm_stokes-cartan}证明简洁.

\index[physwords]{定向!微分流形}

%%%%%%%%%%%%%%%%%%%%%%%%%%%%%%%%%%%%%%%%%%%%%%%%%%%%%%%%%%%%%%%%%%%%%%%%%%%%%%
\section{单位分解定理}\label{chdf:sec_partition-unity}
单位分解是近代数学中才产生的一个概念,它可将局部拼接成整体.
我们将略去定理的证明,可参考\textcite[\S 3.3]{cc2001-zh}.
%文献\parencite[\S 16]{munkres-1997-Analysis}提供了一份较易理解的证明,它
%只讨论了$\mathbb{R}^n$空间问题,没有讨论一般流形.

\index[physwords]{单位分解定理}

\begin{definition}
    设光滑流形$M$上存在实函数$f:M\to \mathbb{R}$,函数$f$的{\heiti 支撑集}(support set)是指
    使$f$取非零值的点集的闭包,
    记作${\rm supp} f \equiv \overline{\{p\mid f(p)\neq 0, \forall p\in M \}}$.
\end{definition}
如果$x\notin {\rm supp} f$,那么$x$必然有一邻域使得函数$f$在该邻域内恒为零.

\begin{definition}
    设$\Sigma$是光滑流形$M$的一个开覆盖;如果$M$的任意一个紧致子集只与$\Sigma$中
    有限个成员相交,则$\Sigma$是$M$的{\heiti 局部有限开覆盖}.
\end{definition}

%\begin{definition}\label{chdf:def_paracompact}
%    若光滑流形$M$任意开覆盖都有局部有限子覆盖,则$M$是{\heiti 仿紧}的.
%\end{definition}

\begin{theorem}\label{chdf:thm_partition-unity}
    (单位分解定理)设$\Sigma$是光滑流形$M$的一个开覆盖,则在$M$上{\heiti 存在}一族
    光滑函数$\{g_\alpha\}$,满足如下条件:
    {\bfseries (1)} 对每一个$\alpha$,$0\leqslant g_\alpha \leqslant 1$,
    支撑集${\rm supp}g_\alpha$是紧致的,并且有开集$W_i \in \Sigma$使
    得${\rm supp}g_\alpha \subset W_i$;
    {\bfseries (2)} $\forall p\in M$,存在$p$点的邻域$U$,它只与有限多个支撑集${\rm supp}g_\alpha$相交;
    {\bfseries (3)} $\sum_{\alpha} g_\alpha =1$.
\end{theorem}
条件(2)一般称为{\kaishu 局部有限性};这说明条件(3)左边的$g_\alpha$只有有限个不为零,
故和式是平凡收敛的.{\kaishu 函数族$\{g_\alpha\}$称为从属于开覆盖$\Sigma$的单位分解}.
%这个定理也表明任何光滑流形都是仿紧的.





%%%%%%%%%%%%%%%%%%%%%%%%%%%%%%%%%%%%%%%%%%%%%%%%%%%%%%%%%%%%%%%%%%%%%%%%%%%%%%
\section{流形上的积分}\label{chdf:sec_integral-on-manifold}
在数学分析中讲述过重积分定义与计算,以及重积分有变量替换定理,
设$\Omega$和$\Omega'$是$\mathbb{R}^m$中的开集,且有微分同胚
映射$\phi:\Omega'\to \Omega$.若标量函数$f$在其定义域是连续的,则有
\begin{equation}\label{chdf:eqn_multiple-integral}
    \int_{\Omega} f(x) {\rm d}x^1 \cdots {\rm d}x^m = \int_{\Omega} f\circ\phi(y)
    \left|\frac{\partial (x^1, \cdots, x^m)}{\partial (y^1,\cdots, y^m)}
     \right|{\rm d}y^1 \cdots {\rm d}y^m .
\end{equation}
变量替换后出现了坐标变换的Jacobi行列式的绝对值,这是多重积分中的基本定理.
下面我们会把积分定义推广到流形上,关键问题是如何进行大范围的、与局部坐标选取无关的处理;
积分是联系局部与整体的有力工具,在数学、物理上都有着重要应用.

\index[physwords]{积分}
\index[physwords]{积分!基本公式}

设$M$是$m$维的、已定向的、$C^\infty$微分流形.设$\omega \in A^m(M)$,
则$\omega$的{\heiti 支撑集}(support set)定义为
\begin{equation}
    {\rm supp}\omega = \overline{\{p\in M \ |\ \omega(p) \neq 0 \}}.
    \qquad \text{注意是闭包}
\end{equation}
我们将积分学的讨论限制在有{\kaishu 紧致支撑集的外微分型式场},要定义的积分是一个
{\kaishu 线性映射},即$\int : A^m(M)\to \mathbb{R}$.

先从简单情形开始,假定用一个坐标域就能覆盖住流形$M$,并设其坐标为$(M,\varphi;x^i)$.
这样$\omega$的紧致支撑集自然是$M$的子集;假设有局部坐标表达(在此例中局部即整体)
\begin{equation}\label{chdf:eqn_tmp01}
    \omega_{a_1 \cdots a_m} = f(x) ({\rm d}x^1)_{a_1} \wedge \cdots \wedge({\rm d}x^m)_{a_m},
    \qquad f(x)\in C^\infty(M).
\end{equation}
在数学分析中,标量函数$f(x)$的黎曼积分早有定义,并且是一个有限实数;
在此处,我们可以直接将流形$M$上外微分型式场$\omega$的积分定义成黎曼积分,即
\begin{equation}\label{chdf:eqn_def-integral-omega}
    \int_{M} \omega \equiv \int_{M} f(x) ({\rm d}x^1)_{a_1} \wedge \cdots \wedge({\rm d}x^m)_{a_m} \overset{def}{=}
    \int_{\varphi(M)} (f\circ\varphi^{-1}) {\rm d}x^1 \cdots {\rm d}x^m 
\end{equation}
下面需要验证,这个定义与局域坐标卡选取无关,设$(M,\psi;y^i)$是此流形的另一套坐标系,$\omega$表达式变为
\setlength{\mathindent}{0em}
\begin{equation*} %\label{chdf:eqn_tmp02}
    \omega_{a_1 \cdots a_m} = g(y) ({\rm d}y^1)_{a_1} \wedge \cdots \wedge({\rm d}y^m)_{a_m}
    =g(y)\frac{\partial (y^1,\cdots, y^m)} {\partial (x^1, \cdots, x^m)}
    ({\rm d}x^1)_{a_1} \wedge \cdots \wedge({\rm d}x^m)_{a_m}.
\end{equation*} \setlength{\mathindent}{2em}
由不同坐标系下的不同表达式\eqref{chdf:eqn_tmp01}和上式,可说明有下式成立
\begin{equation}\label{chdf:eqn_tmpxy1}
    f\circ\varphi^{-1} = (g\circ \psi^{-1} )\circ (\psi \circ \varphi^{-1})
    \frac{\partial (y^1,\cdots, y^m)} {\partial (x^1, \cdots,x^m)} .
\end{equation}
因坐标卡$\psi$和$\varphi$是定向相符的,故坐标变换的Jacobi行列式处处为正,即
\begin{equation}
    \frac{\partial (y^1,\cdots, y^m)} {\partial (x^1,\cdots, x^m)} > 0  \quad \text{和} \quad
    \frac{\partial (x^1,\cdots, x^m)}{\partial (y^1,\cdots, y^m)} > 0 .
\end{equation}
有了这些准备,容易验证积分定义与局部坐标卡无关.
\begin{align*}
    \int_{M} \omega & = \int_{\psi(M)} (g\circ\psi^{-1}) {\rm d}y^1 \cdots {\rm d}y^m \\
    &=\int (g\circ\psi^{-1})\circ(\psi \circ \varphi^{-1})
    \frac  {\partial (y^1,\cdots, y^m)} {\partial (x^1, \cdots, x^m)}
    {\rm d}x^1 \cdots {\rm d}x^m  \\
    &=   \int_{\varphi(M)} (f\circ\varphi^{-1}) {\rm d}x^1 \cdots {\rm d}x^m .
\end{align*}
从上式可以看出,如果两个坐标卡定向不同,Jacobi行列式会产生一个额外的负号.
从上式还可以看出,$m$阶外微分型式场的积分定义与式\eqref{chdf:eqn_multiple-integral}相符合.
如果不是,比如是一个二阶对称张量场,那么积分定义都成问题;当作变量替换时,二阶张量场无法由
自身变换产上一个Jacobi行列式,也就无法与基本定理\eqref{chdf:eqn_multiple-integral}相符合.
微分流形中的积分定义(以及其它任何定义、公理等)原则上应能退化到$\mathbb{R}^m$中的积分定义
(以及相应定义、公理等),除了$m$阶外型式场(标量函数场与其同构),其它张量场都很难给出积分定义.

\index[physwords]{积分!流形}

下面考虑一般情形,设微分流形$M$有定向相符合的坐标卡开覆盖集合$(U_\alpha,\varphi_\alpha;x^i_\alpha)$;
$m$阶外微分型式场$\omega \in A^m(M)$的局部坐标表达式为
\begin{equation}
    \omega_{a_1 \cdots a_m}|_{U_\alpha} = f_{\alpha} ({\rm d}x^1_{\alpha})_{a_1}
       \wedge \cdots \wedge({\rm d}x^m_{\alpha})_{a_m},  \qquad f_{\alpha}\in C^\infty(U_\alpha).
\end{equation}
根据单位分解定理\ref{chdf:thm_partition-unity},在$M$上有从属于$\{U_\alpha\}$的单位
分解$\{h_\alpha\}$,其中${\rm supp}h_\alpha \subset U_\alpha$,$h_\alpha \in C^\infty(M)$,
$\sum_{\alpha}h_\alpha =1$,$1 \geqslant h_\alpha \geqslant 0$,所以有
\begin{equation}
    \omega = \omega \cdot \sum_{\alpha} h_\alpha =  \sum_{\alpha}  h_\alpha \cdot \omega,
\end{equation}
需注意,上式中的“$\cdot$”就是普通的乘法,不是复合映射.
因${\rm supp}\omega$是紧致的,所以上式右端只有有限项非零,即上式右端平凡收敛.
\begin{equation}\label{chdf:eqn_def-integral-omega-partition}
    \int_{M} \omega \overset{def}{=} \int_{M} \sum_{\alpha}  h_\alpha \cdot \omega =
    \sum_{\alpha}\int_{\varphi_\alpha(U_\alpha)}
    \bigl( (h_\alpha \cdot f_\alpha )\circ\varphi^{-1}_\alpha\bigr)
    {\rm d}x^1_\alpha \cdots {\rm d}x^m_\alpha .
\end{equation}
自然需要验证这个定义与局坐标系选取无关,以及与单位分解选取无关.
设$(V_\lambda,\psi_\lambda;y^i_\lambda)$是$M$的另一个定向相符的坐标卡,同时
它也是$M$的局部有限开覆盖,再设$\{g_\lambda\}$是从属于$\{V_\lambda\}$的单位分解;
同时假定$\{U_\alpha\}$与$\{V_\lambda\}$的定向相符合.
$m$阶外微分型式场$\omega \in A^m(M)$在$\{V_\lambda\}$中的局部坐标表达式为
\begin{equation}
    \omega_{a_1 \cdots a_m}|_{V_\lambda} = w_{\lambda} ({\rm d}y^1_{\lambda})_{a_1}
    \wedge \cdots \wedge({\rm d}y^m_{\lambda})_{a_m},  \qquad w_{\lambda}\in C^\infty(V_\lambda).
\end{equation}
当$U_\alpha \cap V_\lambda \neq \varnothing$时,两个坐标系定向是相符合的,
仿照式\eqref{chdf:eqn_tmpxy1}可得其坐标变换
\begin{equation}\label{chdf:eqn_tmpxayl104}
    f_\alpha\circ\varphi_\alpha^{-1} = (w_\lambda \circ \psi^{-1}_\lambda )
      \circ (\psi_\lambda \circ \varphi^{-1}_\alpha)\cdot
    \frac{\partial (y^1_\lambda,\cdots, y^m_\lambda)}
    {\partial (x^1_\alpha, \cdots, x^m_\alpha)} .
\end{equation}
由于$\{h_\alpha\}$和$\{g_\lambda\}$是单位分解,所以有
\begin{equation}
    \left. \left(\sum_{\alpha}h_\alpha\right) \right|_{V_\lambda} = 1, \qquad
    \left. \left(\sum_{\lambda}g_\lambda\right) \right|_{U_\alpha} = 1 .
\end{equation}
我们来计算积分定义\eqref{chdf:eqn_def-integral-omega-partition}右端的表达式:
\begin{align*}
    &\sum_{\alpha}\int_{\varphi_\alpha(U_\alpha)}   \bigl( (h_\alpha \cdot f_\alpha )
    \circ\varphi^{-1}_\alpha\bigr)\cdot  {\rm d}x^1_\alpha \cdots {\rm d}x^m_\alpha \\
    =&\sum_{\alpha}\int_{\varphi_\alpha(U_\alpha)}  \left(\sum_{\lambda}g_\lambda \circ
    \varphi^{-1}_\alpha \right) \cdot
    \bigl( (h_\alpha \cdot f_\alpha )\circ \varphi^{-1}_\alpha\bigr)
    \cdot {\rm d}x^1_\alpha \cdots {\rm d}x^m_\alpha  \\
    =&\sum_{\alpha\lambda} \int_{\varphi_\alpha(U_\alpha\cap V_\lambda)}  \left( (g_\lambda \cdot
       h_\alpha \cdot f_\alpha ) \circ \varphi^{-1}_\alpha \right) \cdot
    {\rm d}x^1_\alpha \cdots {\rm d}x^m_\alpha \\
    =&\sum_{\alpha\lambda} \int_{\varphi_\alpha(U_\alpha\cap V_\lambda)}  \left( (g_\lambda \cdot
    h_\alpha \cdot  w_\lambda \circ \psi^{-1}_\lambda )
    \circ (\psi_\lambda \circ \varphi^{-1}_\alpha)
    \cdot\frac{\partial (y^1_\lambda,\cdots, y^m_\lambda)}
    {\partial (x^1_\alpha, \cdots, x^m_\alpha)}  \right) \cdot
    {\rm d}x^1_\alpha \cdots {\rm d}x^m_\alpha \\
    =&\sum_{\alpha\lambda} \int_{\psi_\lambda(U_\alpha\cap V_\lambda)}
    \left( (g_\lambda \cdot
    h_\alpha \cdot w_\lambda) \circ \psi^{-1}_\lambda
     \right) \cdot  {\rm d}y^1_\lambda \cdots {\rm d}y^m_\lambda \\
   =&\sum_{\lambda} \int_{\psi_\lambda(V_\lambda)}  \left( (g_\lambda
     \cdot w_\lambda ) \circ \psi^{-1}_\lambda \right)
     \cdot  {\rm d}y^1_\lambda \cdots {\rm d}y^m_\lambda
\end{align*}
证明过程请参考定义式\eqref{chdm:eqn_f-cdot-g-circo-gamma}.
上面这个长式推导说明积分定义\ref{chdf:eqn_def-integral-omega-partition}与局部坐标选取
以及从属单位分解选取无关;即此定义是大范围的(或整体的).
除此以外,还需验证本节开头提到的:积分是一个线性映射.
\begin{align*}
    \int_{M} (\omega + k\cdot \eta ) =&
    \sum_{\alpha}\int_{\varphi_\alpha(U_\alpha)}  \bigl( (h_\alpha \cdot (\omega_\alpha+k\cdot \eta_\alpha ) )
    \circ\varphi^{-1}_\alpha\bigr)\cdot  {\rm d}x^1_\alpha \cdots {\rm d}x^m_\alpha \\
    =&\sum_{\alpha}\int_{\varphi_\alpha(U_\alpha)}  \bigl( (h_\alpha \cdot \omega_\alpha )
    \circ\varphi^{-1}_\alpha\bigr)\cdot  {\rm d}x^1_\alpha \cdots {\rm d}x^m_\alpha \\
     &+\sum_{\alpha}\int_{\varphi_\alpha(U_\alpha)}  \bigl( (h_\alpha \cdot k\cdot\eta_\alpha )
    \circ\varphi^{-1}_\alpha\bigr)\cdot  {\rm d}x^1_\alpha \cdots {\rm d}x^m_\alpha \\
    =&\int_{M} \omega  +k\cdot \int_{M} \eta .
\end{align*} %\setlength{\mathindent}{2em}
其中$k\in \mathbb{R}$,$\omega_\alpha$、$\eta_\alpha$分别是$\omega$、$\eta$的分量指标;
这便验证了积分是线性的.

\begin{example}
    利用单位分解定理求积分一例$\int_{-\pi}^{\pi}x {\rm d}x$.
\end{example}
我们先给实数轴$\mathbb{R}$建立一个单位分解.
令$f:\mathbb{R}\to \mathbb{R}$是由下式定义的函数
\begin{equation}
    f(x)=\begin{cases}
        (1+\cos x)/2, & -\pi \leqslant x \leqslant \pi ; \\
        0, & \text{其它} .
    \end{cases}
\end{equation}
$f$是$C^1$类函数.对每个整数$n\geqslant 0$,令$\phi_{2n+1}(x)=f(x-n \pi)$;
对每个整数$n\geqslant 1$,令$\phi_{2n}(x)=f(x+n \pi)$;
那么,函数组$\{\phi_i\}$构成了$\mathbb{R}$上的一个单位分解.
$\{\phi_i\}$的支撑集$S_i$是一个形如$[k\pi,(k+2)\pi]$的闭区间,
这当然是紧致集,并且$\mathbb{R}$上任一点都有一个小邻域$U$,
$U$至多与集合$\{S_i\}$中的三个相交;容易验证$\sum_i \phi_i =1$.
下面计算积分.
\begin{align*}
    &\int_{-\pi}^{\pi}x {\rm d}x = \int_{-\pi}^{\pi}x\sum_i\phi_i {\rm d}x
    =\int_{-\pi}^{\pi}x \phi_1 {\rm d}x 
    +\int_{-\pi}^{\pi}x \phi_2 {\rm d}x
    +\int_{-\pi}^{\pi}x \phi_3 {\rm d}x \\
    &=\int_{-\pi}^{\pi} \frac{x}{2} (1+\cos x) {\rm d}x 
    +\int_{-\pi}^{0}\frac{x}{2} \bigl(1+\cos (x+\pi)\bigr) {\rm d}x
    +\int_{0}^{\pi}\frac{x}{2} \bigl(1+\cos (x-\pi)\bigr) {\rm d}x \\
    &= 0+\left(-1-\pi^2/4\right) + \left(1+\pi^2/4\right) =0.
\end{align*}
就本例来说直接计算比单位分解更简单;
上面只是展示一下如何用单位分解定理来计算积分,
可以初步感性认知单位分解定理下的积分.\qed


\index[physwords]{Stokes--Cartan定理}

下面叙述Stokes--Cartan定理,这是流形论积分学中一个极为重要的定理,
现在所用表述形式是由Cartan给出并证明的.
\begin{theorem}\label{chdf:thm_stokes-cartan}
    设$M$是$m$维、已定向、\CJKunderwave{带边}光滑流形,$\omega \in A^{m-1}(M)$,则
    \begin{equation}\label{chdf:eqn_stokes-cartan}
        \int_{M} {\rm d} \omega =  \int_{\partial M} \imath^{*} \omega .
    \end{equation}
    其中$\partial M$是带边流形$M$的边界,具有从$M$诱导的定向(见\S\ref{chsm:sec_induced-VE}末尾).
    映射$\imath:\partial M \to M$是包含映射,
    使$(\imath,\partial M)$成为$M$的正则嵌入闭子流形.
\end{theorem}
\begin{proof}
    证明过程可参考任一黎曼几何教材,
    如文献\parencite[\S 3.4]{cc2001-zh}或\parencite[\S 8.6]{spivak-dif-1}.
%    需注意$\omega \in A^{m-1}(M)$是$m$维光滑流形$M$上的$m-1$阶外微分型式场,无论是在
%    流形$M$上,还是在$\partial M$上,积分定义都略显含混;
%    需要用包含映射将其拉回到$\partial M$上,
%    使其成为$\partial M$上的外微分型式场,然后才有良性积分定义.
%    等号右端的正负号与流形的定向有关.
\end{proof}


\begin{example}
    设$M$是$\mathbb{R}^2$上有界开子集的闭包,其边界是$C^r(r>0)$简单闭曲线.
    $\omega_b = f ({\rm d}x)_b + g ({\rm d}y)_b$是$M$上的1型式场;
    其外微分是${\rm d}_a\omega_b= (\frac{\partial g}{\partial x}
    -\frac{\partial f}{\partial y})({\rm d}x)_a \wedge ({\rm d}y)_b$,
    由Stokes--Cartan定理\eqref{chdf:eqn_stokes-cartan}可得
    \begin{equation}
        \int_{M} {\rm d}_a \omega_b = \iint_{M} \left(\frac{\partial g}{\partial x}
        -\frac{\partial f}{\partial y}\right) ({\rm d}x)_a \wedge ({\rm d}y)_b
        =  \int_{\partial M} \left(f {\rm d}x + g {\rm d}y \right).
    \end{equation}
    这是经典的Green公式. \qed
\end{example}

\begin{example}
    设$M$是$\mathbb{R}^3$上有界开子集的闭包,其边界是$C^r(r>0)$简单闭曲面$S$.
    $\omega_{ab}$是$M$上的2次型式场,其表达式
    \begin{align}
       \omega_{ab} =& P({\rm d}y)_a\wedge ({\rm d}z)_b +
          Q({\rm d}z)_a\wedge ({\rm d}x)_b+ R({\rm d}x)_a\wedge ({\rm d}y)_b . \\
       {\rm d}_c\omega_{ab} =& \left(\frac{\partial P}{\partial x}
        + \frac{\partial Q}{\partial y} +\frac{\partial R}{\partial z} \right)
       ({\rm d}x)_c\wedge ({\rm d}y)_a\wedge ({\rm d}z)_b  .
    \end{align}
    上面同时给出了$\omega_{ab}$的外微分表达式.
    由Stokes--Cartan定理\eqref{chdf:eqn_stokes-cartan}可得
    \begin{equation*}
        \iiint_{M} \left(\frac{\partial P}{\partial x}
        + \frac{\partial Q}{\partial y} +\frac{\partial R}{\partial z} \right)
        ({\rm d}x)_c\wedge ({\rm d}y)_a\wedge ({\rm d}z)_b
        =  \oiint_{S} \left(P {\rm d}y {\rm d}z +
        Q {\rm d}z {\rm d}x + R {\rm d}x {\rm d}y \right).
    \end{equation*}
    这是经典的高斯散度公式. \qed
\end{example}

\begin{example}
    设$M$是$\mathbb{R}^3$中一块有向曲面,其边界$\partial M$是光滑简单闭曲线,
    具有从$M$诱导的定向(见\S\ref{chsm:sec_induced-VE}末尾).
    $P$、$Q$、$R$是包含$M$在内的一个区域上的光滑函数,命
    $ \omega_b= P({\rm d}x)_b+  Q({\rm d}y)_b + R({\rm d}z)_b $.
    对此式取外微分,有
    \begin{align*}
        {\rm d}_a\omega_{b} = &
        \left(\frac{\partial R}{\partial y}-\frac{\partial Q}{\partial z}\right) ({\rm d}y)_a\wedge ({\rm d}z)_b +
        \left(\frac{\partial P}{\partial z}-\frac{\partial R}{\partial x}\right) ({\rm d}z)_a\wedge ({\rm d}x)_b \\
        &+ \left(\frac{\partial Q}{\partial x}-\frac{\partial P}{\partial y}\right) ({\rm d}x)_a\wedge ({\rm d}y)_b .
    \end{align*}
    从Stokes--Cartan定理可以得到
    \begin{align*}
        \int_{\partial M} P{\rm d}x+  Q{\rm d}y + R{\rm d}z =  &    \iint_{M} 
        \left(\frac{\partial R}{\partial y}-\frac{\partial Q}{\partial z}\right) {\rm d}y {\rm d}z +
        \left(\frac{\partial P}{\partial z}-\frac{\partial R}{\partial x}\right) {\rm d}z {\rm d}x \\
        &+\left(\frac{\partial Q}{\partial x}-\frac{\partial P}{\partial y}\right) {\rm d}x {\rm d}y .
    \end{align*} %\setlength{\mathindent}{2em}
    上式正好是经典的Stokes公式. \qed
\end{example}


\section*{小结}
本章主要参考了\parencite{cc2001-zh}、\parencite{chenwh2001}相应章节.

积分的定义\eqref{chdf:eqn_def-integral-omega-partition}是不涉及度规的,
故它适用于正定以及不定度规场.


%%%%%%%%%%%%%%%%%%%%%%%%%%%%%%%%%%%%%%%%%%%%%%%%%%%%%%%%%%%%%%%%%%%%%%%%%%%%%%%%%%%%%
%%%%%%%%%%%%%%%%%%%%%%%%%%%%%%%%%%%%%%%%%%%%%%%%%%%%%%%%%%%%%%%%%%%%%%%%%%%%%%%%%%%%%
%%%%%%%%%%%%%%%%%%%%%%%%%%%%%%%%%%%%%%%%%%%%%%%%%%%%%%%%%%%%%%%%%%%%%%%%%%%%%%%%%%%%%
\printbibliography[heading=subbibliography,title=第\ref{chdf}章参考文献]
\endinput
