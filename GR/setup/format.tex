% !Mode:: "TeX:UTF-8"
% 此文件从2018.11.10开始写作
%%%%%%%%%%%%%%%%%%%%%%%%%%%%%%%%%%%%%%%%%%%%%%%%%%%%%%%%%%%%%%%%%%%%%%%%%%%%%%%%%%%%%%%%%%%%%

\setcounter{tocdepth}{3}     %%设置在 目录 的显示的章节深度
\setcounter{secnumdepth}{3}  %%设置章节的编号深度
\allowdisplaybreaks[4]  %%允许公式跨页,即可跨4页


% 设置水印内容
%\SetWatermarkLightness{0.97}
%\SetWatermarkText{27257957}

%% 
%\newcommand{\dbar} {\mathrm{d}\kern{-4.3pt}\bar{\small\phantom{q}}\kern{-0.7pt}}
%\newcommand{\bbar} {\mathrm{b}\kern{-4.3pt}\bar{\small\phantom{q}}\kern{-0.7pt}}

% 定义小号圆圈
%\newcommand{\circo}{~\raisebox{1pt}{\tikz \draw[line width=0.5pt] circle(1.0pt);}~}
% 定义 overset 后牛顿中的Gamma
\newcommand{\oversetmy}[2]{\overset{\mathclap{\scriptscriptstyle #1}}{#2}\vphantom{#2}}

% 定义 (p,q) 型张量
\newcommand{\Tpq}[2]{\binom{#1}{#2}}

%% A Lie derivative % 定义Lie导数
\renewcommand*{\mathsterling}{%
    \text{%
        \fontencoding{T1}%
        \fontfamily{lmss}%
        \fontshape{it}%
        \selectfont
        \pounds
    }%
}
\newcommand{\Lie}{\pounds} 


\numberwithin{equation}{section}
\renewcommand{\theequation}{\thesection-\arabic{equation}}  %将公式中的 . 换成 -

%\newcommand{\rqed}{\hfill $\blacksquare$}
\renewcommand{\qed}{\hfill $\blacksquare$}

%定义版面
%\geometry{top=2.5cm, bottom=2.5cm,left=3cm,right=3cm}

%定义超链接
\hypersetup{colorlinks,bookmarksnumbered=true,%bookmarks=true,
            pdfstartview=FitH,linkcolor=black,anchorcolor=black,%
            citecolor=black}

%\theoremstyle{definition}
%\theoremstyle{plain}
%\theoremstyle{remark}


\newtheoremstyle{mystyle} % name
{3pt} % Space above, empty = `usual value'
{3pt} % Space below
{\fangsong } % Body font
{0pt} % Indent amount(empty = no indent, \parindent = para indent)
{\bfseries } % Theorem head font
{\ } % Punctuation after theorem head
{0.2em } % Space after theorem head
{} % Theorem head spec (can be left empty, meaning 'normal' )
\theoremstyle{mystyle}

\renewcommand{\proofname}{\heiti{证明} }

\newcounter{countthm} [chapter]
\renewcommand{\thecountthm}{\thechapter.\arabic{countthm}}
\newtheorem{axiom} [countthm] {公理}
\newtheorem{corollary}[countthm]{推论}
\newtheorem{lemma}[countthm]{引理} 
\newtheorem{proposition}[countthm]{命题}
\newtheorem{theorem}[countthm]{定理}
\newtheorem{remark}[countthm]{注}
\newtheorem{definition}[countthm]{定义}
\newtheorem{example} [countthm] {\rm \noindent \bfseries 例 }


% 定义习题、问题、解答、例题编号
\newcounter{countEX}[chapter]
\renewcommand{\thecountEX}{\thechapter-\arabic{countEX}}
\newtheorem{exercise} [countEX]  {\rm \noindent \bfseries 练习 }



%定义公式新环境
%\newcounter{myeq}
%\newenvironment{myeq}[0]
%{\refstepcounter{myeq}\equation}
%{\tag{\themyeq}\endequation}

%定义选读环境
%\newcounter{myxdnum}[chapter]
%\newenvironment{myxd}[0]
%{ %新环境定义开始
%	\stepcounter{myxdnum}
%    \vspace{5mm}
%	\leftline{{\rule{5ex}{1.7ex}} 
%		\heiti{[选读 \thechapter-\arabic{myxdnum}]}
%		{\rule{40ex}{1.7ex}}
%	}
%	\kaishu
%}
%{ %新环境定义结束
%    \par  %换行
%	\rightline {
%		\rule{30ex}{1.7ex}
%		\heiti{[选读\thechapter-\arabic{myxdnum}结束]}
%		\rule{10ex}{1.7ex}
%       \vspace{5mm}
%	}
%}





%%
%%索引
%\makeatletter
%%%define index items layout
%\def\@idxitem{\par\addvspace{7\p@ \@plus 3\p@ \@minus 3\p@}\hangindent 17\p@}
%%%define index subitems layout
%\def\subitem{\par\hangindent 0\p@ \hspace*{0\p@}}
%%%define index subsubitems layout
%\def\subsubitem{\par\hangindent 0\p@ \hspace*{0\p@}}
%%%define vspace above lettergroup name
%\def\indexspace{\par\addvspace{12\p@ \@plus 2\p@ \@minus 2\p@}} 
%\patchcmd\theindex{\indexname}{\indexname\vspace{5pt}}{}{}
%\makeatother




    
\endinput 
%%%%%%%%%%%%%%%%%%%%%%%%%%%%%%%%%%%%%%%%%%%%%%%%%%%%%%%%%%%%%%%%%%%%%%%%%%%%%%%%%%%%%%%%%%%%%